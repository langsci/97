\BackTitle{Die Sprachwissenschaft}
\BackBody{\textit{Die Sprachwissenschaft} ist das Hauptwerk des deutschen Sinologen und Sprachwissenschaftlers Georg von der Gabelentz (1840–1893), eines späten Vertreters der humboldtschen Richtung der Sprachforschung im 19. Jahrhundert. Als es erschien, wurde das Buch von einer Fachlinguistik, die sich schon längst auf den positivistischen Weg der Junggrammatiker begeben hat, nur kühl rezipiert. Ab der Mitte des 20. Jahrhunderts wurde das Buch jedoch immer öfter als Bindeglied zwischen den Denkströmungen des 19. Jahrhunderts und dem aufkommenden Strukturalismus des 20. Jahrhunderts betrachtet.

Das Buch erschien in zwei Auflagen; die erste zu Gabelentz' Lebzeiten, im Jahr 1891, und die zweite posthum im Jahr 1901, die von Albrecht Graf von der Schulenburg (1865–1902), einem Neffen und Schüler Gabelentz', erheblich überarbeitet und erweitert wurde. Diese kritische Ausgabe bietet zum ersten Mal den Text der beiden Auflagen in einer Form dar, die es dem Leser ermöglicht, die beiden Auflagen leicht zu vergleichen. Dadurch stellt diese Ausgabe eine wertvolle Quelle für historiographische Forschung zum Stand der Linguistik an der Schwelle zum 20. Jahrhundert dar.

}
% \dedication{Change dedication in localmetadata.tex}
\typesetter{James McElvenny, Sebastian Nordhoff}
% \proofreader{Change proofreaders in localmetadata.tex}              
\renewcommand{\lsSeries}{classics}  
\renewcommand{\lsSeriesNumber}{4}  
\renewcommand{\lsURL}{http://langsci-press.org/catalog/book/97} 
\author{Georg von der Gabelentz\newlineCover \vspace*{8cm}{\Large herausgegeben von}\newlineCover Manfred Ringmacher\newlineCover James McElvenny}
\title{Die Sprachwissenschaft}
\subtitle{Ihre Aufgaben, Methoden und bisherigen Ergebnisse}
	 
\renewcommand{\lsISBNdigital}{978-3-946234-34-0}
\renewcommand{\lsISBNhardcover}{978-3-946234-35-7}
\renewcommand{\lsISBNsoftcover}{978-3-946234-36-4}
\renewcommand{\lsISBNsoftcoverus}{978-1-530457-64-9}


