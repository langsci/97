
Gehen wir aber nun nach Anleitung der Agglutinationstheorie auf die Vorzeit der uns bekannten Sprachstämme zurück, suchen wir hinter den Agglutinationen alte syntaktische Gebilde: so ergeben sich manche überraschende Beobachtungen. Der Indogermane hat vor Alters das Verbum stets vor seinem logischen Subjecte genannt, also nach Art der Semiten und Malaien das verbale Prädicat zum psychologischen Subjecte gemacht; *\textit{ed-mi} = ich esse. Der Semite dagegen nannte die vollendete That vor, 
die noch unvollendete dagegen nach ihrem Subjecte: arabisch:
% \corr{1891 und 1901}{\<\setarab  \vocalize qatalta>}{\<\setarab  qatalta>}
\textit{qatal-ta}, du hast getödtet, 
% \marginpar{{\<\setarab  qatalt"a>}}
dagegen: \retro{\arabictext{taqtulu}}{\arabictext{taqtlu}} \textit{ta-qtulu} du wirst tödten; mit anderen Worten: von der fertigen Thatsache sagte er aus, \fed{{\textbar}357{\textbar}}\phantomsection\label{fp.357} wer ihr Urheber war, vom thätigen Subjecte aber sagte er aus, was es im Werke hatte. \sed{Und so bieten überall die Composition und die Agglutination fossile Stellungsgesetze.} Es lohnte sich wohl, unter diesem Gesichtspunkte die wunderlichen Conjugationsgebilde amerikanischer Sprachen zu betrachten.




% \arabictext{qatalt}
% 
% \arabictext{qatalta}
% 
% {\<\setarab \novocalize qatalt>}
% 
% {\<\setarab \novocalize qatalta>}

nothing {\<\setarab  qatalta>}

quotation {\<\setarab  qatalt">}

par {\<\setarab  qatalt§>}

\<\setarab  qatalta> \marginpar{ {\<\setarab  qatalta>} } 


\<\setarab  qatalt> \marginpar{ {\<\setarab  qatalt>} } 

% 
% {\<\setarab \novocalize t>}
% \arabictext{qatal} 