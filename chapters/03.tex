\chapter*{Drittes Buch.}
\pdfbookmark[0]{Drittes Buch.}{drittesbuch}
\pdfbookmark[1]{Einleitung.}{III.I.Einleitung}
\cehead{{{\large III.}} Die genealogisch-historische Sprachforschung.}
\cohead{Einleitung.}
\section*{Die genealogisch-historische Sprachforschung.}
\section*{Einleitung.}\phantomsection\label{III.I}

\largerpage%longdistance
\fed{{\textbar}145{\textbar}}\phantomsection\label{fp.145} \sed{{\textbar}{\textbar}136{\textbar}{\textbar}}\phantomsection\label{sp.136}

Es handelt sich \update{hier}{bei der geschichtlichen Sprachforschung} zuvörderst nicht um „Prinzipien der Sprachgeschichte“, wie sie \textsc{Paul} in seinem so betitelten Buche und früher \textsc{Whitney} (Life and Growth of Language) aufgestellt haben. Die Entdeckung solcher allgemeiner Grundsätze gehört weder der einzelsprachlichen noch der historisch-genealogischen Forschung. Jede Sprache und jeder Sprachstamm ist autonom, nur beschränkt durch die Grenzen des Menschenmöglichen: die Weite dieser Schranken, was schlechthin \update{nothwendig}{nothwendig,} oder schlechthin unmöglich ist, das zu erörtern gebührt der allgemeinen Sprachwissenschaft. \sed{Mit Recht aber haben die Indogermanisten je länger je mehr diesen Prinzipienfragen ihre ganze Aufmerksamkeit zugewandt. Inwieweit aber diese Grundsätze, so wie sie von ihnen für ihren Bereich aufgestellt worden sind, auch in anderen Sprachfamilien sich halten, kann nur die Erfahrung lehren.} Der Zweig der Sprachforschung, der uns hier beschäftigt, hat es \update{zunächst}{als solcher} mit den trockensten Einzelthatsachen zu thun: Sind die Sprachen \textit{A} und \textit{B} miteinander verwandt, und in welchem Grade? Giebt es dieses Wort oder jene Form in der und der Sprache oder in der und der Zeit der Sprachgeschichte? wie lautet es da? Welche Gesetzmäßigkeit herrscht in den lautlichen Abweichungen? Besteht im einzelnen Falle Urgemeinschaft oder Entlehnung? Was ist alles Gemeingut, was neu \update{hinzu erworben}{hinzuerworben}? u.~s.~w. Alles das klingt und ist auch wirklich sehr trocken. Was die menschliche Rede im Innersten bewegt, was sonst die Wissenschaft von den Sprachen der Völker zu einer der lebensvollsten macht, das tritt hier zunächst zurück; nur einige ihrer Ausläufer ranken in das Seelen- und Sittenleben der Völker hinüber. Der einzelsprachliche Forscher kann gar nicht schnell genug die fremde Sprache in’s eigene Ich aufnehmen; der Sprachhistoriker steht draussen vor seinem Gegenstande: hier der Anatom, da der Cadaver. Ich übertreibe wohl nicht, wenn ich behaupte, \textsc{Bopp} und \textsc{Schleicher} hätten ihre ver\sed{{\textbar}{\textbar}137{\textbar}{\textbar}}\phantomsection\label{sp.137}gleichenden Grammatiken ganz ebensogut schreiben können, wenn sie \fed{{\textbar}146{\textbar}}\phantomsection\label{fp.146} auch nicht einer einzigen der darin bedachten Sprachen mächtig gewesen wären. Was treibt sie und ihre Fachgenossen zu philologischer Thätigkeit? \update{entweder}{Entweder} Liebhaberei für das verwandte frischere Nebenstudium, oder das Bedürfniss der eigenen Wissenschaft, der sie bessere Quellen zuführen wollen.

Die Geschichte der Linguistik, diesmal der vergleichenden Indogermanistik, ist hier wie immer lehrreich. Erst zügellos kühnes Zusammenstellen ähnlich klingender Vocabeln; dann Ringen nach sicherer Methode. Methode hiess aber in diesem Falle genügsame Selbstbeschränkung auf das Greifbarste. Die Gefahr lag nahe, dass daraus beschränkte Selbstgenügsamkeit \sed{nach der Art eines platten Materialismus} wurde: nur das Greifbare, Stoffliche schien Werth zu haben, nur die Scheidekunst an Lauten geübt schien wahre, beweisende Wissenschaft. Dass sie dies ist, hat sie glänzend bewährt, auch da, wo ihre Ergebnisse umstritten sind: sie tastet eben die Grenzen des Beweisbaren aus. Nun aber drängt es sie über die selbstgezogenen Schranken hinaus, – \textsc{Pott}’s souveräner, allseitiger Geist hatte sich ohnehin diesen Schranken nie unterworfen. Kurz die Sprachgeschichte ist nur zum Theile aus lautmechanischen Vorgängen zu erklären; das lernte man um so tiefer empfinden, je schärfer man die entdeckten Lautgesetze zuzuspitzen strebte. Gesetze wollen ausnahmslos gelten. Werden sie von Ausnahmefällen durchbrochen, so sind sie entweder zu weit gefasst, oder sie werden von anderen Mächten überwunden. Glaubte man also die Erscheinungen des gesetzmässigen Lautwandels erschöpfend auf allgemeine Formeln zurückgeführt zu haben, so musste man für den unerklärbaren Rest eine fremde, nicht lautmechanische Macht verantwortlich machen, und diese konnte nur seelischer Art sein. Es schien, als wäre das Prinzip der Analogie auf einem weiten Umwege ein zweites Mal entdeckt, das grosse Gesetz der lebendigen Sprache. Nicht die Entdeckung, sonderen die Anwendung dieses Gesetzes auf dem Gebiete der geschichtlichen Forschung ist das Verdienst der neueren Indogermanistik. Urtheile ich recht, so hat in dieser Schule die sprachgeschichtliche Wissenschaft eine wahre Verjüngung, eine Rückkehr zum frischen Leben gefeiert. Jetzt schwelgt ein Theil ihrer Anhänger im Entdecken „falscher Analogien“, manchmal wohl zur Schadenfreude ihrer Gegner, und doch selbst in ihren Übereilungen anregend.

Ein zweiter Fortschritt in gleicher Richtung ist zu verzeichnen. Die syntaktische Vergleichung hatte sich lange nur auf einzelne Zweige \fed{{\textbar}147{\textbar}}\phantomsection\label{fp.147} des Sprachstammes erstreckt, zudem, mehr für die Zwecke des Schulunterrichtes, auf Lateinisch und Griechisch untereinander und mit der Sprache des Lernenden. Erst in neuerer Zeit ist man darauf verfallen, einzelne Theile der Satzlehre weiterhin vergleichend zu verfolgen; die Grammatiken aber von vorwiegend linguistischer Tendenz, die jetzt sammlungsweise erscheinen, hören nach wie vor da auf, wo die Syntax anfangen sollte. Allenfalls wird da der Formenlehre etwas syntak\sed{{\textbar}{\textbar}138{\textbar}{\textbar}}\phantomsection\label{sp.138}tische Zuthat wie Schmuggelgut beigepackt; sonst aber ist es, als sollte durch solche Programme bewiesen werden, dass Verständniss und Handhabung einer Sprache nicht zu den Dingen gehören, die eine Grammatik zu lehren hat. Zum Glücke liegt der Fehler nur in dem missbräuchlich angemassten Titel: statt Grammatik sollte es heissen: Laut- und Formenlehre. Auch die geschichtliche Grammatik muss alle Theile ihres Gegenstandes erfassen, und ihre Vertreter sind gewiss die Letzten, die dies verneinen möchten.

Aber der besondere Gesichtspunkt ergiebt besondere Aufgaben. Welches sind die Aufgaben der historisch-genealogischen Sprachforschung?

Alle Sprachen erleiden Wandelungen in Stoff und Form, viele erleiden zudem Spaltungen, wohl auch Mischungen. Unter Spaltungen der Sprachen aber verstehen wir dies, dass die Veränderungen auf verschiedenen geographischen Gebieten verschieden geschehen. Solche Spaltungen geringfügigster Art haben wir schon in den leisesten mundartlichen Abschattungen, in rein localen Wortgebräuchen und Redensarten, ja in den individuellen Eigenheiten der Lautbildung und des Sprachgebrauchs zu erkennen. Untermundarten desselben Dialektes, Dialekte derselben Sprache, Sprachen derselben Familie, Familien desselben Sprachstammes sind weiter nichts als Spaltungs- und vielleicht Mischungsergebnisse.

So hat die genealogisch-historische Forschung, mag sie ihr Gebiet so eng oder so weit ziehen wie sie will, doch eigentlich immer die Geschichte einer einzigen Sprache zum Gegenstande, vielleicht sogar nur die Geschichte einer einzelnen Mundart. Der Indogermanist fragt: Was ist aus der Sprache unserer indogermanischen Urahnen geworden? Wie hat sie sich verzweigt und in den Verzweigungen gestaltet? wie mag sie selbst beschaffen gewesen sein? Ähnlich der Germanist, der Slavist, der Romanist, der Keltist u.~s.~w., ähnlich auch der, der etwa die nordfriesischen oder schwäbischen Mundarten unter einander vergleichen wollte. Die Vergleichung halte sich ganz auf der Oberfläche, besage zunächst \fed{{\textbar}148{\textbar}}\phantomsection\label{fp.148} nicht mehr, als dass dieser Erscheinung der einen Sprache jene in der anderen entspreche, z.~B. dem deutschen \textit{d} ein englisches \textit{th}, dem deutschen Worte „Knabe“ = Kind männlichen Geschlechts, das englische \textit{knave} = Schurke, dem lateinischen \update{accusativus}{Accusativus} cum infinitivo der deutsche Objectssatz mit „dass“, – immer liegt mindestens stillschweigend der Gedanke zu Grunde, dass die gemeinsame Ursprache sich hüben so, drüben so weiter entwickelt, oder auch, dass sie sich in der einen ihrer Verzweigungen unverändert bewahrt habe.

Man sieht, im Grunde hat es die historisch-genealogische Forschung ebensogut mit Einzelsprachen zu thun, wie die einzelsprachliche selbst. Worin besteht also der wesentliche Unterschied Beider?

Er besteht nicht im räumlichen Umfange des Untersuchungsobjectes. Einen Sprachstamm als solchen und in seinen Verzweigungen kann man freilich \sed{{\textbar}{\textbar}139{\textbar}{\textbar}}\phantomsection\label{sp.139} nicht einzelsprachlich betrachten; denn die einzelsprachliche Forschung ist ihrer Natur nach an die sprachgemeindlichen Grenzen gebunden. Dafür bietet aber wiederum eine Mundart in ihren Abschattungen weit mehr Anlass zur historisch-genealogischen Untersuchung, als zur einzelsprachlichen.

Auch der zeitliche Umfang ist natürlich nicht entscheidend. Cultursprachen haben ihre epochemachenden classischen Zeitalter, deren Meisterwerke viele Jahrhunderte lang die Geister und die Sprache beherrschen mögen, und immer leben die Epigonen mit den Classikern in Sprachgemeinschaft. Umgekehrt können die Sprachgemeinschaften recht kurz und jäh abgebrochen werden, und es steht der Nachwelt frei, ganze Perioden von der Gemeinschaft auszuschliessen. \textsc{Luther}’s grobschrötige Frische steht unserm Sprachgefühl näher, als die unausstehliche Ziererei und Ausländerei, darin man sich nach dem dreissigjährigen Kriege gefiel. \textsc{Boccaccio} gehört in die Gemeinschaft der lebenden \update{lingua}{Lingua} toscana; der zwanzig Jahre jüngere \textsc{Franco Sacchetti} dagegen wird wie ein Fremdling verdolmetscht. Jenem, der \fed{als} ein halbes Jahrtausend hindurch als Muster edler Sprache gegolten, gebührt auch in der einzelsprachlichen Wissenschaft ein hervorragender Platz; der jüngere Novellist dagegen, dessen Werke viel später im Druck erschienen und wohl nie in ähnlichem Umfange nationales Gemeingut geworden sind, kann nur in der Sprachgeschichte Berücksichtigung beanspruchen. 

Der Unterschied zwischen beiderlei Forschungen ist in der That ein artlicher. Die Einzelsprache ist ein Vermögen, das aus seinen Äusserungen \fed{{\textbar}149{\textbar}}\phantomsection\label{fp.149} begriffen, in diesen nachgewiesen werden will. Diese Aufgabe setzt sich die einzelsprachliche Forschung, und sie darf innerhalb ihres Kreises jenes Vermögen als ein sich im Wesentlichen gleichbleibendes \update{behandeln; denn}{behandeln. Denn} das ist es in der That. In \textsc{Luther}’s Rede wurden der Hauptsache nach dieselben Stoffe von denselben Kräften beherrscht, nach denselben Gesetzen bearbeitet, wie in der Sprache irgend eines unserer Zeitgenossen. Und das Gleiche gilt von den Mundarten verschiedener Gaue: im Wesentlichen gleicht das Sprachvermögen des \retro{Erzgebirgers}{Erzgebirges} dem des Schwarzwäldlers oder Oberbaiern, mag sich auch unter den äusseren Hüllen die Wesensgleichheit verbergen.

Dieses Vermögen also soll der Einzelsprachforscher erkennen, beschreiben\fed{,} und aus ihm heraus soll er die Äusserungen der \update{Einzel\-sprachen}{Einzel\-sprache} erklären. Thatsächlich ist nun aber jenes Vermögen ein gewordenes und immer weiter werdendes, sich veränderndes und verschiebendes, und auch das will erklärt werden: durch welche Veränderungen ist die Sprache zu ihrem jeweiligen Zustande gelangt? womöglich auch, – wenn die Frage nicht in alle Zukunft \update{unbeant\-wortbar}{unbeant\-wortet} bleibt: warum ist die Sprache gerade so geworden und nicht anders? Auf alles dies kann die Einzelsprachforschung von ihrem Standpunkte aus und mit ihren Mitteln keine Antwort geben; hier stehen wir auf dem Gebiete der Sprachgeschichte.

\sed{{\textbar}{\textbar}140{\textbar}{\textbar}}\phantomsection\label{sp.140}

\sed{In der Wissenschaft gelten als Eintheilungsgrund nicht die Werkzeuge, mit denen gearbeitet, auch nicht die Quellen aus denen geschöpft, sondern die Erkenntnissziele, denen zugestrebt wird. In unserem Falle, bei der einzelsprachlichen und der sprachgeschichtlichen Forschung, sind aber die ersteren kaum weniger verschieden, als die Letzteren. Beide Forschungszweige verhalten sich zu einander gegensätzlich und sich ergänzend; von ihrem Standpunkte aus und mit ihren Mitteln erstrebt und vermag die Eine gerade das, was der Anderen unzugänglich ist. Die einzelsprachliche Forschung erklärt die Sprachäusserungen aus dem jeweiligen Sprachvermögen und thut sich genug, wenn sie dieses Vermögen, wie es derzeit in der Seele des Volkes ist oder war, in seinem inneren Zusammenhange systematisch begreift. Sie wird dabei hin und wieder gern Anleihen bei der Sprachgeschichte machen; sie kann aber auch ohnedem leben; denn was dem jeweiligen Sprachgefühle gegenüber zufällig ist, darf es auch ihr bleiben. Wie und warum jenes Vermögen und dieses Gefühl so geworden, begreift sie nicht. Dagegen will die Sprachgeschichte \so{als solche} eben weiter nichts als dies erklären. Das heisst: die Lebensäusserungen der Sprache, die Rede, begreift sie gar nicht. Will sie sie begreifen, so muss sie eben auf den einzelsprachlichen Standpunkt übertreten. Somit ist sie in viel weiterem Masse auf Borg angewiesen, als die einzelsprachliche Forschung; jene Gebietsüberschreitungen, – denn das sind sie nun einmal, – sind ihr so unentbehrlich und gewohnt geworden, dass sie im besten Glauben auch drüben, jenseits ihrer Grenze ihre Flagge hissen möchte und Dinge, mit denen sie nichts anzufangen weiss, vielfach wegzuräumen versucht. Denn das masst sie sich an, wenn sie z.~B. den Satz aufstellt: eine Sprache, von der wir weder frühere Phasen noch seitenverwandte oder dialektische Verzweigungen kennen, sei überhaupt kein Object für die Sprachwissenschaft. Als ob die Sprachwissenschaft nichts weiter zu erforschen hätte, als wenn und wo diese oder jene Veränderungen im Laut- und Formenwesen, in den Wörtern und Bedeutungen eingetreten sind; als wären die grossen Wirkungen dessen, was sich Jahrhunderte hindurch im Wesentlichen gleich geblieben ist, weniger interessant, als jene kleinen Wandelungen, die sich im Laufe der Jahrhunderte vollzogen haben; als müsste der, der die Gesetze einer vereinzelten Sprache in einer systematischen Grammatik darzustellen weiss, nicht mindestens ebensoviel Verständniss vom Wesen der menschlichen Sprache haben, als Jener, der das Lautinventar der indogermanischen Ursprache um ein paar neue Nummern bereichert. Keiner von beiden hat das Recht, auf den Anderen herabzublicken, denn Beider Leistungen können sich an aufgewendetem Fleisse und Scharfsinne die Waage halten. Gilt es aber nicht dieser oder jener Sprache, sondern dem menschlichen Sprachvermögen überhaupt, so denke ich, ein neuer Sprachtypus hat doch etwas mehr Erkenntnisswerth, als ein paar neue alte Laute.}

\sed{{\textbar}{\textbar}141{\textbar}{\textbar}}\phantomsection\label{sp.141}

Die Sprache oder Mundart, deren Geschichte untersucht werden soll, wollen wir relativ eine \so{Ursprache} oder einen \so{Urdialekt} nennen, – relativ, das \update{heisst:}{heisst,} wohl wissend, dass diese sogenannten Urformen ihrerseits doch nur Phasen einer vielleicht langdauernden früheren Entwickelung sind. In diesem Sinne reden wir von Ur-Indogermanisch, Urgermanisch, Urskandinavisch, Urschwäbisch u.~s.~w., immer in Hinsicht auf spätere Phasen und Spaltungen.

Sind nun diese Spaltungen alt und weitklaffend genug, so hört die sprachliche oder mundartliche Einheit auf, Entfremdung tritt ein, Dialekte werden zu Sondersprachen, Einzelsprachen werden zu Oberhäuptern ganzer Sprachstämme. Dann mag es wohl geschehen, wie manchmal in der Pflanzenwelt, bei Stockausschlägen und Wurzelausläufern. Der alte Stamm ist längst verfault, und man muss unter dem Boden nachgraben, um die gemeinsame Wurzel blosszulegen. Und wenn es nur immer so stünde, dass jede Sprache bloss eine Wurzel hätte! Oft allerdings ist die Ähnlichkeit einer Sprache mit anderen so augenfällig, dass es kaum eines besonderen Verwandtschaftsnachweises bedarf. Oft aber \fed{{\textbar}150{\textbar}}\phantomsection\label{fp.150} auch sind die Ähnlichkeiten so verwischt, die Verwandtschaften so entfernt, dass jener Nachweis viel Mühe und Scharfsinn erfordert.

\sed{Es ist leicht erklärlich, dass uns diejenige Sprache der Ursprache am Nächsten zu stehen scheint, die uns über die übrigen den reichsten Aufschluss giebt. Oft wird dies die sein, die wir am Frühesten oder am Genauesten kennen gelernt haben, und von deren Standpunkte aus wir nun, ganz menschlicher Weise, die anderen betrachten. So können sich in die Forschung Zufälligkeiten und Einseitigkeiten einschleichen, die erst in der Folge berichtigt werden. Vom Sanskrit ausgehend hatte man die indogermanische Sprachverwandtschaft entdeckt, nun mass man die übrigen Glieder der Familie am Sanskrit, bis man einsehen lernte, dass die europäischen Sprachen in manchen Dingen den Urtypus reiner bewahrt haben\textsc{. R. H. Codrington} (The Melanesian Languages, Oxford 1885) ist der Meinung, wir würden über das Verhältniss der melanesischen Sprachen zu den malaischen ganz andere Ansichten haben, wenn wir jene früher als diese kennen gelernt und die malaischen Sprachen an den melanesischen gemessen hätten, wie er es thut. Die Bemerkung ist an sich sehr fein und zutreffend; nur das möchte ich bezweifeln, dass der ausgezeichnete Forscher von seinem neuen Standpunkte aus zu befriedigenderen Ergebnissen gelangt wäre. (Vergl. meine Besprechung im Journal of the R.~A.~S.~of Gr. Br. \& Ireland, XVIII., pt. 4). Solange man der indochinesischen Sprachvergleichung das Neuchinesische zu Grunde legte, kam man nicht recht von der Stelle. Das Tibetische hilft weiter; und welche Aufschlüsse vom Siamesischen und von den agglutinirenden Gliedern der Familie zu erwarten sind, lässt sich noch gar nicht übersehen.}

Wir werden, um Missverständnisse zu vermeiden, gut thun, zwischen äusserer und innerer Sprachgeschichte zu unterscheiden. Die äussere Geschichte \sed{{\textbar}{\textbar}142{\textbar}{\textbar}}\phantomsection\label{sp.142} einer Sprache ist die Geschichte ihrer räumlichen und zeitlichen Verbreitung, ihrer Verzweigungen und etwaigen Mischungen (Genealogie). Die innere Sprachgeschichte erzählt und sucht zu erklären, wie sich die Sprache in Rücksicht auf Stoff und Form allmählich verändert hat.

Es leuchtet ein, dass man, solange man nur sprachgeschichtliche Zwecke verfolgt, nur genetisch verwandte Sprachen miteinander vergleichen darf. Und umgekehrt ist es einleuchtend, dass der Beweis der Verwandtschaft, wo er nöthig ist, nur im Wege der Vergleichung geführt werden kann. So scheint es, als drehten wir uns im Kreise. In der That ist aber die vergleichende Arbeit, die nur die Familienzugehörigkeit erweisen will, summarisch im Gegensatze zu jenen minutiösen Untersuchungen, die die innere Sprachgeschichte erheischt. Zudem ist jene Arbeit die vorbereitende, und schon darum muss sie zuerst betrachtet werden.

\pdfbookmark[1]{I. Theil.}{III.I.I}
\cehead{{{\large III.}} I. Die äussere Sprachgeschichte. Der Verwandschaftsnachweis.}
\section*{Erster Theil.}
\section*{Die äussere Sprachgeschichte. \mbox{Der Verwandtschaftsnachweis.}}
\pdfbookmark[2]{§. 1. Aufgaben der Sprachengenealogie.}{III.I.I.1}
\cohead{§. 1. Aufgaben der Sprachengenealogie.}
\subsection*{§. 1.}\phantomsection\label{III.I.I.1}
\subsection*{Aufgaben der Sprachengenealogie.}

So jung unsere Wissenschaft ist, so viel hat sie bereits in sprachgenealogischer Hinsicht geleistet. Weitaus die meisten Sprachen der alten Welt sind wenigen grossen Familien und engeren Sippen zugeordnet. Vom Ganges bis nach Island erstreckt sich der indogermanische Stamm, von der Mündung des Amur bis nach Lappmarken und der Türkei der ural-altaische, von der Osterinsel bis nach Madagaskar der malaio-polynesische Sprachstamm. Den Semiten vom alten Babylon bis zum heutigen Marokko reihen sich die hamitischen \update{Aegypter}{Ägypter}, die Gallas \fed{{\textbar}151{\textbar}}\phantomsection\label{fp.151} und Berbern vetterschaftlich an; den grössten Theil Afrikas südwärts vom Erdgleicher wissen wir von der stämme- und sprachenreichen Bantu-Familie bewohnt. Wir haben gelernt, die nichtarischen Ureinwohner Vorderindiens in Drâvidas und Kolarier zu scheiden, theilen die Sprachen der kriegerischen Kaukasusvölker in einen nördlichen und einen südlichen Stamm und sind eben dabei, etwa ein Drittheil der Menschen zu einer grossen indochinesischen Sprachenfamilie zu vereinigen. Dass die Sprachen der Ureinwohner Australiens sammt und sonders untereinander verwandt sind, stellt sich immer deutlicher heraus; von jenen der amerikanischen Indianervölker sind nun wohl die meisten einer Anzahl grösserer oder kleinerer Familien eingereiht. Schon bieten die Sprachenkarten ein erfreuliches Bild; man ist versucht, über der \sed{{\textbar}{\textbar}143{\textbar}{\textbar}}\phantomsection\label{sp.143} Menge der gewonnenen Einsichten die noch viel grössere Menge der zu lösenden Räthsel zu übersehen.

Und dabei besagt das Zahlenverhältniss noch am Wenigsten. Ein guter Theil des bisher Erworbenen bot sich von selbst, man brauchte nur hinzuschauen und zuzugreifen. Dass der Wolf zum Hundegeschlechte gehört, lehrt uns ein einziger Blick. Dass aber die Blindschleiche nicht eine Schlange, sondern eine Eidechsenart ist, erfahren wir erst, wenn wir dem Thiere die Haut abstreifen und es anatomisch untersuchen. Beiderlei kommt auch in der Sprachenwelt vor, nur dass hier noch viel öfter die Verwandtschaftsmerkmale unter der Haut zu suchen sind.

Woher noch immer die Menge der Sprachstämme? und woher die grosse Menge der Sprachen, die noch keinem bekannten Stamme zugeordnet sind? War es wirklich so, wie Manche glauben, dass an mehreren Orten der Erde, unabhängig von \update{einander,}{einander} sich sprachlose Anthropoiden zu sprachbegabten Menschen entwickelt haben? Gab es vielleicht gar, – denn gerade das ist behauptet worden, – Anfangs auf der bewohnten Erde eine Menge grundverschiedener Sprachen, deren grosser Theil nachmals im Kampfe um’s Dasein spurlos erloschen ist? Dann freilich wäre unseren sprachvergleichenden Bestrebungen eine jener Schranken gesetzt, vor denen der Verständige Halt macht, und die der Narr mit dem Kopfe durchrennt.

Wie aber, wenn jene Anderen Recht hätten, die da annehmen, die sprechende Menschheit, also auch die menschliche Sprache habe sich aus einer ursprünglichen Einheit differenzirt? Dann müssten wir eben unverdrossen fortfahren in der Arbeit des Vergleichens und Zerlegens, \fed{{\textbar}152{\textbar}}\phantomsection\label{fp.152} immer gewärtig, dass uns am Ende doch der zerkleinerte Stoff wie Streusand durch die Finger rinne, unfassbar und ungestaltbar.

So stehen wir mitten drinnen in einer der heikelsten Fragen der menschlichen Urgeschichte, berufen, wo möglich dereinst selber das entscheidende Wort zu sprechen, darum doppelt verpflichtet zu unbefangenem Verhalten. War nun Mehrsprachigkeit der ursprüngliche Zustand des Menschengeschlechtes, so sind zwei Fälle möglich: entweder leben noch mehrere jener Ursprachen in ihren Nachkommen fort, oder diese sind alle bis auf eine im Daseinskampfe erlegen, und dann wären alle bekannten Sprachen Nachkomminnen einer einzigen Stammmutter. Mir scheint ein Kampf von so verheerend sichtender Wirkung in jener Urzeit nicht wahrscheinlich. In geschichtlicher Zeit siechen wohl Rassen und Sprachen unter dem tödtlichen Einflusse einer einbrechenden überlegenen Gesittung dahin, werden wohl auch einzelne wilde Stämme von anderen ihresgleichen mit Stumpf und Stiel ausgerottet. Doch Letzteres gehört zu den Ausnahmen. Viel öfter weiss die Völkerkunde von sogenannten Autochthonenstämmen zu reden, die in irgend welchen verlassenen Winkeln Art und Sprache \sed{{\textbar}{\textbar}144{\textbar}{\textbar}}\phantomsection\label{sp.144} der Vorfahren fortsetzen. Und wie leicht war das Fliehen und Wandern und Neubesiedeln auf der dünnbevölkerten Erde.

Sind aber die uns bekannten Sprachen in grundverschiedene Stämme vertheilt, so bleibt doch die Möglichkeit, dass mehrere der jetzt noch für geschieden geltenden Sprachstämme sich bei fortgesetzter Vergleichung als urverwandt erweisen, und schon das wäre ein gewaltiger Gewinn. Gesetzt aber, alle uns zugänglichen Sprachen wären weiter nichts als Fortsetzungen der einen Urform menschlicher Rede, so folgte daraus noch nicht mit Nothwendigkeit, dass sich diese Ursprungseinheit jemals müsse wissenschaftlich erweisen lassen. Wie vorhin angedeutet, wäre es ja möglich, dass sich die Wissenschaft am Ende nach den besonnensten Vorarbeiten einem so gestalt- und haltlosen Stoffe gegenüber sähe, dass auch dem Kühnsten der Muth zu weiterer Analyse und Vergleichung verginge. Dann würde also unser Endurtheil lauten: Die Urverwandtschaft aller Sprachen ist unerweisbar, aber auch unwiderlegbar, – und die Anthropologie müsste sich dabei beruhigen; denn den Neanderthalschädel kann sie doch nicht zum Reden bringen.

Nicht davor muss gewarnt werden, dass man mit solchen Vergleichungen zu früh aufhöre, sondern davor, dass man zu früh damit anfange. Vielleicht kommt einmal die Zeit, wo der Beweis einer Ver\fed{{\textbar}153{\textbar}}\phantomsection\label{fp.153}wandtschaft etwa zwischen den irokesischen Sprachen und denen der Maya-Huastecafamilie unumstösslich geführt vorliegt. Wollte ich heute diese Verwandtschaft behaupten, so würde ich für alle Zukunft nicht den Ruhm eines ahnenden Genies, sondern den Vorwurf ur\-theilsloser Voreiligkeit verdienen. Denn alle Voreiligkeit in wissenschaftlichen Dingen ist unmethodisch, und alle Unmethode läuft auf dumme Urtheilslosigkeit hinaus. Und doch liegt die Versuchung auch besseren Geistern so nahe. Von dem „Sprachstamme der Titanen“, den „turanischen Sprachen“ und ähnlichen Versuchen darf ich hier schweigen. Was aber dem grossen \textsc{Franz Bopp} auf seinen sprachvergleichenden Irrfahrten im malaischen und kaukasischen Gebiete widerfahren musste, giebt zu denken. Es war, als hätte er die Richtigkeit seiner Grundsätze nun auch von der Kehrseite beweisen, an sich selbst das \update{argumen\-tum}{Argumen\-tum} ad absurdum liefern sollen: Compass und Karte, die ihn bisher geleitet, hatte er über Bord geworfen, und nun sass sein Fahrzeug auf dem Sande fest. Die Versuchung zu dergleichen Wagnissen muss doch mächtig sein, dass ihr ein solcher Mann unterliegen konnte.

Es \update{gilt,}{gilt} uns durch eine strenge Methode gegen ähnliche Verirrungen zu wappnen, und eben das ist das Schwierige, eine Methode der Entdeckungen zu finden. Auch in den günstigsten Fällen, vielleicht gerade in diesen, hat der Zufall und der glückliche Einfall, der doch selbst ein Zufall ist, sein reichliches Antheil. Wir lernen eine Sprache, die bisher für isolirt galt. Zufällig kennen wir eine andere, die jener entfernt verwandt ist; zufällig lenkt sich unsere Auf\sed{{\textbar}{\textbar}145{\textbar}{\textbar}}\phantomsection\label{sp.145}merksamkeit auf diese oder jene versteckten Übereinstimmungen zwischen Beiden; wir stutzen, fragen: Sollte das auch Zufall sein? und nun folgen wir der Fährte, entdecken des Gemeinsamen immer mehr, und endlich ist der Beweis geführt. Hier gingen also der methodischen Arbeit, diese veranlassend, zwei Dinge voraus, die sich scheinbar nicht lehren und lernen, sonderen nur als Geschenke hinnehmen lassen. Und gleichwohl ist auch dabei eine gewisse Methode möglich, so eine Art Wünschelruthe, die anzeigt, wo Bohrer, Grabscheit und Haue mit Aussicht auf Erfolg ihre Arbeit beginnen können. Wahrscheinlichkeiten und Unwahrscheinlichkeiten giebt es auch hier. Und ist nun erst der forschende Geist auf verheissende Anzeichen gestossen, so muss er bedächtig, Schritt für Schritt vorgehen, um nicht schliesslich doch sein Ziel zu verfehlen. Dafür ist also erst recht eine Methodik \update{nöthig}{nötig}.

\fed{{\textbar}154{\textbar}}\phantomsection\label{fp.154}

\largerpage[1]Die Sprachengenealogie will indessen mehr, als den Globus unter eine Zahl Sprachfamilien vertheilen; sie will auch innerhalb dieser Familien die Verwandtschaftsgrade feststellen. Vergleichen wir die sich verzweigende Ursprache mit einer Pflanze, so gilt es zu wissen, an welchen Stellen die Äste am Stamme, die Zweige an den Ästen, die Blätter an den Zweigen ansitzen, welche Spaltungen die älteren, welche die jüngeren sind, – es gilt, um den beliebten Vergleich mit menschlichen Verwandtschaftsgraden anzuwenden, die auf- und absteigenden Linien von den Seitenlinien, innerhalb dieser gradweise die Geschwister- und Vetterschaften, vielleicht auch voll- und halbbürtige Versippungen zu unterscheiden. In dieser Richtung ist noch unendlich viel zu thun, und wie sauer und heikel die Arbeit sein kann, davon weiss Niemand besser zu reden, als die Indogermanisten.

Nach dem Bisherigen richtet sich unsere Untersuchung auf zwei Hauptfragen:

A. Wie entdeckt und beweist man das Bestehen von Sprachstämmen und die Zugehörigkeit einzelner Sprachen zu solchen und die etwaige Urverwandtschaft mehrerer Stämme?

B. Wie stellt man innerhalb der Sprachstämme die Verwandtschaftsgrade fest?

\pdfbookmark[2]{§. 2. Entdeckung und Erweiterung der Sprachstämme.}{III.I.I.2}
\cohead{§. 2. Entdeckung und Erweiterung der Sprachstämme.}
\subsection*{§. 2.}\phantomsection\label{III.I.I.2}
\subsection*{Entdeckung und Erweiterung der Sprachstämme.}
\subsection*{A.}\phantomsection\label{III.I.I.2A}
\subsection*{Das Aufsuchen von Anzeichen.}
Es handelt sich hier um das, was man im Detectivwesen entfernte Indicien nennt. Der Richtungen, in denen nachgeforscht werden könnte, sind unzählig viele; aber es fragt sich: welche Richtungen versprechen am Ersten zum Ziele zu führen, welche sind also in erster Reihe zu verfolgen? Denn wir wollen nicht mit blindem Umhertapppen und Ausprobiren Zeit und Kräfte vergeuden. \sed{{\textbar}{\textbar}146{\textbar}{\textbar}}\phantomsection\label{sp.146} Es muss eine Kunst des Suchens geben, die sich lehren und lernen lässt; es muss für diese Kunst apriorisch geltende Grundsätze geben, die sich aus der Natur der Sache entwickeln lassen.

Wir setzen den einfachsten Fall: Ein Volk hat sich verbreitet und gespalten, mit ihm auch seine Sprache. Die nationale und sprachliche \fed{{\textbar}155{\textbar}}\phantomsection\label{fp.155} Einheit hat aufgehört, es sind verschiedene Völker und verschiedene Sprachen geworden, deren ursprüngliche Einheit erst wieder entdeckt werden soll. Welche Spuren wird sie hinterlassen haben?

a. Geographische Momente.\phantomsection\label{III.I.I.2Aa}

Die Völker- und Sprachenkarten in unsern Atlanten erzählen ein gut Stück Weltgeschichte und Völkerkunde. Hier kraftvolle Nationen, die sich ausbreiten, andere zurückdrängend, dort jene armen und schwachen, deren Überbleibsel eingeengt oder zerstückelt sind. Zwischen den weithingestreckten Gebieten der Germanen und Romanen auf enge Küstengelände am westlichen Meere beschränkt die Nachkommen jener Kelten, die einst ganz Britannien und ein grosses Stück Galliens beherrschten; ähnlich am biskaischen Meerbusen die Basken, vormals die Besitzer weiter Strecken des südlichen \update{Frankreichs}{Frankreich} und der pyrenäischen Halbinsel. Die Etrusker, Ligurer, Veneter, Messapier sind spurlos verschwunden; vielleicht haben die Pelasger in den Arnauten, diese oder jene der alten Völker Kleinasiens in den schönen Bewohnern des Kaukasus Nachkommen hinterlassen. Jene slavische Sprachinsel im Lüneburgischen, jene keltische in Cornwall sind unlängst erst vom andringenden Germanenthume überfluthtet worden. Gotisch wurde noch vor zweihundert Jahren in einigen Gemeinden der Krim gesprochen. Ein Jahrhundert früher war die Sprache der Preussen und jene der türkischen Kumanen in Ungarn verklungen, und jene der Litauer und Letten siechen vor unseren Augen dahin. Nicht besser steht es um das ober- und niederlausitzer Wendisch und andrerseits um so manche deutsche Einsprenglinge in slavischen und italienischen Sprachgebieten, um die Gottscheer Mundart und jene der \update{sette}{Sette} und der \update{tredeci}{Tredeci} communi. Es ist ein geschichtliches Gesetz, dass die Kleinen sich in der Vereinsamung nicht halten können; sie müssen in den grösseren Nachbarn aufgehen, das heisst untergehen. Heute besitzen die Finnen und Esthen ihre wissenschaftlichen Akademien, sammeln emsig, was sie an heimischen Sagen und Gesängen vorfinden, bearbeiten Grammatik und Wortschatz ihrer Sprachen in mustergültiger Weise, mehren mit immer wachsendem Eifer ihre Literaturen. Zwei Völker, zusammen kaum mehr als drei Millionen Köpfe zählend, arm an irdischen Gütern, – man müsste ihrem idealen Streben zujubeln, wenn ein wenig mehr Pietät gegen ihre deutschen und schwedischen Lehrmeister dabei wäre. Allein was wird all ihr Mühen fruchten? Die Sprachen werden weiter leben, so lange ihnen der \fed{{\textbar}156{\textbar}}\phantomsection\label{fp.156} mächtige Nachbar das Leben gönnt, und ihnen ihre lutherische Kirche ein gewisses nationales Sonderdasein sichert. Die Nach\sed{{\textbar}{\textbar}147{\textbar}{\textbar}}\phantomsection\label{sp.147}kommen der \update{Aegypter}{Ägypter} haben seit Jahrhunderten ihre Sprache mit der arabischen vertauscht und nur Dank ihrer Religion einen Schatten eigenen Volksthums gerettet. Es giebt aber \update{Zuflucht\-stätten}{Zufluchts\-stätten} bedrängter Völker, die einen an den Rändern der Meere, wo die Bedrückten nicht weiter können, die \update{andern}{anderen} in unwirthlichen Gebirgen oder Einöden, wohin die Bedrücker nicht folgen mögen. Das sind jene Völker- und Spracheninseln, bei deren Betrachtung der Sprach- und Geschichtsforscher bald prickelnden Reiz, bald innige Wehmuth empfindet, wie beim Anblick gesunkener Grösse.

Die Einen treibt Noth und Schwäche vom heimischen Herde, die Anderen überströmende Kraft oder unbefriedigte Gier. Und so verschieden wie die Gründe der Wanderungen, der Ausbreitung oder Einengung, sind auch ihre Ergebnisse. Jetzt finden wir grosse zusammenhängende Gebiete von einem einzigen Sprachstamme beherrscht, so zu sagen Continente auf dem linguistischen Globus, jetzt wieder inselartig zerstreute Glieder einer grossen Familie oder kleine sprachlich vereinzelte Völker. So stellt das nordöstliche Asien mit seinen Tschuktschen, Korjäken, Giljäken, Itelmenen, Ainos, Japanern und Bewohnern der Aleuten eine Art sprachlichen Archipel, und umgekehrt jenes Inselgebiet, das sich von der malaischen Halbinsel und Sumatra aus südostwärts in langer Kette bis Neuguinea, nordostwärts über Borneo und die Philippinen bis Formosa und weiterhin gen Osten durch die mikronesischen, melanesischen und polynesischen Inseln bis Rapa-nui verbreitet, – ein continentmässig zusammenhängendes riesiges Sprachgebiet dar. Auch darauf müssen wir gefasst sein, dass eine solche Continuität freiwillig durch Auswanderung oder unfreiwillig durch den überschwemmenden Einbruch Fremder durchrissen worden sei. Die seefahrenden Malaien haben nach den Malediven und Madagaskar Absenker geschickt; die Azteken (Nahuatl) scheinen den Algonkinstämmen, die Australier den vorderindischen Kolariern sprachverwandt. Aus der Benachbarung allein Schlüsse auf die sprachliche Zusammengehörigkeit zu ziehen, ist immer misslich.

b. Anthropologische Momente.\phantomsection\label{III.I.I.2Ab}

Hat ein Volk sich verzweigt, so ist zu erwarten, dass seine versprengten Nachkommen im Wesentlichen die ursprüngliche Leibes- und Geistesart bewahrt haben. Die Juden sind ein classisches Beispiel hiefür. \fed{{\textbar}157{\textbar}}\phantomsection\label{fp.157} Je ähnlicher der Typus, desto enger die Rassenverwandtschaft, desto näher die genealogische Zusammengehörigkeit.

So mag die Anthropologie schlussfolgern, nicht aber die Linguistik. Dem Scandinavier steht der Finne geistig und leiblich näher, als der arische Hindu. Und umgekehrt: Finnen, Esthen, Magyaren und osmanische Türken tragen kaukasischen Rassetypus im Gegensatze zu ihren \update{mongoloiden}{mongolischen} Sprachverwandten in Asien. Nigritische Melanesier reden Sprachen, die den \retro{malaisch-polynesischen}{malaisch-polynesichen} verwandt sind, und die Neger der Republik \update{Haiti}{Hayti} sprechen französisch. Der \sed{{\textbar}{\textbar}148{\textbar}{\textbar}}\phantomsection\label{sp.148} Sprachforscher muss immer mit der Möglichkeit rechnen, dass sich Völker gemischt, oder dass sie fremde Sprachen angenommen haben, und so besitzen in seinen Augen Übereinstimmungen im geistleiblichen Typus immer nur den Werth entfernter Indicien.

Das \update{gleiche}{Gleiche} gilt

c. von den ethnographischen und culturgeschichtlichen Momenten.\phantomsection\label{III.I.I.2Ac}

Trachten und Geräthe, Sitten, Religionen und sonstige Überlieferungen aller Art pflanzen sich nur zu gern von Nachbarn zu Nachbarn fort. Vielleicht mehr noch als das Christenthum hat der Buddhismus, und viel mehr noch der Muhammedanismus culturausgleichend gewirkt. Indische Spiele und Märchen sind durch Vermittelung der Perser, Araber und Türken nach Europa und Afrika, durch buddhistische Pilger bis Ostsibirien, Japan und in die malaische Inselwelt gedrungen, und man braucht nur \textsc{R. Andree}’s Ethnographische Parallelen und Vergleiche zu lesen, um zu sehen, wie die wunderlichsten Bräuche und Anschauungen in den entlegensten Winkeln der Erde wiederkehren. Ähnlichkeiten in Cultur und Uncultur beweisen nichts für die genetische Zusammengehörigkeit der Völker, und vollends \update{nichts}{nicht} für die Verwandtschaft ihrer Sprachen.

d. Sprachliche Momente.\phantomsection\label{III.I.I.2Ad}

Das einzig untrügliche Mittel, eine Verwandtschaft zu erkennen, liegt in den Sprachen selbst. Die Sprachen aber bieten verschiedene Seiten, und diese scheinen von verschiedenem Werthe zu sein. Sprachen sind untereinander verwandt, das besagt ein Doppeltes: Erstens, dass sie einander in gewissen Beziehungen ähnlich sind; denn sonst trügen sie nicht mehr die Merkzeichen der gemeinsamen Herkunft; – und zweitens, dass sie in anderen Beziehungen \update{von einander}{voneinander} verschieden sind; denn \fed{{\textbar}158{\textbar}}\phantomsection\label{fp.158} sonst wären sie nicht mehrere Sprachen, sondern eine einzige. Es fragt sich: welche Merkmale sind die dauerhaftesten, daher zuverlässigsten?

\subsection*{\sed{α.} Ähnlichkeiten im Lautwesen}\phantomsection\label{III.I.I.2Adalpha}
\update{aa.}{}wollen wenig besagen. Das Vorwiegen der Zischlaute in den slavischen Sprachen, der Vocale in den polynesischen, die Abwesenheit der \update{mediae}{Mediae} in letzteren, der gutturale Klang vieler amerikanischer Sprachen, und anderwärts mancherlei Anderes gehört allerdings zum Familientypus. Dafür ist aber auch an entgegenstehenden Beispielen kein Mangel. Unter den romanischen Sprachen steht das Französische mit seinen nasalirten Vocalen vereinzelt da und nähert sich insoweit den schwäbisch-deutschen Mundarten. Das Annamitische hat Wortaccent, der dem verwandten Khmêr (Cambodjanischen) und den weiterhin verwandten kolarischen Sprachen fehlt. Die gleiche Erscheinung trennt sogar innerhalb des Tibetischen einen Dialekt von allen übrigen. Unter den finnisch-ugrischen \sed{{\textbar}{\textbar}149{\textbar}{\textbar}}\phantomsection\label{sp.149} Sprachen ist eine einzige, die das Gesetz der Vocalharmonie nicht kennt: die syrjänische. Sie ist hierin wahrhaft aus der Art geschlagen; denn jenes Gesetz, \update{wornach}{wonach} sich die Vocale der Suffixe nach jenen des Stammes richten, gehört recht eigentlich zum ural-altaischen Typus.

Die Verzweigung der Sprachen beruht ja mit zum grossen Theile in der verschiedenen Entwickelung ihres Lautwesens, also in der allmählichen Erzeugung neuer Laute. In der indogermanischen Ursprache hat man bisher noch keine Spur von \textit{ö}, \textit{ü}, \textit{š}, \textit{ž}, \textit{χ}, \textit{f} und manchen anderen Lauten entdecken können, die heute in verschiedenen Familien ihrer Nachkommen verbreitet sind.

\subsection*{\sed{β.} Ähnlichkeiten im Sprachbaue.}\phantomsection\label{III.I.I.2Adbeta}
\update{bb.}{}In den meisten der bisher erforschten Sprachfamilien herrscht eine gewisse Gleichmässigkeit des grammatischen Baues. In den einen ist die Wortformung ausschliesslich suffigirend, – so in den ural-altaischen und drâvidischen Sprachen; in anderen ist sie prä- und suffigirend, so in den malaischen und in den kongo-kaffrischen (Bantu-)Sprachen. Der semitische Sprachtypus mit seinem Triconsonantismus und seiner wunderbar \update{mannich\-faltigen}{mannig\-faltigen} und doch gesetzlichen Vocalisation ist vielleicht der am schärfsten ausgeprägte. Ähnlich pflegt es mit der Morphologie des Satzes zu sein, mit dem Aufbaue und der Reihenfolge seiner Glieder, der Art seiner Verknüpfungen. Steht das Attribut voran, wie in den \fed{{\textbar}159{\textbar}}\phantomsection\label{fp.159} uralaltaischen und drâvidischen Sprachen? oder folgt es nach, wie in den malaio-polynesischen, semitischen und kongo-kaffrischen? \update{steht}{Steht} das Verbum hinter dem Subjecte, oder darf es auch diesem vorangehen? Geschieht die Satzverbindung durch Conjunctionen oder durch participiale und gerundiale Suffixe?

Übereinstimmungen in solchen Dingen sind immer bedeutsam, aber sie sind nicht entscheidend. Erstens sind wohl bisher in den meisten Fällen die Grenzen der Sprachstämme zu eng umschrieben, indem nur die einander ähnlichsten Sprachen als verwandt erkannt wurden. Zweitens bestehen doch auch in einigen der schon bekannten Sprachstämme sehr bedeutende bauliche Verschiedenheiten. Von den urindogermanischen Vocalabstufungen trägt das Lateinische nur noch dürftige Spuren. Vorgefügte Formwörter, zuweilen wahre Präfixe, verdrängen stellenweise die Suffixe. Und wo die von Hause aus bewegliche Wortstellung in enge grammatische Regeln gebannt ist, da können auch nahe verwandte Sprachen, wie das Deutsche und Englische, sehr verschiedene Bilder bieten.

Das ist aber noch nichts im Vergleiche zu jenen Verschiedenheiten, die andere Sprachstämme aufweisen. Der indochinesische begreift unter anderen in sich das Chinesische und die Thai-Sprachen (Siamesisch, Shan, Lao, Khamti, Ahom, Aitom), die zu den reinsten Vertretern des isolirenden Baues gehören, – dann das Barmanische, Arakanische (Rukheng), die Kuki- und Nagasprachen, die mehr oder minder agglutinirend sind, – ferner am Himâlaya die Kirânti\sed{{\textbar}{\textbar}150{\textbar}{\textbar}}\phantomsection\label{sp.150}sprachen, deren Agglutination an Polysynthetismus zu streifen scheint, – endlich das Tibetische, das mit einer ziemlich losen Agglutination wunderbare, wahrhaft flexivische innere Veränderungen der Verbalstämme vereint. Dass das Annamitische, gleichfalls eine streng isolirende Sprache, sich den reich agglutinirenden kolarischen Sprachen verwandtschaftlich anschliesst, hat \textsc{Ernst Kuhn} nachgewiesen. Im westlichen Sudân, an den Küsten von Senegambien und Guinea und weiter landeinwärts, wohnt eine Menge Völker, die man früher als \update{ächte}{echte} Neger von den Bantus schied: die Woloffen, die Mande, Susu, Vei, Bambara, Mende, Ibo, Nupe, Temne, Ewhe, Akra, Aschanti, Grebo, Kru u.~s.~w. Ihre Sprachen sind zum Theil untereinander und sämmtlich von der Bantufamilie baulich so verschieden, dass man versucht war, sie für vereinzelt zu halten, höchstens sie in kleine, einander fremde Sippen zusammenzuordnen. Neuerdings aber gewinnt die Anschauung an Boden, dass wir es hier mit einem losen \fed{{\textbar}160{\textbar}}\phantomsection\label{fp.160} Schwarme entfernterer Verwandter des grossen Bantustammes zu thun haben, die sich zu diesem ähnlich verhalten mögen, wie die melanesischen Sprachen zu dem malaio-polynesischen.

\subsection*{\sed{γ.} Übereinstimmung in der inneren Sprachform.}\phantomsection\label{III.I.I.2Adgamma}
\update{cc.}{} Jede Sprache stellt gewisse Denkgewohnheiten dar, auf denen sie beruht, und die sich von Geschlechte zu Geschlechte fortpflanzen. Der äusseren Form entspricht die sogenannte innere, das heisst, um \textsc{Steinthal}’s glücklich gewählten Ausdruck zu gebrauchen, die Anschauung von Anschauungen. Diese begreift ein Doppeltes in sich: erstens die Art, wie die einzelnen Vorstellungen mit den vorhandenen Hülfsmitteln dargestellt werden, z.~B. Mond, μήν, als messender, luna als leuchtende, – und zweitens die Art, wie die Vorstellungen geordnet, geschieden und zu gegliederten Gedanken verknüpft werden. Man sollte meinen, wenigstens Letzteres, die den Sprachbau beherrschende innere Form, müsse besonders dauerhaft in der Vererbung, daher entscheidend für die Verwandtschaft sein. Und in der That gehören zu den charakteristischen Merkmalen vieler Sprachfamilien gewisse Eigenthümlichkeiten der inneren Form. Die substantivischen Classen der Bantusprachen, die zwei Geschlechter der hamito-semitischen, die drei der indogermanischen, die vorwiegend nominale Auffassung des Prädicates, die Vorliebe für die passivische Redeweise in den Sprachen der malaischen Familie u.~s.~w., sind Beispiele solcher typischen Eigenschaften. In Amerika aber herrscht eine geistige Verwandtschaft unter vielen Sprachfamilien, deren leibliche Verwandtschaft im glücklichsten Falle sehr entfernt ist.

Auch geschieht es wohl, dass Sprachen sehr wichtige Eigenthümlichkeiten der inneren Form im Laufe der Zeit abstreifen oder annehmen. Den hottentottischen Dialekten mit ihren drei grammatischen Geschlechtern sind Buschmannsprachen verwandt, die keinerlei Genuszeichen kennen. Zu der indo\sed{{\textbar}{\textbar}151{\textbar}{\textbar}}\phantomsection\label{sp.151}chi\-ne\-si\-schen Familie gehören das Thai \update{(siame\-sische)}{(Siame\-sische)} und seine Verwandten, in denen die prädicative, – das Tibetische und Barmanische, in denen die attributive Anschauungsweise vorherrscht, und das Chinesische, das beide Kategorien scharf auseinanderhält. Unter den melanesischen Sprachen und, soviel ich weiss, unter allen Sprachen des Erdballes, steht die von Annatom (Aneiteum), einer Insel der Neuen \update{Hebriden}{Hebriden,} in Rücksicht auf die innere Form der Rede ganz vereinzelt da. In ihr wird nicht das Verbum, sondern das Pronomen personale conjugirt. Dies \fed{{\textbar}161{\textbar}}\phantomsection\label{fp.161} eröffnet den Satz, zeigt an, ob von der ersten, der zweiten oder einer dritten Person Singularis, Dualis, Trialis oder Pluralis die Rede ist, ob es sich um ein Gegenwärtiges, Vergangenes, Zukünftiges, Gesolltes u.~s.~w. \update{handele;}{handelt;} dann folgt das Verbum mit seinen näheren Bestimmungen, zuletzt das \update{Subject.}{Subject,} \sed{dem sich allerdings noch weitere adverbiale Betimmungen anschliessen können.} Das Verbum selbst ist ganz \retro{ungeformt,}{umgeformt,} und das conjugirte Fürwort keineswegs als eine Art Verbum substantivum aufzufassen. \sed{Beispiele:}

\tabcolsep=0.04cm
\begin{tabular}{ l l l l l }
\sed{\textit{namu}} & \sed{\textit{džim}} & \sed{\textit{taiṅ}} & \sed{\textit{aiek}} \\[-1.5ex]
 & & & & \sed{= weine nicht.} \\[-1.5ex]
 \sed{du \textit{opt.}} & \sed{nicht} & \sed{weinen} & \sed{du} \\
\end{tabular}

\begin{tabular}{ l l l l l}
 & & & & \\
\sed{\textit{is}} & \sed{\textit{atahaidžeṅ}} & \sed{\textit{ra}} & \sed{\textit{aien}} \\[-1.5ex]
 & & & & \sed{= er hörte sie.} \\[-1.5ex]
\sed{er \textit{praeter}.} & \sed{hören} & \sed{sie (\textit{pl. obj.})} & \sed{er} \\
\end{tabular}

\begin{tabular}{ l l l l l }
 & & & & \\
\sed{\textit{eris}} & \sed{\textit{atṅa}} & \sed{\textit{iran}} & \sed{\textit{atimi}} \\[-1.5ex]
& & & & \sed{= Menschen traten darauf.} \\[-1.5ex]
\sed{sie \textit{praet.}} & \sed{treten} & \sed{darauf} & \sed{Mensch} \\
\end{tabular}

\begin{tabular}{ l l l l l l l l }
 & & & & & & & \\
\sed{\textit{is}} & \sed{\textit{um}} & \sed{\textit{tas}} & \sed{\textit{n'atimi}} & \sed{\textit{eseṅe}} & \sed{\textit{is}} & \sed{\textit{eθi}} \\[-1.5ex]
 & & & & & & & \sed{= und es sprach ein Lehrer.} \\[-1.5ex]
\sed{er \textit{praet.}} & \sed{und} & \sed{sprechen} & \sed{der Mensch} & \sed{lehren} & \sed{er \textit{praet.}} & \sed{ein} \\
\end{tabular}

\noindent \sed{(Vgl. \textsc{H. C. v. d. Gabelentz}, Die melanesischen Sprachen (I) S.~65–124).} – Eine uns gleichfalls befremdende Vertheilung der Functionen, \update{wornach}{wonach} sozusagen nicht das Verbum, sondern das Subject Träger des \update{genus}{Genus} verbi ist, findet sich auf zwei sehr entfernten Punkten, im Tibetischen und in australischen Sprachen. Da kann man von einem Casus activo-instrumentalis und von einem neutro-passivus reden.

So erhellt, dass die Übereinstimmung in der inneren Sprachform, gleich der in der äusseren, wohl ein beachtliches, aber keineswegs ein untrügliches Anzeichen der leiblichen Verwandtschaft ist. Weder herrscht überall, wo jene besteht, auch diese, noch auch ist Letztere allemal mit \sed{der} Ersteren verbunden; verwandte Sprachen können in Bau und Geist recht verschieden sein, und baulich und geistig einander ähnelnde Sprachen können einander in genealogischer Hinsicht bis zur völligen Fremdheit fern stehen. Zudem gehört die innere Sprachform zu den Dingen, über die man nur nach tieferer Einsicht urtheilen soll.

\subsection*{\sed{δ.} Übereinstimmungen in Wörtern und Formativen.}\phantomsection\label{III.I.I.2Addelta} 
\update{dd.}{} Wer den Verwandtschaften neuentdeckter Sprachen nachforscht, sieht sich in den meisten Fällen auf sehr mageres Material angewiesen, auf kleine Wörter\-\sed{{\textbar}{\textbar}152{\textbar}{\textbar}}\phantomsection\label{sp.152}sammlungen, wie sie von Reisenden in der Eile aufgerafft werden. Ein Glück noch, wenn sie wenigstens zuverlässig sind, – wie es damit gehen kann, haben wir früher gesehen. Ein Glück \update{noch,}{aber auch,} dass gerade die lexikalischen Übereinstimmungen für die Verwandtschaft der Sprachen die entscheidendsten sind.

Solche Übereinstimmungen zwischen entfernteren Verwandten zu erkennen, ist allerdings oft sehr schwer, und dann müssen sich wohl „Verdienst und Glück verketten“, damit die Entdeckung zu Wege komme: Bekanntschaft mit den anderen Verwandten, ein sicherer Tact, ein rechtzeitiger Einfall, dass dies zu Jenem stimme, dann ein mühsam methodisches Weiterforschen auf der gefundenen Spur.

Wie wechselvoll können die Schicksale eines Wortschatzes sein! Ausdrücke kommen ausser Gebrauch oder ändern ihre Bedeutungen, \fed{{\textbar}162{\textbar}}\phantomsection\label{fp.162} Fremdwörter werden eingeführt, und Alles war im Laufe der Zeiten der zerstörenden Macht des Lautwandels ausgesetzt. Es wäre gut, wenn sich von \update{vorn herein}{vornherein} sagen liesse, welcherlei Wörter am meisten Aussicht haben, in den verschiedenen stammverwandten Sprachen ihren Platz zu behaupten; leider aber dürfte dies nur in sehr beschränktem Masse möglich sein.

Die Zahl- und Fürwörter halten noch am häufigsten Probe. Allein die ersteren stehen und fallen mit dem Bedürfnisse, und dies Bedürfniss kann sehr tief sinken. Bei den Chiquito-Indianern muss es ganz geschwunden sein; denn die können nicht einmal bis zwei zählen. Wird es dann einmal geweckt, so liegt es nahe, fremde Zahlwörter zu entlehnen, die dann den Forscher auf falsche Fährte lenken. Besser steht es nach den bisherigen Erfahrungen um die Fürwörter. Allein ganz vor Verfall und Verlust gesichert sind auch sie nicht. Meist sind es kleine Lautkörper, ein Consonant und ein Vocal, die leicht durch lautliche Veränderungen unkenntlich werden können. Auch mag wohl die höfliche Sitte Fürwörter der ersten und zweiten Person abschaffen und durch neue Mittel ersetzen. So ist es im Holländischen mit dem Du geschehen, und so wird es wohl mit der Zeit dem englischen \textit{thou} ergehen, das jetzt noch von der Kirche und den Dichtern eine Art Altersversorgung bezieht. Im Niedermalaischen pflegt \so{ich} durch \textit{sāya} (sanskrit \textit{sahāya}, Gefährte), \so{du} durch \textit{tuwan} (arabisch \textit{tuhan}, Herr) ersetzt zu werden. Doch das sind Ausnahmen; anderwärts gehören die Pronomina und Zahlwörter zu den bereitwilligsten und lautestredenden Zeugen. Die Erkenntniss der hamito-semitischen Verwandtschaft gründet sich fast ausschliesslich auf sie. Ihnen und gewissen Formativen zuliebe, glaube ich die australischen Sprachen mit den kolarischen, das Mexikanische (Nahuatl) mit der Algonkinfamilie verbinden zu dürfen.

\begin{sloppypar}Verbalstämme und Adjectiva erhalten sich vielleicht im Stofflichen besser, als in der Bedeutung; denn die mit Thätigkeiten und Eigenschaften verbundenen Vorstellungen haben meist flüssige Grenzen. Unser „sehen“ entspricht nach Laut \sed{{\textbar}{\textbar}153{\textbar}{\textbar}}\phantomsection\label{sp.153} und ursprünglicher Bedeutung dem lateinischen \textit{sequi}, griechischem ἕπεσθαι; im Italienischen und Französischen ersetzt kauen, \textit{manducare}, \textit{mangiare}, \textit{manger}, das alte \textit{edere}. Lateinisch \textit{dicere} hiess ursprünglich zeigen, δεικνύναι. Zu \fed{deutsch} „blau“ stimmt lautlich lateinisch \textit{flavus}.\end{sloppypar}

\largerpage[-1]Besser scheint es mit den handgreiflichsten, nächstliegenden Sub\fed{{\textbar}163{\textbar}}\phantomsection\label{fp.163}stantiven zu stehen, mit den Namen für Menschen und Thiere, für Verwandtschaftsgrade, Körpertheile, Gestirne, Elemente und was dessen mehr ist. Hier werden sich wenigstens die Bedeutungen nicht so leicht verschieben, und zur Annahme neuer Ausdrücke liegt scheinbar wenig Anlass vor. Die Erfahrung bestätigt dies im Allgemeinen, spielt uns doch aber auch manchen verblüffenden Streich. Das Wort für „Hand“ lautet sanskrit \textit{hasta}, slavisch-litauisch \textit{rãka}, griechisch χείρ, lateinisch \textit{manus}. Vollends bunt sind die Wörter für „Mädchen“ in den romanischen und germanischen Sprachen: lateinisch \textit{puella}, ital. \textit{ragazza}, span. \textit{chica}, \textit{muchacha}, franz. \textit{fille}, dann, auf germanischer Seite, deutsch Mädchen, Dirne, englisch \textit{girl}, dänisch \textit{pige}, norweg. \textit{jenta}, schwed. \textit{flicka}. Unserm „sehr“ entpricht engl. \textit{sore}, wund, schmerzhaft. Mond, \textit{luna}, σελήνη, \textit{çaçin} bezeichnen dasselbe Ding nach verschiedenen Merkmalen.

Sehr willkommen sind Übereinstimmungen in den Formativlauten. Allein auch sie können trügen. Entweder bleiben sie aus, wo man sie erwarten sollte, weil die Affixe abgeschliffen oder durch neue ersetzt sind. Beispiele dafür bietet unser eigener Sprachstamm die Hülle und Fülle. Oder aber sie treten da auf, wo sonst keinerlei Verwandtschaft nachweisbar ist. So z.~B. sind Genitivpartikeln mit \textit{n}, Dativ- oder Locativ- und Illativzeichen mit \textit{d} oder \textit{t} nicht nur im uralaltaischen Sprachstamme heimisch, sondern auch sonst weit verbreitet. So zeigt die magyarische Conjugation und selbst die der Yunga-Sprache in Süd-Amerika in den Pronominalelementen auffallende Ähnlichkeiten mit der indogermanischen. Affixe bestehen ja meist aus wenigen, leichtwiegenden Lauten. Da hat der Zufall leichtes Spiel; und doch möchte man fürwitzig fragen, ob uns in solchen Fällen nicht letzte Spuren einer Ursprungseinheit aller menschlichen Sprachen entgegendämmern, oder ob sich die Dinge unabhängig \update{von einander,}{voneinander,} nur durch Gleichheit der Anlage, an den verschiedensten Punkten so ähnlich entwickelt haben. Es ist das bekanntlich die Frage, die in der vergleichenden Völkerkunde immer wiederkehrt, manchmal bei noch überraschenderen Anlässen.

So ist es nun auch mit manchen der gebräuchlichsten Wörter. \textit{Papa}, \textit{baba}, \textit{mama}, \textit{ama}, \textit{tata}, \textit{tete}, \textit{nana}, \textit{nunu} und ähnliche kehren weit und breit wieder als Ausdrücke für die frühesten Bedürfnisse des Kindes: Vater, Mutter, Zitze, saugen; und diese \retro{Bedeutungen}{Bedürfnisse} sind auf die Laute hier so, dort anders vertheilt. Im Japanischen heisst \textit{fafa} \update{(\textit{papa})}{(\textit{haha})} Mutter, \textit{titi} Vater; \textit{papilla}, \textit{mamilla}, τίτθη, spanisch \fed{{\textbar}164{\textbar}}\phantomsection\label{fp.164} \update{\textit{teta}}{\textit{teta,}} \sed{Montagnais (Athapaskisch) \textit{tthuthi}} sind Namen für Mutterbrust \sed{{\textbar}{\textbar}154{\textbar}{\textbar}}\phantomsection\label{sp.154} und Zitze. Den ersten Lauten des Kindermundes wurden von den Eltern Bedeutungen beigelegt, die natürlich der engen Vorstellungswelt des kleinen Wesens entsprechen. Das wird wohl ein allgemein menschlicher Hergang sein.

Wo die Wörter durch Schallnachahmung gebildet sind, – und das waren sie eigentlich schon in dem eben besprochenen Falle, – da ist ihre Ähnlichkeit natürlich gleichfalls von der Verwandtschaft der Sprachen unabhängig. Seltsam ist es nun aber doch, wie der Zufall da spielen kann, wo man es am Mindesten vermuthen sollte. Ganz stilles Verhalten ist doch eigentlich geräuschlos, also so wenig wie möglich zur onomatopoetischen Darstellung geeignet. Der Mandschu liebt aber diese Darstellungsweise: Alles soll reden. Die Glocken sagen \textit{kilang kalang}, aufflatternde Vögel sagen: \textit{bur bar}, Pferde, die über ein Steinpflaster traben, sagen \textit{pis pas}, und wer sich ganz ruhig verhält, der sagt \textit{cib} (sprich \textit{tsib}). Wenn der Schwede will, dass sein Kind hübsch schweigsam und gerade bei Tische sitze, so ermahnt er es: „Du musst \textit{sipp} sagen!“ \textit{Du bör säga sipp!} \sed{Im Deutschen bedeutet „nicht Zipp sagen“ soviel wie: nicht das leiseste Wörtchen äussern, und im Siamesischen ist \textit{sip} (alte Aussprache wohl \textit{j’ip}) Onomatopöie für ein leises Geräusch.}

Lehnwörter fallen selbstverständlich ausser Betracht, und darum sind solche Wörter, bei denen Entlehnung wahrscheinlich ist, von vorn herein bedenklich. Der Verdacht kann sich auf Zweierlei gründen: entweder auf den ausgedrückten Begriff, den man nicht für landesheimisch halten mag, wie bei exotischen Naturerzeugnissen oder Ideen, die vermuthlich einer fremden, höheren Gesittung entstammen; – oder zweitens auf die äussere Erscheinung der Wörter. Auch hier sind mehrere Fälle denkbar: die Wörter können einander gar zu ähnlich klingen, während sonst die Übereinstimmungen verborgener liegen. Oder sie können Laute enthalten, die sonst der Sprache fremd sind, wie anlautendes \textit{p} im Deutschen. Oder endlich, es mag ihre Bildungsweise schon bei oberflächlicher Betrachtung fremdartig erscheinen. Für alles das eignet man sich wohl einen gewissen Tact an; das Wichtigste ist aber doch eine wohlbedachte Methode.

\pdfbookmark[2]{§. 2. B. Zur Methodik der Sprachenvergleichung.}{III.I.I.2B}
\cohead{§. 2. B. Zur Methodik der Sprachenvergleichung.}
\subsection*{B.}\phantomsection\label{III.I.I.2B}
\subsection*{Zur Methodik der Sprachenvergleichung.}
\subsection*{Der Verwandtschaftsbeweis.}
\begin{sloppypar}Es ist schrecklich verführerisch, in der Sprachenwelt umherzuschwärmen, drauf los Vocabeln zu vergleichen und dann die Wissenschaft mit einer Reihe neu entdeckter Verwandtschaften zu beglücken. Es kom\fed{{\textbar}165{\textbar}}\phantomsection\label{fp.165}men auch schrecklich \update{viele}{viel} Dummheiten dabei heraus; denn allerwärts sind unmethodische Köpfe die vordringlichsten Entdecker. Wer mit einem guten Wortgedächtnisse begabt ein paar Dutzend Sprachen verschiedener Erdtheile durchgenommen hat, – studirt \sed{{\textbar}{\textbar}155{\textbar}{\textbar}}\phantomsection\label{sp.155} braucht er sie gar nicht zu haben, – der findet überall Anklänge. Und wenn er sie aufzeichnet, ihnen nachgeht, verständig ausprobirt, ob sich die Anzeichen bewähren: so thut er nur was recht ist. Allein dazu gehört \update{folgerich\-richtiges [\textit{in den Berichtigungen, S.~502}: folgerich\-tiges]}{folgerichtiges} Denken, und wo das nicht von Hause aus fehlt, da kommt es gern im Taumel der Entdeckungslust abhanden. So ging es, wie wir sahen, dem grossen \textsc{Bopp}, da er es versuchte, kaukasische und malaische Sprachen dem indogermanischen Verwandtschaftskreise zuzuweisen. Das Schicksal hatte es merkwürdig gefügt. Es war, als hätte er die Richtigkeit seiner Grundsätze doppelt beweisen sollen, erst positiv durch sein grossartiges Hauptwerk, das auf ihnen beruht, – dann negativ, indem er zu Schaden kam, sobald er ihnen untreu wurde.\end{sloppypar}

\textsc{Bopp}’s Verirrung ist die lehrreichste ihrer Art; leider ist sie nicht auch die verhängnissvollste. Sie wurde schnell erkannt, und wenn sie einem \update{geringerem [\textit{in den Berichtigungen, S.~502}: geringeren]}{geringeren} Manne widerfahren wäre, hätte man sie längst vergessen. Von Anderen, die Schlimmeres verbrochen haben, liest man höchstens noch in antiquarischen Katalogen. Nur ein Fehlgriff dieser Art hat dauernde Verwirrung geschaffen. Noch immer liest man, zumal in englischen Werken, von turanischen Sprachen. Der Name ist bekanntlich von \textsc{Max Müller} in einer geistreichen Jugendarbeit „On the Classification of the Turanian Languages“ eingeführt worden und sollte alle die Sprachen der alten Welt in sich begreifen, die weder semitisch, noch hamitisch noch indogermanisch sind: die uralaltaischen, kaukasischen, indochinesischen, malaio-polynesischen, drâvidischen u.~s.~w., also eine ganze Reihe verschiedener Sprachfamilien.\footnote{\sed{Später hat \textsc{Max Müller} selbst seine Überkühnheit eingesehen und anerkannt. Aber das Unglück war einmal geschehen, und es mag ihm ergangen sein, wie Goethe’s Zauberlehrling, – wenn er sich überhaupt sehr mit Beschwörungsversuchen angestrengt hat.}} Von den Speculationen, die der Verfasser daran knüpft, darf ich schweigen. Genug, ein brauchbarer linguistischer Begriff war mit dem neuen Namen nicht gewonnen, eher, für bescheidene Ansprüche, eine Art Compromiss mit den biblischen Überlieferungen, und das mochte der Sache Liebhaber werben. Nun wurde es bei \update{Vielen}{vielen} Glaubenssatz: wer nicht von Sem, Ham oder Japhet stammt, der gehört zu Tur’s Geschlechte. Anders ausgedrückt: Wenn man von einer Sprache nichts weiter weiss, als dass sie weder hamito-semitisch noch indogermanisch ist, so rechnet man sie zur turanischen Familie.

\fed{{\textbar}166{\textbar}}\phantomsection\label{fp.166}

~~~~Denn eben, wo Begriffe fehlen,

~~~~Da stellt ein Wort zur rechten Zeit sich ein.

\noindent In diesem Sinne mag man sich den Ausdruck gefallen lassen, mag ihn auch auf alle anderen Sprachen ausdehnen, über die Einer schreibt, ohne etwas von ihnen zu verstehen. Schlimmer ist es, wenn \update{Manche}{manche} den Namen auf den ural-altaischen oder finno-tatarischen Sprachstamm \update{ein\-schränken}{ein\-schränken,} und somit zum Ausdrucke eines verständigen wissenschaftlichen Begriffes missbrauchen. \textsc{Max Müller}’s Verdienste liegen bekanntlich auf anderen Gebieten, als denen der ural-altaischen \sed{{\textbar}{\textbar}156{\textbar}{\textbar}}\phantomsection\label{sp.156} Sprachforschung, und es sind andere Männer, die hier zur Namengebung berechtigt waren. Wie kommt jener Fremde zur Pathenschaft?

\sloppypar{
\textsc{Bopp} hatte gefehlt, indem er die Verschiedenheiten des Sprachbaues und Sprachgeistes übersah. \textsc{Max Müller} fehlte, indem er vermeintliche und wirkliche Ähnlichkeiten in der äusseren und inneren Form seiner turanischen Sprachen überschätzte. Jener trieb Wortvergleichungen ohne Methode; dieser suchte methodisch zu verfahren, ersparte sich aber die Wortvergleichungen da, wo sie am Nöthigsten gewesen wären.
}

\sed{Als ein rechtes kakographisches Beispiel wähle ich unter einer ganzen Schaar gleichwerthiger ein Buch: „Etymologisches Wörterbuch der magyarischen Sprache, genetisch aus chinesischen Wurzeln erklärt“. Hier ist für jeden halbwegs Verständigen gleich der Titel ein Todesurtheil. Magyarisch gehört bekanntlich dem finnisch-ugrischen, Chinesisch dem indochinesischen Sprachstamme an: es werden also zwei Sprachen verschiedener Familien miteinander verglichen, statt dieser Familien selbst. Das möchte zur Noth noch angehen, wenn die beiden Sprachen besonders alterthümlich und somit je die besten Vertreterinnen ihrer Stämme wären. Nun aber gehört das Magyarische schon zu den abgeschliffeneren Gliedern seiner Familie, und vollends Neuchinesisch zu den abgeschliffensten Sprachen der Welt. Darin ist es allerdings mit dem Englischen zu vergleichen, aber auch nur darin. Ich könnte aber auch ein Buch nennen, wo allen Ernstes chinesische Wörter mit ähnlich klingenden englischen zusammengestellt werden, um die Urverwandtschaft beider Sprachen nachzuweisen. Es ist eben nichts so verkehrt, dass nicht immer noch ein Verkehrteres möglich wäre.}

Wer entdecken will, muss den Muth haben zu irren. In der Wissenschaft irrt aber nicht allein der, der für eine Thatsache hält, was nicht thatsächlich ist, sondern auch Jener, der vorschnell für bewiesen ansieht, was noch des Beweises ermangelt, oder für wahrscheinlich ausgiebt, wofür noch keine hinlänglichen Anzeichen vorliegen. Von diesen Anzeichen haben wir vorhin gesprochen. Jetzt fragt es sich: Wie wird der Beweis der Verwandtschaft geliefert? und auch hierfür lassen sich gewisse Grundsätze aufstellen.

\phantomsection\label{III.I.I.2B1}1. Sind verwandte Sprachen einander so unähnlich, dass ihre Verwandtschaft nicht ohne Weiteres in die Augen fällt, so ist diese Verwandtschaft eine entferntere, also seit der vormaligen Einheit eine sehr lange Zeit verstrichen. Daraus folgt, dass man bei der Vergleichung immer auf die ältesten erkennbaren Lautformen und Bedeutungen der Wörter und Formative zurückzugehen hat. Das Nähere ergiebt sich bei einigem Nachdenken von selbst.

\largerpage[1]\phantomsection\label{III.I.I.2B1a}a) Laute gehen leichter verloren, als dass sie neu hinzukommen. Folglich hat zunächst die vollere Lautgestalt die Vermuthung der grösseren Alterthümlichkeit für sich. Diese Vermuthung wächst, wenn sich aus den volleren Lauten der einen Sprache die dürftigeren der anderen erklären lassen. Dafür ein Beispiel. \sed{{\textbar}{\textbar}157{\textbar}{\textbar}}\phantomsection\label{sp.157} Zu dem Gemeingute des indochine\fed{{\textbar}167{\textbar}}\phantomsection\label{fp.167}sischen Sprachstammes gehören unter Anderem auch die Zahlwörter. Unter diesen pflegen die Ausdrücke für Acht und für Hundert gleichen Anlaut zu haben:

\begin{table}[h]
\centering
\begin{tabular}{ l l l}
 & Acht & Hundert \\
Chinesisch & \textit{pat} & \textit{pek} \\
Newar, Pahi & \textit{čya} & \textit{či} \\
Barmanisch & \textit{rhač} & \textit{ra} \\
Singpho & \textit{ma)tsat} & \textit{la)tsa} \\
Gyarung & \textit{o)ryet} & \textit{pa)ryē} \\
Horpa & \textit{rhiēē} & \textit{rhyā} \\
Sērpa & \textit{gyē} & \textit{gyā} \\
Thāksya & \textit{bhrē} & \textit{bhrā} \\
Tibetisch & \textit{brgyad} & \textit{brgya} \\
 & u.~s.~w. & \\
\end{tabular}
\end{table}

\noindent Sehen wir zunächst vom Tibetischen ab, so finden wir als Anlaute \textit{b}, \textit{bh} oder \textit{p}, – \textit{r} oder \textit{rh}, \textit{g} und \textit{y} und zwar theilweise in Verbindungen: \textit{bhr}, \textit{ry} oder \textit{rhy}, \textit{gy}, – während \textit{č} und \textit{ts} als secundär gelten mögen. Offenbar hat hier der zungenbrecherische tibetische Anlaut \textit{brgy} als Generalnenner zu gelten, das heisst als derjenige, welcher die gemeinsame Urform verhältnissmässig am Getreuesten bewahrt hat: die vollsten Laute sind die alterthümlichsten.

\phantomsection\label{III.I.I.2B1b}b) Die Sprachgeschichte lehrt aber, dass dem nicht immer so ist. Wie das kommt, hat sie zu erklären; der genealogischen Sprachvergleichung genügt die Thatsache. Ein Beispiel ist das deutsche hund-ert im Gegentheil zu sanskrit \textit{çata}, slavisch \textit{sŭto}, griechisch ἑκατόν, lateinisch \textit{centum}. Hier entscheidet die Stimmenmehrheit gegen die deutsche Form. Das Gleiche gilt von jenen unorganischen \textit{ă}, die das Madegassische an auslautende Consonanten fügt, von dem Alif prostheticum des Arabischen und von dem \textit{e}, das im Spanischen, Portugiesischen und Französischen vor ein anlautendes \textit{s} mit folgenden Consonanten tritt: \textit{estar} = \textit{stare}, \textit{échelle} = \textit{scala} u.~s.~w. \sed{Lateinisch \textit{cognomen}, \textit{ignominia} scheinen dafür zu sprechen, dass \textit{nomen} aus \textit{gnomen} entstanden sei und zu \textit{noscere} (\textit{cognoscere}, \textit{ignorare}) gehöre. Dagegen erheben aber die übrigen indogermanischen Sprachen einstimmig Widerspruch, und man muss vielmehr annehmen, dass das Lateinische durch eine Verschiebung der Etymologie auf einen Abweg gerathen ist.}

\phantomsection\label{III.I.I.2B1c}c) Unter den Bedeutungen der Wörter und Wortformen gelten in der Regel diejenigen als die ursprünglichsten, aus denen sich die übrigen am Besten herleiten lassen, auch wenn zufällig die ältesten Literaturdenkmäler Anderes anzeigen sollten. 

\phantomsection\label{III.I.I.2B2}2. Dass nicht alle Wort- und Lautähnlichkeiten gleichwerthig sind, haben wir vorhin gesehen. Neben den verdächtigen Zeugen giebt es \fed{{\textbar}168{\textbar}}\phantomsection\label{fp.168} aber auch solche, \sed{{\textbar}{\textbar}158{\textbar}{\textbar}}\phantomsection\label{sp.158} die in ganz hervorragendem Grade beweisend sind, Prärogativinstanzen, um mit \textsc{Bacon} zu reden, bei denen der Zufall so gut wie ausgeschlossen ist. Mehrere Wörter \textit{A}, \textit{B} ... einer Sprache sind einander lautähnlich, in ihren Bedeutungen aber so verschieden, dass an einen etymologischen Zusammenhang nicht zu denken ist. Das Nämliche wiederholt sich nun in der zu vergleichenden Sprache. Jenes Beispiel von Acht und Hundert mag hierher gehören; andere aber sind noch bedeutsamer. In den indochinesischen Sprachen lauten in der Regel die Wörter für „ich, fünf, Fisch“: \textit{nga}, \textit{ngya}, oder ähnlich, und jene für „Du, zwei, Ohr“: \textit{na}, wohl auch \textit{nang}, \textit{no}, \textit{ni}; endlich treffen „Feuer“ und „Auge“ in Lauten wie \textit{mig}, \textit{mit}, \textit{mi} zusammen. Finden sich diese Übereinstimmungen oder ein grösserer Theil derselben in einer Sprache, so mag man diese ohne Weiteres für indochinesisch ausgeben.

\phantomsection\label{III.I.I.2B3}3. Auf diese Art entdeckt man nun mehr oder minder regelmässige Lautvertretungen und kann schliesslich sagen: Kehrt das und das Wort in jener Sprache wieder, so muss es so und so lauten. Das Mafoor von Neuguinea zeigt den malaischen Sprachen gegenüber einen argen Verfall. Das ursprüngliche \textit{k} ist verschwunden, das \textit{t} in \textit{k}, vor \textit{i} in \textit{s}, – \textit{p} in \textit{f}, \textit{l} in \textit{r} verwandelt, die Auslautsvocale sind abgefallen. Dies ergiebt sich aus einer Vergleichung der Zahlwörter: 3 \textit{tōru} : \textit{kior}; 5 \textit{līma} : \textit{rim}; 7 \textit{pītu} : \textit{fīk}; 10 \textit{pūlu} : \textit{fūr}; dann aus anderen Wörtern: essen, \textit{kan} : \textit{\=an}; Laus, \textit{k\=utu} : \textit{uk}; weinen, \textit{tāngis} : \textit{kianes} u.~s.~f. – Wo solche Regelmässigkeit herrscht, da steht die Verwandtschaft ausser Zweifel. Die Sprachen sind verschieden, denn die Lautentwickelung hat verschiedene Wege eingeschlagen. Hüben und drüben aber ist sie ihre Wege folgerichtig gegangen; darum herrscht in den Verschiedenheiten Ordnung, nicht Willkür. Sprachvergleichung ohne Lautvergleichung ist gedankenlose Spielerei.

\pdfbookmark[2]{§. 3. Arten und Grade der Verwandtschaft.}{III.I.I.3}
\cohead{§. 3. Arten und Grade der Verwandtschaft.}
\subsection*{§. 3.}\phantomsection\label{III.I.I.3}
\subsection*{Arten und Grade der Verwandtschaft.}
Alle dermalen näher bekannten Sprachfamilien stellen sich als Verzweigungen je eines Stammes dar, ihre Angehörigen sind vollbürtige Geschwister oder Nachkommen solcher. Vereinzelte fremde Bestandtheile gelten als Lehngut und für zu unbedeutend, als dass sie am Wesen der Familieneinheit und \update{Ächtheit}{Echtheit} etwas ändern könnten. Man hat sich lange daran gehalten, dabei beruhigt; man hat flottweg verneint, dass es eigent\fed{{\textbar}169{\textbar}}\phantomsection\label{fp.169}liche Mischsprachen gebe. Es war das eine jener vielen Voreiligkeiten, die zu den Entwickelungskrankheiten unserer jungen Wissenschaft gehören. Man hatte leichtes \update{Spiel}{Spiel,} zu beweisen, dass das Englische, trotz der romanischen Beimischungen, eine germanische, das Neupersische, trotz der arabischen Zuthaten, eine arische Sprache \update{sei;}{sei,} jenen missgestalteten Creolensprachen wandte man vornehm den Rücken, von anderen Mischlingen konnte \sed{{\textbar}{\textbar}159{\textbar}{\textbar}}\phantomsection\label{sp.159} man damals wohl kaum etwas ahnen. Dazu kam jene Anthropologie der amerikanischen Schule, die möglichst viele Menschenrassen mit möglichst schroffen artlichen Unterschieden aufstellte. Und ebenso artverschieden sollten die Sprachen sein. Konnte man die \update{Mulatten-}{Mulatten} und die Creolensprachen nicht aus der Welt leugnen, so verneinte man frisch drauf los, dass die Ersteren untereinander fortpflanzungsfähig, und die Letzteren vollberechtigte Menschensprachen seien. „Die Natur will keine Bastarde“, lautete das Stichwort.

Ich behalte es mir für eine spätere Stelle vor, eingehender über Sprachenmischung und Mischsprachen zu reden. Genug einstweilen: könnten wir die Geschichten aller Sprachen verfolgen, so würden wir wahrscheinlich alle erdenklichen Stufen und Arten der Sprachenmischung beobachten; – das ist a priori zu vermuthen. Und rechnen wir mit unseren bescheidenen Erfahrungen, so finden wir solcher Stufen und Arten schon eine erkleckliche Zahl vertreten. Die genealogische Sprachforschung muss auf halbbürtige Verwandtschaften ebenso gefasst sein, wie auf vollbürtige, sie muss mit der Möglichkeit rechnen, dass Sprachen, vielleicht ganze Sprachfamilien, durch Vermischungen anderer, unter sich verschiedener erzeugt worden sind. Man ahnt, in welche dunkelen Tiefen sie dabei geführt werden kann.

Über den Grad, das heisst die Nähe oder Ferne der Verwandtschaft entscheidet in der Regel die grössere oder geringere Ähnlichkeit der Sprachen. Und das mit Recht. Denn je später eine Sprache sich gespalten hat, desto mehr Gemeinsames werden ihre Zweige haben. Am \update{Meisten}{meisten} beweisen hierbei wohl weitgehende lexikalische Übereinstimmungen und demnächst gemeinsame Neubildungen, wie das neuromanische Futurum, die neuindischen Casussuffixe, die slavischen Praeterita durch das Participium auf \textit{–lŭ}, \textit{–la}, \textit{–lo}. Weniger Werth haben Ähnlichkeiten in der Entwickelung des Lautwesens, wie etwa die der Gutturalen im Indisch-Irânischen und im Litauisch-Slavischen; denn zwei Sprachen können gerade hierin sehr wohl ganz unabhängig voneinander auf parallelen \fed{{\textbar}170{\textbar}}\phantomsection\label{fp.170} Wegen gewandelt sein. Ein wunderliches Beispiel anderer Art liefert das Holländische. Die Vorfahren der heutigen \update{nicht friesischen}{nichtfriesischen} Niederländer, die alten Bataver, galten für einen fränkischen Stamm, müssen also, wenn diese Überlieferung Glauben verdient, vor Alters Dialektgenossen unserer oberdeutschen Franken gewesen sein. Das war aber vor Eintritt der hochdeutschen Lautverschiebung. An dieser nahm das Oberfränkische Theil. Jene batavischen \update{Nieder\-franken}{Nieder\-länder} aber bewahrten, gleich den Niedersachsen, den alten Lautzustand; und so steht jetzt das Holländische dem Plattdeutschen äusserlich näher, als seiner vorgeschrittenen hochdeutschen Schwester. Es müssten sehr hervorstechende lexikalische und grammatische Übereinstimmungen zwischen dem Holländischen und unsern fränkischen Mundarten bestehen, wenn die Sprachforschung allein mit ihren Mitteln zu einer Einsicht in einen so verwickelten Thatbestand hätte \sed{{\textbar}{\textbar}160{\textbar}{\textbar}}\phantomsection\label{sp.160} gelangen sollen. Ob nun jene Überlieferung auf Wahrheit beruht oder nicht, darauf kommt es nicht an. Uns genügt es, dass die Sache möglich ist, dass es Fälle geben kann, wo die ähnlichere Sprache die genealogisch fernerstehende ist.

\pdfbookmark[2]{§. 4. Zusatz I. Zur Anwendung der obigen Lehren.}{III.I.I.4}
\cohead{§. 4. Zusatz I. Zur Anwendung der obigen Lehren.}
\subsection*{§. 4.}\phantomsection\label{III.I.I.4}
\subsection*{Zusatz I.}
\subsection*{Zur Anwendung der obigen Lehren.}
\subsection*{I.}\phantomsection\label{III.I.I.4.I}
\subsection*{Die hamito-semitische Sprachfamilie.}
Die Verwandtschaft der semitischen Sprachen untereinander ist so eng, wie in wenigen indogermanischen Sprachfamilien. Völlige Gleichheit des Sprachbaues, – vielleicht des absonderlichsten, den es giebt, – mit seinen dreiconsonantigen Wurzeln und seinem organischen Vocalwandel, mit der Zweiheit des grammatischen Geschlechtes, der suffigirenden perfectischen und der präfigirenden imperfectischen Conjugation, der Gleichmässigkeit seiner Stellungsgesetze und so manchem Anderen, – weitgehende Gleichheit des Lautwesens und des Wortschatzes: das sind Dinge, die selbst dem blöden Auge einleuchten müssen. Um so seltsamer, dass zu einer vergleichenden Grammatik dieser Sprachfamilie eben erst die Anfänge gemacht worden sind. Bücher wie \textsc{Paul de Lagarde}, Übersicht über die im Aramäischen, Arabischen und Hebräischen übliche Bildung der Nomina (Göttingen 1889) und \textsc{J. Barth}, Die Nominalbildung in den semitischen Sprachen. I. (Leipzig 1889) gehören hierher.

\fed{{\textbar}171{\textbar}}\phantomsection\label{fp.171}

Minder nahe stehen einander die drei Gruppen der hamitischen Sprachen: die aegyptische (Altaegyptisch und Koptisch),\footnote{\sed{\textsc{Ad. Erman}, Das Verhältniss des Ägyptischen zu den semitischen Sprachen (Z. D. D. M. G. XLVI, S.~128), gelangt zu folgendem Ergebnisse: „So wäre denn das Ägyptische gegenüber den semitischen Sprachen als ein Idiom starker lautlicher Zersetzung und Entartung anzusehen; es} \sed{spielte neben ihnen etwa die Rolle, die das Englische neben dem Deutschen, das Französische neben dem Italienischen spielt.“ Ich füge hinzu: Dann lägen hier die Dinge, wie in der indochinesischen Familie, wo die älteste Cultursprache, die chinesische, schon weit abgenutzter erscheint, als viele ihrer jüngeren Schwestern.}} die aethiopische (Galla, Somali, Dankali, Bischari, Bilin, Saho, Agau, Kunama, Barea u.~s.~w.) und die berberische (Altlibysch, Kabylisch, Tuareg oder Tamaschek, Ghat u.~s.~w.). Für ihre Zusammengehörigkeit untereinander und mit den semitischen Sprachen zeugen aber zwei sehr augenfällige Übereinstimmungen:

1. Ähnlichkeiten im grammatischen Baue, sogar in gewissen Formativlauten, zumal dem \textit{t} des Femininums.

\sed{{\textbar}{\textbar}161{\textbar}{\textbar}}\phantomsection\label{sp.161}

2. Die wesentliche Übereinstimmung in den Personalpronominibus und Zahlwörtern:

\begin{table}[h]
\centering
\begin{tabular}{l || l | l | l !{\vrule width 1.2pt} l | l | l} 
% \lsptoprule
\hline 
\hline
& \multicolumn{3}{l!{\vrule width 1.2pt}}{Hamitisch} & \multicolumn{3}{l}{Semitisch}\\
\cline{2-7} & Ägypt. & Galla u.~s.~w. & Berberisch & Arabisch & Äthiop. & Hebräisch\\
% \lsptoprule
\hline 
\hline
ich & \textit{\r{ḁ}nn\textsubring{u}k} & \textit{ani} & \textit{nek} & \textit{anā} & \textit{ana} & \textit{ānōkhi}\\
Du m. & \textit{\textsubring{e}nt\textsubring{u}k} & \multirow{2}{*}{\LARGE \} \normalsize \textit{ati}} & \textit{kai} & \textit{anta} & \textit{anta} & \textit{attah}\\
Du f. & \textit{\textsubring{e}nt\textsubring{u}t} &  & \textit{kem} & \textit{anti} & \textit{antī} & \textit{att}\\
er & \textit{\textsubring{e}nt\textsubring{u}f} & \textit{ini} & \textit{enta} & \textit{huwa} & \textit{u’tū} & \textit{hū’}\\
sie & \textit{\textsubring{e}ntus} & \textit{išin} & \textit{entat} & \textit{hiya} & \textit{i’tī} & \textit{hī’}\\
wir & \textit{anon} & \textit{unu (anno)} & \textit{nekkeniḏ} & \textit{naḥnu} & \textit{neḥna} & \textit{anaḥnū}\\
ihr& \textit{\textsubring{e}nt\textsubring{u}t\textsubring{e}n} & \textit{izin} & (m.) \textit{kaueniḏ} & \textit{antum} & \textit{ant\textsubring{e}mu} & \textit{attem}\\
&  &  & (f.) \textit{kametiḏ} & \textit{antunna} & \textit{ant\textsubring{e}n} & \textit{atten}\\
\hline
1 & \textit{ua} & \textit{toko} & \textit{iien, iiet} & \textit{aḥadu} & \itshape ‛aḥadū & \textit{ehadh}\\
2 & \textit{son} & \textit{lama} & \textit{sin} & \textit{iθnāni} & \textit{kel’etū} & \textit{šenayim}\\
3 & \textit{χemet} & \textit{sadi} & \textit{keraḏ} & \textit{θalāθu} & \textit{salastu} & \textit{šālōš}\\
4 & \textit{fetu} & \textit{afur\textsubring{i}} (\textit{faḍig}) & \textit{okkoz} & \textit{arba‛u} & \textit{arbā‛} & \textit{arba‛}\\
5 & \textit{ṭua} & (Bilin:) \textit{ankua} & \textit{semmus} & \textit{χamsu} & \textit{χam\textsubring{e}s} & \textit{ẖāmēš}\\
6 & \textit{sas} & \textit{dja, tša} & \textit{sedis} & \textit{sittu} & \textit{s\textsubring{e}d\textsubring{e}s} & \textit{šēš}\\
7 & \textit{seχef} & \textit{torbḁ} & \textit{essaa} & \textit{sab‛u} & \textit{s\textsubring{e}b‛e} & \textit{šebha‛}\\
8 & \textit{sesennu} & \textit{sadēt\textsubring{i}} & \textit{ettam} & \textit{θamāni} & \textit{samāni} & \textit{š\textsubring{e}moneh}\\
9 & \textit{peseṭ} & \textit{sagalḁ} & \textit{tezzaa} & \textit{tis‛u} & \textit{t\textsubring{e}s‛e} & \textit{teša‛}\\
10 & \textit{met} & \textit{kundan\textsubring{i}} & \textit{merau} & \textit{‛ašru} & \textit{‛\textsubring{e}šr} & \textit{‛eser}\\
&  & (Bilin:) \textit{šika} &  &  &  & \\
&  & (Somali:) \textit{toban} &  &  &  & \\
\end{tabular}
\end{table}

\fed{{\textbar}172{\textbar}}\phantomsection\label{fp.172}

Es ist interessant, dass sich ein Theil dieser Übereinstimmungen bis tief hinein in die Sprachen von Negervölkern erstreckt. \textsc{Friedrich Müller}, Grundriss der Sprachwissensch. I, \textsc{2}, S.~236, sagt mit Recht: „Diese tiefgreifenden Übereinstimmungen des Hausa und anderer afrikanischer Idiome können nach unserer Überzeugung ohne die Annahme eines tiefgreifenden vorhistorischen Einflusses der Hamito-Semiten auf die Neger nicht erklärt \update{werden.}{werden.“} \textsc{R. Lepsius }hat bald darauf den gleichen Gedanken in der Einleitung zu seiner Nubischen Grammatik weiter ausgeführt.

\sed{{\textbar}{\textbar}162{\textbar}{\textbar}}\phantomsection\label{sp.162}

\sloppypar{
Eine Vergleichung der possessiven und verbalen Pronominalaffixe würde weitere, nicht weniger augenfällige Übereinstimmungen ergeben, und zur Herstellung einer recht eingehenden vergleichenden Grammatik der hamito-semitischen Sprachen liegt wohlgesicherter Einzelstoff in Menge vor. Noch wichtiger aber ist meiner Meinung nach eine gründliche lexicalische Vergleichung, wie sie eben von \textsc{Leo Reinisch} in seinen Arbeiten zur Kunde der nordostafrikanischen Sprachen angebahnt wird. Der starre Schematismus im semitischen Wurzel- und Stammbildungswesen findet in den hamitischen Sprachen nicht Seinesgleichen. Es fragt sich: gehört er zum ursprünglichen Gemeingute der Familie, ist er drüben von der Urzeit her erhalten, hüben verloren gegangen? oder hat er sich nicht vielmehr erst nach der Trennung der Hamiten von den Semiten bei diesen entwickelt? Und wenn das: wie war die ursprüngliche Gestaltung, wie sind die Wandelungen geschehen? Wir haben hier vorgegriffen, in die innere Sprachgeschichte hinein; die wird aber hier wie oft der äusseren Sprachgeschichte die Leuchte halten müssen.
}

Statt aber den langweilig verständigen Weg vom Näheren zum Entfernteren einzuschlagen, konnte man es nicht erwarten, Japhet’s Nachkommen mit denen Sem’s auch sprachlich vervettert zu sehen und ging, zuweilen mit wahrem Scharfsinn, mit fein ersonnener Methode, an ein verfrühtes Werk. Gegenüber der Culturgemeinschaft der beiden Rassen hatte man die tiefgehende Geistesverschiedenheit der Sprachen unterschätzt.

Von mehr linguistischem Treffer, aber freilich recht mangelhafter Methode zeugt ein kleines Buch des englischen Missionars \textsc{D. Macdonald}, Oceania, Linguistic and Anthropological (Melbourne und London 1889). Der Verfasser behandelt die Malaien als Absenker der Semiten. Eine unleugbare, wenn auch rohe Geistesverwandtschaft zwischen den Sprachen \fed{{\textbar}173{\textbar}}\phantomsection\label{fp.173} leuchtet ihm ein, und nun versucht er, – freilich ganz ohne feste lautgesetzliche Methode, – auch die leibliche Verwandtschaft der beiden Sprachstämme nachzuweisen.

\subsection*{II.}\phantomsection\label{III.I.I.4.II}
\subsection*{Verwandtschaft des Nahuatl mit den Algonkin-Sprachen.}
Im Baue und in der inneren Form zeigen die amerikanischen Sprachen oft auch da auffällige Gleichheit, wo mit den bisherigen Mitteln eine leibliche Verwandtschaft kaum erweisbar scheint. Um so schätzbarer, wenigstens als Fingerzeige zu weiteren Untersuchungen, sind lautliche Übereinstimmungen, wie sie mir zwischen dem Nahuatl (Mexicanischen) und den Algonkinsprachen aufgestossen sind. Davon einige Beispiele:

\sed{{\textbar}{\textbar}163{\textbar}{\textbar}}\phantomsection\label{sp.163}

\begin{table}[h]
\centering
\begin{tabular}{ c || c | l} 
% \lsptoprule
\hline 
\hline
& Nahuatl & \multicolumn{1}{c}{Algonkinsprachen}\\
% \lsptoprule
\hline 
\hline
ich & \textit{ne}, \textit{newa} & Lenape: \textit{ni} Kri \textit{nita}, \textit{niya}\\
du & \textit{te}, \textit{tewa} &~„~~~~„~~~~: \textit{ki}~~„~~\textit{kita}, \textit{kiya}\\
er, sie, es & \textit{e}, \textit{yewa} &~„~~~~„~~~~: \textit{neka}~„~\textit{wita}, \textit{wiya}\\
wir & \textit{tewan} & Algonkin, Odschibwe: \textit{kinawin}\\
sie (pl.) & \textit{yewan} &~„~~~~~~~„~~~~~~~„~~~~~~~„~~~~~~~: \textit{winawa}\\
vier & \textit{nahui} & Kri: \textit{newo}, Mikmak \textit{neu}\\
zehn & \textit{matlaktli} & Kri: \textit{mitātat}, Mikmak \textit{metelen}\\
\end{tabular}
\end{table}

Das ist nun für amerikanische Sprachenverhältnisse schon recht beachtlich; man muss nur wissen, wie weit die Algonkinsprachen unter sich schon in einem Theile der Zahlwörter auseinandergehen. Dass aber eine methodische Vergleichung dieser Sprachen auf ihren Wortschatz hin, zunächst untereinander und dann mit dem Nahuatl, guten Erfolg verspräche, wage ich schon jetzt zu behaupten.

\pdfbookmark[2]{§. 5. Zusatz II. Stammbaum- und Wellentheorie.}{III.I.I.5}
\cohead{§. 5. Zusatz II. Stammbaum- und Wellentheorie.}
\subsection*{§. 5.}\phantomsection\label{III.I.I.5}
\subsection*{Zusatz II.}
\subsection*{Stammbaum- und Wellentheorie.}
Der Streit, den die beiden obigen Stichwörter bezeichnen, galt ursprünglich nur den verwandtschaftlichen Verhältnissen der indogermanischen Sprachen, hat aber zu grundsätzlichen Erörterungen geführt, die ein weiteres Interesse beanspruchen.
\largerpage[-2]

\fed{{\textbar}174{\textbar}}\phantomsection\label{fp.174}

\textsc{August Schleicher} (Compendium der vergl. Gramm. der indogerman. Sprachen. 3. Aufl. S.~5 flg.) theilt unsern Sprachstamm in drei Hauptäste: 1. den asiatischen oder arischen, indo-eranischen, dessen eranischem Hauptzweige er auch das Armenische \update{zuzählt;}{zuzählt:} 2. den südwesteuropäischen, gräcoitalokeltischen; endlich 3. den slavodeutschen. Den Sitz des Urvolkes und der Ursprache sucht er in Asien; als dem Urtypus am ähnlichsten gelten ihm die arischen Sprachen. Und nun nimmt er (S.~7 flg.) an, je stärker die Abweichungen von jenem Urtypus, desto älter seien die Abzweigungen: „Die \so{indogermanische Ursprache} theilte sich zuerst durch ungleiche Entwickelung in verschiedenen Theilen ihres Gebietes in zwei Theile; es schied nämlich von ihr aus das \so{Slavodeutsche} (die Sprache, welche später in Deutsch und Slavo-Litauisch auseinanderging); sodann \update{theilte}{teilte} sich der zurückbleibende Stoff der Ursprache, das \so{Ariograecoitalokeltische}, in \so{Graecoitalokeltisch} und \so{Arisch}, von denen das Erstere in \so{Griechisch} (-Albanesisch) und \so{Italokeltisch} sich schied, das Letztere, das Arische, aber noch lange vereint blieb ... Je östlicher ein indogermanisches \sed{{\textbar}{\textbar}164{\textbar}{\textbar}}\phantomsection\label{sp.164} Volk wohnt, desto mehr Altes hat seine Sprache erhalten, je westlicher, desto weniger Altes und desto mehr Neubildungen enthält sie. Hieraus, wie aus anderen Andeutungen folgt, dass die Slavodeutschen zuerst ihre Wanderung nach Westen antraten, dann folgten die Graecoitalokelten; von den zurückbleibenden Ariern zogen sich die Inder südostwärts, die Eranier breiteten sich in der Richtung von Südwest aus. Die Heimath des indogermanischen Urvolks ist somit in Centralasien zu suchen.“ – Dies im Wesentlichen seine Theorie. Er hat sie bequem und geschickt in Form eines Stammbaumes versinnlicht, dem sie \retro{nun}{nur} ihren Namen verdankt. Wieweit sie sonst in ihren Einzelheiten bestritten worden, dürfen wir hier übergehen; nur den einen verhängnissvollen Punkt müssen wir erwähnen.

In der Behandlung der Gutturalen stimmen die lituslavischen Sprachen mit den arischen auffällig, wenngleich nicht ganz ausnahmslos überein. Schon \textsc{Bopp} war dadurch zur Annahme einer engeren Verwandtschaft zwischen den beiden Familien gedrängt worden. \textsc{Schleicher} betrachtete das Zusammentreffen als ein zufälliges, auf beiderseits selbständiger paralleler Entwickelung beruhendes. Dem konnte \textsc{Johannes Schmidt} (Die Verwandtschaftsverhältnisse der indogermanischen Sprachen. 1872) nicht beipflichten. So entschieden sich die lituslavischen Sprachen durch gewisse Merkmale den germanischen nähern, ebenso sicher sind sie durch \fed{{\textbar}175{\textbar}}\phantomsection\label{fp.175} jenes mit den arischen verbunden. Diese nun wieder andrerseits dem Griechischen, weiterhin das Griechische den italischen, diese den keltischen, endlich diese den germanischen, – sodass jede Familie in einer Art Ringelreigen nicht einen, sondern zwei nächste Verwandte habe. Der Urzustand wäre der ununterbrochener Übergänge gewesen. Woher nun nachmals die Grenzen? Wo die Dialekte unmerklich ineinander übergehen, da herrscht Sprachgemeinschaft. So sind wir auf die Analogie der Einzelsprache hingewiesen, deren Beispiel allerdings jener Übergangstheorie zu statten kommt. Aber auch das lehrt die Sprachgeschichte, dass mächtigere Mundarten, das heisst die Mundarten mächtigerer Gemeinden, mit der Zeit ihre Nachbarinnen verschlingen können. Sie fressen um sich; und in dem Masse, wie sie dies thuen, werden fernerstehende, also minder ähnliche, ihre Nachbarinnen. Dann mag ihnen wohl auch von einem anderen Mittelpunkte aus entgegengearbeitet werden, und nun stösst aneinander, was sich früher sehr fern stand, und was sich heute noch scharf unterscheidet; schroffe Grenzen sind an die Stelle der leisen Übergänge getreten. Auch fremdsprachige Völker mochten sich keilartig eindrängen, die alten Nachbarn seitab treibend, vielleicht sie zum Theile vernichtend, zum Theile in sich aufnehmend, das heisst Zwischenstufen wegräumend.

Doch nicht hierin, nicht in dem unbestreitbar möglichen Eingreifen fremder Mächte, beruht das Schwergewicht der \textsc{Schmidt}’schen Theorie (der sog. Wellentheorie), sondern darin, dass sie erklärt, wie es möglich ist, dass ein ununter- \sed{{\textbar}{\textbar}165{\textbar}{\textbar}}\phantomsection\label{sp.165}brochener Zusammenhang der Mundarten, zunächst ohne politische Scheidung der Volksgemeinschaft, von \update{Innen}{innen} heraus zerrissen werden kann. Insoweit ist sie meiner Meinung nach unanfechtbar, und darin liegt ihr prinzipieller \update{Werth,}{Wert,} gleichviel wie es um die Verwandtschaftsverhältnisse der indogermanischen Sprachen stehe.

\pdfbookmark[2]{§. 6. Zusatz III. Die Sprachen von Kabakada und von Neu-Lauenburg etc.}{III.I.I.6}
\cohead{§. 6. Zusatz III. Die Sprachen von Kabakada und von Neu-Lauenburg etc.}
\subsection*{§. 6.}\phantomsection\label{III.I.I.6}
\subsection*{Zusatz III.}
\subsection*{Die Sprachen von Kabakada und von Neu-Lauenburg, – ein Ausnahmefall.}
Wo sonst in der Welt Sprachen oder Mundarten einander besonders nahe stehen, da pflegt ihr Hauptunterschied im Lautwesen zu beruhen, Grammatik und Wortschatz dagegen bis auf Kleinigkeiten übereinzu\fed{{\textbar}176{\textbar}}\phantomsection\label{fp.176}stimmen; man spricht hüben und drüben mehr oder weniger dasselbe, man spricht es nur anders aus. Man mache den Versuch mit Deutsch und Holländisch, mit Italienisch und Spanisch oder Portugiesisch, mit Finnisch und Esthnisch, mit zwei beliebigen Sprachen der slavischen, semitischen, polynesischen Familie, so wird man das bestätigt finden.

Ein einziges Mal ist mir fast das gerade Gegentheil vorgekommen. Die Sprache von Kabakada auf der Gazelle-Halbinsel von Neu-Pommern (Neu-Britannien) gleicht jener von Neu-Lauenburg (Duke of York Island) im Lautlichen bis auf einen vielleicht mehr orthographischen Unterschied vollständig; was verwandt ist, stimmt in der Regel auch buchstäblich überein, und wo es das nicht thut, da beruht der Unterschied in der Art der Wortbildung. Lautvertretungsgesetze giebt es zwischen den Beiden nicht. Auch das Grammatische, Wortformungsmittel, Formwörter und Satzbau, ist in der Hauptsache auf beiden Seiten gleich. Um so grösser sind die Verschiedenheiten in den Stoffwörtern. Was sich da nicht entweder völlig oder bis auf die \update{Bildungs\-sylben}{Bildungs\-silben} gleicht, ist in der Regel grundverschieden; und dessen ist erstaunlich viel. Man muss sich schon besinnen, um einen italienischen Satz zu finden, dessen spanische Übersetzung nicht zum guten Theile aus denselben Wörtern besteht. Man wird aber erst recht, \fed{man wird} vielleicht vergebens suchen müssen, einen Kabakada-Satz von nur einem Dutzend Wörtern aufzustellen, der sich mit Anwendung derselben Stoffwörter in die Duke of York-Sprache übersetzen liesse.

\begin{sloppypar}Allem Vermuthen nach kann es verhältnissmässig nicht lange her sein, dass beide Völker noch völlig gleichsprachig waren: sonst wäre das gemeinsame Sprachgut im Lautwesen weiter auseinandergegangen. Nun müssen aber hüben oder drüben, wo nicht auf beiden Seiten, sehr heftige Störungen eingetreten, fremde Elemente eingemischt worden sein. Und das ist allerdings in den melanesischen Sprachen besonders leicht möglich.\end{sloppypar}

\sed{{\textbar}{\textbar}166{\textbar}{\textbar}}\phantomsection\label{sp.166}

Erstens mag wohl auch hier das Tabuwesen den Wortgebrauch ändern. Scheinbar willkürlich, aus abergläubischen Gründen, werden gewisse Ausdrücke verpönt und conventionell durch andere ersetzt. Ob sie dann, nach Erlöschen des Verbotes, wieder auftauchen, hängt von den Umständen ab.

Zweitens sind die Melanesier das Gegentheil von Puristen. Unsere Gewährsmänner klagen darüber, wie schnell Fremdes, Fehlerhaftes, bloss weil es von radebrechenden Fremden gesagt worden, angenommen und \fed{{\textbar}177{\textbar}}\phantomsection\label{fp.177} nachgeahmt wird. Dies Laster mag alt sein. Die Wörter, in denen die melanesischen Sprachen am Meisten mit den \update{malaischen}{Malaischen} übereinstimmen, sind gerade solche, die man überall sonst am Spätesten aufgiebt: Pronomina, Zahlwörter, die gebräuchlichsten Substantiva und gewisse Partikeln. Es sind das aber auch die Wörter, die man den Fremden am Ersten ablauscht, die man also, wenn man sonst will, sich am Schnellsten aneignen kann. Nun kam es nur darauf an, bei wem jeder Theil, und was er borgte, und in kurzer Zeit konnten sich die ärgsten Verschiedenheiten einstellen.

\pdfbookmark[2]{§. 7. Zur Technik.}{III.I.I.7}
\cohead{§. 7. Zur Technik.}
\subsection*{§. 7.}\phantomsection\label{III.I.I.7}
\subsection*{Zur Technik.}
\subsection*{Collectaneen zum Verwandtschaftsnachweise.}
Gilt es, zu ermitteln, welchem Verwandtschaftskreise eine Sprache zugehöre, so ist dem Gesagten nach die \update{lexikalische}{lexicalische} Vergleichung die nächst notwendige. Um diese zu erleichtern, legt man am Besten eine Sammlung an, die ich noch nicht ein vergleichendes Wörterbuch, sondern nur ein \so{Wörterbuch zur Vergleichung} nennen möchte. Zettelcollectaneen sind hier besonders zu empfehlen. Es fragt sich, wie sie am Zweckmässigsten zu ordnen seien?

Hat man, wie ich es empfehle, schon das einzelsprachige Wörterbuch auf Zetteln angelegt, so ist viel Arbeit gespart: man braucht die Zettel nur umzuordnen, hat nicht die Mühe der doppelten Schreiberei. Die neue Ordnung aber muss für ihren Zweck möglichst übersichtlich sein.

Nun haben die verwandten Wörter in verschiedenen Sprachen nicht allemal die gleiche, sondern oft nur eine ähnliche Bedeutung. Also müssen die Wörter thunlichst nach ihren Bedeutungen, mit anderen Worten encyklopädisch geordnet sein. Es leuchtet ja ein, dass es umständlich wäre, wenn man etwa „wollen, wünschen, begehren, verlangen, streben“ an fünf verschiedenen Orten aufsuchen müsste, dass es ärgerlich wäre, wenn man an vier Flecken auf einen fünften verwiesen würde.

Ein Schema für ein solches Wörterbuch, dessen Bequemlichkeit ich erprobt habe, will ich nun mittheilen.

I. \so{Pronomina}. A. Personalia. B. Demonstrativa, reflexiva, determinativa, \sed{{\textbar}{\textbar}167{\textbar}{\textbar}}\phantomsection\label{sp.167} indefinita. C. Possessiva. D. Fragwörter (einschliesslich der Fragadverbien, die ja in der Regel pronominal sein werden).

\fed{{\textbar}178{\textbar}}\phantomsection\label{fp.178}

II. \so{Zahlwörter}, bestimmte und unbestimmte.

III. \so{Substantiva}. A.~Gott, Himmel, Gestirne. B.~Himmelsgegenden. C.~Zeit. D.~Wetter. E.~Erde (Land, Feld, Ebene, Weg, Ort, Berg u.~s.~w.). F.~Stein, Metall. G.~Feuer (Funke, Flamme, Rauch, Asche, Kohle). H.~Wasser. I.~Pflanzen und ihre Theile. K.~Thiere. a.~Säugethiere. b.~Vögel u.~s.~w. L.~Mensch. M.~Körpertheile. a.~Kopf. b.~Hals, Rumpf. c.~Extremitäten. d.~Sonstige Körpertheile, Ausscheidungen u.~s.~w. (Haut, Knochen, Ader, Blut). – Anhang: Geist, Schatten, Name, Stimme, Wort. N.~Wohnung. O.~Schiff. P.~Waffen und Geräthe. Q.~Gefässe. R.~Kleidung, Schmuck. S.~Nahrung. T.~Allgemeines (Ding, Stück, Theil, Masse u.~dgl.).

IV. \so{Adjectiva}. A.~Gross u.~s.~w. (lang, stark, dick, hoch, alt, schwer \update{...).}{...),} B.~Klein (kurz ...). C.~Gestalt, Consistenz. D.~Farben. E.~Eigenschaften des Gefühls, Geschmackes, Geruches, Gehöres. F.~Körperliches Befinden. G.~Gemüths- und Verstandeseigenschaften. H.~Allgemeine (wahr, gleich, ähnlich, ganz, fertig u.~s.~w.).

V. \so{Adverbien}. A.~Der Zeit. B.~Des Ortes. C.~Der Art und Weise.

VI. \so{Conjunctionen}.

VII. \so{Präpositionen} oder \so{Postpositionen}. Casusaffixe.

VIII. Verba. A.~Sagen, sprechen u.~s.~w. B.~Denken u.~s.~w. (wollen, lieben, hassen, vergessen ...). C.~Leben, Körperfunctionen. D.~Gehen, kommen u.~s.~w. (laufen, treten, folgen, steigen, fliessen, schwimmen, fallen, tröpfeln ...). E.~Da sein, verweilen (stehen, sitzen, liegen ...). F.~Andere Verba. (Schwer zu classificirende, für die die alphabetische Ordnung als Nothbehelf dienen muss.)

Dies Schema ist gewiss noch sehr verbesserungsfähig und erspart natürlich das Hin- und Herblättern nicht ganz, verringert es aber doch. Andere Gruppirungen sind ja wohl denkbar und können sich unter Umständen bewähren, z.~B. 

Auge, sehen, blind;

Sonne, Tag, hell, leuchten.

Allein erstens können die Ideenverbindungen von einem Punkte aus nach sehr verschiedenen Seiten verlaufen; und zweitens wären solche aus allen Redetheilen zusammengestellte Gruppen kaum zu einem übersichtlichen Ganzen zu vereinigen.

Hat man nun ein solches Wörterbuch angelegt, so hält man sich \fed{{\textbar}179{\textbar}}\phantomsection\label{fp.179} zunächst an die Sprachen, die nach dem früher Gesagten in erster Linie der Verwandtschaft verdächtig sind, trägt ähnlich Klingendes ein und sieht zu, wie weit man damit kommt. Geht es gut, so ergeben sich bald gewisse Regelmässigkeiten in der \so{Lautvertretung}, für die man nun weitere Collectaneen anlegen muss.

Nun fragt man sich: Können die Übereinstimmungen nicht auch auf Ent- {\sed{{\textbar}{\textbar}168{\textbar}{\textbar}}\phantomsection\label{sp.168} lehnung beruhen? Denn dass sie nicht zufällig sind, das haben eben jene Lautvergleichungen ergeben. Die Frage betrifft sowohl die \update{Menge}{Menge,} als die Art des Verwandten; es kann sehr Vieles entlehnt sein, wie im Englischen aus dem Altfranzösischen, – und doch gerade das Wesentlichste nicht. Hier muss die Vergleichung des Sprachbaues, der Wortformen und Formwörter entscheiden; und somit reiht sich an die lexikalische und phonetische Vergleichung die \so{grammatische} an.

Das ganze hier geschilderte Verfahren ist scheinbar rein mechanisch und ist es oft auch wirklich. Allein in vielen Fällen wird neben einem guten Gedächtnisse, das die Arbeit verkürzt, auch ein gewisser Tact erfordert, der den Forscher vor thörichten Combinationen behütet, also ein Verständniss für das, was in der Sprachgeschichte möglich und wahrscheinlich ist.

\pdfbookmark[1]{II. Theil.}{III.II}
\cehead{{{\large III,}} II. Die innere Sprachgeschichte.}
\cohead{I. Allgemeines. §. 1. Ihre Aufgaben.}
\pdfbookmark[2]{I. §. 1. Ihre Aufgaben.}{III.II.2}
\subsection*{Zweiter Theil.}
\section*{Die innere Sprachgeschichte.}
\subsection*{Erstes Hauptstück.}
\section*{Allgemeines.}
\subsection*{§. 1.}\phantomsection\label{III.II.1}
\subsection*{Aufgaben der inneren Sprachgeschichte.}
Alle Sprachen sind dem Wandel ausgesetzt, alle unterliegen ihm in höherem oder geringerem Grade, schneller oder langsamer. Und zwar in allen ihren Theilen. Hatten wir früher gelernt, zwischen Sprachschatz und Sprachbau und bei beiden wieder zwischen den zu deutenden Erscheinungen und den anzuwendenden Mitteln zu unterscheiden, und für \fed{{\textbar}180{\textbar}}\phantomsection\label{fp.180} die Grammatik letzte Elemente und oberste Gesetze anzuerkennen, die weder mehr zu analysiren noch durch Synthese zu gewinnen waren: so tritt nunmehr ein neues Moment hinzu, nämlich der Wandel aller dieser Dinge im Laufe der Zeit. Es ist ein Leichtes, für alles dies Beispiele aufzufinden, wenn man etwa dem Lateinischen das Französische gegenüberstellt.

Damit wäre jedoch nicht mehr geleistet, als wenn man etwa die deutschen Zustände des Jahres 1800 mit jenen des Jahres 800 vergleichen wollte. Gewisse allgemeine Gesichtspunkte liessen sich wohl hüben und drüben aufstellen; denn das Gleichzeitige muss ja organisch \update{zusammen\-hängen; aber}{zusammen\-hängen. Aber} in welcher Reihenfolge das Eine nach und aus dem Anderen geworden, welches dabei die \sed{{\textbar}{\textbar}169{\textbar}{\textbar}}\phantomsection\label{sp.169} treibenden und hemmenden Kräfte gewesen, das bliebe verschwiegen; man hätte zwei parallele Schilderungen statt einer fortlaufenden Geschichte.

Diese aber soll gewonnen werden, und zwar eine möglichst wissenschaftliche, die die \update{Thatsachen,}{Thatsachen} nicht nur aufzählt, sondern auch in ihrem Werden erklärt. Und will sie dies erschöpfend thun, so muss sie alle Theile und Seiten der Sprache mit gleichem Interesse erfassen, weil alle ein organisches Ganzes bilden. Der Wortschatz wie die Grammatik, die Laut- und Formenlehre wie die Syntax, die äussere Gestalt der Wörter und grammatischen Mittel wie ihre Bedeutungen und daraus sich ergebenden Anwendungen, sie alle wollen in ihren Wandelungen verfolgt, beschrieben, wo möglich erklärt werden.

Jede Sprache, auch die beständigste, ist in einem fortwährenden Werden begriffen. Die Veränderungen geschehen in der Regel nicht hüpfend, sondern, dass ich es so nenne, schlüpfend, jeden Punkt der Entwickelungslinie berührend, keinen überspringend. Krisen, Metamorphosen, epochemachende, vielleicht verhängnissvolle Schicksalsfälle kennt das Leben der Sprache nicht weniger, als das der natürlichen Organismen und der Völker; und gerade solche Zeiten gesteigerter Lebensthätigkeit oder acuten Leidens mögen besonders lehrreich sein, weil sie das Walten der treibenden und störenden Kräfte recht handgreiflich vorzuführen scheinen. Nur fragt es sich: Kommen dabei auch alle Kräfte zum Vorscheine, oder etwa nur die jeweilig stärksten? mit anderen Worten: \update{kann}{Kann} das, was die Krisen herbeiführt, nicht etwas ganz Anderes sein, als das, was bis dahin mit lindem Drucke die Sprache vorwärts \update{geschoben}{gehoben} hatte? Sind alle bewegenden Mächte derart, dass sie einmal zu plötzlicher Heftigkeit ge\fed{{\textbar}181{\textbar}}\phantomsection\label{fp.181}steigert werden können? und wenn nicht: sind die steigerungsfähigsten auch dieselben, die in Zeiten ruhiger Entwickelung am \update{Mäch\-tigsten}{mäch\-tigsten} treiben? Diese Fragen drängen sich uns von selbst auf, sobald wir der Sache etwas ernster nachdenken. Dann aber leuchtet auch ein, erstens, dass die Antwort nur auf dem Wege der Erfahrung zu finden, und zweitens, dass vor der Hand hierzu unsere Erfahrung noch viel zu jung und zu beschränkt ist. Die Mehrzahl der historischen Sprachforscher widmet sich der Indogermanistik oder dem einen oder anderen Theile derselben. Aber die indogermanischen Sprachen machen knapp ein Zwanzigstel der Sprachen unseres Erdballes aus. Und wenn sie die besterforschten sind, so stehen sie doch wieder ihrem Baue und vermuthlich auch ihren Schicksalen nach recht vereinzelt da. \update{Aber wieweit}{Und wie weit} ist man denn mit der Vergleichung und mit der Geschichtsforschung? Man untersucht den Wandel der Laute und der Formen; – der Etymologie, der Syntax und der Lehre vom Wechsel der Wortbedeutungen tritt man nur zögernd nahe, begnügt sich in der Regel mit einem Theile des Theiles. Und dann: wie schnell waren innerhalb eines halben Jahrhunderts die prinzipiellen Fortschritte und Meinungswechsel! Die drei vergleichenden Grammatiken von \textsc{Bopp}, \textsc{Schleicher} und \textsc{Brugmann} mögen als Marksteine dienen: \sed{{\textbar}{\textbar}170{\textbar}{\textbar}}\phantomsection\label{sp.170} Wie verschieden sind sie; und wenn nach weiteren fünfundzwanzig Jahren ein neues Werk dieser Art erscheint: wie wird das sich ausnehmen?

Gerade in diesen zwei Punkten aber, in ihrer Selbstbeschränkung und in ihrem bis zum Umsturze raschen Fortschreiten, beruht der unvergleichliche Lehrwerth der Indogermanistik. Ihrem geschäftigen Treiben sollte Jeder beiwohnen, der sich mit geschichtlicher Sprachvergleichung befassen will, beiwohnen, nicht in der Arena, – das ist nicht Jedermanns Sache, – aber wenigstens von der Galerie aus, wo er den Kampf verfolgen kann, ohne den Staub schlucken zu müssen. Schon das wird anregend sein, hintereinander ein paar entsprechende Abschnitte aus \textsc{Bopp}’s, \textsc{Schleicher}’s und \textsc{Brugmann}’s vergleichenden Grammatiken zu lesen.
\largerpage

Denn in der That mag die Geschichte der Indogermanistik für alle sprachgeschichtliche Forschung einigermassen als typisch gelten. Sinnfällige Übereinstimmungen in den Wortstämmen und -formen leiteten \textsc{Bopp} zu einer Arbeit, die in ihren Zielen vorwiegend etymologisch war. Woher die Formativelemente? was bedeuteten sie, als sie noch selbständig waren? Darauf richtete sich die Neugier zuerst. \textsc{Pott}’s Verdienst bleibt \fed{{\textbar}182{\textbar}}\phantomsection\label{fp.182} es, zum ersten Male die indogermanische Etymologie auf die lautgesetzliche Grundlage gerückt zu haben. \textsc{Schleicher}, der scharfsinnige Schematiker, versuchte zuerst die Stammesursprache rückschliessend wieder aufzubauen. Ihr Lautwesen galt ihm für sehr einfach, Sandhigesetze sollte sie nicht gekannt haben, die volllautigsten Formen wurden als die alter\-thümlichsten angesehen, die Lautverschiebungen gestatteten manche Freiheiten, scheinbare Willkürlichkeiten und Ausnahmen, die noch unerklärt blieben. Damit war den Nachfolgern eine Aufgabe gestellt, von deren Schwierigkeit eine fast unübersehbar grosse Literatur zeugt. \sed{Der alte \textsc{Pott} musste sich schelten lassen, dass er nicht frischweg die \textsc{Schleicher}’schen Lautreconstructionen als ein Evangelium in sein etymologisches Riesenwerk aufnahm. Hatte er so Unrecht?} In der kurzen Spanne Zeit von \textsc{Schleicher} bis \textsc{Brugmann} hat sich die indogermanische Ursprache bis zur Unkenntlichkeit verändert! \sed{\textsc{Schleicher} liess seine Urindogermanen eine Fabel erzählen: Das Ross und das Schaf, \textit{akvas avis ka}. Hätten die heutigen Urindogermanen noch zum Fabuliren Laune, sie würden stattdessen in ihren verschiedenen Mundarten etwas sagen, was dem lateinischen \textit{equos ovis-que} sehr ähnlich klänge.}

\largerpage
\sed{Mir ist es zum Vorwurfe gemacht worden, }\corr{1901}{dass}{das} \sed{ die Discussionen der Indogermanisten über die Principien der sprachgeschichtlichen Forschung „doch kaum mein Denken in der Tiefe berührt haben.“ Auch „die Energie zielbewusster Methode“ ist von indogermanistischer Seite bei mir vermisst worden. Diese Energie besitzen die Indogermanisten in hohem Grade und bethätigen sie mit glänzendem Erfolge. Sie haben auch recht daran gethan, ihrem eigenen Forschungsgebiete ihre Heische- und Lehrsätze zu entnehmen, ihre Methode an}{\textbar}{\textbar}171{\textbar}{\textbar}\phantomsection\label{sp.171}\sed{zupassen. Und gerade hierin ist mir ihr Thun mustergültig. Dasselbe Recht, dieselbe Pflicht haben Jene, die andere Sprachfamilien historisch vergleichen, auch. Wollten sie unbesehen die Aufstellungen der Indogermanisten auf ihre Gebiete übertragen, so wäre dies nicht sehr methodisch und sicher unkritisch. Denn die Kritik verlangt Voraussetzungslosigkeit, und die Methode der inductiven Forschung verlangt, dass man den Thatsachen ihre Gesetze abfrage, nicht dass man die Gesetze als gegeben annehme und ihnen die Thatsachen zwangsweise, auf Biegen oder Brechen unterordne. Sie duldet auch nicht, dass man Gegeninstanzen, die jene Gesetze zu erschüttern drohen, durch unwahrscheinliche Hypothesen hinwegbugsire. Den Sinn für das Gesetzmässige, den Trieb nach Erkenntniss der Gesetze stelle ich so hoch, wie nur Einer. Aber ich verlange auch rückhaltlose Anerkennung der sich bietenden Thatsachen, der willkommenen wie der unwillkommenen; ich verlange ein folgerichtiges Denken, das den weiten Raum des Möglichen austastet, und einen in Seelen- und Lebenskunde geschulten Sinn, der unter der Menge der Möglichkeiten das Wahrscheinliche herausfindet. In Alledem weiss ich mich mit den Indogermanisten im Einverständniss: sie verlangen dasselbe. Wo sie sich aber streiten, und sie streiten sich noch immer recht viel, da scheint oft, ausgesprochener- oder unausgesprochenermassen, Einer am Anderen etwas von jenem Gefühle für das Wahrscheinliche zu vermissen, dem der Eifer des Systematisirens und des Schematisirens Gewalt angethan hat. Uns Forschern auf minder gepflegten Gebieten steht es am Wenigsten an, sie um dieses Eifers willen zu tadeln; ihm verdankt ihre Methode, trotz aller Zweifel in Einzelheiten, eine Exactheit, um die wir sie beneiden, die wir uns nach Kräften aneignen müssen.}

\begin{styleAnmerk}
\sed{Anmerkung 1. Übrigens böte die Literatur der finnisch-ugrischen Sprachvergleichung, wenn man ihr folgen wollte, für die Methodik fast ebensoviel Belehrung, wie die indogermanische, von der sie sich keineswegs in’s Schlepptau nehmen lässt. Ein dem \textsc{Vernes}’schen vergleichbares Gesetz hat der Finne \textsc{Alex. Castrén} (Om Accentens inflytande i Lappiskan) schon im Jahre 1844 nachgewiesen.}
\end{styleAnmerk}

\begin{styleAnmerk}
\sed{Anmerkung 2. Wann und wie haben sich in den jüngeren indogermanischen Sprachen die Wortstellungsgesetze verengt und gefestigt? In wie weit hat dieser Prozess mit dem Verblassen der Wortformen Schritt gehalten? Soviel mir bekannt, hat sich bisher die geschichtliche Forschung nur sehr ausnahmsweise auf diese Fragen eingelassen. Es wäre aber ein grosser Gewinn, wenn die Sache auch nur an einer einzelnen Sprache, etwa der englischen oder französischen, in ihrer historischen Entwickelung dargestellt würde.}
\end{styleAnmerk}

\subsection*{Zusatz.}\phantomsection\label{III.II.zusatz}
\textsc{Delbrück}, Einleitung in das Sprachstudium, 1. Aufl. S.~44–45, zieht in wahrhaft classischer Schärfe eine Parallele zwischen \textsc{Bopp} und \textsc{Schleicher}. Er sagt: „\textsc{Schleicher}’s Compendium steht als der Abschluss einer Periode in der \sed{{\textbar}{\textbar}172{\textbar}{\textbar}}\phantomsection\label{sp.172} Geschichte der Sprachwissenschaft den einleitenden Arbeiten \textsc{Bopp}’s gegenüber. Darum ist denn auch der Totaleindruck, den die vergleichende Grammatik einerseits und das Compendium andrerseits hervorbringen, so ausserordentlich verschieden. \textsc{Bopp} musste die wesentliche Gleichheit der indogermanischen Sprachen beweisen, \textsc{Schleicher} setzte sie als bewiesen voraus, \textsc{Bopp} erobert, \textsc{Schleicher} organisirt. \textsc{Bopp} wendete seine Aufmerksamkeit vorzüglich auf dasjenige, was allen indogermanischen Sprachen gemeinsam ist; für \textsc{Schleicher} ergab sich die Aufgabe, die einzelnen indogermanischen Sprachen auf dem gemeinsamen Hintergrunde hervortreten zu lassen. Deshalb ist die vergleichende Grammatik eine zusammenhängende Schilderung, während das Compendium ohne grosse Mühe in eine Anzahl von Einzelgrammatiken auseinandergenommen werden könnte. Der Verfasser der Grammatik giebt der Darstellung des Einzelnen überwiegend die Form der Untersuchung, die er mit grosser natürlicher Anmuth handhabt; das Compendium dagegen bewegt sich fast nur in dem knappen und gleichförmigen Stil der Behauptung. Das ältere Werk lässt sich mit der Darstellung eines interessanten Processes vergleichen, das jüngere mit den Paragraphen einer Gesetzsammlung.“ \sed{– Soweit \textsc{Delbrück}.}

\sed{\textsc{Schleicher} besass, wie Wenige, die Tugend der Selbstdisciplin, und ich glaube, dieser verdankte er zum grossen Theile den gewaltigen Einfluss, den er auf die Weiterentwickelung der indogermanischen Wissenschaft geübt. Er ermass seine Kräfte und Mittel, bemass darnach sorgsam das Gebiet dessen, was ihm erreichbar schien. Jede seiner Schriften macht den Eindruck sauberer Durch- und Ausarbeitung, – ganz anders als bei \textsc{Pott}, der das Licht seines gewaltigen Geistes unter den Scheffel eines wüsten Stiles stellte. Bei dem Jenenser baut sich alles klar und deutlich vor den Augen des Lesers auf, man hat den Plan und die Methode des Verfassers fast ebenso beständig und klar vor Augen, wie den Gegenstand selbst, den Mörtel fast ebenso wie die Steine. Was für den Gelehrten zur Schwäche und zur Gefahr werden konnte, was ihn bei aller Vor- und Umsicht gelegentlich zu vorschnellen Constructionen verleiten mochte: die Neigung des Hegelianers zu mechanischem Schematismus, gab den Worten des Lehrers die Macht einzuleuchten, zu überzeugen und zu haften. Und die kritische Methode, zu der er seine Jünger erzog, gab diesen die Kraft, indem sie sein Werk fortführten, seine Aufstellungen Stück für Stück umzureissen und durch neue zu ersetzen. Es ist, als übte der Todte noch fort und fort seine unerbittliche Selbstkritik.}

\fed{{\textbar}183{\textbar}}\phantomsection\label{fp.183}

\pdfbookmark[2]{I. §. 2. Alte und neuere Sprachen.}{III.II.2}
\cohead{I. Allgemeines. §. 2. Alte und neuere Sprachen.}
\subsection*{§. 2.}\phantomsection\label{III.II.2}
\subsection*{Alte und neuere Sprachen.}
\begin{sloppypar}Es ist sehr erklärlich, dass \update{der Forscher sich}{sich der Forscher} da am Wohligsten fühlt, wo es am Meisten zu entdecken giebt. Dass das Deutsche mit dem Holländischen, \sed{{\textbar}{\textbar}173{\textbar}{\textbar}}\phantomsection\label{sp.173} Englischen, Dänischen, Schwedischen, das Italienische mit dem Spanischen, Portugiesischen, Französischen, das Hebräische mit dem Arabischen, Syrischen, \update{Aethio\-pischen}{Äthio\-pischen} verwandt ist, brauchte eigentlich gar nicht entdeckt zu werden, weil es nie verdeckt war; das zeigte sich von selbst, sobald man nur hinsah. Dass aber im fernen Indien eine altehrwürdige Verwandte des Griechischen, Lateinischen, \update{Gotischen,}{Gothischen,} Litauischen weilte, das war eine entzückende Neuigkeit. Welche Erwartungen knüpften sich daran für die jugendlich hoffnungsfrohe Wissenschaft, welche Ausblicke in ein Alterthum, das noch weit hinter dem homerischen liegen musste, – aber auch welche Fülle neuer Räthsel! Die beiden \textsc{Grimm} durchwühlten heimischen, germanischen Boden. Was sie zu Tage förderten, mochte es auch nur Bronze oder Eisen sein, – der Dankbarkeit eines patriotisch fühlenden \update{Geschlechtes}{Volkes} dünkte es doch eitel Gold. Die Indianisten aber zeigten der Welt im fernen Osten \update{ächte}{echte} Gänge des edelsten Metalles, dessen Adern sich bis in unsern Erdtheil verästeln. So wirkten hüben und drüben gleich mächtige und doch sehr verschiedenartige Reize. \fed{\textsc{Pott}’s unsterbliches Verdienst ist es, dass er zuerst die Methode strenger Lautvergleichung auf das weite Gebiet der indogermanischen Forschung übertragen hat. Er hat damit allen Späteren den Weg gewiesen.}\end{sloppypar}

Mittlerweile nahm die Zweigforschung ihren Fortgang. \textsc{Diez} bearbeitete die romanischen, \textsc{Zeuss} die keltischen, \textsc{Miklosich} die slavischen Sprachen, und je länger je mehr vertiefte man sich in die Dialekte, d.~h. in die lebende Rede des gemeinen Mannes. \textsc{Pott} hatte noch das Motto gewählt:

Literis suus honos esto, litera animi nuncia.

Jetzt hatte man es nicht mehr mit steifen Buchstaben zu thun, sondern mit flüssigen Lauten, nicht mehr mit wohlgeschulten Texten, sondern mit naturwüchsigem Geplauder; nicht mehr drunten in \update{dunkelen}{dunklen} Tiefen hatte man zu arbeiten, sondern bei hellem Tageslichte. Und bald sollte man erfahren, wie auch hier die Blumen auf der Oberfläche von den verborgenen Erzgängen zu erzählen wissen. Dass die todten Sprachen von denselben \fed{{\textbar}184{\textbar}}\phantomsection\label{fp.184} Mächten beherrscht waren, die in den lebenden walten, konnte man eigentlich nie bezweifeln. Aber die Natur dieser Mächte schien man zu verkennen. Schon \textsc{Schleicher}, und vielleicht schon Mancher vor ihm, hatte es ausgesprochen, dass neben dem mechanischen Lautwandel noch eine seelische Kraft, die Analogie, in der Sprachentwickelung \mbox{wirke}. Allein das hat ihn nicht \update{abgehalten,}{aufgehalten,} die Sprachwissenschaft den Naturwissenschaften zuzuzählen; und anderen, schwächeren Geistern der materialistisch gerichteten Zeit ging \update{das}{dies} nur zu gut ein.

\textsc{Schleicher}’s Problem, die indogermanische Ursprache zu erschliessen, blieb und bleibt bestehen, mag man auch heute weniger hoffnungsvoll davon reden, als vor einem Vierteljahrhundert. Es ist mit den wissenschaftlichen Zweifeln wie mit den Köpfen der Hydra: für jeden, den man beseitigt hat, erwachsen ihrer zwei neue. Was man heute als Ursprache bezeichnet, jene Wörter mit dem Sternchen davor, sind doch ausgesprochenermassen nichts weiter, als formelmässige Ausdrücke für das, was nach der jetzigen Meinung des Sprachpalaeon\sed{{\textbar}{\textbar}174{\textbar}{\textbar}}\phantomsection\label{sp.174}tologen für wahrscheinlich gilt. Es ist damit, wie mit jenen Bildern \update{vorsünd\-fluthlicher}{vorsünd\-flutlicher} Thiere, deren Knochen man mit Fleisch und Haut bekleidet hat. Man malt sie braun an; sie können aber auch schwarz oder grau gewesen sein. \sed{Gewiss wird mit jedem Fortschritte, den die Wissenschaft macht, das Bild seinem Originale ähnlicher, dass es aber je so recht lebenswahr werde, darf man kaum hoffen. Gesetzt, wir besässen nur die neuromanischen Sprachen in ihrer jetzigen Gestalt, keine mittelalterlichen Denkmäler, geschweige denn Spuren der lingua rustica und des Lateinischen, – und nun würde uns die Aufgabe gestellt, die gemeinsame Stammessprache zu reconstruiren: wieweit würden wir wohl kommen, wieweit würden wir im Übereifer am Ziele vorbeischiessen? Und doch stehen Italienisch, Spanisch, Französisch, Rumänisch u.~s.~w. einander näher, als jene alten und alterthümlichen Sprachen, auf die unsere Indogermanisten angewiesen sind.}

Nehmen wir indessen einmal an, es gelänge, die Ursprache unseres Stammes wieder herzustellen, wie sie geleibt und gelebt: was wäre damit gewonnen? Die erste Antwort lautet recht bescheiden: Man hätte zu tausend bekannten Sprachen noch eine tausend und erste; man hätte auf mühsamstem Wege zum Lateinischen, Griechischen, \update{Gotischen}{Gothischen} u.~s.~w. das erobert, was für das Dänische, Schwedische und Isländische bequem in den Liedern der Edda und auf Runensteinen zu erholen ist; man hätte einen Granitblock aus dem Schachte gefördert, statt ihn vom Felde aufzulesen. \sed{Und schauen wir weiter zurück, nach der Zeit vor der Trennung der Indogermanen: so wissen wir gar nicht, eine wie lange Vorentwickelung diese Sprache gehabt hat, welche Veränderungen und Verschiebungen schon damals in ihr vorgegangen sein mögen. Wer weiss es? Könnten wir ihre Geschichte noch weitere zehntausend Jahre zurück verfolgen, so würde sie uns in dem Zustande, in dem wir sie vor der Spaltung antreffen, vielleicht recht modern, d.~h. recht verschliffen erscheinen.} Das ist die erste Antwort. Die zweite ist scheinbar schon tröstlicher: Bisher haben wir von der Oberfläche nach dem Centrum gebohrt; jetzt befinden wir uns im Centrum, und an die Stelle \update{inductiver}{induktiver} Wahrscheinlichkeit darf hinfort deductive Gewissheit treten; jetzt bohren wir vom Mittelpunkte aus nach beliebigen Punkten der Peripherie. Freilich wohl. Wüssten wir nur auch, was uns unterwegs Alles zustossen kann. Wir wären Götter, wenn es keinen Zufall mehr für uns gäbe. Den sichersten Gewinn nenne ich an dritter Stelle. Er beruht in der gethanen Arbeit selbst und in den dabei gesammelten Erfahrungen. In der Wissenschaft bleibt kein Umweg ungelohnt, und auch Jenen ge\fed{{\textbar}185{\textbar}}\phantomsection\label{fp.185}bührt Dank, deren vestigia terrent, den Tollkühnen, die den Hals gebrochen haben, und den Allzuschüchternen, die stecken geblieben sind. So ist der Hauptgewinn ein methodologischer, und die gewonnenen Lehren haben um so mehr Werth, je mühsamer sie erkämpft werden mussten. \sed{Man denke sich ein altes }{\textbar}{\textbar}175{\textbar}{\textbar}\phantomsection\label{sp.175}\sed{ schönes Glasgemälde in Hunderte winziger Stücke zerschlagen. Die müssten nun aus dem Staube zusammengelesen und in kunstvoller Mosaik zum alten Bilde wieder vereinigt werden: da hat man eine ungefähre Vorstellung von jener reconstructiven Arbeit, von dem Fleisse, aber auch von dem künstlerischen Verständnisse, das sie erheischt. Es gilt, Zerrissenes wieder organisch zu verbinden, Todtes neu zu beleben. Das kann nur der, der ein warmes, sicheres Gefühl besitzt für jenes Seelenleben, das sich in der Sprache, in der Rede äussert. Denn das ist es eben: Die Splitter sollen nicht nur mit den Kanten aneinanderpassen, sondern sie sollen sich auch zu einem lebenswahren Gesammtbilde zusammenschliessen. Jenes Gefühl, jenes Verständniss ist dem Forscher zunächst Mittel: Sein Zweck ist überall die Reconstruction und die Entdeckung der Formeln, nach denen sich die Wandlungen gesetzlich, – der Störungen, schuld deren sie sich regelwidrig vollzogen haben. Er hat es, solange er nicht in das Gebiet der allgemeinen Sprachwissenschaft hinüberschreiten will, mit der Entdeckung von Einzelthatsachen und particularen Gesetzen zu thun.}

Für die \sed{allgemeinen} Thatsachen und Gesetze des sprachgeschichtlichen Werdens hingegen sind natürlich die Erkenntnissquellen die besten, die zugleich am Reichsten und am Klarsten fliessen, und das sind die neueren und neuesten. Hier handelt es sich doch um Fragen, die belangreicher sind als jene: Wann ist indogermanisches \textit{p} zu germanischem \textit{f} geworden? Welches war der ursprüngliche Anlaut von „sechs, \textit{sex}, ἓξ, \textit{šaš}“? Wieweit ist der albanesische Wortschatz bodenwüchsig, wieweit entliehen? Das Alles sind doch nur Einzelheiten, Äusserungen von Gesetzen und Kräften, die wir zu erkennen verlangen. Wieweit eine solche Erkenntniss zu erlangen sei, muss die Zukunft lehren. Das aber ist ohne Weiteres einzusehen, dass die geschichtliche Sprachforschung ebensosehr in die Weite, wie in die Tiefe gehen muss, um gesicherte, gemeingültige Ergebnisse zu erzielen.

\begin{sloppypar}Nicht als Erkenntnissziele, sondern als Erkenntnissquellen sind die alten Sprachen wichtig. Um aber als Quellen zu gelten, müssen sie erschlossen sein, am liebsten sich selbst erschliessen. Und hier offenbart sich der Werth jener viel\-geschmähten historischen Orthographien für die sprachgeschichtliche \update{Forschung, der Mumien der Laut\-geschichte.}{Forschung der Lautgeschichte.} \sed{Überall haben wir auf die erreichbar ältesten Lautformen zurückzugehen, ehe wir eine Vergleichung wagen dürfen. Und wo solche Formen nicht urkundlich belegt sind, da gilt es, sie durch Vergleichung zu ermitteln. Hierbei verlangt die Methode, dass man von den engeren Kreisen, zu den weiteren fortschreite, sonst kann es geschehen, dass man Dinge für Petrefacten hält, die nur Putrefacten sind, dass man für urgemeinsam und uralt hält, was einander nur zufällig in seinem jetzigen Zersetzungszustande gleicht. Lateinisch iterum und deutsch \so{wiederum} sind einander sehr ähnlich und doch bekanntlich grundverschieden. Ägyptisch \textit{χeper} passt nun gar fast auf den Buch}{\textbar}{\textbar}176{\textbar}{\textbar}\phantomsection\label{sp.176}\sed{staben zu dem deutschen \so{Käfer}. Den schützt jedoch die Indogermanistik mit ihren erbarmungslosen Lautgesetzen. Wo aber unmethodische Entdecker ihre Füllhörner über die Wissenschaft ausschütten, da kann man Dinge, die um nichts besser sind, als wahre Perlen rühmen hören.}\end{sloppypar}

\sed{Manchmal, namentlich beim Beginn der vergleichenden Forschung, ist ein gewisser Leichtsinn recht heilsam. Man arbeitet eine Weile mit Fictionen, thut als wäre die Sprache, die des Alterthümlichen am Meisten zu bieten scheint, die Mutter oder doch der Urtypus der ganzen Familie. Über Schwierigkeiten, Unregelmässigkeiten, die sich ergeben, schlüpft man wohlgemuth hinweg und überlässt das Aufklären und Berichtigen der Zukunft. So wird schnell ein geräumiges, für den ersten Bedarf wohnliches Gebäude aufgeführt und die Ernte unter Dach und Fach gebracht, – wahrscheinlich viel Unkraut unter vielem Weizen. Wer sich dessen schämt, der wage sich nicht auf ein neues Gebiet; wem davor bangt, dass er in Einzelheiten irre, der verzichte darauf, im Grossen zu entdecken. Ich denke daran, wie gutgläubig unsere Altmeister \textsc{Bopp} und \textsc{Pott} die indogermanischen Sprachen am Sanskrit gemessen, wie zuversichtlich sie bis in die schwärzesten Tiefen hinein etymologisirt haben. Heute weiss man so Vieles besser; aber was wüsste man ohne sie? Und was sie in gutem Glauben gefehlt haben, das müssen wir ihnen heute noch mit offenen Augen nachmachen, wenn wir der historischen Sprachvergleichung ein neues Feld hinzuerobern wollen. Handelt es sich um das Δός μοι ποῦ στῶ, um den Punkt, wo wir den Hebel ansetzen sollen, so wählen wir frischweg die scheinbar besterhaltene Sprache des Stammes, zumal wenn sie leidlich erforscht ist, und messen an ihr die übrigen. Wir wissen, diese können stellenweise dem Urtypus näher kommen, aber dabei halten wir uns für’s Erste nicht auf. Wir kommen mit den kleinen Irrthümern rascher vom Flecke; wir setzen die Ludolphische Zahl = 3 und ein Bischen, und überlassen es Anderen, dies Bischen bis in die Decillionstel zu berechnen; sie werden die Kreise genauer quadriren, dafür quadriren wir derweile mehr Kreise als sie. Wir haben vorhin, S.~157, gesehen, wie uns das Tibetische bei der Vergleichung der indochinesischen Sprachen einen Generalnenner lieferte. Der Fall war vereinzelt; er war aber so hervorragend, dass wir uns bis auf Weiteres diese Sprache zur Führerin nehmen dürften; sie würde uns wahrscheinlich am Weitesten führen.}

\pdfbookmark[2]{I. §. 3. Die vereinzelte Sprache.}{III.II.3}
\cohead{I. Allgemeines. §. 3. Die vereinzelte Sprache.}
\subsection*{§. 3.}\phantomsection\label{III.II.3}
\subsection*{Die vereinzelte Sprache.}
Die volkswirthschaftliche Theorie hat sich gelegentlich mit der Vorstellung eines isolirten Staates beschäftigt, der dem internationalen Verkehre entrückt, übrigens aber mit den Errungenschaften unserer materiellen Gesittung ausgerüstet wäre. Wie würden sich da die wirthschaftlichen Dinge ordnen und entwickeln? \sed{{\textbar}{\textbar}177{\textbar}{\textbar}}\phantomsection\label{sp.177} Man operirte so zu sagen am Phantome, noch dazu an einem Phantome, dessen Urbild nirgends anzutreffen \update{ist;}{ist:} denn alle isolirten Völker stehen auf niederer \update{wirthschaft\-licher}{wirtschaft\-licher} Stufe. Gleichwohl durfte, musste vielleicht die Wissenschaft ihre Zuflucht zu einem solchen Phantasiegebilde nehmen, um daran einen Theil ihrer Gesetze in unvermischter Reinheit darzustellen.

\fed{{\textbar}186{\textbar}}\phantomsection\label{fp.186}

Die Sprachwissenschaft ist hierin günstiger gestellt, denn sie weiss aus den Sprachen der Wilden nicht weniger Lehren zu schöpfen, als aus den Cultursprachen. Freilich Völker, die des internationalen Verkehres auf die Dauer \update{entbehren}{entbehren,} oder sich dessen erwehren, giebt es sehr wenige. Vielleicht gehören dazu in \update{Afrika}{Africa} jene zwischen den Bantustämmen eingesprengten Pygmäen, die Akka, Tiki-tiki, in Amerika die ἒσχατοι ἀνδρῶν, die Feuerländer, dann etwa dieser oder jener Jägerstamm der brasilianischen \update{Urwälder,}{Urwälder;} ferner, man weiss nicht seit wie langer Zeit, die Osterinsulaner und diese oder jene unter den Hyperboräern und den Nigritiern der südöstlichen Inselwelt. Diesen mögen sich dann jene kriegerischen Melanesier anreihen, deren einziger mündlicher Verkehr mit den Nachbarvölkern in gegenseitigem Auffressen besteht. Doch auf eine vollständige Aufzählung der Beispiele kommt wahrlich nicht viel an. Denn immer bleiben die Fragen: Seit wann die Vereinsamung? mit wem früher der Verkehr? Dazu kommt ein Zweites. Man müsste solche vereinzelte Sprachen mehrere Geschlechter hindurch sehr genau in ihren Wandelungen verfolgen. Das ist aber bei keiner geschehen, und wenn es unternommen würde, so würde voraussichtlich die fortgesetzte Anwesenheit weniger Europäer unter einer kleinen Horde Wilder genügen, um den Bann der Vereinsamung zu durchbrechen. Denn werden die Weissen in Güte geduldet, so werden sie auch wie höhere Wesen verehrt, und dann gilt natürlich ihr Einfluss Alles. Und erzwingen sich die Weissen die Herrschaft, so ist es natürlich erst recht um die Isolirtheit von Volk und Sprache geschehen.

Allein auch ohnehin würden solche seltene Untersuchungsobjecte nicht die Bevorzugung verdienen, deren man sie \fed{etwa} von vornherein würdigen könnte. Denn in weitaus den meisten Fällen haben an der stätigen Entwickelung der Sprache die einheimischen Mächte unendlich mehr Einfluss, als die von auswärts kommenden Anregungen. Wo diese am Kräftigsten wirken, da bewirken sie eben nicht ruhige Entwickelung, sondern Umsturz oder Vernichtung, und dann werden ihre Spuren nicht so leicht verwischt werden. Dies gilt von den grelleren Erscheinungen der Sprachenmischung und von den eigentlichen Mischsprachen. Wo aber, wie in den grossen Culturvölkern, Jahrhunderte lang die Dialekte einander über den Rain pflügen, da hat eben die sprachgeschichtliche Forschung Methoden zu finden, um das Einheimische vom Eingeschleppten zu sondern. Sie darf dabei so zu sagen parlamentarisch \fed{{\textbar}187{\textbar}}\phantomsection\label{fp.187} verfahren, nach Stimmenmehrheit; denn die Majorität wird in der Regel national sein.

\sed{{\textbar}{\textbar}178{\textbar}{\textbar}}\phantomsection\label{sp.178}

Jene allein auf inneren Kräften und Anlagen und auf den heimischen Lebensbedingungen beruhende Entwickelung wollen wir die \so{freie} nennen; und von ihr ist mit Nothwendigkeit vorauszusetzen, dass sie völlig einheitlich sei. Wir können nicht anders, wir müssen wieder die Sache auf die Spitze treiben. Lautlehre, Formenwesen, Syntax, Wortschatz einer Sprache oder Sprachenfamilie, Alles muss denselben Quellen entflossen sein. So ahnen wir zwischen dem scheinbar Fremdartigsten Zusammenhänge, die wir uns heute nicht vorstellen, die wir vielleicht nimmermehr nachweisen können. Einzelnes der Art leuchtet allerdings bald ein. Starker Verschliff des Lautwesens z.~B. wird zu Verlust oder Unkenntlichmachung der Formen, zur Überhandnahme der Homonymen und somit zur Beschaffung neuer Verdeutlichungsmittel führen. Mit solchen allgemeinen Sätzen ist aber die ideale Aufgabe noch lange nicht gelöst. Es fragt sich: Welchen Weg hat im einzelnen Falle der Lautwandel eingeschlagen, und welche Wege das Formenwesen und die übrigen Bestandtheile der Sprache? und warum das alles? Vergleichen wir z.~B. die germanischen Sprachen mit ihren Schwesterfamilien, so finden wir, dass in jenen das Laut-, Casus- und Tempuswesen u.~s.~w. die und die Sonderschicksale gehabt hat. Man sagt, in diesen Besonderheiten beruhe der Charakter der germanischen Sprachen. Sollte man nicht mit gleichem Rechte sagen dürfen: in ihnen äussere sich dieser Charakter, sie also beruhen auf ihm? Und nun gälte es, die Einheit des Charakters in seinen verschiedenen Äusserungen nachzuweisen.

Stecken wir zunächst das Ziel nicht zu hoch. Lassen wir die historischen Sprachvergleicher noch eine geraume Weile weiterarbeiten und die anderen grossen Sprachfamilien ähnlich sorgfältig untersuchen, wie die unsere. Dann darf man hoffen, auf Grund eines reichen Inductionsmateriales Erfahrungssätze zu gewinnen, die besagen, dass im Leben der Sprachen die und die Tendenzen einander parallel gehen: „je mehr so, desto mehr oder weniger so“. Ist dies geglückt, dann erst mag man fragen: Woher dieses regelmässige Zusammentreffen?

Hier, freilich in sehr, sehr weiter Ferne, glaube ich den wichtigsten Gewinn zu erkennen, den die allgemeine Sprachwissenschaft von der genealogisch-histori\-schen erhoffen mag. Es wäre zu beklagen, wenn so glänzender Scharfsinn und so rastloser Fleiss in alle Zukunft nichts \fed{{\textbar}188{\textbar}}\phantomsection\label{fp.188} weiter zu Tage förderte, als die Erkenntniss, dass aus den und den Lauten hier die und dort jene geworden, dass in der einen Sprache diese, in der anderen Sprache jene Formen verloren gegangen und durch Neubildungen ersetzt seien, und, wenn es hoch kommt, dass es sich mit der Etymologie der Wörter und der Bildungselemente so oder so verhalte.

Soweit, dass wir sagen könnten: In der Sprachgeschichte ist dies nothwendig, jenes unmöglich, – soweit sind wir noch lange nicht. Aber die Erfahrung lehrt schon jetzt, dass erstaunlich Vieles möglich ist, und dem sorgsamen Beobachter gelingt es oft, die Gründe dieser Möglichkeiten zu entdecken. Somit \sed{{\textbar}{\textbar}179{\textbar}{\textbar}}\phantomsection\label{sp.179} lässt sich mindestens zum Theile feststellen, worauf der Erforscher der Sprachgeschichte gefasst sein muss. Und dabei erfordert allerdings die Methode, dass wir von dem einfacheren Falle, von der vereinzelten Sprache ausgehen, ehe wir die Wechselwirkungen der Sprachen und Mundarten aufeinander betrachten.

\pdfbookmark[2]{I. §. 4. Die Etymologie.}{III.II.4}
\cohead{I. Allgemeines. §. 4. Die Etymologie.}
\subsection*{§. 4.}\phantomsection\label{III.II.4}
\subsection*{Die Etymologie.}
Die Frage nach dem Ursprunge der Wörter und grammatischen Formen lässt zwei Gesichtspunkte zu, natürlich wieder die der Erscheinung und des Zweckes. Alle Etymologie ist ihrer Methode nach analytisch, sie setzt eine Scheidekunst voraus, die die letzten, nicht weiter lösbaren Bestandtheile der Wörter aufweisen soll. Man kann nun von diesen Elementen (Wurzeln) ausgehen und fragen: wie und in welchen Bedeutungen werden sie miteinander verbunden? Dies ist die Aufgabe der etymologischen Wörterbücher. Oder man kann die auszudrückende Vorstellung zum Ausgangspunkte nehmen und fragen: mit welchen Mitteln hat sie die Sprache zum Ausdrucke gebracht? Dies sollte in den synonymischen Wörterbüchern mit bedacht werden; denn Bedeutung und Anwendung eines Wortes müssen im letzten Grunde auf seiner Etymologie beruhen.

Im weitesten Sinne verstanden, fängt jene Scheidekunst schon da an, wo sie noch überhaupt gar keine Kunst ist. Sage ich: Wohnhaus ist ein attributives Compositum aus dem Verbalstamme wohn– und dem Substantivum Haus: so ist das eigentlich auch schon ein etymologischer Ausspruch. Ebenso, wenn ich sage: das Wort „wohntest“ besteht aus dem Verbalstamme wohn–, dem Imperfectsuffix –te und dem \fed{{\textbar}189{\textbar}}\phantomsection\label{fp.189} Suffix der 2. Pers. Sing. –st. In diesem Verstande fassen die Engländer die ganze Formenlehre unter den Begriff \retro{Etymology.}{Etymologie.}

Aber der forschende Geist strebt weiter, am Kühnsten bekanntlich in der Jugend der Wissenschaft. Da möchte er am Liebsten gleich die höchsten Höhen erklettern, die letzten Tiefen ergründen; denn von den Gefahren, die ihm unterwegs drohen, ahnt er noch nichts. Platon stellte die prinzipielle Frage: Wie kommen die Dinge zu ihren Namen, durch ihre eigene Natur, oder durch menschliche Satzung? Eine inductive Lösung war nur auf etymologischem Wege zu finden. Die Frage nach der Herkunft der Wörter hat wohl zu allen Zeiten denkende Köpfe gereizt. Und so sind uns denn auch aus dem Alterthume genug etymologische Versuche überliefert worden, – die meisten sind auch darnach. Das zügellose Treiben hat fortgedauert bis zur Begründung der heutigen Sprachwissenschaft; es wird noch heute von freien Künstlern geübt und wahrscheinlich nie ganz aufhören. Die Männer der Wissenschaft aber werden immer bescheidener und nüchterner. \textsc{Franz Bopp }durfte noch hoffen, \sed{{\textbar}{\textbar}180{\textbar}{\textbar}}\phantomsection\label{sp.180} den Ursprung unserer grammatischen Formen zu ergründen; auch \textsc{Schleicher} hat diese Hoffnung nicht ganz aufgegeben. Der geniale \textsc{Pott} glaubte einen Theil der indogermanischen Wurzeln aus Zusammensetzungen erklären zu können, – auch mit den semitischen dreiconsonantigen Wurzeln sind ähnliche Versuche angestellt worden. Jetzt verhält sich die Indogermanistik zaghafter. Was hinter der Verzweigung des Urstammes in den einzelnen Familien zurückliegt, nimmt sie in der Regel als gegeben hin und wagt sich nur da ein Stück weiter, wo die Ursprache selbst einen klareren Blick in ihre Vorgeschichte zu gestatten scheint. Dass diese Fälle nicht zu häufig vorkommen, dafür sorgt die immer strenger werdende Methode und die eifersüchtige Aufsicht, die die Mitforscher üben.

\largerpage[1]Manche schütten wohl das Kind mit dem Bade aus, wollen überhaupt von Etymologie nichts hören, verweisen sie in das Gebiet speculativer Spielerei und machen damit unserer Wissenschaft eine ihrer anziehendsten Aufgaben streitig. Soweit wir die Sprachgeschichte an der Hand beglaubigter Thatsachen verfolgen können, beruhen alle äusserlichen Mittel der Wort- und Formenbildung auf Agglutination, das heisst auf der Anfügung ursprünglich selbständiger Wörter. Dies \update{verall\-gemeinern,}{verall\-gemeinern} heisst nur: von der Gegenwart auf das Vergangene und von der bekannten jüngeren Vergangenheit auf die unbekannte ältere zurück\fed{{\textbar}190{\textbar}}\phantomsection\label{fp.190}schliessen. Das und weiter nichts thut jene Theorie, die man die Ag\-glutinationstheorie genannt hat. Zu weit würde sie nur dann gehen, wenn sie die Möglichkeit innerer Formungen neben jenen äusseren oder der Neuschöpfungen durch falsche Analyse verneinte. Man müsste die Analogie aus dem Organon der inductiven Wissenschaften streichen, man müsste annehmen, dass die Gesetze des Werdens ebenso wandelbar sind, wie die Erscheinungen, wenn man der \so{Agglutinationstheorie} den Vorwurf der Unbedachtsamkeit machen wollte. \sed{Der Vorwurf beschränkter Unduldsamkeit würde sie aber treffen, wollte sie Alleinherrschaft beanspruchen und jene anderen Kräfte, die an der Wortformung Theil haben können, kurzweg verneinen.} – Über die von \textsc{Alfred Ludwig} entgegengestellte s.~g. \so{Adaptationstheorie} möge man \textsc{Delbrück}, Einleitung in das Sprachstudium, 2. Aufl., S.~66 flg. vergleichen.

Nicht die Etymologie dürfen wir verbieten, auch nicht jene tiefstgehende, die die Schleier der Urzeit lüften möchte: – das hiesse die Welt oder unsere Köpfe mit Brettern vernageln. Oder welche Frage ist interessanter: wie ein Wort oder eine Wortform vor so und sovielen tausend Jahren geklungen hat, – oder jene andere, wie der menschliche Geist dazu gekommen ist, seine Vorstellungen in die und die Lautformen zu kleiden?

Nicht das ist unsere Aufgabe, dem Streben des Forschers verbietende Schranken zu setzen, sondern ihm Wege zu weisen, – die Wege, auf denen die Sprachen selbst gewandelt sind, – und ihm die Abwege zu zeigen, die ihn irreleiten können, die Störungen, von denen die Sprachen auf dem Wege ihrer \sed{{\textbar}{\textbar}181{\textbar}{\textbar}}\phantomsection\label{sp.181} Entwickelung bedroht werden. Denn die Methode ist hier wie überall durch die Natur des Gegenstandes bedingt. Sie verlangt es aber, dass wir den Ausnahmen, – jenen Störungen, – gleiche Aufmerksamkeit, darum viel mehr Raum widmen, als den mechanisch regelmässigen Vorgängen; denn diese sind verhältnissmässig einfach, jene sehr \update{mannich\-faltig}{mannig\-faltig} und vielleicht bis in’s Unendliche combinationsfähig. \sed{Wir werden finden: es ist mit den Einflüssen, die die Sprachen verändern, wie mit jenen Krankheitskeimen, von denen die jetzige Arzneikunde redet. Sie umgeben uns immer, mit jedem Athemzuge saugen wir sie ein. Aber die wenigsten fassen Wurzel. Und wenn wir im einzelnen Falle die eingetretene Veränderung erklären wollen, so genügt es nicht, die Art des Keimes festzustellen, sondern wir müssen auch fragen: Warum hat gerade dieser Keim, warum hat er gerade auf diesem Boden wuchern können? Sonst machen wir es uns zu leicht, haben für alle Fälle Erklärungen, für jeden Nothfall einen Unterschlupf bereit, wollen Alles beweisen und beweisen nichts.}

\begin{styleAnmerk}
Anmerkung. Der Unterschied zwischen den \update{so genannten}{sogenannten} Junggrammatikern und ihren Gegnern beruht wohl zum grössten Theile im Verhalten jenen Störungen gegenüber. Keiner von beiden Theilen leugnet sie, aber wann und wie ihnen Rechnung zu tragen sei, ist streitig. Es sind das Interna; dem Draussenstehenden kommt es nicht zu, Partei zu ergreifen.
\end{styleAnmerk}

\pdfbookmark[2]{II. §. 1. Deutlichkeit und Bequemlichkeit.}{III.II.II.1}
\cohead{II. §. 1. Deutlichkeit und Bequemlichkeit.}
\subsection*{Zweites Hauptstück.}
\section*{Die sprachgeschichtlichen Mächte.}
\subsection*{§. 1.}\phantomsection\label{III.II.II.1}
\subsection*{Deutlichkeit und Bequemlichkeit.}
Wir halten uns an die vereinzelte Sprache in dem Sinne, den wir vorhin festgestellt haben; wir denken an alle die Veränderungen, die sie erleiden \update{kann,}{kann} aber auch an alles das, was sich in ihr unverändert erhalten mag, es sei im Lautwesen, im Sprachbaue oder im Wortschatze, es betreffe die äussere Erscheinung oder den geistigen Inhalt. Wir denken an Lautschwund, Euphonik, Analogie, \update{System\-zwang}{System\-zwang,} und wie sonst die Begriffe heissen, mit denen die sprachgeschichtliche Forschung hantiert, und fragen: Woher dies Alles? auf welche letzten Ursachen und Kräfte lässt es sich zurückführen?

Regelmässig dient die Sprache dem Verkehre, das heisst zweien Parteien, zwischen denen sie vermitteln soll, dem Ich und dem Du. Darum ist sie von beiden Parteien abhängig: ich muss so reden, dass Du es verstehst, sonst verfehlt meine Rede ihren Zweck. Mit anderen Worten: Deine Sprache muss auch \sed{{\textbar}{\textbar}182{\textbar}{\textbar}}\phantomsection\label{sp.182} die meine sein, ich muss annähernd so reden, wie Du zu reden und reden zu hören gewohnt bist. Diese Gewöhnung beruht auf Überlieferung, an diese Überlieferung sind wir beide gebunden.

Das Gewöhnliche ist immer auch bequem; je öfter wir etwas geübt haben, desto weniger empfinden wir die Kraftanstrengung. Aber vorhanden ist diese Anstrengung immer. Daher ist auch immer das Bestreben vorhanden, sie zu verringern, Kräfte zu sparen, es uns noch bequemer zu machen. Es ist leicht, dies in der Art, wie wir unsere Muttersprache handhaben, zu beobachten. Wir halten es damit, wie mit der \update{Kleidung}{Kleidung,} und erlauben uns nur gar zu gerne ein Négligé, wenn wir uns einbilden, dass es uns von Anderen erlaubt werde. Jetzt articuliren wir mangelhaft, nuscheln, murmeln, muffeln mit träger Mundbewegung. Jetzt wieder werfen wir ein paar abgerissene Worte hin statt eines rechtschaffenen Satzes: – in beiden Fällen ist es eine körperliche Kraftersparniss, die wir uns gönnen. Andere Male fallen wir unbedenklich aus der Construction oder ergehen uns in gedankenlos breitem Geplausche \fed{{\textbar}192{\textbar}}\phantomsection\label{fp.192} und huldigen somit einer geistigen Trägheit. Sehen wir von jenen Ausnahmefällen ab, wo ein spielerischer oder künstlerischer Sinn sich in der Formung der Rede gefällt, so gilt für die Sprache der wirthschaftliche Grundsatz, dass der Zweck mit möglichst geringem Aufwande erreicht werden soll.

Es ist nun wohl einzusehen, in welchen Richtungen die beiden Factoren wirken: der Zweck der Verständlichkeit und die Neigung zur Kraftersparniss.

Beide wirken zunächst erhaltend; denn das Überkommene pflegt gewohnt, darum zugleich für den Hörer verständlich und für den Redner verhältnissmässig bequem zu sein. Aber, wie gesagt, es ist darum noch nicht nothwendigerweise das Bequemste.

Immer neigt das Bequemlichkeitsbedürfniss, die Trägheit dahin, sich auch das Unerlässliche zu erlassen. Und das heisst soviel, wie durch Vernachlässigung zerstören; denn in der Sprache bleibt nur das erhalten, was gebraucht wird. Nun kann freilich der häufige Gebrauch zur Abnutzung führen. Alltägliche Redensarten werden undeutlich ausgesprochen, zur Hälfte verschluckt. Formwörter und Wortformen werden schwach betont, in ihrem Lautbestande geschädigt, endlich wohl ganz unterdrückt. Was früher verschieden klang, klingt jetzt einerlei, und so wird die Sprache in ihrem Äusseren ärmlicher. Das wäre an sich eher ein Gewinn, als ein Fehler. Denn die Sprache ist ein Mittel, und unter den verschiedenen Mitteln, die zum Zwecke führen, ist in der Regel das einfachste das beste.

Es kann aber eine Zeit kommen, wo das vereinfachte Mittel nicht mehr seinem Zwecke genügt, – sei es, dass der Zweck höher gesteckt worden, sei es, dass die Vereinfachung in Verwüstung ausgeartet ist. Da greift dann der Verständigungszweck wie ein mahnender Gläubiger ein und fordert Ersatz für \sed{{\textbar}{\textbar}183{\textbar}{\textbar}}\phantomsection\label{sp.183} das, was ihm verdorben worden ist. Wo der Unterjochte die Sprache des Überwinders annimmt, da mag er sie erst nach Herzenslust verstümmeln; für die bescheidenen Bedürfnisse des Verkehres zwischen Zwingherren und Hörigen genügt eine armselige Sprache von der Art der Creolenjargons. So war dem Schoosse der lateinischen Sprache als wüster Bastard die Lingua rustica entsprossen, die nachmals einen grossen Theil Europas erobern sollte. Sie war dazu besser geeignet, als ihre formenreichere Mutter; denn sie war einfacher und darum bequemer. Eine Zeit wahrer Dürftigkeit muss sie aber schon \fed{{\textbar}193{\textbar}}\phantomsection\label{fp.193} früher durchlebt haben: das Casuswesen war verkümmert, das Futurum, das Passivum und manches der schätzbarsten Hülfswörter \sed{war} verloren gegangen, ehe es durch Neues ersetzt wurde. Und nun sollte die Bauernsprache in Italien, Gallien und auf der iberischen Halbinsel auf dem Boden einer neuen Gesittung, geweckt durch höhere \update{Aufgaben}{Aufgaben,} zu Gebilden erblühen, die an Pracht und Kraft mit der römischen Ahnin wetteifern dürfen. Denn im Punkte der Ausdrucksfähigkeit und Anschaulichkeit nehmen es die neuromanischen Sprachen mit den besten auf. Diese Vorzüge aber verdanken sie offenbar dem Deutlichkeitsbedürfnisse, das in den Trümmern der verfallenen alten Sprache ein trefflich gefügiges Baumaterial vorfinden mochte. Diesmal war der Fall acut, und die Sprachentwickelung vollzog sich nicht in der Vereinzelung, sondern unter dem Einflusse fremder Mächte. Das Beispiel zeigte aber, wie jenes Bedürfniss nicht nur nachbessernd, sondern sogar umschaffend, verjüngend wirken kann.

Eine andere seiner Wirkungen ist scheinbar mehr vorübergehend und doch nachhaltig genug. Wo es uns auf Deutlichkeit ankommt, da sprechen wir mit besonderer Anstrengung der Sprachorgane. Ist dann die Articulation normal, im Gegensatze zu der flüchtigeren, so weckt sie im Redenden wie im Hörenden die Erinnerung an das ursprüngliche Lautbild. Es kann aber auch geschehen, dass sie übertreibt oder aus Unkenntniss des \update{ächten}{echten} Lautbildes fehlgeht. Solche Übertreibungen sind es \update{z.~B.}{z.~B.,} wenn Vocale zu Diphthongen zerdehnt oder Consonanten über Gebühr verschärft werden, \textit{i} zu \textit{ai}, \textit{u} zu \textit{au}, \textit{d} zu \textit{t} u.~s.~w., wie in der germanischen Lautentwickelung.

Ähnliches ist in allen Theilen der Sprache denkbar und nachweisbar. Das Bedürfniss nach Deutlichkeit und Anschaulichkeit, nach Eindringlichkeit der Rede thut sich nicht leicht genug. Einfache grammatische Beziehungen werden umschrieben, um recht scharf hervorzutreten, Casus werden durch Präpositionen, diese durch sinnverwandte Substantiva ersetzt, und was dessen mehr ist. Flüche, Schwüre, rhetorische Fragen treten an die Stelle einfacher Versicherungen; gedankenlos werden überschwängliche Prädicate gebraucht, und alles das kann Mode und somit Rechtens werden.

\begin{sloppypar}Das Gleiche kann aber auch mit jenen Flüchtigkeiten geschehen, die sich die Bequemlichkeit gestattet, mit undeutlichen Articulationen, Ellipsen und syn- \sed{{\textbar}{\textbar}184{\textbar}{\textbar}}\phantomsection\label{sp.184} taktischen Nachlässigkeiten aller Art. Wir müssen \fed{{\textbar}194{\textbar}}\phantomsection\label{fp.194} immer daran denken, dass jede Neuerung in einer Sprache von Hause aus ein Fehler war. Ist aber ein Fehler geringfügig genug, um übersehen oder geduldet zu werden, so hat er schon den Process halb gewonnen. Und wird er von einer tonangebenden Persönlichkeit oder Classe begangen, so hat das zehnte Mal die Menge nicht den Muth, ihn zu missbilligen. Die eigenthümlich schnarrend näselnde Sprache, die man früher nur in preussischen Officierskreisen hörte, kommt immer mehr in Übung. Für das militärische Commando ist sie zweckmässig und bequem, weil weithin vernehmbar, und so erklärt es sich wohl, dass sie auf dem Exercierplatze in Aufnahme gekommen und von da zu einer Art Standessprache geworden ist. Die Officiere entstammen aber den höheren und höchsten Ständen und zählen überall zur ersten Gesellschaft. Innerhalb dieser bilden sie eine durch ihre Geschlossenheit mächtige Körperschaft, der anzugehören den Ehrgeiz reizt. Wer es kann, wird mindestens \update{Reserve\-officier}{Reserve\-offizier} und kehrt dann auch im Civilleben die militärische Seite heraus. Dahin gehört auch der militärische Ton der Rede, den jetzt schon Leute nachäffen, die nie einen Degen getragen haben. Interessant ist es, dass man Ähnliches auch in \update{Oesterreich}{Österreich} beobachten kann.\end{sloppypar}

Hier sind wir nun eigentlich in einen neuen Gedankenkreis eingetreten; denn jene Ziererei beruht weder auf der Bequemlichkeit noch auf dem Streben nach Deutlichkeit, sondern auf dem eitelen Verlangen, für etwas zu gelten, was man nicht ist. Das aber heisst im vorliegenden Falle soviel als: die Gewohnheit eines fremden Standes annehmen, um zu thun, als gehörte man diesem an. Und eben diese Gewohnheit beruhte ursprünglich auf Bequemlichkeit und Deutlichkeit.

Nirgends versteht man es besser, Bequemlichkeit mit Schnelligkeit und Sicherheit zu verbinden, als in den grossen Mittelpunkten des geschäftlichen Verkehres. Die Menge der Aufgaben und Bedürfnisse drängt zu rascher und möglichst müheloser Erledigung, und Einer lernt vom Anderen. So auch in der Sprache, die hier wie die landläufige Münze nicht nach der Schönheit und Reinheit des Gepräges, sondern nach ihrem Gehalte und ihrer Geführlichkeit geschätzt wird. Da schleift sich das Lautwesen ab, und die Phraseologie entwickelt sich zu ausdrucksvoller Knappheit. Wer die Sprachen nur vom archäologischen Gesichtspunkte aus betrachtet, dem mag dabei das Herz bluten: denn was ihm als Unart und Verderb gilt, das strömt von diesen grossen Centren aus in \fed{{\textbar}195{\textbar}}\phantomsection\label{fp.195} immer weitere Kreise. Auch ist es wahr, viel Gutes und Schönes kann dabei verderben, zumal da, wo der Verkehr allzu vorwiegend kalt geschäftlich ist. Die Sprache wird dann wohl glatt, aber auch hart.

\begin{sloppypar}Nicht immer jedoch ist das Deutlichkeitsbedürfniss seinem Grunde und Zwecke nach geschäftlich: es kann auch gemüthlich und ästhetisch sein, und dann redet man wohl lieber von ausdrucksvoller, anschaulicher, eindringlicher \sed{{\textbar}{\textbar}185{\textbar}{\textbar}}\phantomsection\label{sp.185} Sprache, als von deutlicher. Und doch ist es im Grunde immer die Deutlichkeit, auf die es dabei ankommt. Es fragt sich nur: Was soll angedeutet werden, was wird bedeutet? Auch jene Formen und Wendungen in der Rede dienen der Deutlichkeit, in denen der Redende seine Subjectivität äussert oder auf die Stimmung des Hörers einwirken will, jene Partikeln und Phrasen, die der Rede das Gepräge breiter Gemüthlichkeit, bedächtiger Überlegung oder heftiger Erregung verleihen, die Äusserungen der Bescheidenheit und Höflichkeit, Umschreibungen aller Art, Euphemismen und ihr Gegentheil, die der Sache besondere Merkmale abgewinnen, – man denke an die vielerlei Ausdrücke für sterben, betrunken sein u.~s.~w. Deutlich in diesem Sinne ist das Persönliche und Zarte nicht minder, als das Sachliche und Derbe.\end{sloppypar}

Würzende Zuthaten, wie die eben erwähnten, machen natürlich die Rede umständlich und insofern lästig. Da geschieht es dann wohl, dass das Bequemlichkeitsbedürfniss Abhülfe schafft durch allerlei Kürzungen, die der sprachgeschichtlichen Untersuchung zu rathen geben.

\pdfbookmark[2]{II. §. 2. Der Lautwandel.}{III.II.II.2}
\cohead{II. §. 2. Der Lautwandel.}
\subsection*{§. 2.}\phantomsection\label{III.II.II.2}
\subsection*{Der Lautwandel.}
Mit Recht misst die sprachgeschichtliche Forschung den Erscheinungen der organisch regelmässigen Lautveränderung geradezu fundamentalen Werth bei. Ihre Methode verlangt es, zuerst diese Erscheinungen festzustellen, also nur das zu vergleichen, dessen Verschiedenheiten ausschliesslich in der verschiedenen lautlichen Entwickelung derselben Urform beruhen. Es könnte scheinen, als bewegte sie sich damit im Kreise, machte sich einer Petitio principii schuldig. Es sollen gewisse Regeln ermittelt werden. Es wird zugegeben, dass diese Regeln nicht alle Erscheinungen der sprachgeschichtlichen Veränderungen erklären, dass also nicht alle Erscheinungen zur Induction gleich brauchbar sind: wie sind nun die brauchbaren von den unbrauchbaren zu unterscheiden, \fed{{\textbar}196{\textbar}}\phantomsection\label{fp.196} ehe man die Regel kennt? Die Geschichte der Indogermanistik ist hier wie immer lehrreich, auch in ihren Um- und Irrwegen.

Zunächst verglich man unter Zugrundelegung des Sanskrit Alles, was sich in der Bedeutung glich und in den Lauten ähnelte: \textit{pit\textsubring{r}}, πατήρ, \textit{pater}, Vater; \textit{mât\textsubring{r}}, \corr{1891 und 1901}{μήτηρ,}{μητήρ,} \textit{mater}, Mutter; \textit{pra}, πρὸ, \textit{pro}, für; \textit{asti}, ἐστι, \textit{est}, ist, – aber auch \textit{h\textsubring{r}d}, καρδία, \textit{cor}(\textit{d}), Herz; \textit{yak\textsubring{r}t}, ἧπαρ, \textit{jecur}, Leber; \textit{jihvā}, \textit{lingua}, Zunge \update{u.~s.~w.}{u.~s.~w.,} \sed{an deren Zusammengehörigkeit man später, da man es mit den Lauten genauer nahm, zweifeln lernte.}

Nun inducirte man: die Laute, die in solchen Fällen an der gleichen Stelle stehen, haben sich aus den nämlichen Urlauten entwickelt. So gelangte \textsc{Schleicher} dazu, ursprachliches \textit{k} in den Sanskritlauten \textit{k}, \textit{č}, \textit{ç}, seltener \textit{p}, noch seltener \textit{h} \sed{{\textbar}{\textbar}186{\textbar}{\textbar}}\phantomsection\label{sp.186} wiederzufinden. Es sind Regeln mit Ausnahmen, in denen allerdings manchmal neue Regeln erkannt, aber nicht schlechthin verlangt werden.

Dieser Zustand schien auf die Dauer unerträglich. Was sich mechanisch entwickelt, muss sich regelmässig, das heisst gleichmässig entwickeln. Was sich verschieden entwickelt hat, dessen Verschiedenheiten wollen gleichfalls erklärt sein; und der Erklärungsmöglichkeiten fanden sich mehrere.

\phantomsection\label{III.II.II.2a}a. Entweder die Ähnlichkeit war trügerisch, die zu Grunde liegenden Urwörter selbst waren verschieden, das griechische θεὸς z.~B. war anderen Ursprunges als das sanskrit \textit{deva}, das sanskrit \textit{h\textsubring{r}d} hatte von Hause aus mit καρδία, \textit{cor}, Herz nichts zu thun.

\phantomsection\label{III.II.II.2b}b. Oder es lagen schon in der Ursprache verschiedene Laute zu Grunde. So entdeckte man in dieser neben dem kurzen \textit{ă} noch kurze \textit{ĕ} und \textit{ŏ}, von denen die europäischen Sprachen zeugten, ferner tönende \textit{\textsubring{r}}, \textit{\textsubring{l}}, \textit{\textsubring{m}}, \textit{\textsubring{n}} oder nach Anderen einen dumpfen Vocal, der diesen Liquidis voranging: \textit{\textsubring{e}r}, \textit{\textsubring{e}l}, \textit{\textsubring{e}m}, \textit{\textsubring{e}n}, und eine zweifache Gutturalreihe, aus denen sich scheinbare Unregelmässigkeiten erklärten.

\phantomsection\label{III.II.II.2c}c. Oder drittens: derselbe Laut der Ursprache musste sich unter verschiedenen Bedingungen, die es zu ermitteln galt, verschieden entwickeln. So hat das bekannte \textsc{Verner}’sche Gesetz mit einem Schlage über eine scheinbare Wüstenei Licht und Ordnung verbreitet.

\phantomsection\label{III.II.II.2d}d. Oder viertens: Die Wörter oder Wortformen waren durch einen sehr natürlichen seelischen Vorgang aus ihrem ursprünglichen Verwandtschaftskreise in einen anderen hinübergezogen. Das ist der Fall der so genannten falschen Analogien, deren es wieder verschiedene Unterarten giebt. Das lateinische \textit{lingua} verdankt seinen Anlaut dem sinnverwandten \fed{{\textbar}197{\textbar}}\phantomsection\label{fp.197} \textit{lingere}, lecken. Die Form „frug“ statt „fragte“ beruht auf der Analogie von „trug, schlug“; und im Sanskrit und in einem Theile der slavischen Sprachen haben ein paar der gebräuchlichsten Verben die Conjugation auf \textit{–mi} zur \retro{allein\-herschenden}{allein\-herrschenden} gemacht.

\phantomsection\label{III.II.II.2e}e. Endlich, fünftens, konnten ja auch Entlehnungen von Sprache zu Sprache stattgefunden haben; und dann fragt es sich: auf welcher Stufe der Lautentwickelung in der entlehnenden Sprache? wieviel Zeit hatte der Fremdling, sich dem einheimischen Lautwesen anzupassen? Man denke an jene Wörter, wo lateinischem \textit{c} ein deutsches \textit{k} entspricht: \textit{carcer} – Kerker, \textit{cicer} – Kicher-Erbse, \textit{cerasus} – Kirsche; ferner an die, wo lateinisches \textit{p} noch als \textit{p} erhalten oder schon in \textit{pf} verwandelt ist: \textit{palatium} – Palast – Pfalz.

\phantomsection\label{III.II.II.2axiom}Unter Vorbehalt aller dieser Möglichkeiten hat man nun den Satz aufgestellt: Die Lautgesetze sind unverbrüchlich; wo sie durchbrochen scheinen, da ist entweder die Störung nur scheinbar (a), oder sie beruht auf noch unerkannten Lautgesetzen (b, c), oder sie ist durch fremde, nicht lautmechanische Mächte verursacht (d, e).

Wäre dieser Satz so schroff, wie er dasteht, richtig, so gäbe es keine \sed{{\textbar}{\textbar}187{\textbar}{\textbar}}\phantomsection\label{sp.187} schwankenden Articulationen, und dann wäre überhaupt nicht zu begreifen, wie sich die Laute verschieben konnten. Es sei an dem: alle Mundartsgenossen haben genau die gleiche Aussprache: wie kann sich dann diese ändern? Denn Einer muss doch mit der Änderung anfangen, ehe diese um sich greifen und zur Herrschaft gelangen kann. Schon der Satz ist bedenklich, dass der Einzelne unter gleichen Umständen dasselbe immer gleich aussprechen werde. Nehmen wir die Sache genau, auf die Gefahr hin, sylbenstecherisch zu scheinen.

Bin ich heute derselbe, der ich vor vierzig Jahren war? derselbe, der ich vor zwanzig, vor zwei Jahren, vor drei Wochen, gestern war? Derselbe wohl, aber nicht mehr dasselbe. Und auch ob meine Aussprache noch dieselbe ist, bleibt zweifelhaft. Ich brauche nicht erst an den Fall zu denken, dass ich ein paar Tage lang mit Leuten von einer anderen Mundart verkehrt und nun, wie es Vielen geht, unbewusst \update{etwas von deren Mundart}{von deren Aussprache etwas} angenommen habe. Jede pathologische Änderung in meinen Sprachorganen, ein Katarrh, das Abbrechen eines Stückchens Zahn, ja selbst eine allgemeine körperliche Ermüdung oder gemüthliche Erregung, ändert etwas an der Art meiner Lauterzeugung. Und ferner: was heisst gleiche Umstände? Wenn ich zu verschiedenen Personen, \fed{{\textbar}198{\textbar}}\phantomsection\label{fp.198} wenn ich zu derselben Person aus verschiedenen Entfernungen, wenn ich das Gleiche jetzt aus freiem Antriebe, jetzt als Antwort auf eine Frage oder im Wortwechsel als Gegenbehauptung sage, so sind die Umstände verschieden. Endlich: Was heisst gleiche Aussprache? Gehört dazu bloss die Lautbildung oder auch der Ton und das Tempo? gehört zur Lautbildung nur die akustische Wirkung, oder auch die Art, wie diese durch Bewegungen der Sprach- und Athmungsorgane hervorgebracht wird? Doch wohl dies Alles zusammen. Wo Sprachen laterale Laute bilden, da wird das Gehör nicht unterscheiden können, ob die Zunge sich nach links oder nach rechts bewegt hat. Wenn wir ein Pferd zum Galopp antreiben wollen, so schnalzen wir lateral, und das klingt genau wie der Lateralschnalzlaut der Hottentotten und Buschmänner. Diese aber erzeugen den Laut an den rechten Backenzähnen, wir in der Regel an den linken. Und wenn wir das in jenen Sprachen thun wollten, so würden die Eingeborenen uns tadeln. Als man einen gelehrten Araber fragte, ob sein laterales \textit{s} nicht auch auf der linken Mundseite gebildet werden dürfte, antwortete er witzig: „Il n’y a pas d’exemple d’une telle gaucherie!“ Also nicht bloss die Ohren, sondern auch die Augen urtheilen über die Richtigkeit der Lauterzeugung. Erwägt man alles dies, so darf man fragen: Wann treffen die Voraussetzungen des Satzes von der gleichmässigen Aus\-sprache ein? trifft der Satz überhaupt zu? Und dann macht ja nicht der Einzelne die Sprache, sondern die Gemeinschaft, und schon mein Bruder oder Dorfnachbar spricht nicht ganz genau so wie ich.

\largerpage[-1]Wir mussten allen diesen Bedenken Rechnung tragen, um sie zu überwinden. Ich habe schon früher, S.~33, darauf hingewiesen, dass das Sprachge\sed{{\textbar}{\textbar}188{\textbar}{\textbar}}\phantomsection\label{sp.188}fühl der Völker die Laute anders, weiter fasst, als die Lautphysiologie; es gestattet einen gewissen, grösseren oder geringeren Spielraum in der Lauterzeugung und in der Schallwirkung; erst wenn dieser überschritten wird, erhebt es Einspruch. Jeder Einzelne macht es naturgemäss annähernd so, wie er es von seinen Nächsten machen hört und sieht; seiner Laune und seiner Nachlässigkeit sind Schranken gesetzt: der Vogel flattert am Faden. Nun kann es wohl geschehen, dass allmählich innerhalb dieser Schranken gewisse Richtungen bevorzugt werden. Es handle sich z.~B. um den Laut \textit{a}. Der darf sich ein kleines Stückchen in der Richtung nach \corr{1891 und 1901}{\textit{ä}}{ä [\textit{nicht kursiv}]} und ein ebenso kleines Stückchen nach \textit{å} oder \textit{ö} hin bewegen; das ideale \textit{a} bildet den Mittelpunkt eines Kreises, der seine erlaubten Ausspracheweisen umschreibt. Innerhalb dieses \fed{{\textbar}199{\textbar}}\phantomsection\label{fp.199} Kreises entscheidet sich der Gebrauch für die Richtung nach \textit{å} hin. Damit verschiebt sich zunächst der Schwerpunkt; denn die anderen Richtungen werden seltener ausgemessen. Dann aber wird sich auch der Mittelpunkt verschieben, der Kreis sich in der Richtung nach \textit{å} erweitern, in den Richtungen auf \textit{ä} und \textit{ö} verkleinern. Ob er dabei weiter oder enger wird, ist eine Sache für sich. Die Hauptsache aber, die Lautverschiebung, dürfte sich somit ziemlich einfach erklären, und nun wird man gestehen, dass der Name gar nicht treffender gewählt sein konnte.

Was entscheidet nun über Enge oder Weite jener Kreise? Zunächst natürlich die grössere oder geringere Empfindlichkeit des Lautgefühles: dem Einen mag etwas noch richtig scheinen, was dem Anderen schon fehlerhaft klingt. Hierin werden nun auch ganze Nationen verschieden beanlagt oder durch das Schicksal verschieden erzogen sein. Von Sprachen, die die Tenues von den Mediis nicht unterscheiden und es erlauben, im gleichen Falle beliebig \textit{k} oder \retro{\textit{g},}{\textit{g} [\textit{am Zeilen\-ende}]} \textit{t} oder \textit{d} zu sprechen, lesen wir so oft, dass wir doch kaum ein für allemale an der Lautauffassung der Schriftsteller zweifeln dürfen. Wo lebhafter Verkehr zwischen mundartlich verschiedenen Menschen stattfindet, da weiss man wohl die Mundarten voneinander zu unterscheiden, aber man ist gegen das Fremde duldsam, weil es kaum mehr ein Fremdes ist, macht es wohl unwillkürlich einmal selbst mit. In Deutschland herrschen, wenn ich recht beobachtet habe, die bilabiale und die labiodentale Ausspracheweise des \textit{f} und \textit{w} zonenweise, letztere wohl im ganzen Norden, erstere mehr auf alt-oberdeutschem Gebiete. Den wenigsten Deutschen fällt der Unterschied \update{auf;}{auf,} wenn man aber im Holländischen und Schwedischen ein labiodentales \textit{w} spricht, so wird man getadelt. Dafür mögen dort wieder andere Freiheiten gestattet sein.

Es ist allemal bedenklich, statistische Behauptungen aufzustellen, von Immer oder Nie, von Meistens oder Selten zu reden, so lange man nicht den statistischen Beweis führen kann. Und wie in unserem Falle die Dinge liegen, mag man nunmehr zweifeln, welcher Satz der richtigere sei: der, dass dasselbe Individuum unter den nämlichen Umständen dasselbe immer auf den Punkt gleich\sed{{\textbar}{\textbar}189{\textbar}{\textbar}}\phantomsection\label{sp.189}mässig ausspreche, – oder jener, dass nur selten und durch Zufall Dasselbe mehrere Male ganz genau gleich articulirt werde, weil die Individuen, die Umstände und die Aussprache unzähligen winzigsten Wechseln unterliegen. Gesetzt aber, Letzteres träfe zu, so wäre damit noch keineswegs ausgesprochen, dass die \fed{{\textbar}200{\textbar}}\phantomsection\label{fp.200} Lautentwickelung ganz regellos geschehen könne. Und umgekehrt: gesetzt, es liessen sich überall so scharf begrenzte und unverbrüchliche Lautgesetze \retro{nachweisen,}{nach weisen, [nach \textit{am Zeilen\-ende}]} wie sie auf indogermanischem Sprachgebiete theils schon entdeckt sind, theils noch gesucht werden: so wäre damit noch nicht die Behauptung widerlegt, dass überall die Articulation einen gewissen, wäre es auch einen noch so engen Spielraum gestattet.

Solange aber nicht für die Lautgeschichte der anderen Sprachfamilien Ähnliches geleistet ist, wie für die indogermanische zum Theile erst noch geleistet werden soll: solange mag ich den Satz von der Ausnahmslosigkeit nicht als Dogma, geschweige denn als bewiesenen Lehrsatz gelten lassen, sondern nur als ein methodologisches Prinzip, das besagt: „Denke, es wäre so; richte Deine Forschungen darnach ein; beruhige Dich nicht, ehe Du das Lautgesetz oder den Grund, warum es im einzelnen Falle durchbrochen scheint, entdeckt hast: dann gehst Du so sicher, wie es nach Lage der Sache möglich ist.“ \sed{Das ist vor der Hand eine heilsame Fiction, im günstigsten Falle ein ansprechender Heischesatz. Will man Ernst mit ihm machen, ihn auf die Probe stellen, so fange man da an, wo die Quellen der Sprachgeschichte am Reichsten und verhältnissmässig Reinsten fliessen, etwa bei den romanischen Sprachen. Vorläufig thut man besser, von einer oft bewährten Vorschrift, als von einem unumstösslichen, bewiesenen oder beweisbaren Lehrsatze zu reden. Und würde er für die indogermanischen Sprachen bewiesen, so wäre er es darum noch lange nicht für die übrigen.}

In diesem Sinne, der wohl auch von vielen unserer fortgeschrittensten Indogermanisten gebilligt wird, darf der Grundsatz der neueren Indogermanistik auch anderwärts gelten. Gegen Überstürzungen wird er sich als Hemmschuh erweisen, der auf abschüssigen Bahnen vor Unfall behütet, auf ebenem Wege aber das Vorwärtskommen erschwert, auf ansteigendem es vereitelt. \sed{Dass \so{Vater}, \so{Mutter}, \so{Bruder} mit lateinisch \textit{pater}, \textit{mater}, \textit{frater} gleichen Stammes sind, konnte man auch vor der Entdeckung des \textsc{Verner}’schen Gesetzes verständigerweise nicht leugnen, mochte man auch den augenscheinlichen Unregelmässigkeiten in der Lautvertretung rathlos gegenüberstehen. Wer aber vor lauter Drang nach exacter Gewissheit vergisst, nach dem Wahrscheinlichen zu fragen, von dem gilt das Wort des Faust: „Daran erkenn’ ich den gelehrten Herrn ...“ der verkümmert sich jenen Sinn für das bunte Leben, der keinem Historiker fehlen darf.} Des Unerklärlichen wird immer genug bleiben, immer wird es vorkommen, dass uns die Lautgesetze einmal im Stiche lassen, ohne dass wir er\sed{{\textbar}{\textbar}190{\textbar}{\textbar}}\phantomsection\label{sp.190}klären könnten, was sie durchbrochen hat. Italienisch und spanisch \textit{gato}, \update{Katze,}{Katze} ist offenbar vom lateinischen \textit{catus} nicht zu trennen; kein Lautgesetz aber erklärt den Wechsel von \textit{k} in \textit{g}, und schwerlich wird man nachweisen können, warum er gerade hier stattgefunden habe.\footnote{\sed{Vergl. \textsc{Meyer-Lübke}, Gramm. der roman. Spr. I. 353.}} Im Germanischen trat \textit{θ} (\textit{þ}) an Stelle des indogermanischen \textit{t}; im Hoch- und Niederdeutschen wurde daraus \textit{d}; im Englischen aber spaltete sich der Fricativlaut in einen harten und einen weichen, und im Dänischen und Schwedischen entspricht ersterem \textit{t}, letzterem \textit{d}: Du denkst – \textit{thou thinkest} – \textit{du tänker}. Woher das?

Lautverschiebungen greifen nur allmählich um sich, nicht nur in örtlicher, sondern auch in sachlicher Hinsicht. Das Beispiel der hochdeutschen Mundarten ist hierfür lehrreich. Die Niederrheinländer und ihre Stammverwandten im siebenbürgischen Sachsen- und Burzenlande sagen noch: \textit{et}, \textit{dat}, \textit{wat}, statt: es, das, was, haben aber im Übrigen das hochdeutsche \textit{s} (\textit{ʃs}) statt des auslautenden \textit{t} angenommen. \sed{Das Wort „\so{bloss}“ weist noch das Allemannische in der älteren und niederdeutschen Form \so{blutt} auf (Vergl. \textsc{Kluge}, Etymol. Wb. der D. Spr. v. bloss). „\so{Zwerch}“ = quere, und „\so{Zwehle}“ = Handtuch, sind erst im Neuhochdeutschen aus mittelhochdeutschen \textit{twerch}, \textit{dwerch}, \textit{twehele}, \textit{dwehele} entstanden. Der Wandel von \textit{t} in \textit{z}, die althochdeutsche Lautverschiebung, hat sich also hier nach Jahrhunderten, ein zweites Mal abgespielt.} Im Japa\fed{{\textbar}201{\textbar}}\phantomsection\label{fp.201}nischen lässt sich beobachten, wie seit einem Jahrtausende die Neigung, unbetonte mit \textit{m} oder \textit{n} anlautende Sylben in \textit{u} zu verwandeln, immer neue Opfer fordert. \sed{\textsc{H.~Oldenberg}’s Warnung, mit solchen Annahmen vorsichtig zu sein (Zeitschr. f. deutsche Philol. XXV, 116), ist allerdings beherzigenswerth. Jenes doppelte Schicksal des germanischen \textit{th} und das \textit{et}, \textit{dat}, \textit{wat} im Niederrheinischen scheinen doch eine mehr oder minder befriedigende lautgesetzliche Erklärung gefunden zu haben. Dass auch die anderen, von mir angeführten Ausnahmeerscheinungen auf lautmechanischen Gründen beruhen mögen, kann ich nicht in Abrede stellen, so schwer ich es mir vorstellen kann. Doch das ist auch hier von geringerem Belange, als die grundsätzliche Frage: was überhaupt im Sprachleben möglich, was nothwendig oder unmöglich sei.} Und so \update{ist wohl nirgends die Möglichkeit schlechthin}{ganz schlechthin ist wohl nirgends die Möglichkeit} zu verneinen, dass Lautverschiebungen an gewissen Stellen in’s Stocken gerathen, anderwärts weiter gedrungen seien, dass sie wohl auch nach langen Pausen wie atavistische Anlagen von Neuem zum Durchbruche kommen. \sed{Dass Lautgesetze nur gewisse Theile der Sprache ergreifen, andere unberührt lassen können, dafür bietet das Baskische gute Beispiele in Fülle. Dass sie nicht immer mit einem Schlage, sondern auch allmählich zur Herrschaft gelangen können, beweist z.~B. lateinisch \textit{quintus} für \textit{quinctus}, während \textit{junctus}, \textit{punctum} u.~s.~w. erst im Italienischen das \textit{c} eingebüsst haben: \textit{giunto}, \textit{punto}.} Das \sed{{\textbar}{\textbar}191{\textbar}{\textbar}}\phantomsection\label{sp.191} Italienische hat in gewissen Fällen auslautendes \textit{s} in \textit{i} verwandelt: \textit{noi}, \textit{voi}, \textit{poi}, \textit{sei} = \textit{nos}, \textit{vos}, \textit{post}, \textit{sex}. Gesetzt, eine ähnliche Neigung wäre schon in indogermanischer Urzeit vorhanden gewesen, so würden sich sanskrit \textit{tē}, nom. pl. masc. des pron. demonstr. \textit{sa}, \textit{sā}, \textit{tad}, und die lateinischen und griechischen Formen \textit{equi}, ἳπποι neben sanskrit \textit{açvās}, vielleicht auch – mit Hülfe des Reflexivpronomens – die Medialendungen des Sanskrit und Griechischen, sowie der sanskr. Imperativ \textit{ēdhi} = sei, erklären. Doch das ist und bleibt voraussichtlich im günstigsten Falle eine Hypothese; – genug schon, wenn sie nicht allzukühn erscheint.

\sed{Kein Laut trägt in sich die Tendenz, sich nach einer bestimmten Richtung hin zu verschieben; jedem stehen, innerhalb des den Sprachorganen Möglichen, alle denkbaren Richtungen offen. Erinnern wir uns an das früher Gesagte. Jede, auch die geringste Veränderung in der Haltung oder Bewegung eines Theiles der Sprachorgane erzeugt ein neues Lautindividuum, und zwischen je zwei verschiedenen Lauten liegen unzählige Mittelstufen. Darum kann man kühnlich behaupten: \so{jeder Laut kann im Laufe der Zeit, auf längerem oder kürzerem Wege, in jeden anderen Laut übergehen}. Es ist nicht nöthig, vielleicht nicht einmal möglich, dies bis in’s Einzelnste zu belegen. Nur einige allgemeinere Beobachtungen mögen hier Platz finden.}

\sed{\so{Je schwächer eine Sylbe oder ein Wort betont ist, desto mehr sind ihre Laute der Verflüchtigung und dem Schwunde ausgesetzt}. Dies zeigt sich z.~B. an den Sub- und Praefixen unserer Sprachen. Dagegen kann \so{eine kräftige Betonung zu übertreibender Lautbildung führen}: Vocale werden gedehnt, zu Diphthongen gedehnt, Consonanten verdoppelt, verhärtet, aspirirt u.~s.~w. Hier stehen wir auf der Schwelle der sogenannten Euphonik. Und fragen wir weiter: Wodurch wird die stärkere oder schwächere Betonung verursacht? so werden wir gar in’s psychologische Gebiet hinübergewiesen.}

\sed{Als Regel, wenn auch kaum als ausnahmslose Regel darf man annehmen, \so{dass in der Lautverschiebung einer Sprache eine gewisse Folgerichtigkeit herrsche}, dass also verwandte Laute auch verwandte Schicksale erfahren. Handelte es sich um die Schrift, so würde ich sagen: nicht sowohl die einzelnen Buchstaben, als vielmehr der gesammte Ductus verändere sich, werde spitziger oder runder, markiger oder dünner u.~s.~w. Was man hier Ductus nennt, das sind in der Rede alle jene Gewohnheiten, welche die Laut- und Tonbildung beherrschen: Lage und Bewegungsweise der Sprachorgane, Haltung der Lippen, der Zunge, des Gaumensegels, des Kehlkopfes, heftigere oder gelindere Thätigkeit der Athmungsorgane und was dessen mehr ist. Solche Gewohnheiten können sich im Wechsel der Geschlechter unvermerkt ändern, und diese Änderungen sind entscheidend für die Richtung der Lautverschiebung. Wir und zumal die Ungebildeten, die in solchen Dingen sehr empfindlich sind, }{\textbar}{\textbar}192{\textbar}{\textbar}\phantomsection\label{sp.192}\sed{ haben einen reichen Vorrath darauf abzielender Ausdrücke. Da heisst es, die Sprache klinge hart oder weich, träge oder hastig, dehnig oder polternd, schnarrend, schnärchelnd, gurgelnd, quäkend, näselnd, lispelnd, zischend, muffelnd, – und jedes dieser Wörter enthält ein vielsagendes Signalement, ein Gesammtbild, dem so und soviele Einzelzüge entsprechen werden. Es mag nicht immer leicht sein, die Thatsachen einer Lautverschiebung auf einen solchen Generalnenner zurückzuführen; versucht musste es aber allemal werden. Wenn das Germanische die Tenues in Aspiraten, dann in Fricative, die nicht aspirirten Mediae in Tenues, die aspirirten in nicht aspirirte verwandelt hat: so muss dies unter der Herrschaft einer einheitlichen Tendenz geschehen sein.}

\sed{So folgerecht haben sich nun allerdings die Dinge nicht überall gestaltet. Viele deutsche Dialekte haben das \textit{g} im Munde vorwärts gerückt, zu \textit{j} palatalisirt, das \textit{k} aber auf seiner alten Stelle gelassen. Von den Tenues des Lateinischen hat im Altfranzösischen und in der Florentiner Mundart nur das \textit{k} fricativen Wandel (in~χ) erlitten, während \textit{t} und \textit{p} unverändert geblieben sind. Im Madegassischen hat sich anlautendes \textit{k} in \textit{h}, \textit{p} in \textit{f} verwandelt, \textit{t} dagegen (bis auf gewisse Ausnahmen) sich behauptet. Auch das gälte es zu erklären.}

\sed{\so{Seltene Laute und Lautverbindungen haben die Tendenz, ganz zu verschwinden} und durch geläufigere ersetzt zu werden; denn das Ungewöhnliche wird unbequem. So setzt \textsc{Joh. Schmidt} (Pluralbildungen der indogerm. Neutra, S.~198–199), um deutsch Leber, armenisch \textit{leard}, altpreussisch \textit{lagno} mit sanskrit \textit{yak\textsubring{r}t}, lat. \textit{jecur}, griech. ἧπαρ, litauisch \textit{jeknos} in Einklang zu bringen, ein ursprüngliches \textit{ljḗk\textsubscript{e}rt} an und vermuthet auch für griechisch εἴβω, λείβω denselben seltenen Anlaut \textit{lj}.}

\sed{Doch noch weiter in die Tiefe sollte die Wissenschaft zu dringen streben. Warum hat der Lautwandel gerade hier, in dieser einen unter den vielen verschwisterten Mundarten oder Sprachen diesen Weg eingeschlagen, warum gerade dort jenen, und dort wieder jenen dritten? Denken wir nur an die indogermanischen Sprachen, an die Schicksale ihrer Vocale, ihrer Gutturale, Palatale, Aspiraten und \textit{s}-Laute: woher die verschiedenen Tendenzen bei den Ariern, den Griechen, den Italern, den Slaven? Klima, Lebensgewohnheiten, Berührung mit Nachbarstämmen, vielleicht Temperament und das eine oder andere Mal ein gewisses ästhetisches Gefühl (Mode), alles das wird mit bestimmend gewesen sein, – aber in welchem Masse und in welcher Richtung? Es scheint, nur eine ganz riesenhafte inductive Arbeit könne hier zum Ziele führen, jetzt mikroskopisch, die nächsten, kleinsten Mundarten untersuchend, jetzt wieder in der Weite, bei allen Sprachstämmen in allen Erdtheilen Umfrage haltend. Für eine Wissenschaft, die die Gesetze des Geschehens ergründen will, sind jene Generalnenner, sie seien noch so sauber ausgearbeitet, doch nur Vorstufen; vereinfachte Beschreibungen sind es, aber keine Erklärungen.}

\sed{{\textbar}{\textbar}193{\textbar}{\textbar}}\phantomsection\label{sp.193}

\pdfbookmark[2]{II. §. 2. Der Lautwandel. Zusatz.}{III.II.II.2.Z}
\cohead{II. §. 2. Der Lautwandel. Zusatz.}
\subsection*{Zusatz.}\phantomsection\label{III.II.II.2zusatz}
\subsection*{\so{Beispiele zur Lehre von der Articulation und der Lautverschiebung.}}
1. \so{Samoanisch} (\textsc{G. Pratt}, \foreignlanguage{english}{A Grammar and Dictionary of the \retro{Samoan}{Somoan} language, 2\textsuperscript{d} ed. p. 1): „The Samoan alphabet proper consists of only fourteen letters: – \textit{a}, \textit{e}, \textit{i}, \textit{o}, \textit{u}; \textit{f}, \textit{g} (= \textit{ṅ}, \textit{ng}), \textit{l}, \textit{m}, \textit{n}, \textit{p}, \textit{s}, \textit{t}, \update{\textit{v}}{\textit{v},} .... \textit{K} is found only in \textit{puke}, catch you! and its compound: \textit{pukētā}!}“ Hierzu bemerkt \textsc{Whitmee}: „\foreignlanguage{english}{To a person now for the first time visiting Samoa this would appear to be incorrect. He would hear \textit{k} used by most of the natives in their ordinary conversation in place of \textit{t}. When I went to Samoa in 1863, I heard \textit{k} used only on the island of Tutuila, and on the eastern portion of Upolu. Now it is used all over the group. It is difficult to say how this change commenced, but its spread has been noted, and every attempt has been made to arrest it, but without effect. Many of the people now seem unconscious of the difference. The more intelligent (even although they may fall into the careless habit of using \textit{k} in conversation) use \textit{t} quite correctly in reading and in public speaking. But the practice of transposing \textit{k} and \textit{t} in reading is rapidly \fed{{\textbar}202{\textbar}}\phantomsection\label{fp.202} growing: e.~g., in introduced words where \textit{k} occurs, many read \textit{t}. In the same way \textit{n} and \textit{g} (\textit{ng}) are transposed.}“

(Daselbst S.~2): „\foreignlanguage{english}{Many natives are exceedingly careless and incorrect in the \corr{1891 und 1901}{pronun\-ciation}{pronoun\-ciation} of consonants, and even exchange or transpose them without confusion, and almost unnoticed by their hearers; as \textit{mānu} for \textit{nāmu}, a scent; \textit{lagoga} for \textit{lagona}, to understand; \textit{lava‛au} for \textit{vala‛au}, to call; \so{but they are very particular about the} \corr{1891 und 1901}{\so{pronun\-ciation}}{pronoun\-ciation} \so{of the vowels}.}“

2. \so{Batta}. (\textsc{H. N. van der Tuuk}, kurzer Abriss einer Batta’schen Formenlehre im Toba-Dialekte S.~6 flg.) §. 25. „Durch ihre trillernde Bewegung verursachen \textit{r} und \textit{l} Tonversetzungen. I. Die Vocale zweier Sylben werden verwechselt: \textit{lote} = \textit{leto}, \textit{biruran} = \textit{buriran}. II. Die Anlaute von aufeinanderfolgenden Sylben wechseln ihre Stelle: \textit{laba} = \textit{bala}, \textit{derém} = \textit{redém}, \textit{talgáng} = \textit{tanggál}. III. Der Triller ist bald Anlaut, bald Auslaut: ... \textit{alpís} = \textit{lapís}. §. 27. I. \sed{(Solche Metathesen werden auch im Baskischen beobachtet. \textsc{W. J. van Eys}, Grammaire comparée des dialectes basques, p. 21.).} Die vorletzte Sylbe, wenn sie betont ist, wird oft durch \textit{s}, \textit{r}, oder \textit{l} geschlossen, mit Verdrängung eines anderen Schlussconsonanten: \textit{hurtut} neben \textit{huttut}, \textit{listun} neben \textit{littun}, \textit{orgos} neben \textit{ogos}. §. 28. I. In der vorletzten Sylbe, wenn sie unbetont ist, und also die letzte den Ton hat, können alle Vocale stehen: \textit{gumír}, \textit{gamír} und \textit{gomír} nebeneinander, ebenso \textit{depé}, \textit{dopé}, \textit{dapé}, oder \textit{biltáng}, \textit{boltáng}.“ – Der Verfasser führt noch eine Menge anderer Lautvertauschungen an, z.~B. \textit{hurbit} = \textit{surbit}, \textit{nop} \update{–}{=} \textit{nok}, \textit{teptep} = \textit{tektek}, \textit{lusup} = \textit{lusut}. Ähnliche Erscheinungen finden sich auch in anderen malaischen Sprachen, ohne dass immer einer Dialektmischung auf die Spur zu kommen wäre. Jedenfalls zeugen sie von stumpfem Articulationsgefühle. Vergl. noch \textsc{A. Hardeland}, Versuch einer Grammatik der \sed{{\textbar}{\textbar}194{\textbar}{\textbar}}\phantomsection\label{sp.194} \so{Dajack}schen Sprache S.~57 fl. \textsc{J. J. de Hollander}, Handleiding tot de Beoefening der \so{Maleische} Taal- en Letterkunde, 3de druk, §. 10.

3. \so{Australische Sprachen}. (\textsc{C. G. Teichelmann} u. \textsc{C. W. Schürmann}, \foreignlanguage{english}{Outlines of a Grammar, Vocabulary, and Phraseology of \retro{the}{de} Aborig. Language of South Australia, spoken by the Natives in and for some Distance around Adelaide. pg. 3:) „... a few letters which are frequently changed or omitted, even amongst one and the same tribe: ... \textit{B} is confounded with \textit{p}; \textit{d} with \textit{t}; and \textit{g} with \textit{k}}“ u.~s.~w. – Ähnliches sagt \textsc{W. Ridley} (\foreignlanguage{english}{Kámilarói, and other Australian Languages}, 2\textsuperscript{d} ed. pg. 5) von anderen australischen Sprachen.

4. \so{Amerikanische Sprachen}. Auch hier wird der willkürliche \fed{{\textbar}203{\textbar}}\phantomsection\label{fp.203} Wechsel zwischen Tenuis und Media vieler Orten bezeugt. So vom \so{Pima} (\foreignlanguage{english}{Grammar of the Pima or Névome, a Language of Sonora ... ed. by B. Smith}, pg. 9). \sed{\textsc{A. S.~Gatschet}, gewiss ein zuverlässiger Beobachter, versichert (Phonetics of the Kayowē Language: American Antiquarian IV, 283–284): \foreignlanguage{english}{Speaking of the languages that came to my notice, I can state that an Indian pronounces almost every word of his tongue in six, ten or twelve different ways ... A few examples taken from Káyowē ... will illustrate this curious feature better than any grammatic rules can; it appears from them, that interchange exists, for no apparent cause, between the \so{gutturals} \textit{k}, \textit{g}, \textit{gg}, χ, \textit{ḵ} and the spirant \textit{h}; between the \so{dentals} \textit{t}, \textit{d}, \textit{nd}, \textit{md} ...; between the labials \textit{p}, \textit{b}, \textit{f}, \textit{mb}. Among the \so{vowels} alternation is observed between \textit{a}, \textit{ä}, \textit{o}, \textit{u}, and their long sounds; between \textit{e}, \textit{i}, \textit{ä}, and their long sounds; also between the nasalized and the \corr{1901}{non-nasalized,}{non – nasalized,} and between the \corr{1901}{clear}{dear} and dumb-sounding vowels}. – Ähnliches bezeugt er von der \so{Klamath}-Sprache (The Klamath Indians of Southwestern Oregon. Contrib. to N. Amer. Ethnol. Vol. II, \textsc{i}, p. 223–227).} Im \so{Hidatsa} wechselt \textit{d} mit \textit{n}, \textit{r} und \textit{l} (\textsc{W. Mathews}, \foreignlanguage{english}{Ethnography and Philology of the Hidatsa Indians}, pg. 89). \sed{Im \so{Zapotekischen} findet gleichfalls Wechsel der Tenues und Mediae und mancher Vocale statt (\textsc{J. de Cordova}, Arte del idioma Zapoteco. Morelia 1886, p. 73. Autor anónimo, Gram. de la l. Zap., Mexico 1887, p. 15–16). Im \so{Anti} oder \so{Campa} (Arte de la l. de los Indios A. o. C., Paris 1890, p. 15). Im \so{Bakaïrí} (\textsc{K. von den Steinen}, die Bakaïrí-Sprache, Leipzig 1892, S.~254).} Vom \so{Chilenischen} sagt \textsc{Havestadt} (Chilidugu I, pg. 8): „Sumit sibi lingua Chilensis licentiam usurpandi unam literam pro alia; idque ... 3tiò: quia aucupantur verborum concinnitatem, orationis cultum, famamque eloquentiae: vel etiam ad cuiusvis arbitrium ac libitum. Hinc sunt synonyma: \textit{huif}, \textit{huiv}, \textit{huib}, ordo; \textit{chollob}, \textit{chollof}, \textit{chollov}, tegula; ... \textit{ruca}, \textit{duca}, \textit{suca}, domus; \textit{huera}, \textit{hueda}, \textit{huesa}, malus, a, um; \textit{carù}, \textit{cadù}, \textit{casù}, viridis, crudus ...“ Dem Verfasser darf man zutrauen, dass, wenn dies örtlich mundartliche Verschiedenheiten wären, er dies hervorgehoben haben würde.

Folgenden Einwand könnte man erheben: Die \sed{meisten} Gewährsmänner waren \sed{{\textbar}{\textbar}195{\textbar}{\textbar}}\phantomsection\label{sp.195} nicht zu wissenschaftlicher Lautbeobachtung geschult; sie beurtheilten die fremden Laute nach denen ihrer Muttersprache, und Zwischenstufen zwischen diesen schienen ihnen bald nach der einen, bald nach der anderen Seite zu neigen. Darauf ist zu entgegnen, dass mindestens ein Theil jener Männer lange genug unter den Eingeborenen gelebt, um ihr Ohr an die fremde Sprache so zu gewöhnen, wie es vordem an die Muttersprache gewöhnt gewesen. Dieser, oder richtiger ihrer mehrsprachigen Schulung, verdankten sie eben das feinere Gehör, das sie jene unsicheren, schwankenden Articulationen empfinden liess.

\sed{Übrigens brauchen wir vielleicht gar nicht in der Ferne zu suchen; auch die indogermanischen Lautgesetze weisen noch ein langes Schuldconto auf, wenn auch jährlich einzelne Posten davon getilgt werden. Wenn es wahr ist, dass unsere germanischen Sprachen zuweilen An- und Auslaut der Wortstämme vertauschen, so leisten sie selber in unsicherer Articulation das Möglichste. Anscheinende hierher gehörige Doubletten sind: \so{Pott} – \so{Topf}, \so{Ziege} – \so{Gais} (\textit{haedus}); \so{Zicke} – \so{Kitze} = junge Ziege; \so{kitzeln} = englisch \textit{to tickle}; vielleicht \so{Pfote} – \so{Dope}, \so{Tape}; \so{Kahn} – \so{Nachen}, \so{Hübel} – \so{Bühel}. Räthselhafte Lautverschiebungen zeigen z.~B. finster, althochdeutsch \textit{finstar}, neben \textit{dinstar}, mhd. \textit{dinster}, die auf einen alten Anlaut \textit{th} deuten. Auch sonst erscheint im Germanischen zuweilen \textit{f} neben \textit{th}, z.~B. \so{fliehen}, got. \textit{thliuhan}; \so{flehen}, got. \textit{ga-thlaihan}, Feimen neben \textit{Dimmen}, \textit{Diemen}. Unerklärt ist wohl auch \so{sollen}, neben \so{Schuld}, gotisch \textit{skulan}. Althochdeutsch \textit{pfrimma}, neben \textit{brimma} = Ginster; \so{flach} neben \so{Blachfeld}; \so{pflücken}, neben schweizerischem blueken, deuten auf einen Wechsel von \textit{p} und \textit{b}. In \so{heikel} neben \so{ekel} und \so{heischen} neben \corr{1901}{ahd.}{\textit{ahd.}} \textit{eiskôn} scheint unorganisches anlautendes \textit{h} vorzuliegen. Italienisch \textit{nespola} ist wohl nicht von lat. \textit{mespilum}, \corr{1901}{pl.}{\textit{pl.}} \textit{mespila} zu trennen. Warum aber ist dort das \textit{m} durch \textit{n} ersetzt? \so{Keck} und \so{queck} (erquicken, engl. \textit{quick}), \so{kommen}, \so{kam}, \so{bequem}, engl. \corr{1901}{\textit{to come},}{\textit{tocome},} \textit{came}, holländisch \textit{komen}, \textit{kwam}, ah. \textit{chuman}, \textit{queman}, got. \textit{qiman}, \textit{kirr} neben altnord. \textit{kvirr}, got. \textit{qairrus}, umgekehrt: \so{Quelle}, neben altnord. \textit{kelda} dürften gleichfalls vereinzelt dastehen und kaum auf ein Gesetz zurückzuführen sein. Sonst bleibt \textit{qu} entweder unverändert, oder wechselt mit \textit{tw}, \textit{zw}. In anderen Fällen ist es freilich um so bewunderungswürdiger, wie folgerichtig bei uns der Lautwandel vor sich gegangen ist; und so darf man vielleicht auch hier nicht daran verzweifeln, dass sich dereinst das unerklärlich Scheinende doch noch aufhellen werde.}

\sed{Unregelmässiger Wandel zwischen Tenuis und Media findet sich auch im \so{Lateinischen}. Anscheinende Lehnwörter wie \textit{buxus}: πύξος, gubernare: κυβενᾶν wollen vielleicht weniger besagen: Die griechischen Tenues mögen weniger hart gewesen sein, als die lateinischen. Kaum erklärbar aber dürften Erscheinungen sein, wie \textit{viginti} neben \textit{vicesimus}, \textit{septingenti} neben \textit{ducenti} und \textit{gloria}, das man vom sanskrit \textit{çravasyam} nicht wohl trennen kann.}

\largerpage[-1]{\textbar}{\textbar}196{\textbar}{\textbar}\phantomsection\label{sp.196}

\sed{Ferner zählt man im Lateinischen etwa ein Dutzend Fälle, wo \textit{l} der allgemeinen Regel entgegen \textit{d} vertritt: \textit{lacrima} – δάκρυ, \textit{olere} neben \textit{odor} u.~s.~w. – Auch die aspirirten Tenues in manchen Sanskritwörtern, z.~B. in der Wurzel \textit{sthā} = stehen, scheinen der lautgesetzlichen Erklärung noch zu spotten. Und woher das \textit{h} aspiré im französisch \textit{haut} = \textit{altus}, \textit{huit} = \textit{octo}? Dem Sanskrit giebt man in ein paar Fällen schwer \corr{1901}{erklärlichem}{erklärlichen} Abfall anlautender Dentale Schuld: \textit{açru}, Thräne, ist von griechisch δάκρυ, gotisch \textit{tagr} u.~s.~w. nicht wohl zu trennen. Aber auch \textit{ahan}, \textit{ahas}, \textit{ahar}, Tag, soll mit gotisch \textit{dags} verwandt sein.}

Mechanische Mächte, die nur an ganz vereinzelten Punkten zum Durchbruch kommen, sind vielleicht nie voll zu erweisen, aber auch nicht wegzustreiten. Lehrreich ist es jedenfalls zu sehen, mit welch ausdauerndem Fleisse und geschultem Scharfsinne die Indogermanisten diesem dornenvollsten Theile ihrer Aufgabe zu Leibe gehen, und wie glänzend oft ihre Mühe gelohnt wird. Ihr Glaube, dass in unserer Sprachfamilie und, füge ich hinzu, noch in mancher anderen, der Lautwandel durchweg in völlig gesetzmässiger Weise vor sich gegangen sei, scheint sich von Jahr zu Jahr mehr bewahrheiten zu wollen.

\begin{styleAnmerk}
\sed{\so{Anmerkung}. Der Güte \textsc{H. Oldenberg}’s verdanke ich folgende Literaturnachweise: Sollen – \textit{skulan}, \textsc{v. Firlinger}, Kuhn’s Ztschr. XXVII, 190 flg. \textit{C} und \textit{g} im Lateinischen, \textsc{Thurneysen}, ebenda XXVI, 309 flg. \textit{Açru} – δάκρυ, \textsc{Brugmann}, Grundriss der vergl. Gramm. der indogerm. Sprachen II. 303. An Erklärungsversuchen fehlt es nicht, aber auch nicht an Streit über sie; und auch dieser ist lehrreich für die Draussenstehenden. Adhuc sub iudice lis est; solange es noch auf indogermanischem Gebiete Streit über die Gegeninstanzen giebt, ist auch dort der Prozess nicht entschieden, viel weniger ein Präjudiz für die ganze übrige Sprachenwelt gewonnen.}
\end{styleAnmerk}

\pdfbookmark[2]{II. §. 3 a. Die Euphonik (Sandhi).}{III.II.II.3a}
\cohead{II. §. 3 a. Die Euphonik (Sandhi).}
\subsection*{§ 3. \sed{a.}}\phantomsection\label{III.II.II.3a}
\subsection*{Die Euphonik (Sandhi).}
\sed{Die lautgeschichtliche Forschung muss nach möglichster Exactheit streben; das Ziel, unverbrüchliche Lautgesetze zu gewinnen, hat sie solange zu verfolgen, bis sie sich überzeugt, dass in der gegebenen Sprache solche Gesetze allein die Schicksale des Lautwesens nicht bestimmt haben können. Sie wird also zunächst annehmen, dass jeder Laut der Ursprache in jeder Tochtersprache immer die gleiche Wandelung erlitten habe, dass also, wo die Tochtersprachen unterscheiden, auch die Ursprache unterschieden haben müsse. In diesem Sinne mag sie vorerst ganz unvorgreiflich schematische Zeichen einführen: \textit{a}\textsuperscript{1} \textit{a}\textsuperscript{2} \textit{a}\textsuperscript{3} u.~s.~w. Der Zukunft bleibt es dann überlassen, den Sinn dieser Ziffern zu bestimmen. Vielleicht war die Articulation der Ursprache unsicher, – dann muss sich eben die Forschung insolvent erklären. Oder es lag doch von Hause aus ein wirklicher Unterschied vor; und da sind nun zwei Fälle möglich. Der Unterschied mochte in den Lauten an sich beruhen, es waren wirklich von Hause aus ver}{\textbar}{\textbar}197{\textbar}{\textbar}\phantomsection\label{sp.197}\sed{schiedene Laute. So haben die Indogermanisten neben dem \textit{ă} der \textsc{Schleicher}’schen Ursprache ein \textit{\u{e}} und ein \textit{\u{o}}, statt deren einfacher Gutturalreihe eine doppelte nachgewiesen. Oder der Unterschied war nicht durch die Beschaffenheit des Lautes selbst, sondern durch den Einfluss seiner Nachbarn bedingt, er war euphonischer Natur. – Von anderen störenden Mächten haben wir jetzt noch nicht zu reden.}

Mit dem sogenannten Wohllaute ist früher von den Philologen viel Unfug getrieben worden. Man empfand es, wie angenehm die Dichtungen und Reden der Alten, ihre Lautfolge und Rhythmik, in’s Ohr fielen, man hatte auch ganz recht, den Alten eine gleiche Empfindung zuzuschreiben, und nun missbrauchte man die Euphonik als ein bequemes Auskunftsmittel, um eine Menge scheinbarer Unregelmässigkeiten zu erklären: die Alten sollten ihrer Sprache ich weiss nicht welchen Zwang angethan haben, bloss damit sie hübsch klänge. So verlegte man den Sitz der Euphonik in das Ohr, statt in den Mund.

Das Wohllautsgefühl der Völker ist verschieden, sowohl nach dem \fed{{\textbar}204{\textbar}}\phantomsection\label{fp.204} Grade der Empfindlichkeit, wie nach der Richtung, nach seinen Vorlieben und Abneigungen. Wo es aber vorhanden ist, da wird man ihm lieber schmeicheln, als es verletzen, und der Zwang, den es der Sprache auferlegt, kann für diese sogar heilsam werden. Im Chinesischen erscheint die Rhythmik als ein sehr wirksamer grammatischer Factor, Anmuth und Kraft, Kürze und Deutlichkeit in sich vereinigend. Kein Wunder also, dass man sie erstrebt; kein Wunder auch, dass sie sich als eine Sache der Gewohnheit oft ungewollt einstellt. Die Möglichkeit also, dass Wohllautszwecke auch die gewöhnliche, ungekünstelte Rede nachhaltig beeinflussen mögen, ist nicht zu leugnen. Nur das ist zu vermuthen, dass in weitaus den meisten Fällen nicht Zwecke, sondern Ursachen, nicht vorgestellte akustische Wirkungen, sondern mechanische Vorgänge in den Sprachorganen das erzeugen, was man Wohllaut nennt.

Und Wohllaut mag man es immerhin nennen. Jeder liebt es, seine Sprache fliessend und ohne merkliche Anstrengung geredet zu hören. Stockender Vortrag ermüdet, ungewohnte, heftige Bewegungen des Mundes lassen fratzenhaft. Was nun ungewohnt, was schwierig ist, das hängt eben von der lautlichen Beschaffenheit der Einzelsprache ab. Es giebt Sprachen, die die Organe in einer sehr vielseitigen Gymnastik üben; – die slavischen und die kaukasischen gehören dahin. Es giebt Sprachen, die gewisse Übungen, z.~B. in den Gutturalen, stark bevorzugen, andere, z.~B. in stärkeren Consonantenhäufungen, nicht veranlassen; das Gesagte gilt vom Arabischen, vom Ketschua und vielleicht, mutatis mutandis, von der Mehrzahl der Sprachen. Endlich giebt es Sprachen, die durch die Weichlichkeit und Armuth ihres Lautwesens der Trägheit der Sprachwerkzeuge unglaublichen Vorschub leisten, – so die polynesischen. Allen aber, den reichsten wie den ärmsten, werden gewisse Lautverbindungen geläufiger sein als andere, manche werden ihnen gänzlich fehlen; und was in der \sed{{\textbar}{\textbar}198{\textbar}{\textbar}}\phantomsection\label{sp.198} Sprache selten vorkommt, das kommt gern vollends abhanden, eben weil es unbequem ist.

Offenbar nun hat in diesem Falle die Bequemlichkeit ihren Sitz in den Sprachorganen. Gewisse Laut- und Tonfolgen und Rhythmen sind diesen geläufig, andere nicht, und das eben ist nach der Eigenart der Sprachen verschieden.

\sed{Dass die Laute voneinander, dass sie durch den Wort- oder Satzaccent beeinflusst werden, ist wohl in den allermeisten Sprachen zu beobachten, vielleicht von vornherein in allen zu vermuthen. Nach Mass und Art dieser Beeinflussungen aber verhalten sich die verschiedenen Sprachen, selbst die einer einzelnen Familie sehr ungleich. Lautgesetze sind in der Regel örtlich und zeitlich beschränkt. Jene Gesetze aber, die der bequemeren Aussprache dienen, werden wohl ebenso oft, wo nicht öfter, erlaubende sein, wie befehlende. Wie weit reichen die Freiheiten? wo fangen die bindenden Vorschriften an? Und wenn wir auf diese Fragen die Antwort suchen: wie weit reichen unsre Quellen? Ich glaube, eben weit genug, um uns wieder einmal vor verfrühten Verallgemeinerungen zu warnen.}

Wir fassen den Begriff der Euphonik sehr weit, sodass jeder Art gegenseitige Beeinflussung von Lauten oder Betonungen unter ihn fällt. An eigentlichen Wohllaut ist dabei am \update{Wenigsten}{wenigsten} zu denken, und man \fed{{\textbar}205{\textbar}}\phantomsection\label{fp.205} hat sehr verständiger Weise den anspruchsvollen griechischen Namen durch den nüchternen indischen ersetzt; das Wort \textit{sandhi} ist nachgerade Gemeingut der Sprachwissenschaft geworden.

\sed{Meist wirkt der Sandhi störend, verwischt das ursprüngliche Lautbild mehr oder weniger. Er kann aber auch, wie im Französischen, erhaltend wirken und Laute, die sonst geschwunden sind, ausnahmsweise wieder zur Geltung bringen („\textit{il-s-on-t-eu}, \textit{ils ont eu}“ u. dergl.). Und dann kann es weiter geschehen, dass solche Laute allmählich im Sprachgefühle die Rolle rein euphonischer Einschiebsel annehmen und auch da angewandt werden, wo sie eigentlich nicht hingehörten. Das νῦ ἐφελκυστικόν des Griechischen hat dies Schicksal gehabt, und das Sanskrit weist Ähnliches auf. In französischen Patois erscheint -\textit{s}- (-\textit{z}-) geradezu als Polsterlaut zwischen Aus- und Anlautsvocalen, natürlich nach Analogie des pluralischen \textit{–s}. –} Ich will es versuchen, die hierher gehörigen Erscheinungen unter den \retro{verschiedenen}{verschieden} möglichen Gesichtspunkten zu classificiren.

1. Richtung der Beeinflussung.

a) Das Frühere wirkt auf das Spätere. In den malaischen Sprachen, zumal im Batta und Madegassischen, ist dieser Hergang der gewöhnliche; aber auch anderwärts ist er zu beobachten. So wird in oberdeutschen Dialekten \textit{s} nach \textit{r} im Auslaute zu \textit{sch}: Gieb mir’sch, erscht; ähnlich im Schwedischen: \textit{föršt} = erst. Im Sanskrit wird inlautendes \textit{s} hinter \textit{i}, \textit{u}, \textit{e}, \textit{o}, \textit{r} und \textit{k} zu \textit{š}: masc. \textit{tēšām}, fem. \textit{tāsām} = deren, \corr{1891 und 1901}{\textit{nadīšu},}{\textit{nadīšū},} in den Flüssen, \textit{dvēkši}, du hassest. Solche Nach\sed{{\textbar}{\textbar}199{\textbar}{\textbar}}\phantomsection\label{sp.199}wirkungen können sich weithin erstrecken. Im Sanskrit wird oft \textit{n} nach vorausgehendem \textit{r} oder \textit{š} lingual (\textit{ṇ}), wenn nicht Dentale, Linguale oder Palatale dazwischen treten und die Nachwirkung aufheben: \update{\textit{brahmaṇya},}{\textit{brāhmaṇya},} den Brahmanen zugethan. Besonders wichtig ist das Gesetz der Vocalharmonie in den uralaltaischen Sprachen, vermöge dessen der Stammvocal die Vocale der Suffixe bestimmt: mandschu \textit{alaha} = habe erzählt, aber \textit{genehe} = bin gegangen, \textit{toktoho} = habe geordnet. \sed{Verwandt, doch meines Wissens vereinzelt, ist im Arabischen die Regel, wonach die Objects- und Possessivsuffixe \textit{-hu} = ihn, sein, \textit{-humā} = sie zwei, ihrer beider, \textit{-hum} = sie, ihr (plur. masc.) und \textit{-hunna} = sie, ihr (plur. fem.) nach auslautendem \textit{i}, \textit{ī} und \textit{ai} ihr \textit{u} in \textit{i} verwandeln: Nom. \textit{nafsu-hu}, Acc. \textit{nafsa-hu} = seine Seele, aber Genit. \textit{nafsi-hi} = seiner Seele u.~s.~w.}

b) Das Spätere wirkt auf das Frühere, der Process ist vorgreifend. Dies bildet in den indogermanischen Sprachen die Regel: λείπω : ἐλείφθην. Wir sprechen: an-sehen, An-denken, aber am-binden, Ang-kunft. Die Sandhigesetze des Sanskrit beruhen zumeist auf der Anähnlichung des Vorhergehenden an das Folgende; die Epenthese des Zend, der Umlaut im Deutschen gehören gleichfalls hierher.

c) Früheres und Folgendes beeinflussen einander gegenseitig. Im Sanskrit wird \textit{labh} + \textit{ta} zu \textit{labdha} = genommen, \textit{mahat çakaţam}, grosser Wagen, wird \textit{mahač čhakaţam}. Anlaut und Auslaut dürfen nicht gleichzeitig aspirirt sein: \textit{duh}, melkend, statt \textit{dhuh}, lautet im nom. sing. \textit{dhuk}, im gen. sing. \textit{duhas}, aber im loc. plur. \textit{dhukšu}. Natürlich gehören auch die Vocalverschmelzungen hierher. Im Sanskrit wird \textit{a} + \textit{i} zu \textit{ē}, \textit{a} + \textit{u} zu \textit{ō}, im Griechischen ε + ο zu ου: γένεος : γένους. Im Japanischen und Kaffrischen wird umgekehrt \textit{i} + \textit{a} zu \textit{e}.

\largerpage[-1]2. Was übt den Einfluss, und was wird beeinflusst: Vocale, Consonanten oder \update{Betonungen?}{Betonungen,} \sed{der Anlaut auf den Auslaut oder umgekehrt? In gewissen chinesischen Dialekten, z.~B. dem von Canton, wird bei labialem Anlaute der Auslauts-Labial in den entsprechenden Dental verwandelt: \textit{fap} in \textit{fat}, \textit{fam} in \textit{fan}, \textit{pim} in \textit{pin} u.~s.~w.} Ich will die sich hier ergebenden Möglichkeiten nicht weiter durch Beispiele verfolgen und nur an die Fortschritte \fed{{\textbar}206{\textbar}}\phantomsection\label{fp.206} erinnern, die die Indogermanistik gemacht hat, seit sie den Accenterscheinungen schärfere Aufmerksamkeit zuwendet. \sed{Das Gesetz, dass anlautende Media den Tiefton bewirkt oder nach ihrer Umwandlung in die Tenuis zurücklässt, herrscht im Chinesischen, im östlichen Dialekte des Tibetischen und im Thai (Siamesischen).}

\begin{sloppypar}3. Geschieht die Beeinflussung innerhalb des einzelnen Wortes (innerer Sandhi), oder zwischen benachbarten Wörtern (äusserer Sandhi)? \sed{Es kann geschehen, dass der eine Allerhand zulässt, was der andere verbietet, oder dass der eine andere Lautwechsel fordert, als der andere. Im Sanskrit ist sogar der äussere Sandhi unduldsamer als der innere. Dieser duldet ein \textit{s} zwischen zwei Vocalen: \textit{asi}, du bist, \textit{āsam}, ich war, oder er verwandelt es nach \textit{i}, \textit{e}, \textit{u}, \textit{o} in }{\textbar}{\textbar}200{\textbar}{\textbar}\phantomsection\label{sp.200}\sed{ \textit{š}: \textit{vidušī}, \textit{vidušas}, statt \textit{vidusī}, \textit{vidusas}. Im äusseren Sandhi dagegen wird \textit{açvas asti} (= \textit{equus est}) zu \textit{açvo} \corr{1901}{\textit{’sti},}{\textit{’sti}} \textit{ravis iva} (wie die Sonne) zu \textit{ravir iva}. Die Verbindung Tenuis + Nasal ist im Inneren des Wortstammes zulässig: \textit{ratna} Edelstein, \textit{ātman}, Seele, \textit{āpnōmi}, ich erlange. Im äusseren Sandhi dagegen müssten die Tenues durch die entsprechenden Mediae oder Nasale ersetzt werden: \textit{t} durch \textit{d} oder \textit{n}, – \textit{p} durch \textit{b} oder \textit{m}.}\end{sloppypar}

4. Was ist das Ergebniss: Anähnlichung, Entähnlichung, Verschmelzung, Platzwechsel (Metathesis) oder theilweise Tilgung, diese mit oder ohne Hinterlassung von Nachwirkungen (Ersatzdehnung, geschliffener Accent, Ab- oder Umlaut u. dgl.). \sed{Die Anähnlichung kann sowohl den Ort wie die Art der Lauterzeugung betreffen. Den Ort z.~B. bei der Behandlung der auslautenden Nasale im Griechischen: ἐμ-πίπτειν, συγ-κοπή, statt ἐν-, συν-. Die Art, das heisst Angleichung zwischen Tenuis, Media, Nasal u.~s.~w. Lateinisch \textit{scriptum} st. \textit{scrib}-, griech. ἐλείφ-θην, st. λειπ-, sanskrit \textit{văg-bhiḥ} st. \textit{vāk-}, \textit{āsīn Madrešu} st. \corr{1901}{\textit{āsīt}.}{\textit{āsīt-}.} Beides combinirt, z.~B. sanskrit \corr{1901}{\textit{abhavaǰ ǰaṃtuḥ}}{\textit{abhavnǰ ǰamtuḥ}} statt \textit{abhavat}. Zur Entähnlichung mag man wohl auch die Mittel zur Vermeidung des Hiatus, ephelkystische Consonanten rechnen und jene Mediä, die z.~B. das Griechische zwischen den Nasal und \textit{r} oder \textit{l} schiebt: ἀνδρός, ἄμβροτος.}

\sed{5. Wie verhalten sich die Laute gegeneinander in Rücksicht auf ihre Wirkungskraft? Wenn das Sanskrit \textit{çvaçura} statt \textit{svaçura}, Schwäher, und \textit{çaça} statt \textit{çasa}, Hase aufweist, so hat dort der spätere Laut den früheren, hier der frühere den späteren, in beiden Fällen das \textit{ç} das \textit{s} nach sich umgestaltet. Also war nicht die Stellung, sondern die Qualität der Laute entscheidend, der palatale Zischlaut war kräftiger als der dentale.}

\begin{sloppypar}\sed{Um nun wieder einmal den Unterschied zwischen einzelsprachlicher und sprachgeschichtlicher Auffassung zu zeigen, wähle ich ein Beispiel aus der Sanskrit-Grammatik. Diese lehrt, dass, wenn eine aspirirt auslautende Wurzel durch Sandhi die Aspiration im Auslaute verliert, letztere auf den \corr{1901}{Anlauts\-konsonanten,}{Auslauts\-konsonanten,} wenn dieser dessen fähig ist, zurückspringt: \textit{budh}: \textit{bodhāmi} ich bemerke, aber \textit{bhotsye}, ich werde mir merken. Diese Regel ist so im Sinne der Einzelsprache ganz richtig; denn dem Sprachgefühle des Inders wird die gewöhnliche Lautform \textit{budh}, \textit{bodh} als die normale, der aspirirte Anlaut als etwas Secundäres, Ausnahmsweises erscheinen, und zwar als ein Ersatz für die verlorene Aspiration im Auslaute. Spricht der Grammatiker dies aus, so erklärt er den Vorgang im Sinne des redenden Inders, das heisst im Sinne der Einzelsprache, und thut somit bloss seine Schuldigkeit. Nun kann aber die griechische Wurzel πυθ nur aus φυθ, nicht aus βυθ entstanden sein. Die gemeinsame Wurzel ist mithin als \textit{bhudh} anzusetzen; und in sprachgeschichtlicher Forschung wird die Regel lauten: Aspiration im An- und Auslaute zugleich duldet das Sanskrit nicht. Der Anlaut behält seine Aspiration nur dann, wenn der Auslaut dieselbe durch Sandhi ver}{\textbar}{\textbar}201{\textbar}{\textbar}\phantomsection\label{sp.201}\sed{liert; sonst wird er in den entsprechenden nicht aspirirten Laut verwandelt. Dann muss aber der Historiker auch erklären, wie sich in dieser Hinsicht das Sprachgefühl geändert hat.}\end{sloppypar}

\largerpage[-1]
Eigentlich ist jederlei Sandhi ein Opfer, das die Deutlichkeit der Bequemlichkeit bringt. Nur freilich können die Bequemlichkeitsbedürfnisse sehr verschiedener Art sein. Der Eine will schnell zu Ende kommen, spart Sylben und Athem durch zungenbrecherische Consonantenhäufungen; der Andere lässt lieber die Lungen arbeiten, als die Lippen und die Zunge, und fühlt sich in einer vocalreichen Sprache behaglich. Das ist nun vorwiegend mechanisch, abhängig von physiologischer Gewöhnung, wenngleich Gemüthsanlagen dabei einigen Antheil haben mochten. Dem Seelenleben allein aber ist es zuzuschreiben, ob vorgreifend der frühere Laut dem späteren, – oder, einer Nachwirkung folgend, der spätere Laut dem früheren angeglichen wird. Nur in manchen Fällen dürften die Zu- und Abneigungen, oder richtiger die Bedingungen der leichten oder schwierigen Lautfolge, allgemeinherrschend sein. Tenuis und Media oder Media und Tenuis unmittelbar hintereinander befinden sich merkwürdigerweise zuweilen im Anlaute; so tibetisch \textit{dpe-čha}, Buch, \sed{\textit{dpyid}, Frühling, \textit{bkur-ba}, ehren, \textit{bkren-pa}, arm, \textit{dkyil}, Mitte, \textit{dkrigs-pa}, dunkel, \textit{gtos}, Grösse, \textit{dpral-ba}, Stirne;} Karen: \textit{Pgo}, Name eines Volksstammes. Da in dieser Sprache neben \textit{pg} auch \textit{bg} vorkommt, z.~B. ein anderer Stammesname \textit{Bgai} geschrieben wird, so ist anzunehmen, dass zwischen \textit{pg} und \textit{bg} ein genauer Unterschied besteht. Tenuis und Media oder Media und Tenuis derselben Organe hintereinander auszusprechen, z.~B. \textit{dt}, \textit{td}, \textit{kg} u.~s.~w., widerstrebt wohl den Sprachorganen aller Völker. Wo die Schriftsprache dergleichen aufweist, wie in den Wörtern Bettdecke, Strickgarn, Lobpreisung, da wird die flüchtige Aussprache die Sache vereinfachen: „Beddecke, Striggarn, Lōpreisung“ oder umgekehrt: Bettecke, Strickarn, Lobreisung. Höchstens wird der eine der beiden unverträglichen Laute nur halb angedeutet, etwa wie in dem englischen Satz „He ha(d)to do ou(t)doors.“ – Dass der Nasal sich dem folgenden Consonanten organisch anähnelt, ist wohl weit verbreitet, aber keineswegs allgemeingültig. So gut wir \fed{{\textbar}207{\textbar}}\phantomsection\label{fp.207} sagen „nimmt, Amt“, so gut sagt ein Castilianer \textit{blanco} mit deutlichem dentalen \textit{n}.

Es lohnte sich wohl, die Sandhigesetze der verschiedenen Sprachen systematisch zusammenzustellen, wie es \textsc{Schleicher} in seiner Abhandlung über den Zetacismus in Betreff der Palatalisirungserscheinungen unternommen hat. Auffallende Übereinstimmungen in den entlegensten Gegenden würden sich dabei ergeben. Unorganischer Dental im Anlaute findet sich in deutschen Dialekten: derschlagen, derzählen, \fed{– vereinzelt im Holländischen: \textit{tachtig} = achtzig;} aber auch im Malaischen bei den persönlichen Fürwörtern: \update{\textit{dāku}}{\textit{dāku},} statt \textit{āku}, ich, \update{\textit{dīya}}{\textit{dīya},} statt \textit{īya}, er. Im Mafoor von Neuguinea ist dieses \textit{d} so zu sagen Vertreter des \update{spiritus}{Spiritus} lenis im Falle und zur Vermeidung eines Hiatus: \textit{ya-d-ān}, ich esse, statt \textit{ya-ān}. \sed{Anderwärts, z.~B. im Fran}{\textbar}{\textbar}202{\textbar}{\textbar}\phantomsection\label{sp.202}\sed{zösischen, hat die Abneigung gegen den Hiatus manchen, sonst verstummten Laut gerettet: \textit{ils ont eu}, \textit{aime-t-il} u.~s.~w.}

\sed{Euphonische Einschiebsel sind, wo uns die Sprachgeschichte im Stiche lässt, immer besonders schwer zu beurtheilen: Sind es wirklich unorganische Zuthaten, oder hat sich in ihnen nicht etwa die ältere Lautform erhalten? Und wenn dies: wo und wie ist etymologisch abzutheilen. Die baskische Sprache ist hierin, wie überhaupt in den Fragen der Lautgesetzlichkeit, besonders interessant. Vieles ist hier, zumal durch \textsc{van Eys}’ Scharfsinn, aufgehellt. Aber selbst dieser Forscher muthet manchmal der Euphonik Bedenkliches zu. Die Form \textit{daukanerik} = das was er besitzt, enthält zunächst \textit{dauka} = er besitzt, dann folgende erkennbare Elemente: das Relativsuffix, \textit{-n} und das indefinite \textit{-ik}. Dies würde die Form \textit{daukanik} ergeben, an der das baskische Lautgesetz schwerlich Anstand nehmen dürfte. Der Verfasser der trefflichen Grammaire comparée des dialectes basques nimmt nun an, es sei das \textit{-er-} aus euphonischen Gründen eingeschaltet (S.~40 das.). Ein euphonischer Grund lag aber kaum vor. Andere Male (z.~B. S.~36) erklärt er die An- oder Abwesenheit solcher Einschiebsel aus dem Bedürfnisse \corr{1901}{ver\-schie\-dene}{ver\-schie\-dener} Formen lautlich zu differenziren: das Pronomen demonstrativum \textit{ar} bildet mit dem Activsuffixe \textit{-k}: \textit{ark}, dagegen mit dem Pluralsuffixe \textit{-k}: \textit{arek}; das \textit{e} sei zum Zwecke der Unterscheidung eingefügt. Hier würde ich den Nachweis verlangen, dass nicht etwa \textit{-ek} die ursprüngliche Form des Mehrheitssuffixes gewesen wäre.}

Ein entschiedener Fortschritt der jüngeren Indogermanistik besteht darin, dass sie mit Sandhi-Erscheinungen der Ursprache rechnet. Allerdings sind die Rechnungen dadurch verwickelter geworden, und manche unbekannte Grösse findet darin Platz. Dafür dürfte aber auch jetzt manche anscheinende Unregelmässigkeit in der Lautentwickelung die einzig mögliche Erklärung gefunden haben. \sed{Französisch \textit{hors} neben und statt \textit{fors} soll aus \textit{dehors} neben und statt \textit{defors} ausgeschält sein; aber auch so steht es vereinzelt da. (\textsc{Meyer-Lübke}, Gramm. der roman. Sprachen, I, 511–512).} Jene schwierige Doppelform ὗς, σῦς \update{rechne ich}{rechnet man gleichfalls} dahin. Der Spiritus asper entspricht bekanntlich dem allgemeinen Lautgesetze. Woher nun das Sigma? An eine Entlehnung ist hier kaum zu denken, eher an die Nachwirkung eines Sandhi, wornach in vorgeschichtlicher Zeit das anlautende \textit{s} unter gewissen Umständen rein ausgesprochen, nicht zu \textit{h} verflüchtigt wurde. \sed{Schwerer wird man sagen können, warum gerade dies eine Wort dem sauberen Lautgesetze entgegen die Doppelform bewahrt hat, warum nicht auch die vielen anderen, die wir nur mit dem Anlaute \textit{h} kennen. Ein solches Asyl für bedrängte Lautgesetze ist fast zu bequem und geräumig.} Ein \update{gutes}{anderes, besseres} Beispiel liefert das Pronomen der ersten Person Pluralis in vielen deutschen Mundarten: mir, m’r, statt wir. Das \textit{w} wurde durch Anähnlichung an das \textit{n}, älter \textit{m}, der Conjugationsform in Verbindungen wie „wollen \sed{{\textbar}{\textbar}203{\textbar}{\textbar}}\phantomsection\label{sp.203} wir, woll’ m’r“ zu \textit{m}, und schliesslich gewöhnte man sich so an diese Form, dass man sie zur allein herrschenden machte, die ursprüngliche darüber vergass; nach der Analogie von „wollen mir“ sagte und sagt man nun auch: „mir wollen“. \sed{So können Sandhigesetze, jetzt erhaltend, jetzt entstellend, auch über ihr ursprüngliches Geltungsbereich hinaus- und nachwirken, – eine unheimlich störende Macht, mit der die Lautgeschichte wohl rechnen, die sie in ihren Nöthen wohl anrufen, deren Walten sie aber nur in vereinzelten Fällen mit Sicherheit controlliren kann.}

Wie sehr Sandhi-Erscheinungen die Lautbilder der Wörter verändern, ja verfälschen können, das ist leicht zu erproben, wenn man eine fremde Sprache durch blossen mündlichen Verkehr erlernt. Im Französischen hört man den Plural von \textit{oeil}: \textit{yeux} fast nie ohne vorausgehendes weiches \textit{s}: \textit{les yeux}, \textit{ses yeux}, \textit{beaux yeux} u.~s.~w. An der Singularform hat \textit{yeux} so gut wie keinen Rückhalt, und so mag man sich lange Zeit mit dem Wahne begnügen, der Plural von \textit{oeil} laute \textit{zieux}. Ich müsste sehr \fed{{\textbar}208{\textbar}}\phantomsection\label{fp.208} irren, wenn ich dieser Form nicht auch in Schreiben ungebildeter Franzosen und in dialektischen Aufzeichnungen begegnet bin. Dass Ortsnamen in ihren Anlauten durch Rester von Präpositionen Zuwachs erfahren haben, ist zumal im Neugriechischen oft nachzuweisen. Der germanische Auslaut der zweiten Person Singularis \textit{st} (dem Gotischen noch fremd) verdankt seinen Ursprung dem oft nachfolgenden persönlichen Fürworte: mittelhochdeutsch: \textit{bistu}, altschwedisch \textit{ästu}.

Zum Glück ist dafür gesorgt, dass der Sandhi nur ausnahmsweise zu solchen dauernden Entstellungen der Wörter führt. Es wird immer zu den Ausnahmen gehören, dass ein Wort vorzugsweise oft im gleichen Sandhi erscheint, ohne dass fort und fort die Etymologie an seine richtige Form erinnerte. Denn gerade von diesem Punkte aus übt das Deutlichkeitsbedürfniss über das allzu schlaffe Gebahren des Bequemlichkeitstriebes eine Art zügelnder Oberaufsicht. Spreche ich „ambinden, eingkehren“, so sage ich wieder andere Male: „ich binde an, kehre ein“, und werde so an die ursprüngliche Lautform erinnert. Und dann treibt mich wohl andere Male mein etymologisches Gewissen, auch in der Zusammensetzung den reinen Dentalnasal zu sprechen. Wo das etymologische Bewusstsein mangelt, da hat das Sandhiwesen freien Lauf, und dann kann es umgekehrt die Etymologie auf Abwege leiten, z.~B. Erreichniss, von erreichen, statt Ereignis, von ereignen, das seinerseits das ältere eräugnen ersetzt hat. Dagegen kann man auch bei Ungebildeten beobachten, dass sie z.~B. die letzten vier Buchstaben des Wortes „Abart“ anders aussprechen, als „Bart“, indem sie hinter dem \textit{b} eine deutliche Sylbentrennung eintreten lassen.

Noch ein Anderes mag den Verwüstungen der Euphonik entgegenwirken: das gelegentliche Stocken in der Rede, das den „äusseren Sandhi“ und manchmal auch den inneren unterbricht. Ich möchte wissen, wie sich ein sanskrit \sed{{\textbar}{\textbar}204{\textbar}{\textbar}}\phantomsection\label{sp.204} redender Inder benommen hat, wenn er sich z.~B. besann, ob sein nächstes Wort nach einem auslautenden \sed{nicht nasalen} Consonanten \textit{bālaḥ}, Knabe, oder \textit{putraḥ}, Sohn, sein sollte; denn in beiden Fällen verlangte der Sandhi ganz verschiedene Vorkehrungen. In solchen Augenblicken und bei gelegentlichen einwortigen Antworten und Ausrufen wird man doch daran erinnert, dass das einzelne Wort eine gewisse Selbständigkeit hat.

\sed{Offenbar ist die lautgeschlichtliche Forschung den Sandhi-Erscheinungen und ihren Nachwirkungen gegenüber besonders übel dran. Die Gesetze, oder richtiger Tendenzen, mit denen sie es hier zu thun hat, können gebietend, schlechthin bindend, – sie werden aber auch oft nur erlaubend sein, das heisst, erklären, dass neben der sorgfältigeren Aussprache die nachlässigere als gleich richtig gilt. Wo sie sich aber dazu verstehen muss, da erkennt sie schon eine unsichere Articulation an, und da wird sie von unverbrüchlichen Lautgesetzen nur noch in sehr uneigentlichem Sinne reden können. Denn ein Gesetz, das gestattet, es so oder auch anders zu machen, pflegt wenigstens von den Juristen nicht ein unverbrüchliches genannt zu werden.}

\largerpage[1]\sed{Man sieht leicht ein, dass das Sprachgefühl über den unverfälschten Lautbestand der einzelnen Satz- und Worttheile um so eifersüchtiger wacht, je lebhafter ihm deren Sonderbedeutung gegenwärtig ist. Isolirende Sprachen und solche, deren lose Agglutination der Isolirung noch nahe steht, werden also dem Sandhi wenig Spielraum gestatten. Allein der Bequemlichkeitstrieb, mag er auch sonst meist verhalten bleiben, ist immer vorhanden und wird bei besonders flüchtiger Rede, z.~B. in allgeläufigen Formeln, zum Durchbruch kommen. Hier äussert sich das, was wir später, im IV. Buche, als innere Articulation verstehen werden. Wo sich Gehör und Verständniss mit einem verflüchtigten Lautbilde, mit der Skizze, statt des Gemäldes begnügen, da thut auch der Redner genug, wenn er nur annähernd den lautlichen Gesammteindruck hervorruft, und da hat der Sandhi freies Spiel.}

\sed{Denn das ist allerdings anzunehmen, dass er ursprünglich, wo er überhaupt gestattet war, auch eben nur facultativen, nicht präceptiven Werth hatte, dass also der Sprechende beliebig die reine oder die getrübte Lautform anwenden durfte: adferre neben afferre, ein-geben neben eing-geben. Und hatten die Hörer anfangs, halb widerstrebend, zu Gunsten der flüchtigeren Form eine Art Rabatt bewilligt, so mochte ihnen später bei dem unverkümmerten Laute zu Muthe sein, als empfingen sie ausnahmsweise ein Aufmass: es trat im Sprachgefühle eine Verschiebung ein, Regel und Ausnahme vertauschten die Rollen.}

\begin{sloppypar}\sed{Aber zunächst war die Regel doch immer nur das Gewöhnlichere, die Ausnahmen waren daneben noch möglich. Auch das konnte sich ändern. Ein durch Übung versteinertes ästhetisches Gefühl konnte die flüssigere, flüchtigere Form geradezu verlangen, die reinere, härtere, wie einen Missklang verabscheuen. }{\textbar}{\textbar}205{\textbar}{\textbar}\phantomsection\label{sp.205} \sed{Beim inneren Sandhi geschah das öfter, als beim äusseren; denn die mechanische Einheit des Wortes ist dichter, als die des Satzes, dessen Glieder mit Synonymen vertauscht, wohl auch, wo die Wortstellung Freiheiten gestattet, beliebig umgeordnet werden können. So ist es, um nur das nächstliegende Beispiel anzuführen, im Griechischen; und doch findet man auch hier dialektische Schreibungen wie τὸμ πρῶτον, τῶγ χρημάτων. Der Inder dagegen behandelte seinen ganzen Satz wie ein untrennbares Ganze, wie ein Conglutinat, wenn man den Ausdruck einmal gelten lassen will: er verlangte geradezu den äusseren Sandhi so gut wie den inneren und, seltsam genug, er war in jenem noch anspruchsvoller, als in diesem. Dass freilich der Volksmund nicht allemal leistete, was die Grammatik verlangte, lag in der Natur der Sache.}\end{sloppypar}

\sed{Anders sind natürlich die Fälle zu beurtheilen, wo der Sandhi ausnahmsweise eine geschwundene Lautform wieder an den Tag hebt, wie das \textit{t} der 3. Person Singularis und Pluralis und das pluralische \textit{s} im Französischen, das \textit{d} von \textit{ed}, \corr{1901}{im Italie\-nischen,}{und, im Italie\-nischen,} und das \textit{s} im Acc. Plur. der Masculinstämme auf \textit{a}, das \textit{t} der dritten Person Plur. im Imperf. und Aorist Act. des Sanskrit: \textit{açvāṃs tu} = \textit{equos} \corr{1901}{\textit{ autem};}{\textit{autem}:} \textit{āsant atra} = \textit{erant ibi}. Hier hat der Sandhi nicht als Verderber, sondern als Retter gewaltet.}

\sed{Wir haben somit das rein lautmechanische Gebiet früher verlassen, das psychologische früher betreten müssen, als zu erwarten stand. Alle nun weiterhin zu betrachtenden Mächte der Sprachgeschichte gehören dem Seelenleben an. Ihr Dasein, das heisst, die Möglichkeit ihres Eingreifens, können wir erweisen. Dass aber in einem gegebenen Falle die eine oder die andere von ihnen wirklich eingegriffen habe, ist nur hie und da, – dass sie habe eingreifen \so{müssen}, ist wahrscheinlich nie bis zur Unumstösslichkeit darzuthun. Nur das blinde Wirken physischer Kräfte gestattet sichere aporiorische Berechnungen. Wo der Menschengeist mit seinen Launen, wo die Gewalt der Individualität eingreift, da ist eben die Sprachgeschichte, so sie nicht auf jeden Erklärungsversuch verzichten will, auf’s Umhertasten angewiesen, ob sie unter den vielen Möglichkeiten eine Wahrscheinlichkeit entdecke; da muss sich zum Scharfsinn jener Tact gesellen, der nur im Verkehre mit dem Leben, – mit den lebenden Sprachen, – zu erwerben ist.}

\pdfbookmark[2]{II. §. 3 b. Bevorzugung und Verwahrlosung in der Articulation.}{III.II.II.3b}
\cohead{II. §. 3 b. Bevorzugung und Verwahrlosung in der Articulation.}
\subsection*{\sed{§ 3 b.}}\phantomsection\label{III.II.II.3b}
\subsection*{\sed{Bevorzugung und Verwahrlosung in der Articulation.}}
\sed{Wir wollen uns an den Heischesatz halten, dass in einer Mundart Gleiches unter gleichen Umständen immer gleich ausgesprochen werde. Natürlich gehört zu jenen gleichen oder ungleichen Umständen auch das Mass des Nachdruckes, den die Seele und demzufolge das Sprachorgan des Redenden auf den Ausspruch oder auf einen Theil desselben legt: die Arbeit des Redenden und der Eindruck,} {\textbar}{\textbar}206{\textbar}{\textbar}\phantomsection\label{sp.206} \sed{den der Hörende empfängt, sind verschieden, jenachdem etwas sorgsam und deutlich oder flüchtig ausgesprochen, jenachdem es schärfer oder gelinder betont wird. Wir haben auch gesehen, dass schwache, flüchtige Betonung den Lautverschliff befördert. Inwieweit hierbei der Satzaccent eine Rolle spielt, pflegt aus schriftlichen Quellen nicht erkennbar zu sein. In dieser Hinsicht, wie in so vielen anderen, kann also die sprachgeschichtliche Forschung bei den lebenden Sprachen, die uns aus dem Munde der Eingeborenen ins Ohr klingen, mehr lernen, als bei den todten.}

\sed{Der flüchtige Vocal wird zu einer Art \textit{\textsubring{e}} verdumpft. Es ist aber auch möglich, dass dies \textit{\textsubring{e}} gewissen Wörtern von Rechtswegen zukommt, und dass es dann, wenn einmal das Wort stärker betont werden soll, willkürlich und missverständlich durch den einen oder den anderen –reineren Vocal ersetzt wird. Bei der Vergleichung der malaischen Sprachen spielt dieser Laut, das sogenannte Pepet, eine wichtige Rolle, erklärt offenbare Unregelmässigkeiten, soweit sie eben zu erklären sind.}

\sed{Offenbar kann an und für sich jedes Wort und jeder Satz so scharf oder so flüchtig betont und articulirt werden, wie es die jeweilige Stimmung des Sprechenden mit sich bringt: man kann das sonst Gleichgültigste einmal sehr nachdrücklich, und das Wichtigste ganz leichthin sagen. Und eben, weil dies möglich ist, wird man in der Regel das scharfe, reine Lautbild auch dann nicht ganz vergessen, wenn man stattdessen im alltäglichen Gebrauche nur ein verwischtes erzeugt: der Berliner, der mit einem „Moajn“, der Thüringer, der mit „Schamster“ grüsst, weiss wohl, dass eigentlich „Guten Morgen“ und „Gehorsamster Diener“ gemeint ist. Es \corr{1901}{scheint}{scheint,} aber auch möglich, wennschon schwer erklärbar, dass Wörter, die meist scharf betont werden, durch häufigen Gebrauch ungewöhnliche Verunstaltungen erfahren. Weit verbreitet ist die dialektische Form \so{nischt} für „nichts“; der Lautwandel aber, der sich hier vollzogen hat, dürfte sonst beispiellos, also auf kein mechanisches Gesetz zurückzuführen sein.}

\sed{Es kann aber auch anders kommen. Der Verkehr, zumal der eigentlich geschäftliche, kann in vereinzelten Fällen die verflüchtigte Lautform schlechthin zur richtigen stempeln, oder er kann, differenzierend, der flüchtigen andere Dienste zuweisen, als der deutlichen. So sind z.~B. in hoch- und niederdeutschen und in manchen romanischen Dialekten flüchtige, pro- und enklitische Nebenformen der Personalpronomina weit verbreitet. Im Englischen und in deutschen Mundarten hat sich das Zahlwort „ein,“ \textit{one} lautlich vom unbestimmten Artikel „\textit{’n}, \textit{ä}, \textit{ă}“, englisch \textit{a} geschieden. Gewaltiges hat in solchen Vermischungen der kirchliche Sprachgebrauch geleistet. So wurde im Englischen aus \textit{presbyter}: \textit{priest}, aus ἐλεημοσύνη: \textit{alms}, sprich \textit{āms}, aus ἐπίσκοπος im \corr{1901}{Fran\-zösischen:}{Fran\-zösichen:} \textit{évêque}, im Dänischen und Schwedischen: \textit{bisp}. Der Handelsverkehr hat den Joachimsthaler Silberling zum „Thaler“, zwei Krüge bairisch Bier zu „zwei Bairisch“,} {\textbar}{\textbar}207{\textbar}{\textbar}\phantomsection\label{sp.207} \sed{im Englischen \textit{genever} zu \textit{gin}, Hochheimer zu \textit{hock}, \textit{cabriolet} zu \textit{cab} verkürzt u.~s.~w. Das mochten anfangs launische, scherzhafte Verstümmelungen sein, aber warum wurden sie in den Sprachschatz aufgenommen? Mit jenen anderen, ernsteren Trümmergestalten ist es aber doch ähnlich geschehen, wie mit den Kieseln im Strombette; je mehr sie dahingerollt wurden, desto mehr wurden sie abgeschliffen.}

\largerpage[-1]\sed{Das Ergebniss ist überall ein unregelmässiger Lautwandel, meist Lautschwund: und darauf kommt es hier an. Gewiss haben wir es auch hier mit Gesetzen zu thun; soweit diese aber lautmechanisch sind, müssen wir sie uns nicht als vorschreibend, sondern wieder als erlaubend denken. Und erlaubend im weiteren Sinne sind auch die einschlägigen psychologischen Gesetze: was Thatsache geworden ist, war nothwendig: aber die Nothwendigkeit beruhte auf uncontrollirbaren, nach Zeit und Umständen wechselnden seelischen Zuständen. Wenn die \corr{1901}{Römer}{Männer} \textit{semisqui} zu \textit{sesqui} und dann weiter \textit{sesquitertius} zu \textit{sestertius} verkürzten, so war ihnen dies durch die Lautgesetze ihrer Sprache nicht geboten, sondern nur gestattet, und wir begreifen, warum der eilige Sprachgebrauch des Marktes unter dem Erlaubten das Bequemste wählte. Wir begreifen überhaupt, wie gerade Verbindungen von und mit Zahlwörtern abnormen Verkürzungen ausgesetzt sein können, – aber eben, wir begreifen es nur insoweit, wie wir seelische Vorgänge begreifen können: als bedingte Nothwendigkeiten, und zwar als solche, deren Bedingungen sich unserer Beobachtung entziehen.}

\sed{Ähnlich wird es mit Rufnamen sein, die sich im Hausgebrauche abnutzen. Manchmal mag das Gelall der Kinder sein Theil mitthun: die Eltern ahmen es, erst scherzend, bald gewohnheitsmässig im Ernste nach: Bob statt Robert; Pepe, Peppo statt Giuseppe. Aber sie selbst können sich auch aus eigenem Antriebe den stündlich gebrauchten Namen bequemer machen: Joseph zu (englisch) Joe, oberdeutsch Sepp, Ludwig zu Lude, Friederike oder Ulrike zu Rike. Verwandt sind Erscheinungen in Titel und Anreden. Dominus konnte zu Don, Monseigneur zu Monsieur werden und in letzterem, den sonstigen Lautgesetzen entgegen, das auslautende r spurlos verschwinden: M’ssieu Armand. Das spanische Pronomen Usted ist bekanntlich aus Vuestra merced = Euer Gnaden entstanden.}

\sed{Man wird aber noch weiter gehen müssen. Bei Sprachen, deren Lautwesen einem raschen, zersetzenden Wandel unterlegen ist, wie z.~B. beim Englischen und Französischen, darf man eine allgemeine Neigung zur Lautverflüchtigung erwarten, daher nicht erwarten, dass alle Lautverschiebungen und Lautverschleifungen nach eisernen Gesetzen in gleichem Schritt und Tritt vor sich gegangen seien. Da kennen wir wohl die herrschenden Mächte, können aber nicht allemal ergründen, warum sie hier gewirkt, dort versagt haben.}

{\textbar}{\textbar}208{\textbar}{\textbar}\phantomsection\label{sp.208}

\sed{Die Sache ist jedoch noch unter einem anderen Gesichtspunkte wichtig. Und unter diesem, daher unter einem anderen Namen, werden wir ihr an einer späteren Stelle nochmals begegnen: als innerer Articulation.}

\fed{{\textbar}209{\textbar}}\phantomsection\label{fp.209}

\pdfbookmark[2]{II. §. 4. Naturlaute, als Ausnahmen von den Lautgesetzen.}{III.II.II.4}
\cohead{\edins{II. §. 4. Naturlaute, als Ausnahmen von den Lautgesetzen.}}
\subsection*{§. 4.}\phantomsection\label{III.II.II.4}
\subsection*{Naturlaute, als Ausnahmen von den Lautgesetzen.}
Die Sprachen haben Bestandtheile, die zäher als andere den Lautgesetzen widerstehen, und die, soweit ich sie übersehen kann, unter den gemeinsamen Begriff der Naturlaute fallen.

a) Wo die Natur uns selbst die Laute vorbildet, da ist es begreiflich, dass wir einfach nachahmen, so gut es unser Sprachorgan erlaubt, und so lange wir bloss nachahmen wollen. Das geht so lange, als die Laute noch dem Munde geläufig sind; erst wenn das Sprachorgan sich weigert Folge zu leisten, oder die Onomatopöie aus ihrer Ausnahmestellung in den Kreis der gemeinen Wörter getreten ist, unterliegen sie dem Wandel. Dies ist mit allen jenen Thiernamen geschehen, die noch jetzt die Spur vormaliger Lautnachahmung tragen, mit \textit{gaus}, βοῦς, \textit{bos}, Kuh, mit Gauch statt Kukuk und mit einer Menge ursprünglich onomatopoetischer Verba. Anders da, wo man sich der Lautnachahmung noch bewusst und ihrer fähig ist. Kikeriki, Kukuk, piepen haben der deutschen Lautverschiebung widerstanden, weil die Natur selbst uns die Laute in’s Ohr rief, so deutlich und mundgerecht, dass wir es ihr immer wieder nachmachen mussten. Wenn dies den Lautwandel nicht gänzlich verhindert, so kann es ihn doch verlangsamen oder gar fehlleiten; Gauch, Gickel und pfeifen sind Beispiele hierfür.

b) Auch jene ersten articulirten Äusserungen des Kindermundes sind Naturlaute, denen die Gesetze des Lautwandels nichts anhaben können. Die Zitze, englisch \textit{teat}, heisst, den hochdeutschen Lautgesetzen entgegen, im Schwäbisch-Allemannischen Tüteli. Auch dieses \textit{t} aber ist dem germanischen Lautgesetze zuwider: denn nach griechisch τίτθη, italienisch \textit{tetta}, spanisch \textit{teto}, \textit{teta} wäre urgermanisch und englisch \textit{th}, hochdeutsch \textit{d} zu erwarten; unser \textit{z} in Zitze ist aber bekanntlich eine Weiterentwickelung von \textit{t}. \sed{Es ist aber ein Naturlaut, der z.~B. auch in afrikanischen Sprachen öfter vorkommt: Bagirmi \textit{dede}, Suaheli \textit{titi}, Sena \textit{didi}, Ngola \textit{teta}; ebenso im Hebräischen \textit{dad}, chaldäisch \textit{tad}.} Man sieht, es kommt hier wie bei den Onomatopöien darauf an, ob und wann solche Wörter in den gemeinen Sprachschatz aufgenommen und so zu sagen in die Uniform gesteckt worden sind. Es ist damit ganz wie mit den Lehnwörtern, \retro{und}{nnd} gewissermassen sind sie ja auch solche.

c) Auch bei eigentlichen Empfindungslauten kommt Ähnliches vor. Unser nachdenkliches Hm! gleicht dem lateinischen \textit{hem} mehr, als es \fed{{\textbar}210{\textbar}}\phantomsection\label{fp.210} nach den Gesetzen der Lautverschiebung sein dürfte. Der Franzose, dem ein reines \textit{m} im \sed{{\textbar}{\textbar}209{\textbar}{\textbar}}\phantomsection\label{sp.209} Auslaute zuwider ist, hat daraus \textit{hein}, \textit{heim} gemacht. In der Regel freilich dürfte bei Naturlauten dieser Art die Vergleichung ziemlich schwierig sein, weil die Laute meist sehr einfach, und ihre Bedeutungen oft sehr unbestimmt sind. Was kann nicht alles ein Oh! Ach! Ei! ausdrücken.

\pdfbookmark[2]{II. §. 5. Die Analogie.}{III.II.II.5}
\cohead{II. §. 5. Die Analogie.}
\subsection*{§. 5.}\phantomsection\label{III.II.II.5}
\subsection*{Die Analogie.}
Darf ich nochmals auf das einzelsprachliche Problem zurückkommen, so formulire ich es jetzt so: Wie setzt das Sprachgefühl die Mittel zu den Zwecken in Beziehung? Concreter gesprochen: Warum wendet die Sprache gerade diesen Ausdruck für diesen Zweck an? Mit anderen Worten: Warum ist dieser Ausdruck zugleich sprach- und sachgemäss?

Nunmehr ist die Antwort fast schon mit der Frage gegeben: Weil die Sprache Ähnliches auf ähnliche Weise ausdrückt. Soweit sie das thut, – denn sie thut es bekanntlich nicht immer, – ist ihr Verfahren analog, herrscht in ihr das Gesetz der \so{Analogie}. Diese ist im Sprachleben der wichtigste Factor, für die Sprachwissenschaft einer der bedeutsamsten Begriffe, aber so beweglich, so vielseitig in ihren Anlässen wie in ihren Wirkungen, anscheinend so launenhaft hier eingreifend, dort \retro{versagend,}{verzagend,} dass die Wissenschaft mit ihr sehr vorsichtig umgehen muss. Als Werkzeug in der Hand des Forschers ist sie so geführlich, dass sie gefährlich werden kann. \sed{Sie leitet uns, wenn wir unsere Muttersprache reden, meist richtig, zuweilen auch fehl. Und diese Fehler können Anklang finden, die falschen Analogien können rechtskräftig werden. Der Erste aber, der sie gemacht hat, richtiger: Der ihnen gefolgt ist, der soll uns Rede und Antwort stehen: Wie bist du darauf verfallen? Wir wissen nur nicht, wer und wie er war. Und jene, die es ihm zuerst nachgemacht haben, sollen uns sagen, warum ihnen das Neue lieber war, als das Überkommene. Könnten sie es sagen, handelte es sich nicht um ganz unbewusste Vorgänge, so wäre es noch immer zweifelhaft, ob die Antworten übereinstimmen würden, und ob nicht die scheinbar albernsten ebenso wahr wären, wie die einleuchtendsten. Nur würden vermuthlich diese letzteren die Stimmenmehrheit für sich haben. Allen aber wäre das gemeinsam, dass die Neuerung dem Sprachgefühle jedes Einzelnen sympathisch, oder das Sprachgefühl insoweit bereits im Einschlafen gewesen ist. Denn eigentlich hat es sich doch überrumpeln lassen, dass es das Gewohnte so leichten Kaufes preisgab. Man sollte nun meinen, dies wäre nur möglich, wenn erstens die Gewohnheit schwach, das heisst wenig geübt, und zweitens die Analogie ganz besonders verlockend ist. Letzteres darf man wohl überall als gewiss annehmen. Ersteres aber ist nicht immer der Fall. Wir werden im späteren Verlaufe dieser Untersuchungen sehen, wie auch das scheinbar Alltäglichste} {\textbar}{\textbar}210{\textbar}{\textbar}\phantomsection\label{sp.210} \sed{unversehens in einen fremden Analogiekreis hinüberschlüpfen kann. Vorsichtig muss man aber immer sein, ehe man eine Analogie, deren innere Wahrscheinlichkeit nicht einleuchtet, zur Erklärung herbeizieht. Um eine scheinbare Ausnahme von einem Lautgesetze zu bestätigen, genügt ein solcher Erklärungsversuch jedenfalls nicht.}

Wir haben gesehen: \update{obschon}{obwohl} sich die Sprache regelmässig in Sätzen bewegt, und der Satz ihre erste wahrhaft organische Einheit darstellt, so haben wir doch oft genug Anlass, des Einzelwerthes der Wörter inne zu werden. In der That reicht unser Sprachgefühl noch weiter: auch den Theilen der Wörter, ihren Stämmen und Formativen misst es besondere Bedeutungswerthe bei.

Alles Sprechen ist ein Aufbauen aus Stoffen und in Formen, die in unserem Geiste vorräthig sind. Dieser Vorrath enthält und bedingt alle Möglichkeiten unserer \update{Sprach\-äusserungen.}{Sprach\-äusserung.} Wir wissen, wie er zu Stande gekommen ist, wir brauchen nur daran zu denken, wie wir unsere Muttersprache erlernt haben. Von dem, was wir reden gehört, hat unser Gedächtniss immer mehr und mehr in sich aufgenommen; Wörter, Wort\fed{{\textbar}211{\textbar}}\phantomsection\label{fp.211}verbindungen, wohl auch Sätze lernten wir nachsprechen. Bald aber vollzogen sich, uns unbewusst, in unserm Innern Vergleichungen und Abstractionen: \update{das}{das,} was im Verschiedenen das Gemeinsame war, schied sich aus, lagerte sich in unserm Geiste ab, theils als Stofftheile, die sich zum Aufbau der Rede zusammensetzen liessen, theils als Regeln, nach denen dieser Aufbau geschehen sollte. Damit wurde allmählich die Rede des Kindes ein freies Erzeugniss, an Stelle des Nachahmens trat ein Nachschaffen. Das Ergebniss mochte nicht immer richtig sein; in seiner Art regelmässig war es aber wohl immer, die Regeln wurden nur noch manchmal falsch angewandt. Das Kind sagte „gedenkt“, wie es gelernt hatte: „geschenkt, gelenkt“. Damit hatte es eine Regel am unrechten Orte angewandt, war durch die Analogie fehlgeleitet worden. Alle natürliche, das heisst ungekünstelte Handhabung der Sprache beruht theils auf unmittelbarer Erinnerung, theils auf jenen Analogien.

Es schien räthlich, hieran nochmals zu erinnern. Als \update{die historische Indo\-germa\-nistik}{der historische Indo\-germa\-nist} anfing, den Analogien und ihrer Einwirkung auf die Umgestaltung der Sprachen eine erhöhte Aufmerksamkeit zuzuwenden, gab es viel Lärm und Streit. Die Einen klagten über eine verderbliche Neuerung, die Anderen jubelten wie über eine grossartige Entdeckung, und Beide hatten Unrecht. Es war keine Neuerung, auch keine Entdeckung; eher möchte ich sagen, man sei zur Besinnung gekommen, habe den einseitig mechanischen Anschauungen entsagt und endlich den seelischen Mächten ihr Recht gegeben.\footnote{\sed{\textsc{W. v. Humboldt} hat es schon im Jahre 1812 (Deutsches Museum II, S.~496) ausgesprochen: „Man kann es als einen festen Grundsatz annehmen, dass Alles in einer Sprache auf Analogie beruht, und ihr Bau, bis in seine feinsten Theile hinein, ein organischer Bau ist. Nur wo die Sprachbildung bei einer Nation Störungen erleidet, wo ein Volk Sprach}{\textbar}{\textbar}211{\textbar}{\textbar}\sed{elemente von einem andern entlehnt, oder gezwungen wird, sich einer fremden Sprache ganz oder zum Theil zu bedienen, finden Ausnahmen von dieser Regel statt. [? Mischsprachen sind meist sehr regelmässig!] Dieser Fall tritt nun zwar wohl bei allen uns jetzt bekannten Sprachen ein – da wir von den Ursprachen und Urstämmen durch Klüfte getrennt sind, über die keine Überlieferung hinüberhilft – ... Allein, wo eine Sprache ein fremdes Element in sich aufnimmt, oder sich mit einer anderen vermischt, da beginnt auch sogleich ihre assimilirende Thätigkeit, und ihr Bemühen, nach und nach denjenigen Stoff, welcher in der Vermischung den kürzern zieht, so viel als möglich in die dem andern eigenthümliche analogische Bildung zu verwandeln, sodass durch diese Mischungen zwar kürzere und längere analogische Reihen entstehen, nicht leicht aber ganz unorganische Masse zurückbleibt.“ – Dass man von der Formenlehre unter dem Titel De analogia handelte, ist bekanntlich noch viel älteren Datums.}} Anders hatten es \sed{{\textbar}{\textbar}211{\textbar}{\textbar}}\phantomsection\label{sp.211} wohl auch die nicht aufgefasst, die man Junggrammatiker nannte. Thatsächlich übten sie nur dasselbe, was uns Anderen, die wir wildfremde Sprachen vom rein einzelsprachlichen Gesichtspunkte aus durchforschen, längst geläufig war.

Es war eine Rückkehr, und doch ein Fortschritt, mochte auch manche Überstürzung dabei vorkommen, die scharfen Widerspruch erfuhr. Darüber masse ich mir kein Urtheil an, das sind innere Streitigkeiten der Indogermanisten. Offenbar aber droht überall die Gefahr, dass man \sed{zu früh vom Versuche einer lautgesetzlichen Erklärung abstehe und} beim Suchen nach Analogien auch einmal zu weit und fehl gehe. Man sollte meinen, eine Analogie müsse nahe liegen, damit sie unsere Rede beeinflussen \update{könne,}{könne;} erst recht nahe aber müsse sie liegen, wenn sie trotz ihrer Neuheit auch bei den Hörenden Anklang finden und hinfort deren Sprache beeinflussen solle. Darin liegt nun aber wohl auch die Macht der Analogie, die Ansteckungskraft der falschen Ana\fed{{\textbar}212{\textbar}}\phantomsection\label{fp.212}logie auf die Sprache der Sprachgenossen, dass sie in der Regel eine Vereinfachung der Sprache darstellt, mithin zugleich der Deutlichkeit Genüge leistet und der Bequemlichkeit fröhnt.

Dies leuchtet so sehr ein, dass man sich wundern möchte, woher denn überhaupt die Unregelmässigkeiten in unsere Sprachen gekommen seien. Sie scheinen gar zu lästig, und jene neuerfundenen Weltsprachen, deren man in drei Wochen vier lernen kann, beruhen immer auf dem verständigen Grundsatze ausnahmsloser Analogie. Volkssprachen aber sind nicht willkürliche Erfindungen einzelner Köpfe, sondern naturwüchsige Gebilde, und ihre Bewahrung und Gestaltung ist nicht den Kindern, auch nicht den Touristen und Handlungsreisenden anvertraut, die sie erlernen müssen, sondern den Erwachsenen und Eingeborenen, denen sie in ihrem überlieferten Zustande geläufig \corr{1891 und 1901}{sind.}{ist.} Und doch: woher jene scheinbaren Launen, die von Hause aus weder der Bequemlichkeit noch der Deutlichkeit dienen konnten? Wir stellen uns auf den Standpunkt der Agglutinationstheorie. Es gab eine Zeit der freiesten und zugleich regelmässigsten Formenbildung; jede Form \sed{hatte ihre besondere Bedeutung, die sie, wenn auch vielleicht nur in schwachen Abschattungen, von allen anderen unter}{\textbar}{\textbar}212{\textbar}{\textbar}\phantomsection\label{sp.212}\sed{schied; jede} konnte \sed{aber} nach Bedürfniss überall und in allen möglichen Verbindungen mit völlig deutlicher Wirkung angewandt \update{werden,}{werden;} die Möglichkeit der \update{Formen\-bildungen}{Formen\-bildung} war nicht durch Gesetze der Sprache, sondern nur durch die Natur der Sache begrenzt, und waren die Formative zahlreich, so mochte der zulässigen und thatsächlich vorkommenden Formungen eine unübersehbare Menge sein. Nun konnte der Gebrauch die Freiheit einschränken, gewisse Verbindungen erschienen besonders häufig, andere seltener; und was in der Sprache selten vorkommt, ist eben dadurch in seinem Bestand gefährdet, weil es weder dem Gebrauche des Redenden, noch dem Verständnisse des Hörenden geläufig ist. Am Ende verbietet es der Sprachgebrauch schlechtweg und engt so die Anwendung der Formen auf bestimmte Fälle ein.

Die Neigung zur Analogie aber bleibt, und sie ist uneingeschränkt sowohl in ihren Anlässen, als auch in ihren Wirkungen. Die Analogie kann nämlich auf Ähnlichkeit sowohl der lautlichen Form, als auch des gedanklichen Inhaltes beruhen, und in beiden Fällen kann sie auf Beides verändernd wirken. Dazu kommt als fernerer Faktor der phraseologische Gebrauch. Wir müssen die Fälle einzeln betrachten.

1. Änderung an der \retro{äusseren}{äüsseren} Form auf Grund von Lautähnlichkeit: gewunken, dialektisch statt gewinkt, nach dem Beispiele von \update{getrunken [\textit{am Zeilenende}]}{getrunken,} gesunken. \sed{Auch Erscheinungen des Accentwandels werden hierher zu rechnen sein: gewisse, wohl nicht mehr zu ermittelnde Vorbilder haben den Hauptton im Tschechischen ein für allemale auf die erste, im Polnischen und einem Theile der malaischen Sprachen auf die vorletzte Sylbe geschoben. Es ist eine \corr{1901}{rein}{reine} mechanische Gewohnheit. Von den uralaltaischen Sprachen haben die finnisch-ugrischen den Accent auf der ersten, die türkischen und mongolischen sowie das Mandschu auf der letzten Sylbe. Nun betonen aber die Mandschu mit besonderer Emphase die einsylbigen Hilfswörter und Suffixe; das Stoffwort brauchte nur den selbständigen Accent zu verlieren, um mit diesen Formenelementen zu einem Oxytonon zusammenzuwachsen, und so ist vielleicht in jenen anderen Sprachen diese Accenttheorie zur Alleinherrschaft gelangt. Denn dass ursprünglich überall die erste Sylbe den Ton trug, dafür spricht die uralaltaische Vocalharmonie, bei der der erste Vocal die weiter folgenden beeinflusst.}

\fed{{\textbar}213{\textbar}}\phantomsection\label{fp.213}

2. Änderung an der äusseren Form auf Grund der ähnlichen oder parallelen Bedeutung: Gutigkeit, statt Güte, nach dem Vorbilde von Schlechtigkeit; ferner Übertragungen grammatischer Constructionen in Fällen wie „sich erinnern, sich entsinnen“: mir eine Sache, – mich einer Sache. \sed{Hier und in anderen Fällen, wo der Sprachgebrauch zwischen accusativischen und genitivischen Regimen schwankt, mag die Analogie zweispännig gefahren sein: erstens ist der Accusativ der gewöhnliche Objectscasus; und zweitens klingt im Neuhochdeutschen der Genitiv von „es“ genau wie der Nominativ und Accusativ: „Ich bin es über\-}{\textbar}{\textbar}213{\textbar}{\textbar}\phantomsection\label{sp.213}\sed{drüssig, \corr{1901}{eingedenk“}{eingedenk} u.~s.~w. Die Genitive der Personalpronomina werden aber weit seltener gebraucht, als die Accusative. Sagt man nun: „Ich habe es vergessen, genossen“, so lag es nahe, als Parallelformen im Masculinum und Femininum „ihn, sie“ statt der ungewohnten „sein, seiner, ihrer“ zu gebrauchen: „Ich habe \so{ihn} genossen, \so{sie} \corr{1901}{vergessen“.}{vergessen.} – Besonders stark wirkt die Gleichheit der grammatischen Function auf die Formenlehre. Im Lateinischen lauten die Accusative Pluralis der Masculina und Feminina auf \textit{–s} (\textit{-os}, \textit{-as}, \textit{-es}, \textit{-us}) aus; im Neufranzösischen ist dieses \textit{s} allgemeines Pluralzeichen geworden. Die zweite Person Pluralis im Lateinischen ist bekanntlich von Hause aus ein Participium Medii: \textit{sequimini} = ἑπόμενοι. Darnach wurden nun \textit{sequebamini}, \textit{sequeremini} gebildet, obschon es natürlich nie entsprechende Participia gegeben hatte. Man sagt: „Der Wagen, die Thüre ist offen oder ist zu.“ Nun wird in attributiver Redeweise das Adverb „zu“ dem Adjectiv „offen“ gleich behandelt: „ein zuer Wagen, eine zue Thüre“. In der chinesischen Grammatik ist geradezu Grundsatz, dass Ausdrücke, die regelmässig als entgegengesetzt gebraucht werden, in allen Fällen gleiche syntaktische Behandlung erfahren.}

3. Änderung an der Bedeutung auf Grund der Lautähnlichkeit. Davon im Abschnitte vom lautsymbolischen Gefühle (S.~\update{217}{218} ff.).

4. Engere Anähnlichung der Bedeutung da, wo schon eine Bedeutungsverwandtschaft vorliegt: spanisch: \corr{1891 und 1901}{\textit{querer},}{\textit{querir},} wollen – lateinisch \textit{quaerere}, suchen. Hiervon ein Mehreres in dem Abschnitte vom Bedeutungswandel (S.~\update{221). [\textit{in den Berichtigungen, S.~502}: 225 flg.)]}{222)}.

5. Auch die Phraseologie im weitesten Sinne des Wortes, das heisst die Gesammtheit derjenigen Verbindungen, in denen die Wörter gebraucht zu werden pflegen, kann sowohl auf ihren Laut, als auch auf ihre Bedeutung Einfluss üben.

a) auf die Lautform. Dahin gehört das bekannte Beispiel, dass Slaven und Litauer in der Übereilung des Zählens den Anlaut der Neun dem der Zehn angeglichen haben. Im Englischen beruht wohl die Endung der 3. Person Singularis in \textit{s} statt des alten \textit{th} auf der Analogie von \textit{is}; „what a man is and does“ ist bequemer und klingt bedeutsamer zusammen, als \retro{„what}{„wath} he is and doeth“. Zu den wenigen ursprünglichen Verben in μι gehören ein paar der allergebräuchlichsten, vorab das Verbum substantivum. Von diesen ausgehend ist im Indisch-Iranischen und im Tschechischen die Conjugation in \textit{mi} zur alleinherrschenden geworden. Alle derartigen Vereinfachungen beruhen vermuthlich auf dem Bedürfnisse, dasjenige, was man in der Regel zu coordiniren pflegt, auch klangähnlich zu machen. Von besonderen Fällen dieser Art werde ich unter der Überschrift „Gebundene Rede“ (S.~\update{223}{225} ff.) handeln.

b) Einfluss auf die Bedeutung und Anwendung. Es ist natürlich, dass wir Ausdrücken, die uns in gewissen Verbindungen besonders geläufig sind, auch sonst zunächst die in diesen Verbindungen eigenen Bedeutungen beilegen und \sed{{\textbar}{\textbar}214{\textbar}{\textbar}}\phantomsection\label{sp.214} sie demgemäss weiter verwenden. Davon ein Mehreres in dem Abschnitte vom Bedeutungswandel (S.~\update{225}{227} ff.).

Ich muss hier einer Erscheinung gedenken, die einigermassen verwandt ist, nämlich

\fed{{\textbar}214{\textbar}}\phantomsection\label{fp.214}

\pdfbookmark[2]{II. §. 6. der falschen Congruenz.}{III.II.II.6}
\cohead{\edins{II. §. 6. der falschen Congruenz.}}
\subsection*{§. 6.}\phantomsection\label{III.II.II.6}
\subsection*{der falschen Congruenz.}
Das Congruenzbedürfniss kann, wo es vorhanden ist, auch wohl auf Irrwege gelangen und am unrechten Platze Befriedigung suchen. Dahin gehört es, wenn in deutschen Mundarten die Conjunction die Personalendungen der Conjugation annimmt: „obst Du hergehst! dassen wir kommen“ u.~s.~w. \sed{Beiläufig bemerkt, bietet das Nama-Hottentottische eine ganz ähnliche Erscheinung, Congruenz der Conjunction mit dem folgenden Subjecte. (\textsc{J. C. Wallmann}, Die Formenlehre der Namaquasprache. Berlin 1857, S.~29.) Und das Gleiche findet im Koptischen und im Somali statt.} Dahin gehört \sed{ferner} die Übertragung der Conjunctionsendung der 3. Person Pluralis auf die entsprechenden Pronomina im Italienischen: \textit{eglino}, \textit{elleno} = \textit{illi}, \textit{illae}, entsprechend \textit{hanno}, sie haben, \textit{vogliono}, sie wollen, u.~s.~w.

Denkbar ist es, obschon ich die Thatsache nicht nachzuweisen wüsste, dass auf diese Weise schliesslich die Formenelemente von dem Redetheile, dem sie ursprünglich zukommen, – in unseren Beispielen dem Verbum, – gänzlich auf einen anderen überspringen, dass etwa, wie im Annatom, das Verbum aller Temporal- und Modalformen entkleidet, und das Pronomen damit belastet würde. Man könnte dann von einer \so{Umladung der Formativa} reden. Die Tragweite einer solchen würde einleuchten: die Redetheile, damit der ganze Satzbau, damit der ganze Sprachbau, die äussere wie die innere Form, wären verschoben, verrenkt, entstellt, vielleicht auch metamorphisch verjüngt.

\begin{styleAnmerk}
Anmerkung \sed{1}. Sollte sich etwa auf diese Weise der sonderbare Activ-Instrumentalcasus des Tibetischen erklären? Der ihn kennzeichnende Laut \textit{–s} erinnert an das sinnverwandte Verbalaffix. Ferner: Wären etwa die Formendoubletten, die wir im lateinisch-griechischen \textit{ego} und im indischen \textit{aham} erhalten sehen, (und vielleicht noch eine dritte Form auf \textit{–mi} und weitere mit den entsprechenden Medialformen), ursprünglich nebeneinander, je nach der entsprechenden Verbalendung angewandt worden?
\end{styleAnmerk}

\begin{styleAnmerk}
\sed{Anmerkung 2. Wäre das Congruenzgesetz unserer Sprache ähnlich zu erklären, so möchte man fragen, ob es wirklich regen Formensinn, oder nicht trägen Schlendrian bekundet? Nur der Charakter unserer Rasse spricht für die günstigere Auffassung. Wie aber mit den Bantu?}
\end{styleAnmerk}

\begin{styleAnmerk}
\sed{Anmerkung 3. Eine verwandte Erscheinung mag es sein, wenn die Berbersprachen das feminine \textit{t} pleonastisch zugleich prä- und suffigiren, z.~B. kabylisch: \textit{agmar}, Pferd: \textit{θagmarθ}, Stute.}
\end{styleAnmerk}

\sed{{\textbar}{\textbar}215{\textbar}{\textbar}}\phantomsection\label{sp.215}

\clearpage\pdfbookmark[2]{II. §. 7. Das etymologische Bedürfniss.}{III.II.II.7}
\cohead{II. §. 7. Das etymologische Bedürfniss.}
\subsection*{§. 7.}\phantomsection\label{III.II.II.7}
\subsection*{Das etymologische Bedürfniss.}
Alles Sprechen ist ein Aufbauen. Die Bausteine liegen in unserm Innern vorräthig, und es besteht zwischen ihnen ein gewisses Gleichmass. In den einsylbigen Sprachen sind sie allzumal von gleichem Gewichte; in anderen Sprachen mögen sich die formativen Laute von den stofflichen Bestandtheilen mehr oder minder scharf unterscheiden, – überall legt die Seele jedem Theile, den sie der Rede einfügt, seinen \fed{{\textbar}215{\textbar}}\phantomsection\label{fp.215} besonderen Werth bei. Das ist ihr gewohnt, darum bequem; und dem Triebe nach Deutlichkeit entspricht es überdies.

\sed{Aber jene Bausteine werden im Laufe der Zeit abgenutzt, wohl gar zerstört. Die Laute verwandeln sich; vielleicht erfährt derselbe Laut je nach seinen Nachbarn oder nach den verschiedenen Betonungsverhältnissen sehr ungleiche Schicksale, und was sich früher gleich, scheint jetzt weit auseinanderzuliegen: französisch \textit{serment} = \textit{sacramentum} erinnert nicht mehr an \textit{sacré}, in \textit{je bénis} ist vom zweiten Theile des lateinischen \textit{benedico} nur noch ein Vocal erhalten. Oder es fristen einst selbständige Wörter nur noch in Zusammensetzungen ein dunkeles, unverstandenes Dasein: deutsch \textit{ali-} = anderer in \textit{ali-land}, Elend, eigentlich Ausland, Verbannung. Oder das gleiche Element hat in verschiedenen Verbindungen so verschiedene Bedeutungen angenommen, dass die ursprüngliche Gleichheit nicht mehr empfunden wird: bei „Leichdorn“ mag man nicht mehr an „Leiche“ denken, seit man darunter nur den todten Körper versteht; und im Worte „gleich“ (gothisch \textit{ga-leika} = σύσσωμος, leibesgemein) ahnt man schon gar nicht mehr die Zusammensetzung und den Zusammenhang; noch weniger in der Bildungssylbe -lich, ähnlich, begreiflich u.~s.~w. Was nun das Sprachgefühl nicht mehr zerlegt, das gilt ihr als Element. Der Bestand dieser Elemente, ihre Zahl, ihr Werth, ihre äussere Gestalt ist stetem Wechsel unterworfen: wir können schon jetzt die Möglichkeit ahnen, dass die Sprache dereinst dem veränderten Stoffvorrathe zuliebe auch den Bauplan der Rede umgestaltete.}

Wie sie aber aufbaut, so zerlegt sie auch wiederum. Und so kann es geschehen, dass sie herausschält, was eigentlich nur in der Verbindung lebensberechtigt ist. In buchhändlerischen Katalogen finden wir die Überschriften: Schilleriana, Lessingiana, Shakespeareana u.~s.~w. Darnach wird nun wohl die ganze Literatur über einzelne Schriftsteller mit dem Gesammtnamen „Ana“ bezeichnet. So abstrahirt man gar nicht übel von Latinismen, Gallicismen, Anglicismen, Germanismen u.~s.~w. eine Kategorie „Ismen“; – auch dieses Wort glaube ich gehört oder gelesen zu haben. Es ist denkbar, – zu belegen wüsste ich es im Augenblicke nicht, – dass hierbei ganz tolle Verirrungen geschehen, und dem Affixe zuliebe das ihm zu Grunde liegende selbständige Wort nun in \sed{{\textbar}{\textbar}216{\textbar}{\textbar}}\phantomsection\label{sp.216} neuer, umgearbeiteter Auflage wieder erscheint. Unser Suffix –keit ist bekanntlich durch Sandhi aus ursprünglichem –heit entstanden, von dem es sich sonst nicht unterscheidet. Lüde diese Sylbe zur Loslösung ein, wäre es reiner Zufall, ob man sich für Heit oder Keit entschiede.

Was ich hier als vereinzelte Fälle angeführt habe, ist bekanntlich seiner Zeit von \textsc{Westphal} zu einer förmlichen Ausschälungstheorie, der s.~g. Evolutionstheorie, ausgebeutet und der hergebrachten Agglutinationstheorie entgegengesetzt worden: der Satz sei das Ganze, seine Theile seien secundäre Abstractionen.\footnote{\sed{Vergl. übrigens schon W. v. Humboldt, Über die Verschied. u.~s.~w. S.~74–75.}} Das war mehr geistreich als \update{zutreffend [\textit{am Zeilenende}]}{zutreffend.} Allerdings lebt die spezifisch menschliche Rede erst im Satze; dieser ist ihre erste organisch eigenlebige Einheit, und seine Theile empfangen ihren besonderen Werth aus jenem Ganzen. Sie würden aber auch diesen Werth nicht empfangen, ja das Ganze würde gar nicht zu Stande kommen können, wenn dem aufbauenden Geiste nicht die allgemeine Bedeutung der Theile, wenn auch nur dunkel und unbewusst, vorgeschwebt hätte.

Dieses Gefühl vom Werthe der Einzeltheile wird nun so oft befriedigt, dass es nachgerade fordernd auftreten kann, sich nicht \update{beruhigt [\textit{am Zeilenende}]}{beruhigt,} wo die Etymologie verhüllt ist, und dann wohl umgestaltend den Thatbestand ändert, um Klarheit zu schaffen. Das ist der Fall der sogenannten \so{Volksetymologie}. Wir Deutschen haben meist leichtwiegende Formative mit stummem \textit{e}: ge–, be–, ver–, zer–, –en, –er, –es, –end, oder ganz \update{vocallos:}{vocallose:} –st, –t, –n u.~s.~w. Wuchtiger sind nur gewisse Suffixe für Abstracta: –heit, –keit, –niss, –thum u.~a. Sonst \fed{{\textbar}216{\textbar}}\phantomsection\label{fp.216} sind wir gewöhnt, schweren Sylben stofflichen Inhalt beizumessen, Spondaeen und Antibacchien in zwei Stämme zu zerlegen. Wo das nicht angeht, befremdet es uns; Wörter wie Ahorn, Hollunder, Wachholder, Eidechse, Ameise, Hornisse sind in ihrer Art Unregelmässigkeiten; indem sie etymologisch unverständlich bleiben, verletzen sie die Analogie. Dagegen wehrt sich die Sprache des Volkes so gut sie kann. Entweder zwängt sie die befremdlichen Wörter in eine Lautform, die den Gedanken an Zusammensetzung nicht mehr aufkommen lässt, nennt die Eidechse Echsel, die Ameise (Seich-)amsel, die Hornisse Hörnse, – oder sie schafft durch Umgestaltung der Laute neue Etymologien, nennt z.~B. das Ahorn Anhorn, die Ameise Armeise, – kehrt wohl auch dabei unvermerkt zum Richtigen zurück, z.~B. wenn sie Hohl-lunder sagt. Wunderliche Verwechselungen kommen aber auch dabei vor. Die Bachstelze stelzt wohl manchmal an den Bächen; ursprünglich aber hatte sie nicht davon ihren Namen, sondern von ihrem wackelnden Sterze: Wachsterze, vergl. plattdeutsch Wippsteert. Die Grasmücke baut wohl ihr Nest im Grase und frisst Mücken, hat aber sonst mit der Mücke nicht mehr gemein, als mit jedem anderen fliegenden Thiere. Besser passt ihr dänischer Name: Graa-smyge, die \sed{{\textbar}{\textbar}217{\textbar}{\textbar}}\phantomsection\label{sp.217} graue Schlüpferin, der wohl auch der ursprünglichere sein wird. \fed{Aus Biscuit wurde, mit Verwendung des einheimischen Präfixes be–, im Holländischen \textit{beschuit}, im westphälischen Dialekte Beschütchen.} So mit einheimischen Wörtern. Wie es Fremdwörtern im Volksmunde ergehen kann, dafür habe ich schon früher ein paar auffällige Beispiele gegeben. Es bedarf aber gar nicht immer der umgestaltenden Nachhülfe, – manchmal hat schon der Zufall das Nöthige gethan. Den Engländer, – ich meine den ungebildeten, der in solchen Dingen der allein massgebende ist, – muss \retro{\textit{notion}}{\textit{nation}} geradezu an das Verbum \textit{to know} erinnern, desgleichen den Holländer das Wort Scandal an \textit{schande}; er hat auch \textit{schandaal} daraus gemacht. \sed{Der Name der Nachtigall, \textit{luscinia}, \textit{lusciniola}, \textit{luscinius}, muss schon in der Lingua rustica sein anlautendes \textit{l} mit \textit{r} vertauscht, stellenweise wohl ganz eingebüsst haben: französisch \textit{rossignol}, portugiesisch \textit{rouxinol}, italienisch \textit{rusignuolo}, \textit{rosignuolo}, \textit{usignuolo}: der Anklang an \textit{ruscum}, Mäusedorn, schien besser zu gefallen, als jener an \textit{luscus}, schielend. Die Spanier aber haben \textit{ruiseñor} daraus gemacht, den Titel \textit{señor} mit dem Rufnamen \textit{Ruy}, allerdings in umgekehrter Ordnung, verbindend. Da war das etymologische Bedürfniss mit sehr Wenigem zufriedengestellt; aber das Sinn- und Sprachwidrige muss ihm doch besser behagt haben, als das ganz Unerklärliche. Statt Allotria habe ich ganz ernsthaft „Hallohdria“ sagen hören; das sollte offenbar auf lärmende Lustigkeit deuten.}

Wo nun Sprachen eines Stammes Wörter aufweisen, die Gleiches bedeuten und ähnlich lauten, deren Lautverschiedenheiten aber sich auf kein Gesetz zurückführen lassen: da sollte man auch an die Möglichkeit solcher etymologischer Verschiebungen denken. Das wären dann Adoptionen, – nur mussten die zu Adoptirenden dem Sprachbewusstsein als Waisen gelten, deren Verwandtschaft unbekannt ist. So könnte ich mir denken, dass sanskrit: \textit{jihvā}, Zunge, \textit{h\textsubring{r}d}; Herz, deutsch: Leber, griechisch θεὸς, ähnlich wie lateinisch \textit{lingua}, doch nur entfremdete Geschwister \fed{{\textbar}217{\textbar}}\phantomsection\label{fp.217} der ihnen \retro{entsprechenden}{entfrem\-deten} Wörter der anderen Sprachen wären. Ob und wie, darüber mögen die Indogermanisten entscheiden. \sed{Schon in der Urzeit scheint es für „Wurm“ zwei lautähnliche Wörter gegeben zu haben, deren eins durch lateinisch \textit{vermis}, gothisch \textit{vaurms}, griechisch, freilich mit \textit{l}: ἕλμινς, ἕλμις, deren anderes durch sanskrit \textit{k\textsubring{r}mis}, litauisch \textit{kirminis} vertreten ist.}

\sed{Die Erfahrung lehrt, dass das etymologische Bedürfniss je nach den Sprachen und Sprachstämmen von sehr verschiedener Stärke sein kann. Am Mächtigsten ist es offenbar da, wo die Rede sich am leichtesten in ihre letzten Bestandtheile zu zerlegen scheint: bei den isolirenden Sprachen und bei jenen, die man vorzugsweise agglutinirende zu nennen pflegt. Man begreift, wie es hier erneuernd wirken musste, sobald und so oft die alten Elemente sich verwischten, wie es aber wohl auch erhaltend wirken und darüber wachen mochte, dass jene Verwischung nicht zu schnell einträte. Immerhin reden jene Sprachen, deren durchsichtiger Bau uns entzückt, von ihrer Vorgeschichte vielleicht weniger} {\textbar}{\textbar}218{\textbar}{\textbar}\phantomsection\label{sp.218} \sed{deutlich, als es scheint. Wie jungen Ursprunges mögen die Anklänge sein? wie mögen sich die Etymologien verschoben haben?}

\sed{Das etymologische Gefühl haftet viel mehr an der Bedeutung, als am Laute. Der leiseste Anklang befriedigt es und kann es wach erhalten. Deutsch „gehen, gegangen, ging“, spanisch \textit{tengo}, ich habe, und \textit{tuvieron}, sie hatten, gehören auch für das naive Bewusstsein noch zusammen; die gemeinsamen Anlaute sind dünne Fäden, aber die verschiedenen Tempora derselben Verben sind starke Ketten. Erst wenn sich die Bedeutungen entähnlichen, schwindet das Gefühl für die Verwandtschaft der Wörter. Bei „Ankunft“ denkt Jedermann an „ankommen“, bei „Vernunft“ werden die Wenigsten an „vernehmen“ denken. An gleichlautenden Wörtern verschiedenen Ursprungs und Sinnes sind viele Sprachen sehr reich; aber meist bleiben die Zweideutigkeiten dem Redner wie dem Hörer unbemerkt. Das Wortspiel, das sie ausnutzt, ist ein Kunststück und wird als solches empfunden. Die Synonymik dagegen ist Jedem in Fleisch und Blut übergegangen, wird überall da geübt, wo man wissentlich das Gleiche mit verschiedenen Worten sagt, bietet der Streitsucht ein beliebtes Spielzeug, der übervollen Seele ein willkommenes Mittel, sich in immer neue Formen zu ergiessen. Jener Ordnungssinn aber, zu dem uns die Sprache erzieht, sähe am liebsten jeden Einzellaut bedeutsam und erklärlich, wie ein Dienstabzeichen. Das eine Mal schafft er, was geschichtlich keine Existenzberechtigung hat; das andere Mal meint er Zusammenhänge da zu sehen, wo geschichtlich keine vorhanden sind, – und beide Male kann er als der Lebende Recht behalten.}

\pdfbookmark[2]{II. §. 8. Das lautsymbolische Gefühl.}{III.II.II.8}
\cohead{II. §. 8. Das lautsymbolische Gefühl.}
\subsection*{§. 8.}\phantomsection\label{III.II.II.8}
\othersrc{{\textbar}26{\textbar}}

\subsection*{Das lautsymbolische Gefühl.}
Jeder Mensch verhält sich zunächst zu seiner Muttersprache naiv: sie ist ihm natürlich, und solange er es nicht erlebt hat, dass anderen Leuten eine andere Sprache ebenso natürlich ist, dünkt es ihm, als könnten die Dinge gar nicht anders heissen, als sie bei ihm daheim benannt werden. Man hat glaubhafte Anekdoten, die darauf hinauslaufen. So die von einem Bauern, der sagte: \arbup{1888 und 1891}{lsLightWine}{„Aber}{Aber} die Franzosen sind närrische Leute, – die nennen ein Pferd \arbup{1888 und 1891}{lsLightWine}{Schewall!“}{Schewall!} Oder die von dem Manne, der sich wunderte, dass drüben in Frankreich schon die kleinen Kinder französisch sprechen. Für solche naive Gemüter besteht in der That der Zusammenhang zwischen Ding und Wort φύσει, nicht θέσει; dieselben Laute erwecken immer dieselbe Vorstellung, und nun erweckt auch umgekehrt derselbe Gegenstand immer \othersrc{wieder} die nämliche Lautvorstellung. Das Ding und sein Name machen auf uns denselben Eindruck, und wo es halbwegs angeht, knüpft unser Gefühl ein Band zwischen dem Klange des Wortes und dem Inhalte der Vorstellung, die das Wort erweckt. Der Laut gilt für symbolisch; \sed{{\textbar}{\textbar}219{\textbar}{\textbar}}\phantomsection\label{sp.219} das Wort „gelind“ scheint einen gelinden Klang zu haben, „hart“ einen harten, „süss“ einen süssen, „sauer“ und „herb“ einen saueren und herben. Ob in „hüpfen, springen, schleichen, hinken, humpeln, schreien, wehen, graupeln, tönen, läuten, schnappen, zerren“ u.~s.~w. geschichtlich Schallnachahmungen zu Grunde liegen oder nicht, ist diesem Gefühle ganz \arbup{1888 und 1891}{lsLightWine}{gleichgiltig,}{gleichgültig,} – ihm dünken die Laute symbolisch. Und ähnlich wird wohl den meisten Deutschen zu \arbup{1888}{lsDarkBlue}{Muthe}{Mute} sein bei einer Menge Substantiva, z.~B. Busch, Strauch, Nuss, Splitter, Faser, Tropfen, Schnecke, Eidechse, Rabe, Eule, Fuchs, Luchs, Säge, Feile, Lappen, Runzel, Sense, Sichel, Zange. Mag unser etymologisches Wissen dazu sagen was es will, für unser Empfinden sind Wörter wie „Blitz“ und „Donner“, „rund“ und „spitz“ so innig und \arbup{1888}{lsDarkBlue}{natur\-nothwendig}{natur\-notwendig} mit ihren Bedeutungen verwachsen, dass wir uns den Fall kaum denken können, es hätten diese beiden Wortpaare ihre Bedeutung ausgetauscht. Statt Hund: Katze, statt Katze: Spatz zu sagen, würde uns nicht so arg zuwider sein, weil hier die Laute dem \arbup{1888}{lsDarkBlue}{symboli\-sirenden}{symboli\-sierenden} Gefühle weniger Anhalt bieten.

\fed{{\textbar}218{\textbar}}\phantomsection\label{fp.218}

Ähnlich geht es uns nun auch mit fremden Sprachen, in die wir uns gründlich eingelebt haben: \textit{piquer} und \textit{pique}\footnote{\fed{Eine ähnliche Vorstellung weckt im Ketschua der Name des Flohes: \textit{p’iki}. Diese Sprache mag noch durch einige weitere Beispiele beweisen, wie ähnlich bei ganz verschiedenen Völkern das lautsymbolische Gefühl angeregt wird und wohl auch wieder schöpferisch angeregt hat. \textit{Curur} bedeutet Knäuel (Krull); \textit{čuku}, zittern (zucken); \textit{mamulla}, ohne Zähne kauen (mampfen); \textit{lluč’a}, schlüpfrig (rutschen), \textit{lluncu}, lecken; \textit{maci}, spülen, begiessen (matschen); \textit{pirutu}, Flöte; \textit{puyllu}, Quaste (span. \textit{borla}); \textit{siuki}, sickern. Viele andere Wörter dieser Sprache, wie \textit{siksi}, kitzlich, muthen uns wenigstens ohne Weiteres lautsymbolisch an.}}, \arbup{1888}{lsDarkBlue}{\textit{bâiller},}{\textit{bailler},} \textit{arracher}, \textit{déchirer}, \textit{pincer}, \textit{frapper}, \textit{battre}, \textit{crier}, \textit{trembler}, \textit{terreur}, \textit{cohue}, \textit{goutte}, \textit{gouffre}, \textit{gueule} und viele andere französische Wörter \arbup{1888}{lsDarkBlue}{muthen}{muten} mich wider besseres Wissen lautsymbolisch an. Erinnert mich „Blitz“ an das plötzliche Aufleuchten, so denke ich bei \textit{foudre} an den zerstörenden Schlag, ob ich gleich weiss, dass \textit{fulgur} auch nur das Aufleuchten bedeutet. Und doch dürfte mir das Sprachgefühl der Franzosen hierin \arbup{}{lsLightWine}{recht geben;}{recht geben,\linebreak{{\tiny1888}}\linebreak rechtgeben,\linebreak{{\tiny1891}}} denn \textit{foudroyer} bedeutet längst schon niederschmettern. Dem Neulinge in einer fremden Sprache mischt sich wohl auch beim Erwachen dieses Gefühles Muttersprachliches mit ein: ἳππος gemahnt ihn an hüpfen, \textit{gladius} an die glatte Klinge, und dass italienisch \textit{caldo} warm heisst und nicht kalt, will ihm gar nicht in den \othersrc{{\textbar}27{\textbar}} Sinn. Alles dies verliert sich mit der Zeit, bei näherer Vertrautheit mit der fremden Sprache; allein jedenfalls zeigt es, wie gern der \arbup{1888}{lsDarkBlue}{Instinct}{Instinkt} da Verbindungen knüpft, wo ähnliche Klänge ähnliche Vorstellungen wecken; das Vereinzelte ist ihm unheimlich, jedem Junggesellen möchte er eine Braut zuführen.

Die Sache wird aber ernsthaft, wenn sie wirksam wird, und das ist sie meiner Meinung nach allerdings.

\sed{{\textbar}{\textbar}220{\textbar}{\textbar}}\phantomsection\label{sp.220}

Je mehr wir in einer Sprache eingelebt sind, desto inniger verknüpfen sich Laut und Sinn ihrer Wörter in unserer Seele, desto mehr sind wir \arbup{1888}{lsDarkBlue}{geneigt}{geneigt,} zwischen lautähnlichen Wörtern Begriffsverwandtschaften zu ahnen. Der Hergang ist ein rein natürlicher, psychologischer: wir finden, empfinden ohne zu suchen, unser Gefühl \arbup{1888}{lsDarkBlue}{etymo\-logisirt}{etymo\-logisiert} so zu sagen ohne sprachgeschichtliches Gewissen, wohl auch geradezu gegen \update{unsre}{unsere} bessere Einsicht, und pfropft \arbup{}{lsLightWine}{aufeinander,}{auf einander\linebreak{{\tiny1888}}\linebreak aufeinander\linebreak{{\tiny1891}}} was aus verschiedenen Wurzeln erwachsen ist. Zu den Relativwörtern „wie, wo, wann, welcher“ u.~s.~w. gesellt sich in dieser naiven Etymologie die Conjunction „weil“, die doch substantivischen Ursprunges ist, – vielleicht sogar „während“ und \fed{{\textbar}219{\textbar}}\phantomsection\label{fp.219} „wegen“. An „stehen“ reiht sich „steif, starr, Stock, Stamm, steil, stopfen, stauen, Stab, stützen, stemmen“, einerlei ob und wieviel sie mit der Wurzel \arbup{1888}{lsDarkBlue}{\textit{sthā}}{\textit{stha}} zu \arbup{}{lsLightWine}{thun}{thun \linebreak{{\tiny1888}}\linebreak thuen\linebreak{{\tiny1891}}} thun haben. Ähnlich ist es mit anderen Gruppen wie

~~~~– zucken, zupfen, zausen, zerren, Zaum;

~~~~– glatt, gleissen, glänzen, glimmen, glühen;

~~~~– klappen, klatschen, – und klaffen, Klammer;

~~~~– Schuft, Schelm, Schurke, Schubiak;

~~~~– straff, streng, stramm, \arbup{1888 und 1891}{lsLightWine}{strotzen;}{strotzen.}

~~~~– \sed{stossen, stampfen, stupfen;}

~~~~– \sed{schweben, schwanken, schwinden (schwindeln);}

~~~~– \sed{knüpfen, knoten, Knollen, knorrig, Knospe.}

So bei gleichem Anlaute, \arbup{1888}{lsDarkBlue}{allite\-rirend.}{allitte\-rierend.} Aber auch Assonanz und Reim, In- und Auslaut können \arbup{1888}{lsDarkBlue}{in’s}{ins} Spiel kommen. Da mag sich dann wohl zucken zu rucken, ducken, mucken gesellen, weinen zu greinen, – flimmern zu schimmern, glimmen, \sed{– stupsen zu schuppen (schupsen),} – schütteln zu rütteln, – Ranke zu schlank, schwanken, wanken, – lügen zu trügen, das wohl jenem zuliebe seinen Vocal verändert hat: früher hiess es triegen. Hier wird buchstäblich Lug und Trug im Spiele gewesen sein. Stemmen verknüpft sich durch \arbup{1888}{lsDarkBlue}{Alliteration}{Allitteration} mit stehen, steif u.~s.~w., durch den Reim aber mit hemmen, klemmen; sinnverwandt ist es nach beiden Richtungen. Unser Gefühl wird nicht entscheiden, ob stemmen = stehend \arbup{1888}{lsDarkBlue}{hemmen}{hemmen,} oder = hemmend stehen, oder \fed{etwa} = durch Hemmen im Stehen erhalten ist, – genug, es empfindet bei „stemmen“ den lautlichen und inhaltlichen Anklang an stehen und hemmen. In Schuft, Schurke, Hund, Lump und einigen anderen Schimpfwörtern, in dumm, stumm, stumpf, dumpf, Dunst, \arbup{1888}{lsDarkBlue}{Wust}{Wust,} hat der tiefe Vocal etwas Stimmungsvolles; dagegen lässt man sich in der Bezeichnung für den geriebenen, gewiegten, pfiffigen Spitzbuben, für den \textit{filou} und \textit{fripon}, das spitzige \textit{i} recht gern gefallen.

Es ist nun sehr leicht, aber auch sehr müssig, für jedes der obigen Beispiele deren ein Dutzend andere beizubringen, wo klangähnliche Wörter auch nicht die Spur von Bedeutungsähnlichkeit haben. Sehr müssig ist es, denn solche Wortpaare bieten eben dem Sprachgefühle nicht den Anhalt, den es \arbup{1888 und 1891}{lsLightWine}{verlangt,}{verlangt.} {\textbar}{\textbar}221{\textbar}{\textbar}\phantomsection\label{sp.221} \sed{und den es eben nur da findet, wo Beides, Laut- und Sinnähnlichkeit, zusammentrifft.} Wichtiger könnte es scheinen, dass dieses Gefühl nicht bei \arbup{1888}{lsDarkBlue}{Allen}{allen} gleich reizbar ist, und dass es nicht bei \arbup{1888}{lsDarkBlue}{Jedem}{jedem} in gleicher Weise berührt wird. \arbup{1888}{lsDarkBlue}{\textsc{Wilh. von Humboldt}}{\so{Wilh. von Humboldt}} sagt (Abh. über die Versch. d. m. Sprachbaues \arbup{1888}{lsDarkBlue}{S. 79}{p. 79}): „Sie (die symbolische Bezeichnung) wählt für die zu bezeichnenden Gegenstände Laute aus, welche \arbup{1888}{lsDarkBlue}{theils an sich, theils}{teils an sich, teils} \othersrc{{\textbar}28{\textbar}} in Vergleichung mit anderen für das Ohr einen dem des Gegen\-\fed{{\textbar}220{\textbar}}\phantomsection\label{fp.220}standes ähnlichen Eindruck hervorbringen, wie \so{stehen}, \so{stätig}, \retro{\so{starr}}{starr} den Eindruck des \arbup{1891 und 1901}{gray}{Festen,}{festen,} das Sanskritische \textit{lî}, schmelzen, auseinandergehen, den des Zerfliessenden, \so{nicht}, \so{nagen}, \so{Neid} den des fein und scharf Abschneidenden. Auf diese Weise erhalten ähnliche Eindrücke hervorbringende Gegenstände Wörter mit vorherrschend gleichen Lauten, wie \so{wehen}, \so{Wind}, \so{Wolke}, \so{wirren}, \so{Wunsch}, in welchen allen die schwankende, unruhige, vor den Sinnen undeutlich durcheinandergehende Bewegung durch das, aus dem an sich schon dumpfen und hohlen \textit{u} verhärtete \textit{w} ausgedrückt wird.“ – Ich muss bekennen, auf die \arbup{}{lsLightWine}{letzteren beiden}{beiden letzten\linebreak{{\tiny1888}}\linebreak beiden letzteren\linebreak{{\tiny1891}}} Gruppen wäre ich nicht verfallen. \so{Wehen} und \so{Wind} gehören an sich zusammen, nach der \arbup{}{lsLightWine}{echten}{echten\linebreak{{\tiny1888}}\linebreak ächten\linebreak{{\tiny1891}}} Etymologie, von der \arbup{1888}{lsDarkBlue}{\textsc{Humboldt}}{Humboldt} überhaupt nur sprechen will; \so{wünschen} hätte ich aber eher mit \so{wollen} verknüpft, und \so{wirren}, \so{wirr} etwa mit \so{wild} und \arbup{1888}{lsDarkBlue}{\so{wüst}}{\so{wüst},} und dann wieder mit \so{irren}, \so{irr}. Die Sinnverwandtschaft liegt hier besser zu Tage und dürfte darum auch der Menge eher einleuchten. Auf das Sprachgefühl der Menge aber kommt es hier wie überall zuerst \arbup{1888}{lsDarkBlue}{an. Und}{an, und} wo die Anklänge so mächtig sind, dass sich die Wenigsten ihrem Eindrucke verschliessen werden, da erlangt die falsche Etymologie Rechtskraft, es geschieht eine Art Annahme an Kindesstatt, die natürlich, gerade wie im bürgerlichen Rechte, von den Vorfahren und Seitenverwandten nicht anerkannt zu werden braucht.

Dass gerade der naive Mensch zu solchen Verknüpfungen sinn- und lautähnlicher Wörter neige, ist von \arbup{1888}{lsDarkBlue}{vorn herein}{vornherein} einleuchtend. Man denke nur an den Hergang bei der Spracherlernung der Kinder. Diesen sind ja auch die verschiedenen Formen derselben Wörter ihrer Muttersprache zunächst nur als sinn- und lautähnliche Wörter erschienen, und so musste zugleich mit der Aneignung der Muttersprache ein lebhaftes etymologisches Gefühl erwachen. Dieses Gefühl ist \arbup{1888}{lsDarkBlue}{noth\-wendiger Bestandtheil}{notwendiger Bestandteil} des \arbup{1888}{lsDarkBlue}{subjectiven}{subjektiven} Sprachgeistes, der die Rede beherrscht, ihre Richtigkeit bedingt und ihr gelegentlich neue Wege weist. Übersetzen wir dies Gefühl in ein \arbup{1888}{lsDarkBlue}{Urtheil,}{Urteil,} so besagt es: Wörter von ähnlichem Klange und ähnlicher Bedeutung sind in der Regel verwandt, also werden sie es wohl immer sein.

Die heutige Linguistik folgt mit Vorliebe der Sprache auf ihren leichtsinnigen Pfaden, dorthin wo der Sprachgeist den Damm des geschichtlich Überlieferten durchbricht, wo die Alten die Köpfe \arbup{1888}{lsDarkBlue}{schütteln,}{schütteln} und die Jungen sich tummeln. Die falschen Analogien werden aber fast \fed{{\textbar}221{\textbar}}\phantomsection\label{fp.221} nur in ihren Wirkungen \sed{{\textbar}{\textbar}222{\textbar}{\textbar}}\phantomsection\label{sp.222} auf grammatische Formen und auf lautliche Unregelmässigkeiten erkannt. Es dürfte an der Zeit sein, diesen fruchtbaren Begriff weiter auszudehnen, nämlich auf die falschen Etymologien, deren Einflüsse man vielleicht noch nicht genug in Rechnung gebracht hat. \footnote{1888 [\textit{als Fußnote 1}]: \othersrc{In einzelnen Fällen ist dies wohl mit Glück geschehen; so hat man z.~B. den Anlaut von \textit{lingua} aus der Analogie mit \textit{lingere} erklärt.}} Es handelt sich um eine Neigung des \arbup{1888}{lsDarkBlue}{Sprach\-gefühles,}{Sprach\-gefühls,} die nicht ganz ohne Wirkung bleiben kann.

Eine dieser Wirkungen suche ich im \arbup{1888}{lsDarkBlue}{\so{Bedeutungs\-wandel}}{Bedeutungs\-wandel [\textit{nicht gesperrt}]} der Wörter. Dem Sprachgefühle ist es angenehm, weil durch Gewohnheit geläufig, mit ähnlichen Klängen auch ähnliche Vorstellungen zu verbinden, ihm ist der Laut bedeutsam, die Etymologie scheint mit der Lautsymbolik vermählt. So beeinflusst in manchen Sprachen das Zusammentreffen von Ähnlichkeit des Lautes und der Bedeutung die \othersrc{{\textbar}29{\textbar}} Bildung von Zusammensetzungen und stehenden Redensarten: blitzblau, \arbup{1891 und 1901}{gray}{fuchs-, feuerroth,}{fuchsfeuer\-rot,} fix und fertig, weit und breit. Im Lateinischen: \textit{bene beateque}, \textit{felix faustum fortunatumque}, \textit{opera et oleum} u.~s.~w. Manchmal gefällt sich auch das Sprachgefühl am Gleichklange Entgegengesetzter. Da werden dann Redensarten gebildet wie: Lust und Leid, Leid und \arbup{1888}{lsDarkBlue}{Freud,}{Freud’,} Wälder und Felder, durch dick und dünn. Aber, und das ist wichtig, sie werden nur gebraucht, um die \arbup{1888 und 1891}{lsLightWine}{Gleich\-gültigkeit}{Gleich\-giltigkeit} des Gegensatzes anzudeuten. Wäre da nicht auch der Gleichklang symbolisch?

\fed{Dass solche Gegensätze auch das Lautwesen der Wörter beeinflussen können, dafür bietet italienisch \textit{greve} (neben \textit{grave}) = schwer, gegenüber \textit{leve}, \textit{lieve} = leicht, ein Beispiel.}

Von \arbup{1888}{lsDarkBlue}{hieraus}{hier aus} komme ich auf den Bedeutungswandel. Was sich in der Vorstellung gern zusammengesellt, verknüpft sich auch gern in der Rede; es entstehen Wahlverwandtschaften, die auch den Sinn der Wörter beeinflussen können. Dafür ein Beispiel: Was ich bedarf, das begehre ich. Nun hat im Englischen \textit{to want}, bedürfen, die Bedeutung von \textit{to wish} und \textit{to will} angenommen, und ebenso im Chinesischen allmählich \textit{yaó}, bedürfen, die Bedeutung von \textit{yuén}, wünschen und \textit{yuk}, wollen. Es ist wohl erlaubt anzunehmen, dass hierbei die Gleichheit der Anlaute eine Art Anziehungskraft geübt habe. Dagegen mag der Reim gewirkt \arbup{1901}{gray}{haben,}{haben} bei schwanken und wanken, die sich begrifflich viel näher stehen, als ihre beiderseitigen Verwandten: schwingen, schwenken und winken. – Wo Stoffwörter den Dienst von Formwörtern übernommen haben, da dürfte zu fragen sein, ob nicht der Laut bei der Wahl mit entscheidend gewesen sei? So etwa bei den oben angeführten \fed{{\textbar}222{\textbar}}\phantomsection\label{fp.222} „weil, wegen, während“, so auch bei den \arbup{1901}{gray}{chinesischen}{chinesichen} Conjunctionen des bedingten Nachsatzes: \textit{tsek}, \textit{tsik}, \textit{tsieú}.

\sed{Vielleicht eröffnet sich uns an dieser Stelle noch ein weiterer Ausblick. Die Laute sollen möglichst symbolisch sein; so will es das Sprachgefühl. Onomatopöie ist Nachahmung natürlicher Töne und Geräusche durch Sprachlaute. Wo nun eine Onomatopöie von Hause aus nicht vorliegt, da kann jenes Gefühl} {\textbar}{\textbar}223{\textbar}{\textbar}\phantomsection\label{sp.223} \sed{sie entweder suchen und finden, oder im Wege des Bedeutungswandels schaffen. Nun geschieht es oft, dass zwei sinngleiche Wörter verschiedener Sprachen einander auffallend gleichen, während doch entweder die beiden Sprachen überhaupt nicht nachweislich verwandt sind, oder doch die Klangähnlichkeit vor den Lautgesetzen nicht Stich hält. Wäre da nicht doch zuweilen mehr als ein blosser Zufall im Spiele? Wir hätten da ein erstes Mal eine Probe von dem, was wir später, §~14, von einer anderen Seite als Spirallauf der Sprachgeschichte kennen lernen werden: eine rückläufige Tendenz der Sprachen in der Richtung nach den symbolischen Lautwurzeln der ältesten menschlichen Rede, – freilich wieder einen uncontrollirbaren Factor. Nur ganz versuchsweise will ich einige Beispiele anführen. Wir lernten früher, §~\corr{1901}{4,}{9,} in deutsch „Zitze“, englisch \textit{teat} eine Rückkehr zu griechisch τίτθη u.~s.~w. kennen. Wie, wenn es sich ähnlich verhielte mit: Kopf – \textit{caput}, tupfen – τύπτειν, Pfote – ποὺς, ποδὸς, holländisch \textit{trecken} – \textit{trahere}, englisch \textit{to call} – καλεῖν? Wäre davon auch nur ein Theil richtig, so stünden wir mitten drin in der Zauberwelt der Wahlverwandtschaften.}

\begin{sloppypar}Auch Neubildungen von Wortstämmen mit symbolischer Abänderung einzelner Laute mögen vorkommen. So klatschen, klitschen; pfeifen, piepen, fiepen; knacken, knicken; kappen, kippen, abkuppen; bimmeln, bammeln, baumeln, bummeln; zwicken, zwacken; \sed{quieken, quäken, quaken; knirschen, knarren, knurren;} scharren, schurren; schnaufen, schniefen, schnüffeln und eine Menge \arbup{1888}{lsDarkBlue}{volks\-thümlicher}{volks\-tümlicher} Onomatopöien. Im Malaischen heisst \textit{boṅkoq}: Buckel; \textit{beṅkoq}: krumm; \textit{beṅkaq}, \textit{buṅkuq}: Geschwulst; \textit{beṅkoṅ}: verbogen; \textit{boṅkaṅ}: lahm; \textit{beṅkil}: geschwollen; \textit{boṅkol}: Buckel, Höcker; \textit{buṅkul}: knorriger Auswuchs am \arbup{1888}{lsDarkBlue}{Baume;}{Baum,} ferner \arbup{1888}{lsDarkBlue}{\textit{bonǰol}}{\textit{bonjol}:}: schwellen; \arbup{1888}{lsDarkBlue}{\textit{benǰol}:}{\textit{benjol}:} Beule; \textit{bantal}: Kissen; \arbup{1891 und 1901}{gray}{\textit{bentol}:}{\textit{bentol},} Beule; \textit{bintul}: Pustel; \textit{bintil}: Warze. \fed{Im Batta giebt es drei, sichtlich verwandte Wörter für „kriechen“: \textit{džarar} im Allgemeinen, \textit{džirir} von kleinen Wesen, \textit{džurur} von grossen oder gefürchteten Thieren gebraucht.} Auch hier scheint Höhe und Tiefe des Vocales bedeutsam für die Grösse des vorgestellten Gegenstandes, während sonst das Malaische einen organischen Vocalwechsel nicht kennt. \sed{Im Arabischen werden vollständige Doppelungen gebildet, indem man das Wort mit verändertem Anlaute wiederholt: \textit{šaiṯānun laiṯānun}, ein Erzteufel; \textit{insānun χab\'{ī}θun nab\'{ī}θun}, ein ganz ruchloser Mensch. (\textsc{Vernier}, Gramm. Arabe II, pg. 510).} Im Annamitischen werden Composita gebildet, deren zweite Hälfte den Anlaut des ersten, eigentlich bedeutsamen Wortes wiederholt, aber ein- für allemal den Auslaut \arbup{1888}{lsDarkBlue}{\textit{iek}, \textit{iet}}{\textit{ieč}} hat. \fed{(\textsc{Trüöng-Vinh-Ky}, Abrégé de gramm. Annamite p. 18.} \sed{Ders., Grammaire de la langue annamite p. 91–92}\fed{).} \sed{Schaffen wir lautspielerische Gebilde wie „Bim-baum, Piff-paff-puff“, so thuen bei den Schan in Berma, Sprachverwandten der Siamesen, die alten Weiber und Kinder nach Gutdünken und doch nach gewissen phonetischen Gesetzen ganz Ähnliches: \textit{ka} oder \textit{ka-ki} = Schleim, \textit{kak} oder \textit{kak-kik}, stottern, \textit{tuṅ}, \textit{tuṅ-taṅ}, Ring; \textit{muṅ}, \textit{muṅ-maṅ},} {\textbar}{\textbar}224{\textbar}{\textbar}\phantomsection\label{sp.224} \sed{überdecken; \textit{lit}, \textit{lit-lat}, träge sein; \textit{lum}, \textit{lum-lam} Luft; \textit{lai}, \textit{lai-l\'ī}, bekommen u.~s.~w. (\textsc{J. N. Cushing}, A Shan and English Dictionary, Rangoon 1881, p. 12 flg.). Auch im Thai selbst findet sich Ähnliches, doch mannichfacher: \textit{čaa}\textsuperscript{1} und \textit{čaa}\textsuperscript{1}-\textit{čat}, anfügen; \textit{čaay}\textsuperscript{2} und \textit{čaay}\textsuperscript{2}-\textit{čit}, winzig; \textit{čen} und \textit{čen}-\textit{čeṅ}, gewohnt; \textit{kl\'īya}\textsuperscript{2} und \textit{kl\'īya}\textsuperscript{2}-\textit{klaam}\textsuperscript{1} locken; \textit{kliṅ}\textsuperscript{2}, rollen; \textit{kliṅ}\textsuperscript{2}-\textit{klaat} entrollen; \textit{kliṅ}\textsuperscript{2}-\textit{klaak}, sich rollen; \textit{kr\'īyeb} und \textit{kr\'īyeb}{}-\textit{kr\'īyem}, trocken; \textit{hlaṅ} und \textit{hlaṅ-hlaa} vergessen; \textit{t\textsubring{r}k} und \textit{t\textsubring{r}k-traaṅ}, nachdenken u.~s.~w.\footnote{\sed{Ich habe die Consonanten des Siamesischen nicht nach ihrer jetzigen Aussprache, sondern nach dem Lautwerthe, der sich aus den indischen Lehnwörtern ergiebt, umschrieben.}} (Vgl. \textsc{Pallegoix}, Gramm., p. 67.).}\end{sloppypar}

\sed{Auf das Recht, an den Lauten seiner Sprache \corr{1901}{augenblick\-liche}{augenblick\-lichen} Stimmungen symbolisirend zu ändern, hat der Mensch nicht überall, hat er vielleicht nirgends gänzlich verzichtet. Viele jener klang- und sinnähnlicher Wurzeldoubletten, die man so gern für dialektische Varietäten erklärt, können ebensogut neben- oder nacheinander in einer und derselben Mundart entstanden sein; und wo ihr Unterschied in einem Laute mehr oder weniger besteht, da ist neben der beliebten Agglutinations- und Compositionstheorie auch die andere Möglichkeit gegeben, dass die Wurzel nach irgendeinem lautsymbolischen Vorbilde abgeändert, erweitert oder verkürzt worden sei. Ja, man muss auch auf den Fall gefasst sein, dass unorganische Doppelungsformen, wie jene des Annamitischen, des Schan und Thai, schliesslich selbständig, ohne Prototypen und statt derer auftreten.}

So scheint es mir auch sehr \othersrc{wohl} denkbar, dass zwei \arbup{1888 und 1891}{lsLightWine}{verschiedene,}{verschiedene} sinnverwandte Gruppen sich zur Schöpfung einer bastardischen Neubildung vermählen. Nehmen wir die Gruppen

{\centering ziehen, zerren, zausen, zucken

und

reissen, raufen, rupfen, raffen, \arbup{1888}{lsDarkBlue}{rucken:}{rucken,}

}

\noindent so wäre es psychologisch erklärlich, wenn das Wort rupfen der zweiten Gruppe in die erste einen Sprössling „zupfen“ verpflanzt hätte. \sed{Das Wort \so{Zaser} erscheint erst im Neuhochdeutschen und scheint in den übrigen germanischen und indogermanischen Sprachen keine Verwandten zu haben. Es wird wohl nach dem gleichbedeutenden \so{Faser} gebildet sein, das sich, wenn nicht etymologisch so doch lautsymbolisch an \so{Faden} anschliesst. Wäre etwa der Anlaut z der Gruppe \so{ziehen} u.~s.~w. entnommen: herausgezupfte Faser? Wie steht es mit: Kopf, Kuppe – Knopf, Knuppe?} Wie kommt griechisch κάπρος zu der Bedeutung Eber, gegenüber lateinischem \textit{caper}, altnordischem \textit{hafr} = Ziegenbock? Dem Eber, \arbup{1888}{lsDarkBlue}{\textit{aper}}{\textit{aper},} würde ἄπρος entsprechen, und die Klangähnlichkeit und vielleicht der weitere Anklang an κόπρος, Koth, mag den κάπρος zum Eintritte in das Schweinegeschlecht veranlasst haben. \fed{Allerdings findet sich noch bei Homer ὗς κάπρος, was darauf hindeuten könnte, dass κάπρος = Bock eine Zeit lang} \sed{{\textbar}{\textbar}225{\textbar}{\textbar}}\phantomsection\label{sp.225} \fed{das männliche Thier im Allgemeinen bezeichnet habe. Selbst dann aber war es schwerlich ein Zufall, das es schliesslich vom Eber allein gebraucht wurde.}

{\textbar}223{\textbar}\phantomsection\label{fp.223}

\fed{Unorganischer Lautwandel mag wohl auch dieser Quelle entfliessen. Französisch \textit{gras} = lateinisch \textit{crassus} dürfte den sinnverwandten \textit{gros}, \textit{grand}, \textit{grave} seinen Anlaut verdanken.}

\othersrc{{\textbar}30{\textbar}}

\arbup{1888 und 1891}{lsLightWine}{Ferner}{Endlich} wäre zu untersuchen, ob nicht die falsche Etymologie gelegentlich auch die grammatische Behandlung der Wörter beeinflusst habe, ihre Formbildung oder syntaktische Construction.

\sed{Endlich wird ein Fremdwort um so leichter volksthümlich werden, je mehr es das lautsymbolische Gefühl zu befriedigen scheint. Das slavische „pomäle“ bekanntlich weiter zu „pomadig“ verunstaltet, hat einen träg sanften Klang. Ähnlich ist es mit „dusemang“ (\textit{doucement}), das nebenbei an duselig erinnert. Auch die Namen mancher Musikinstrumente muthen uns onomatopoetisch an: Posaune, Flöte, Hoboe (\textit{haut-bois}), Cymbel, Cither, Violine, Castagnette. In anderen Fällen mag der Klang des Fremdwortes seine Bedeutung beeinflusst haben. So bei Pöbel, renitentes Frauenzimmer, fidel = lustig (?).}

\largerpage[-1]Was die Etymologie als Wurzeln bezeichnet, waren eben solche bedeutsame Laute und \arbup{1888}{lsDarkBlue}{Laut\-verbin\-dungen. Man}{Laut\-verbin\-dungen; man} darf annehmen, dass im Urzustande der Sprachen für das Gefühl der Redenden alle Wurzeln, aber auch nur die Wurzeln und ihre \arbup{1888}{lsDarkBlue}{Laut\-bestandtheile}{Laut\-bestanteile,} lautsymbolischen \arbup{1888}{lsDarkBlue}{Werth}{Wert} hatten. Im \arbup{1888}{lsDarkBlue}{Laufe}{Lauf} der Zeiten hat sich jenes Gefühl nicht verloren, sondern es hat nur andere Richtungen genommen. Das seelische \arbup{1888}{lsDarkBlue}{Bedürfniss}{Bedürfnis} ist geblieben, und es sucht Befriedigung, wo immer sie von den Thatsachen geboten wird.

\pdfbookmark[2]{II. §. 9. Gebundene Rede.}{III.II.II.9}
\cohead{II. §. 9. Gebundene Rede.}
\subsection*{§. 9.}\phantomsection\label{III.II.II.9}
\subsection*{Gebundene Rede.}
Bei längerem Sprechen müssen wir von Zeit zu Zeit Athem holen, und dies kann auf die Rede zweierlei Wirkung üben. Entweder unterbrechen wir sie und benutzen die Pause zum Einathmen, oder wir sprechen zeitweilig nicht mit \mbox{aus-,} sondern mit einströmendem Athem, wie in einem Seufzer oder schluchzend. Ersteres bildet wohl überall die Regel; aber auch die einathmende Sprechweise ist gar nicht so selten, wie man meinen sollte. Beschäftigte Leute bedienen sich ihrer oft nachlässigerweise in ihren einsylbigen Antworten. Und wo die Rede selbst keine geistige Pause zulässt, wie bei fortlaufendem Zählen, da pflegt die Erscheinung periodisch wiederzukehren. Das kann man z.~B. bei Strickerinnen, die die Maschen abzählen, beobachten.

Jenes Andere aber, das Innehalten in der Rede, die Pause, ist das Wichtigere. Denn hier wirken Körper und Seele recht sichtlich zusammen. Auch \sed{{\textbar}{\textbar}226{\textbar}{\textbar}}\phantomsection\label{sp.226} das Denken verlangt Pausen, der Gedankengang vollzieht sich in Abschnitten, die des Ausdruckes bedürfen, und da ist die negative Ausdrucksweise, die Pause, die einzig sachgemässe.

Die frisch gefüllte Lunge ist natürlich zu kräftiger Tonerzeugung geeigneter und geneigter, als die sich allmählich erschöpfende. Das ist einfach mechanisch. Die Seele aber beherrscht den Mechanismus und \fed{{\textbar}224{\textbar}}\phantomsection\label{fp.224} spart die Kräfte auf für die Kraftstellen. So ist denn der Anfang der Rede durchaus nicht immer am Stärksten betont. Wo hingegen das Reden selbst vorwiegend mechanisch geschieht, beherrscht von Gedächtniss und Gewohnheit, da wird der Hergang roh mechanisch, und eben in der \retro{mecha\-nischen}{mecha\-nichen} Gleichmässigkeit findet die Seele ihr bescheidenes Genüge. Da stellt sich ganz von selbst etwas wie gebundene Rede ein: annähernd gleiche Länge der Abschnitte, annähernd gleiche Vertheilung der Betonungen. Man beobachte nur, wie Schüler ihre Paradigmen hersagen: \textit{ménsa mensae mensae – ménsam mensa mensa}; \textit{ámo amas amat – amámus amatis amant}. Dabei fällt denn wohl noch ein zweiter (Neben-)ton auf die letzten \update{Glieder:}{Glieder;} \retro{denn}{den} der Rest der Luft wird mit Kraft hinausgestossen, um der frisch einzunehmenden Platz zu machen.

Rhythmisches Thun lieben wohl die Menschen aller Länder und Völker. Gesang, Tanz, Taktschlagen mit den Händen, mit Trommeln oder Klappern sind meines Wissens allverbreitet. Und kaum weniger verbreitet ist die Gewohnheit stehender, solenner Sprüche, es seien Sätze der Erfahrung, der Moral oder des Rechtes, Begrüssungsformeln, Gebete oder Beschwörungen. Beachtet man, wie Kinder und Ungebildete dergleichen sprechen, so wird jenes geistleibliche Bedürfniss des Menschen nach Rhythmus recht sinnfällig. Da wird die Rede singend oder plärrend, begleitet von taktmässigen Körperbewegungen, und eine geistlose Betonung schafft etwas wie ein Versmass auch da, wo von Hause aus gar keine Verse waren. Willkommen ist es, wenn in annähernd gleichen Abständen Laut- oder Sinnähnliches wiederkehrt, und so zu sagen der Text selbst die Weise zum Tanze aufspielt. Und wo er das nicht von selbst thut, da ist es menschlich, dass man ihm nachhilft, Gleichklang oder Gleichmass schafft, sei es auch auf Kosten des Sinnes oder der Sprache. So naturwüchsig scheint mir, ihren ersten Anfängen nach, die gebundene Rede mit ihren Licenzen.

Nun ist es zwar nicht Vieles, was solchergestalt zur gebundenen Form drängt, aber das Wenige kehrt häufig wieder, übt also doch in der Sprachgewohnheit eine gewisse Macht. Das Veraltete scheint noch nicht ganz todt zu sein, das Ungewöhnliche nicht schlechthin verpönt. Und wo Wörter in gewisser Reihenfolge hinter einander hergesagt zu werden pflegen, wie die Zahlwörter von den Kindern, die sie lernen müssen, da kann wohl jene urwüchsige gebundene Rede geradezu das Laut- und Formenwesen zerstören. Gewisse unregelmässige Erschei\-\fed{{\textbar}225{\textbar}}\phantomsection\label{fp.225}nungen in dem Zahlwesen der indochinesischen Sprachen \sed{{\textbar}{\textbar}227{\textbar}{\textbar}}\phantomsection\label{sp.227} möchte ich gerade hieraus erklären. Die Sache ist meines Wissens anderwärts noch nicht verfolgt worden, auch in der That schwer verfolgbar; aber als eine Möglichkeit, mit der man rechnen muss, wollte sie erwähnt sein.

Eine gewisse Verwandtschaft des hier Besprochenen mit dem lautsymbolischen Gefühle leuchtet ein, und sie wird doppelt einleuchtend, wenn wir an jene Kunstformen denken, in denen die gebundene Rede in Rhythmus und Ton die Stimmung symbolisirt, wie in den Metren der griechischen Dramen oder in den freien Klangmalereien japanischer Dichtungen. Aber auch schon die Reime, Alliterationen und Assonanzen in ständigen Redensarten sind im weiteren Sinne Formen gebundener Rede.

Doch noch weiter dürfen wir gehen. Unser Denken bewegt sich gern in Antithesen, und nach dem Baue unserer Sprachen sind Wörter von entgegengesetzter Bedeutung oft zum einen Theile einander gleich, zum anderen von einander verschieden. Nun hebt die Betonung den Gegensatz hervor, also in der Antithese die sich unterscheidenden Worttheile, gleichviel ob der regelmässige Accent auf sie fällt oder nicht. So sagen wir der allgemeinen Regel gemäss: „éingehen und áusgehen“, aber: „ausgéhen und ausfáhren“. Wo uns der Gegensatz besonders geläufig ist, da mag wohl die Accentverschiebung ständig werden, und so geschieht es z.~B., dass Leute, die sich viel mit Grammatik beschäftigen, Ímperfectum, Pérfectum, Plúsquamperfectum und die Namen der Casus ein für allemale auf der ersten Sylbe betonen. So kann jenes Gleichklangsgefühl auch da noch einseitig nachwirken, wo der Gegenklang fehlt, aus der zeitweiligen Störung kann eine dauernde werden, und unter den Mächten, die an den Veränderungen der Sprachen mitwirken, gebührt auch der gebundenen Rede ein bescheidener Platz.

\pdfbookmark[2]{II. §. 10. Bedeutungswandel, Verluste und Neuschöpfungen.}{III.II.II.10}
\cohead{II. §. 10. Bedeutungswandel, Verluste und Neuschöpfungen.}
\subsection*{§. 10.}\phantomsection\label{III.II.II.10}
\subsection*{Bedeutungswandel, Verluste und Neuschöpfungen.}
\subsection*{Einleitung.}
Die Wechsel in den Bedeutungen und Anwendungen der Wörter und grammatischen Formen werden wohl in den Wörterbüchern nach Möglichkeit verzeichnet, beziehungsweise in geschichtlichen Grammatiken verfolgt; ihre Ursachen und Gesetze aber sind meines Wissens noch nie mit Erfolg systematisch festgestellt worden. Dieser Theil der allgemeinen \fed{{\textbar}226{\textbar}}\phantomsection\label{fp.226} Sprachwissenschaft gehört noch zu den wenigst gepflegten. Und das mit gutem Grunde. Denn nirgends dürften zuverlässige thatsächliche Unterlagen schwerer zu gewinnen sein, als hier, und kaum irgendwo sonst mögen Zufall und Laune freier walten. ‚Wir schlagen auf gut Glück ein deutsches Buch aus der ersten Hälfte des vorigen Jahrhunderts \sed{{\textbar}{\textbar}228{\textbar}{\textbar}}\phantomsection\label{sp.228} auf, lesen ein paar Zeilen und stossen gleich auf so und \update{soviele}{so viele} Dinge, die unserm Sprachgebrauche zuwider sind. Es ist nicht nur der Stil, der uns befremdet, nein, es ist auch der Gebrauch vieler Wörter, ihre phraseologische und syntaktische Verknüpfung anders als in \update{unseren}{unsern} Tagen. Wenn es gut geht, mögen wir aus dem Grimm’schen Wörterbuche ersehen, wann ungefähr die Änderung eingetreten ist; über das Warum und Woher aber kann selbst ein solcher Thesaurus das zehnte Mal keine Auskunft geben. Es gilt gerade von diesen Neuerungen ganz besonders, dass sie nicht mit einem Male durchbrechen, sondern sich allmählich von einem Punkte aus verbreiten: diese Epidemien sind nicht miasmatisch, sondern contagiös.

\begin{sloppypar}\sed{Jede Sprachgeschichte muss zugleich Bedeutungsgeschichte sein; es gelte den Wörtern oder grammatischen Formen, so muss sie verfolgen, inwieweit sich deren Werth behauptet oder nach Zeiten und Orten verändert habe, sie muss auf den früheren, womöglich mit Hülfe der Etymologie, auf den ursprünglichen Werth zurückgehen. Man weiss, wie wichtig hierbei die Composita werden können. Unser „Marschall, Marschalk“ ist jetzt ein hoher Titel, bedeutete aber vormals einen Pferdeknecht. Und seltsam, gerade umgekehrt in der Werthscala haben sich die beiden Bestandtheile des Compositums bewegt: „Mähre“ schimpft man ein altes hässliches Pferd, und „Schalk“, das wir aus Luthers Bibelübersetzung im übeln Sinne kennen, ist heute eins unsrer kosenden Schimpfwörter: man weiss, dass man damit etwas wie „Gauner, Schurke“, zugleich aber auch begütigend seine Mitfreude an der kleinen Schurkerei und sein Wohlgefallen an dem „schalkhaften“ Menschen ausspricht.}\end{sloppypar}

\sed{Doch wer sich auf dies Forschungsgebiet wagt, der darf sich Glück wünschen, wenn ihm so handgreifliche Dinge begegnen. Sieht er genauer zu, so ist um ihn her Alles schwankend. Er hat es mit Bedeutungen zu thun, will Begriffe feststellen, das heisst umgrenzen. Aber die Grenzen selbst scheinen zitternd zu flimmern, wie vor einem geblendeten Auge: handelt es sich nicht um das Allerconcreteste, so wenden kaum zwei Individuen denselben Ausdruck genau im gleichen Umfange an. Nun sucht er nach dem Kern, nach dem Mittelpunkte, von dem alle jene Bedeutungen und Anwendungen auszustrahlen scheinen. Aber ist dieser bei allen Individuen, ist er auch nur bei der Mehrzahl der Nation der gleiche? Das Körperliche der Sprache ist freilich nicht minder unstät und der Eigenart des Einzelnen unterworfen; aber es ist doch von gröberer Art und leichter zu packen. Die Bedeutungsgeschichte ist recht eigentlich eine Arbeit philologischen Feinsinns und Taktes; aber eine Aufgabe der Sprachwissenschaft ist sie so gewiss, wie die Bedeutung der Lautkörper zur Sprache gehört.}

Wie dermalen die Dinge liegen, kann die Wissenschaft nichts Besseres thun, als die bisher gesammelten Erfahrungen sichten und dann versuchen, wie\-\sed{{\textbar}{\textbar}229{\textbar}{\textbar}}\phantomsection\label{sp.229}weit sich auf apriorischem Wege das Gebiet der Möglichkeiten austasten lässt. Vielleicht gelingt es ihr hierbei, eine gewisse Vollständigkeit zu erzielen; vielleicht glückt es ihr sogar, gewisse Möglichkeiten zu Wahrscheinlichkeiten zu erheben. Zunächst aber liegt es ihr ob, die Thatsachen zu überschauen und zu ordnen, mit denen sie es zu thun haben wird.

\begin{styleAnmerk}
\sed{Anmerkung,. Anregendes bieten \textsc{Kurt Bruchmann}, Psychologische Studien zur Sprachgeschichte. Leipzig 1888. \textsc{Carl Abel}, Sprachwissenschaftliche Abhandlungen. Leipzig 1885.}
\end{styleAnmerk}

\pdfbookmark[2]{II. §. 11. Classification der einschlägigen Thatsachen.}{III.II.II.11}
\cohead{II. §. 11. Classification der einschlägigen Thatsachen.}
\subsection*{§. 11.}\phantomsection\label{III.II.II.11}
\subsection*{Classification der einschlägigen Thatsachen.}
Jedes Wort, jede Wortverbindung (Redensart) und jede grammatische Form, – diese sei Wortform, Formwort oder syntaktische Construction, – mit einem Worte jeder sprachliche Ausdruck ist Vertreter eines Begriffes, das heisst eines Inbegriffes von Vorstellungen, deren Umgrenzung wir die Definition nennen. Die Bedeutung verändert sich, das heisst die Grenzen verschieben sich, werden enger, weiter, rücken wohl auch einseitig von der Stelle. Stellen wir uns diese Umgrenzungen kreisförmig vor, so können wir ihre Veränderungen durch grössere oder kleinere Kreise mit gleichen oder verschiedenen Mittelpunkten darstellen. \fed{{\textbar}227{\textbar}}\phantomsection\label{fp.227} Und allerdings ist das Symbol des Kreises passend; denn alle Bedeutungen des Ausdruckes bilden eine geschlossene Einheit und entfliessen strahlenförmig einem gemeinsamen Mittelpunkte. Diesen zu entdecken fällt manchmal schwer. Es giebt Fälle, wo er verwischt, die Grundbedeutung vergessen ist, und jene Strahlen nur noch an vereinzelten Punkten leuchten. Dann aber ist auch im Sprachbewusstsein die Einheit gelöst. Unser Wort „gerben“ hatte vormals die Bedeutung: zubereiten, fertig machen; vor Alters wurde es ganz im allgemeinen Sinne gebraucht, jetzt bekanntlich nur noch vom Leder. Bei dem Worte „gar“ = fertig gekocht, und nun vollends bei jenem „gar“ in den Redensarten „gar zu viel, gar oft, lieber gar!“ aber denkt Niemand mehr an gerben. Hier ist im Sprachbewusstsein das etymologische Band zwischen zwei verwandten Wörtern gerissen. Es ist aber auch möglich, dass ein und dasselbe Wort in der gleichen Lautform an zwei ganz verschiedenen Punkten der Vorstellungswelt auftritt und dann im Sprachbewusstsein ganz jenen Homophonen verschiedenen Ursprungs wie „sein“ = \textit{esse} und „sein“ = \textit{ejus} gleichgesetzt wird. Man denke an „Bein“ = Knochen und „Bein“ = Gehwerkzeug. Auch das Umgekehrte ist möglich: zwei grundverschiedene Homophone können im Sprachbewusstsein zu einer Art verwandtschaftlicher Gemeinschaft zusammenrücken. Uns Mittel- und Oberdeutschen geht es wohl so mit „ringen“ (niederdeutsch: wringen) und „sich ringeln“. Dort war der Hergang wie bei gewissen \sed{{\textbar}{\textbar}230{\textbar}{\textbar}}\phantomsection\label{sp.230} Infusorien, die sich durch Spaltung vermehren; hier erinnert er an das Zusammenfliessen zweier benachbarter Tropfen.

Wenn nun Ausdrücke ihre Bedeutungen und Anwendungen verändern können, so können sie natürlich auch verlieren, und dann verschwinden sie sie aus der lebenden Sprache, werden vergessen oder günstigstenfalls noch als Archaismen gekannt. Begriff und Namen mögen gleichzeitig verloren gehen, das heisst veralten; – man denke an die Lichtputze, den Falkner. Oder es tritt ein anderer Name, ein schon vorhandener oder neu geschaffener, an Stelle des alten: der Schulze wird Bürgermeister oder Ortsvorsteher, der Drost Landrath, der Frohn oder Büttel Gerichtsdiener genannt, das Retourbillet als Rückfahrkarte verdeutscht u.~s.~w. 

Zwischen dem Bedeutungswandel und der Abschaffung mitteninne, oft letztere vorbereitend, steht als eine besondere Art der Verengung die Erhöhung oder Erniedrigung im Werthe. Der Ausdruck wird nur noch im ehrenden oder im herabwürdigenden Sinne gebraucht, z.~B. englisch \fed{{\textbar}228{\textbar}}\phantomsection\label{fp.228} \textit{queen}, eigentlich = Weib, nur noch von der Königin, deutsch Kerl, eigentlich = Mann, nur noch von gemeinen Leuten. Als deutscher Königsname Karl aber hat es bei den Slaven in den Formen Kral, Krol, Korolj und bei den Magyaren als Király die allgemeine Bedeutung „König“ angenommen, – ganz wie der Eigenname Caesar mit der Zeit dem \update{imperator}{Imperator} gleichbedeutend geworden ist. In China masste sich der gewalt\-thätige Kaiser Schi-hoang-ti im dritten Jahrhunderte v.~u.~Z. ein bis dahin Jedem erlaubtes Pronomen der ersten Person zum ausschliesslichen Gebrauche an. So wurde bekanntlich das Appellativum διάβολος, der Verleumder, zu einem neuen Eigennamen des Satans.

Jeder Begriff verlangt einen Ausdruck. Woher nun die Ausdrücke für die neuen Begriffe? Stammte der neue Begriff aus dem eigenen Lande, so ist es wohl natürlich, dass er mit heimischen Sprachmitteln benannt wurde. So geschah es in der griechischen Philosophie, in der indischen Grammatik und anderwärts. Wo aber, wie im heutigen Europa, die Culturgüter internationales Gemeingut sind, da ist es gerechtfertigt, wenn auch ihre Namen auf das muttersprachliche Gepräge verzichten. In unseren Wissenschaften sind lateinische und griechische Fremdwörter oft bequemer, als ihre puristischen Nachbildungen, und wo solche gesetzlich eingeführt werden, wie in unserm Verkehrswesen, da können Jahre vergehen, ehe sie wahrhaft eingebürgert sind. Man gebraucht sie vielleicht mit Freuden, aber nicht unbefangen, schreibt: Drahtbericht und denkt im Stillen: Tele\-gramm.

Im Erfindungswesen wie in den Wissenschaften bildet die \update{europäisch-amerika\-nische}{europäisch-ameri\-canische} Welt eine Einheit, in der die staatlichen Schranken so wenig wie möglich gelten. Das Griechische, dessen Fähigkeit zur Bildung neuer Zusammensetzungen man dabei nach Gefallen ausnutzt, ist mit einem Theil seines Wort\-\sed{{\textbar}{\textbar}231{\textbar}{\textbar}}\phantomsection\label{sp.231}schatzes fast in den Kreis der gemeinverständlichen Werthzeichen, der mathematischen und chemischen gerückt; und die neuen Begriffe, die die neuen Composita ausdrücken, sind uns anfangs ebenso fremd und werden uns bald ebenso geläufig, gleichviel ob sie aus Leipzig, Paris oder New York stammen.

Anders ist es mit denjenigen Begriffen, die tief im nationalen Gemüths- und Sittenleben wurzeln, zumal mit den religiösen. Mag die neue Lehre noch so begeisterte Aufnahme finden, die Sprache scheint sich ihr nur widerwillig zu fügen. Wir Europäer haben seiner Zeit die griechischen und einige hebräische Ausdrücke des Christenthums un\-\fed{{\textbar}229{\textbar}}\phantomsection\label{fp.229}gekaut wie Pillen \update{herunter\-geschluckt}{hinunter\-geschluckt} und dann nach der Weise unserer Sprachen, oft bis zur Unkenntlichkeit, verdaut. Was ist nicht alles aus ἐπίσκοπος, aus ἐλεημοσύνη geworden! Glimpflicher sind in der Regel die Bekenner des Islâm, die Türken, Perser, Inder, Malaien, \update{Berbern,}{Berber,} mit den arabischen Fremdwörtern umgegangen. Auch in den Sprachen der indisch gebildeten Völker malaischen Stammes, der eigentlichen Malaien, der Javanen, Sundanesen u.~s.~w. ist das fremde Gut noch leicht zu erkennen. Weit freier schalteten die monosyllabisch redenden Hinterinder, die Siamesen, Barmanen, Kambodjaner, als sie sich die Sanskrit- und Paliwörter mundgerecht machten. Die Tibeter nahmen wohl mit dem Buddhismus Sanskritwörter und -Phrasen auf, gefielen sich aber bald in entsetzlichen Umbildungen dieser Fremdlinge, in einer Art zwischenzeiliger Übersetzungen, und das buddhistische Chinesisch ist vollends ein Gräuel. Man muss nun bedenken, dass hierbei die Willkür, das Geschick oder Ungeschick der Sendlinge ebensoviel Antheil hatte, wie das Fassungsvermögen und die Zungenfertigkeit der Neubekehrten.

Mehr als andere Arten des Culturerwerbes pflegen Neuerungen in Glauben und Recht das Alte zu verdrängen; sie sind um so unduldsamer, je tiefer sie in’s Leben und Denken des Volkes eingreifen. Nun wird entweder das Alte ganz \update{vergessen,}{vergessen} oder es erscheint in neuem Lichte. Die zoroastrischen Irânier machten die alten Götter zu argen Dämonen, \textit{daêva}; die germanische Göttin Hulda wurde zur unheimlichen Frau Holle. Oder es bemächtigt sich die Religion gewisser \update{Ausdrücke}{Ausdrücke,} ausschliesslich für ihre Zwecke, so im Deutschen: taufen, eigentlich = tauchen, beichten = gestehen.

Es wäre ein interessantes Schauspiel, wenn man der Bedeutungsgeschichte auch nur einer Sprache, und dann auch nur innerhalb eines Jahrhunderts folgen könnte. Ein Schauspiel fortwährenden Ringens und Drängens, nur, leider, viel zu bewegt für den beobachtenden Blick, einem wimmelnden Ameisenhaufen ähnlich.

\sed{{\textbar}{\textbar}232{\textbar}{\textbar}}\phantomsection\label{sp.232}

\pdfbookmark[2]{II. §. 12. 1. Ähnlichkeit der Vorstellungen.}{III.II.II.12.1}
\cohead{II. §. 12. 1. Ähnlichkeit der Vorstellungen.}
\subsection*{§. 12.}\phantomsection\label{III.II.II.12}
\subsection*{Die bewegenden Mächte.}
\subsection*{1. Ähnlichkeit der Vorstellungen}\phantomsection\label{III.II.II.12.1}
\largerpage[1]Wir fragen: Wie kommt ein Ausdruck zu einer neuen Bedeutung? Diese Frage wird sich in vielen Fällen mit jener anderen decken: Wie \fed{{\textbar}230{\textbar}}\phantomsection\label{fp.230} kommt eine neue Vorstellung zu einem Ausdrucke? Denn einer der gewöhnlichsten Wege ist der der Übertragung.

Wir haben es hier zunächst nur mit Bedeutungserweiterungen zu thun. Das liegt auf der Hand; denn die Verengung erfasst nicht einen neuen Vorstellungskreis, sondern zieht sich auf einen Theil des alten zurück; und die Verschiebung wird in der Regel nach Raupenart durch abwechselnde Streckung und Zusammenziehung vor sich gehen.

Also auf welche neuen Vorstellungskreise wird sich ein Ausdruck zunächst erstrecken? Die Antwort liegt fast schon in der Frage: Auf die nächstliegenden. Welches aber sind diese? Die meisten durch sprachliche Ausdrücke vertretenen Vorstellungen enthalten eine Anzahl Theilvorstellungen, Merkmale, deren Inbegriff sie eben sind. Was an einem dieser Merkmale theilnimmt, steht ihnen nahe, ist ihnen verwandt. Und es ist ihm um so näher verwandt, je zahlreicher und hervorragender die gemeinsamen Merkmale sind. Nun kann die Begriffserweiterung je nach den verschiedenen Merkmalen verschiedene Richtungen einschlagen. Dafür ein paar Beispiele.

Der \so{Flügel} ist ein seitlicher Theil: das ist auch der Flügel eines Hauses oder eines Heeres. Er ist ein bewegender Mechanismus und bewegt sich in der Luft: das gilt auch vom Flügel der Windmühle. Er hat eine dreieckartige Gestalt: darin ähnelt ihm eine Art der Claviere.

\so{Grün} und \so{rosenfarben} gelten für angenehm; darum sagt man: Einem grün (= hold) sein, und redet von rosenfarbener Laune. Grün ist aber auch die charakteristische Farbe des Frühlings, daher der unreifen Jugend; darum redet man geringschätzend von grünen Jungen.

So möge man nun auch die Vieldeutigkeit von \so{anstehen}, \so{Anstand} erklären: anstehende Früchte, Jagd auf dem Anstande; es steht mir nicht an = es behagt mir nicht; ich nehme keinen Anstand = ich trage kein Bedenken; Anstandsregeln = Regeln der guten Sitte.

\fed{Das Wort \so{ob} ist wohl ursprünglich örtliche Präposition für das höher Befindliche; man vergleiche „oben, oberer, über“. Die Wirkung erwächst aus der Ursache, diese ist der Grund, gilt als das Untere; darum \so{ob} = wegen. Der Grund ist oft zweifelhaft, will erforscht sein, und der Zweifel ist Grund des Forschens; daher \so{ob} als fragende und zweifelnde Conjunction, holländisch \textit{of} = oder.} 

\sed{Im Altchinesischen giebt es eine Anzahl lautähnlicher Pronomina der zweiten Person. \textit{ñì} (\textit{r\`{ï}}), \textit{ñî} (\textit{r\^{ï}}), \textit{ñ\`{ü}} (\textit{ž\`{ü}}), \textit{ñok} (\textit{žok}) und \textit{nài}, die auch in anderen Bedeutungen einander nahe kommen: \textit{ñî} und \textit{nài} etwa = und: \textit{ñ\^{ü}} und \textit{ñok} = wenn, wie; \textit{ñî}, \textit{ñ\`{ï}}, \textit{ñ\^{ü}}, \textit{ñok} eine Art Adverbialsuffixe: „-lich.“ Anscheinend sind sie alle einem Adverb \textit{ñì} (\textit{r\`{ï}}) verwandt, das „nahe“ bedeutet, und ihr ursprüng\-}{\textbar}{\textbar}233{\textbar}{\textbar}\phantomsection\label{sp.233}\sed{licher Werth war demonstrativ: „da.“ Diesen Sinn: „da, dabei“ hat noch \textit{ñî}; \textit{nài} dagegen bezeichnet das in der Zeitfolge Nahe: „darauf, dann.“ Der Art nach nahe ist das Ähnliche: „wie“, \textit{ñ\^{ü}}, \textit{ñok}; den Umständen nach nahe ist die Bedingung: „wenn“, \textit{ñok}, \textit{ñǚ}; und dem Redenden nahe ist der Angeredete: „Du.“ Ähnlich ist es, wenn im Lateinischen \textit{iste} = „der da“ das Possessivpronomen \textit{tuus} vertritt: \textit{ista manus} = \textit{tua manus}.}

„Hier“ ist allemal wo ich bin, und was hier ist, nenne ich \update{\so{dieses},}{dieses,} im Gegensatz zu dem und jenem, was da oder dort ist. So erklärt sich \fed{{\textbar}231{\textbar}}\phantomsection\label{fp.231} der lateinische Gebrauch von \textit{hic}, \textit{iste}, \textit{ille} = \textit{meus}, \textit{tuus}, \textit{eius}. \fed{So auch im Chinesischen das Zusammentreffen der Pronomina der zweiten Person mit Conjunctionen für örtliche und zeitliche Nähe und für Ähnlichkeit.}

Wo die übertragene Anwendung die Überhand gewinnt, liegt es nahe, dass sie vollends zur Alleinherrschaft gelangt. Für die ursprüngliche Bedeutung stellt sich dann ein neuer, natürlich auch übertragener Ausdruck ein, die Ämter werden neu besetzt, die Diener mit oder ohne Beförderung versetzt. Das lateinische \textit{caput} und unser „Haupt“ haben früher schon bildlich zur Bezeichnung des Höchsten, Wesentlichsten gedient. Je mehr diese Vorstellung in den Vordergrund trat, desto mehr verdunkelte sich jene des Körpertheiles. So geschah es, dass italienisch \textit{capo}, französisch \textit{chef} und deutsch Haupt nun ausschliesslich oder doch vorzugsweise im übertragenen Sinne gebraucht wurden, und nun dort \textit{testa}, \textit{tête}, hier das sinnverwandte Fremdwort Kopf (= Hirnschaale, Pfanne, schwedisch \textit{panna}) als Namen des Körpertheiles üblich wurden. Ähnlich mag es sich mit Bein und Knochen verhalten.

Eine besondere Art der Ähnlichkeit ist die des Lautes, und die \update{Über\-tragungen}{Über\-tragungen,} die auf ihr beruhen, können wahre Verschmelzungen bewirken. Unser Wort ja vereinigt jetzt in sich die Bedeutungen der französischen \textit{oui} und \textit{si}, \textit{si fait}. In den obersächsisch-thüringischen Mundarten antwortet man auf die affirmativ gefasste Frage mit „ja“ = \textit{oui}, auf die negative mit „jo“ = \textit{si fait} (mittelhochdeutsch \textit{joh} = doch). \sed{„Schwierig“ hiess ursprünglich: mit Schwären behaftet; eine missverständliche Etymologie hat es aber zu „schwer“ hinübergezogen und ihm die Bedeutung „beschwerlich“ gegeben.}

Auch die Laune mag sich dessen bemächtigen. Civilisten erinnert in der Aussprache Zifilisten an den Namen der Philister, den die Studenten den Bürgern geben, und mit dem wir nun weiter platte Alltagsmenschen bezeichnen. Diese Erklärung, meines Wissens von meinem Vater herstammend, dürfte noch immer die einleuchtendste sein.

Eine merkwürdige Wirkung der Klang- und Gestaltähnlichkeiten kann man in der Geschichte der Spielkarten beobachten. Die ältesten Farben waren die spanisch-italienischen: Schwerter, \textit{espadas}, \textit{spadi}, Keulen oder Stöcke, \textit{bastos}, \textit{bastoni}, Kelche, \textit{copas}, \textit{coppi}, und Goldstücke, \textit{oros}, \textit{denari}. Die \textit{espadas}, \textit{spadi} wurden nun auf germanischem Boden zu Spaten, englisch \textit{spades}, oder Schüppen und nahmen deren Form an, die dann der Lanzenspitze, \textit{pique}, oder dem Linden\-\sed{{\textbar}{\textbar}234{\textbar}{\textbar}}\phantomsection\label{sp.234}blatte (Grün, Laub der deutschen Karte) ähnlich wurde. Die \textit{bastos} haben noch in der Baste, Bastille, dem Trefle-As des Lhombrespiels, ihren Namen bewahrt. Dieser \fed{{\textbar}232{\textbar}}\phantomsection\label{fp.232} wurde in’s Englische übersetzt: \textit{clubs}. Im Niederdeutschen erscheint dafür das lautähnliche Klever = Klee, und dessen kreuzartige Gestalt verwandelte sich in der deutschen Karte in die Eichel oder Ecker. Das spanische \textit{dos} oder französische \textit{deux} wurde zum deutschen Daus, englisch \textit{deuce}, und dies dann wieder zu einem klangähnlichen Euphemismus für \textit{devil}: \textit{The deuce!}

Wie seltsam manchmal Zufälligkeiten zusammenwirken können, dafür will ich in Ermangelung eines besseren Platzes hier ein Beispiel erzählen. In Altenburg war früher das Lehrerseminar im Erdgeschosse des Gymnasialgebäudes untergebracht. Ob die Jünglinge, die dort ein- und ausgingen, Gymnasiasten oder Seminaristen, ob Beide verschieden waren, darum kümmerte sich der gemeine Mann natürlich nicht weiter, ihm galten Beide gleich, und er nannte sie Seminasiasten.

\pdfbookmark[2]{II. §. 12. 2. Composition, Construction.}{III.II.II.12.2}
\cohead{II. §. 12. 2. Composition, Construction.}
\subsection*{§. 12.}
\subsection*{2. Composition und Construction}\phantomsection\label{III.II.II.12.2}
Es scheint mir einleuchtend, dass da, wo die Sprache am Beweglichsten, die Herrschaft des Einzelnen über sie am Freiesten ist, auch der Hauptgrund der Bedeutungsveränderungen zu suchen sei. Nun mag eine Sprache noch so beschränkt und arm in ihren Formenmitteln, noch so eingeengt in Regeln sein: dem freien Schaffen lässt sie immer noch einen beträchtlichen Spielraum.

\phantomsection\label{III.II.II.12.2a}a. Grosse Freiheit herrscht natürlich im Stile der Rede, ob er ruhig oder erregt, breit oder zugespitzt, vielleicht durch treffende Vergleiche besonders anschaulich ist. Gerade Letzteres scheint wichtig; denn es spricht an und steckt an. Ein treffendes Gleichniss findet leicht Verbreitung, und wir wissen, wieviel der Mensch in Gleichnissen redet. Findet nun das Gleichniss Anklang, so gewinnt der Ausdruck eine übertragene Bedeutung, die am Ende seine Grundbedeutung ganz verdrängen kann. Die deutschen Wörter „begreifen, durchschauen, ermitteln, verstehen“ sind Beispiele hierfür. In anderen Fällen besteht die Grundbedeutung neben der übertragenen, z.~B. „abtreten, vertreten, verstellen, vorstellen, erfassen, Anstand.“

\phantomsection\label{III.II.II.12.2b}b. Auch die syntaktische und phraseologische Verknüpfung der Wörter lässt der Freiheit Raum. Ich habe die Wahl, ob ich dieses oder jenes Wort mit einem anderen durch „und“ oder „oder“ verbinden, das heisst als sinnverwandt oder entgegengesetzt behandeln will. Werden solche Verbindungen einmal üblich, so wirken die beiden verbundenen \fed{{\textbar}233{\textbar}}\phantomsection\label{fp.233} Wörter in ihren Bedeutungen anziehend oder abstossend aufeinander, das Sinnverwandte wird noch sinnähnlicher, das Entgegengesetzte noch entgegengesetzter. Das Wort „schlecht“ bedeutet ursprünglich schlicht, einfach. Ehe es zu seinem jetzigen übelen Sinne kommen konnte, musste es in Verbindungen üblich gewesen sein, in denen das \sed{{\textbar}{\textbar}235{\textbar}{\textbar}}\phantomsection\label{sp.235} Einfache als das \update{Mindergute}{Mindergut} erschien, etwa: ein schlechtes Haus oder Gewand, im Gegensatze zu einem vornehmen. Heute hat das Wort seinen ursprünglichen Sinn wohl nur noch in den Compositis schlechtweg, schlechthin, schlechterdings, und dann in der Redensart „schlecht und \update{recht“; aber}{recht“. Aber} schon in \update{dieser, [\textit{\mbox{in den} Berichtigungen, S.~502}: dieser]}{dieser} letzteren wird es kaum mehr als das verstanden, was es von Hause aus vorstellen sollte, den Meisten dünkt es wohl eher \retro{wie}{als} ein matter Gegensatz: in der Form unerfreulich, aber in der Sache richtig, – oder gar wie ein Ausdruck der Gleichgültigkeit: wie es eben kommt; kommt es gut, so ist es gut, und kommt es anders, so ist es auch gut. Der Mann, der schlecht und recht lebt, ist eben genügsam; und jener, der sein Gewerbe schlecht und recht übt, mit dem muss man es nicht zu genau nehmen.

Soviel über die coordinirenden Verbindungen, die cumulativen, wie die alternativen. Die prädicativen, adjectivischen, genitivischen, adverbialen und objektiven haben vielleicht nicht ganz soviel Tragweite. Und doch können auch sie den Bedeutungswandel beeinflussen. Zumal mit dem Anstandswerthe der Ausdrücke stehen sie in Wechselwirkung. Weil dies Prädicat oder Attribut in der und der Verbindung gebräuchlich ist, gilt es für ehrend oder herabwürdigend, – und weil es für das gilt, wird es nun weiter nur in verwandten Verbindungen gebraucht. Gewiss wirkt dabei auch die Grundbedeutung. „Schreiten“ lässt an wohlgemessene, feste Schritte denken, wie sie der Vornehmheit geziemen; „gehen“ und nun vollends „schlendern, laufen, trippeln“ haben nichts von diesem Beigeschmacke. Darum heisst es wohl: „Der König oder Minister schritt durch die Ausstellungs\-säle“; von untergeordneten Menschen aber sagt man, sie seien darin umhergegangen. Auch des grossen Napoleon wenig vornehmer Laufschritt, auch das zierliche Trippeln einer Fürstin wird in den Zeitungsberichten als Schreiten bezeichnet. Geschöpfe sind wir Alle, Menschen, Thiere und Pflanzen, die Weltkörper sogut, wie die Sandkörner. Es ist nun auch natürlich, dass man in erster Reihe bei den beseelten Wesen an ihren Schöpfer und an ihre Abhängigkeit von diesem, an ihre Hülfslosigkeit und ihr Elend dachte. Und auch das ist \fed{{\textbar}234{\textbar}}\phantomsection\label{fp.234} natürlich, dass ein Name, der Menschen und Thieren gleichmässig zukommt, nicht für achtungsvoll gilt. Ein hochmüthig vornehmer Sprachgebrauch des vorigen Jahrhunderts hat ihn aber den untersten Volksclassen und zwar, vermuthlich dem französischen \textit{créature} zuliebe, vorzugsweise ihren weiblichen Angehörigen zugetheilt. Ähnlich ist es bekanntlich mit dem Worte „Person“ geschehen. Das Wort „Mensch“ ist im Schriftdeutschen Masculinum, im Schwedischen Femininum und im Dänischen Neutrum. Wegen seiner Allgemeinheit eignet es sich mehr zur geringschätzigen, als zur ehrenden Anwendung; gegen Alter, Geschlecht und Stand verhält es sich ja gleichgültig. So erklärt es sich, dass man wohl von einem liederlichen, verkommenen, dummen Menschen, aber lieber von einem grossen, vortrefflichen Manne redet. Das grammatische Ge\-\sed{{\textbar}{\textbar}236{\textbar}{\textbar}}\phantomsection\label{sp.236}schlecht bringt es nun weiter mit sich, dass wir Deutschen nie eine Frauensperson als einen Menschen und nicht so leicht einen Mann als eine Person bezeichnen. Das Neutrum aber, das Mensch, \retro{Plural:}{Pural:} die Menscher, wird in manchen Theilen des oberdeutschen Sprachgebietes ganz unbefangen und unverfänglich für Dienstmägde und Bauerndirnen gebraucht, während es anderwärts bekanntlich geradezu als schmutzig verpönt ist. Wie willkürlich es \update{aber}{nun} trotz aller Etymologie in solchen Dingen zugehen kann, dafür ein Beispiel. Das Wort „hold“ vereinigt bekanntlich (wie „schnöde“ und „verächtlich“) zwei Bedeutungen in sich, eine active: liebevoll, in der Wendung: Einem hold sein, – und eine passive: lieblich. In letzterem Sinne hat es den Nebengeschmack des Einfachen und Jugendlichen; und so spricht man von einem holden Mädchen, wohl auch von einem holden Knaben oder von einem holden Frühlingstage, holdem Erröthen u.~s.~w. „Huldvoll“ aber sind und handeln nur sehr hohe Herrschaften, – das Wort gilt in der feierlichen Redeweise fast als ein \retro{Superlativ}{Super lativ} von gnädig.

\phantomsection\label{III.II.II.12.2c}c. Die Fähigkeit, eigentliche Composita zu bilden, besitzen bekanntlich manche \retro{Sprachen,}{Sprachen;} wie Sanskrit, in fast unumschränktem Masse, andere, wie die romanischen, die ural-altaischen und semitischen Sprachen, so gut wie gar nicht. Offenbar ist die Zusammensetzung, wo sie sich bietet, das fruchtbarste Mittel der Wortschöpfung. Für den Bedeutungswandel hingegen wird sie ungefähr denselben Werth haben, wie die entsprechende syntaktische Verknüpfung. Gleich dieser ist sie löslich, solange sie noch als Zusammensetzung empfunden wird. Nun aber drängt das Bequemlichkeitsbedürfniss zu Kürzungen, andeutungsweise nennt man \fed{{\textbar}235{\textbar}}\phantomsection\label{fp.235} nur den Theil des Ganzen, auf den es ankommt. Ist die Rede von Schiessgewehren, so sagt man nicht: der Gewehrlauf oder der Lauf des Gewehres, sondern kurzweg: der Lauf.

\phantomsection\label{III.II.II.12.2d}d. Solche Kürzungen sind überall möglich, wo sie mit der Deutlichkeit vereinbar sind, und sie können offenbar die ärgsten Sinnverschiebungen herbeiführen. Es braucht nur das Stichwort in übertragener Bedeutung gemeint zu sein und dann weiter nach anderen Ähnlichkeitsgründen auf Fremdes übertragen zu werden. Das lateinische Wort \textit{axilla} \retro{ist}{isf} ursprünglich ein Diminutivum von \textit{axis}, bedeutet also eine kleine Achse. Nun wurde es auf die Achse des Armes, das ist auf die Achsel angewandt, und diese Bedeutung wurde die ausschliessliche. Der Arm dreht sich in der Achsel, – insofern trifft der Vergleich zu. Nun die weitere Übertragung. Der Zweig sitzt am Stamme der Pflanze, wie der Arm am Rumpfe des Menschen. Darauf hin wird die Stelle, wo der Zweig ansitzt, gleichfalls die Achsel genannt, also gar nicht mehr an Achse und Drehung gedacht. \sed{– Man nimmt an, „sengen“ sei von Hause aus ein Causativum von „singen“ nach Art von setzen : sitzen, sprengen : springen u.~s.~w. (vgl. \textsc{Kluge}, Etymol. Wb. h. v.). Ist das richtig, hat man wirklich das Knistern und Prasseln brennender Körper} {\textbar}{\textbar}237{\textbar}{\textbar}\phantomsection\label{sp.237} \sed{mit unter den Begriff des Gesanges gefasst: so hat man nun weiter anzudeuten unterlassen, dass dieser Gesang durch Brennen verursacht wird. Dann könnte man sich allerdings fragen, was seltsamer sei, die Bedeutungsübertragung oder die Ellipse? Läge etwa ein grausamer Euphemismus für den Feuertod und das Geschrei der Gepeinigten zu Grunde? Im französischen Argot heisst \textit{faire chanter}: Schweigegeld erpressen. Das ist wohl auch schadenfroh gemeint: Einen mit Angst peinigen, dass er schreien möchte. Die Sprachen geschlossener gesellschaftlicher Kreise pflegen in derartigen abgekürzten Übertragungen ganz Erstaunliches zu leisten und liefern gute Typen für das, was mehr oder minder in jeder Volkssprache geschehen konnte. (Vgl. § 25: Standessprachen.)}

Es geht manchen Wörtern, wie es manchen Menschen geht: aus einem Nebenamte, wozu sie sich gebrauchen lassen, wird am Ende ihr ausschliesslicher Beruf. Dies zeigt sich am häufigsten in der Etymologie der Substantiva. \fed{Die starke Kalbe, Färse, wird kurzweg die Starke genannt, die alte Frau kurzweg die Alte.} Jedes Substantivum ist ja ursprünglich beschreibend und greift aus den vielen Merkmalen des Dinges ein besonders hervorspringendes heraus. Dieses wird aber in der Regel auch noch anderen Dingen zukommen, und nur die Gewohnheit entscheidet, ob es etwa schliesslich einem einzigen, und welchem es zugetheilt wird. Ist es erst soweit damit gekommen, dann wird bald das Attribut die Substanz vertreten, das Eigenschaftswort oder Participium wird zum Substantivum.

Dieser Vorgang, bekanntlich einer der ältesten und allgemeinsten, ist, rein logisch betrachtet, immer \so{elliptisch}, und in unsern Geschlechtssprachen lässt oft das Genus des Substantivums ahnen, welches weitere Substantivum ergänzt werden muss: \textit{aurum}, \textit{argentum} u.~s.~w. erinnern an das gemeinsame \textit{aes}. Ferner: die Ellipse ist auch psychologisch vorhanden, solange noch in den Seelen der Redenden das Bewusstsein von der prädicativen Grundbedeutung des Substantivums lebt: dann ergänzt man im Stillen das Gesammtbild des Dinges, fügt zum Gelehrten den Mann, zum Renner das Pferd. Und ähnlich ist es auch mit jenen \fed{{\textbar}236{\textbar}}\phantomsection\label{fp.236} Verben und Adjectiven von übertragener, vergeistigter Bedeutung, solange wir noch das Gleichniss als solches empfinden. Die Seele des Hörenden ergänzt dann, was zur Übertragung nöthig ist. Die des Gebildeten thut dies wortlos; dem Kinde aber muss man wohl durch erklärende Wörter zu Hülfe kommen, bis es die „Wie, Gleichsam“ u.~s.~w. selbst in das Gehörte einzuschalten lernt und schliesslich dieser Mittel nicht mehr bedarf. – Aber gerade bei den ältesten Wörtern wird die Ellipse und ihre Ergänzung auch nur in der Seele vorhanden sein. Denn für die Dinge gab es überhaupt keine anderen Namen, als diese oder jene ihrer Merkmale; die Definition, die sich bemüht, alle Merkmale der Gegenstände aufzuzählen, gehört nicht zu den Aufgaben und Bedürfnissen des naiven Geistes. Und seelische Vorgänge können ein für allemale nur im Wege des Gleichnisses, versinnlicht, zur sprachlichen Verkörperung gelangen.

\sed{{\textbar}{\textbar}238{\textbar}{\textbar}}\phantomsection\label{sp.238}

\pdfbookmark[2]{II. §. 12. 3. Entähnlichung der Bedeutungen bei Doubletten.}{III.II.II.12.3}
\cohead{\edins{II. §. 12. 3. Entähnlichung der Bedeutungen bei Doubletten.}}
\subsection*{3. Entähnlichung der Bedeutungen bei Doubletten.}\phantomsection\label{III.II.II.12.3}
Dem Bequemlichkeitstriebe muss es zuwider sein, das Gedächtniss auf die Dauer mit Überflüssigem zu belasten. Überflüssig aber ist Alles, dessen Zweck schon durch andere vorhandene Mittel genügend erfüllt wird. Giebt es also in einer Sprache zwei sich völlig deckende Synonymen, so ist das eine von ihnen unnütz und, solange es dies ist, gefährdet. Es entsteht \sed{so} eine Art Kampf um’s Dasein, zumal wenn sich fremde Eindringlinge breit machen. Und dies ist natürlich \sed{auch} der häufigste Fall, – man könnte meinen, der einzig mögliche; denn die Muttersprache duldet eben keine müssigen Doppelgänger. Fremd ist aber auch das, was nur einem anderen Dialekte entstammt, und das wird sich natürlich am Leichtesten einnisten. Es geht auch nicht immer an, solche Fremdlinge hinauszuwerfen: der Verkehr hat auch ihnen ein gewisses Freizügigkeitsrecht gesichert. Was ist nun das Ergebniss? Entweder sie verkommen von selbst, oder sie verdrängen, vernichten die einheimischen Concurrenten, – oder endlich, sie vertragen sich mit ihnen nach dem Gesetze der Arbeitstheilung.

Es ist leicht, für alles dies Beispiele aus der Sprachgeschichte zu finden. Was vom Auslande eingeführt wird, das bringt in der Regel auch seinen fremden Namen mit. Der wird dann beibehalten und bürgert sich ein, wenn sich die Sache selbst einbürgert. Die eingeführten Neuerungen können aber auch sehr geringfügig sein und doch Eingang \fed{{\textbar}237{\textbar}}\phantomsection\label{fp.237} finden und ihre Namen behalten. So war es mit den Babuschen und Pantoffeln, mit den Cravatten und manchen anderen Kleidungsstücken und Geräthen. Dann giebt es wohl Verwirrung: allerlei Haus- und Morgenschuhe werden Pantoffeln oder Babuschen, allerlei Halsbinden Cravatten genannt, – und nun hat man jene müssigen Synonymen. Vielleicht verhilft gezierte Ausländerei, Vornehmthuerei, dem Fremdworte zur Alleingeltung, wie es vormals in manchen Ländern Europas und in Japan geschehen ist. Vielleicht gelingt es einem erwachten Volksbewusstsein, sie wieder zu verdrängen, durch Bodenwüchsiges zu ersetzen. Vielleicht auch hat man ihre fremde Herkunft längst vergessen, sie werden weiter gebraucht neben \update{den}{dem} alteinheimischen, aber dann mit feinen Abschattungen. Wo man im Schriftdeutschen und in Obersachsen „sehen“ sagt, da heisst es in Thüringen „gucken“, in Österreich „schauen“, bei den Allemannen „lugen“. Jetzt haben in der Schriftsprache „lugen“ und das weniger gebräuchliche „gucken“ die Nebenbedeutung des Verstohlenen, „gucken“ dabei die des kindisch Neckenden, „schauen“ die des Erstaunten, Bewundernden oder Entsetzten, „blicken“ die des Plötzlichen, während „sehen“ das Allgemeinste ist. Der vornehme Mann speist, andre Leute essen. Wer für die männliche Jugend die Ausdrücke Buben, Knaben, Jungen gebraucht, verbindet mit jedem der drei Wörter gewisse besondere ästhetische Vorstellungen. Das Gleiche gilt von den Wörtern Frauenzimmer, Weib, Weibsperson, Weibsen, von Thier und Beest und wohl von allen wahrhaft eingebürgerten Synonymen: das Überflüssige ist nutz\-\sed{{\textbar}{\textbar}239{\textbar}{\textbar}}\phantomsection\label{sp.239}bar geworden, darum zum Weiterleben berechtigt. Das Gleiche muss aber auch von grammatischen Formen gelten, so von dem ursprünglich wohl vulgären periphrastischen Verbum des Englischen mittels des Hülfsverbums \textit{to do}.

\pdfbookmark[2]{II. §. 12. 4. Verdeutlichung und Verstärkung.}{III.II.II.12.4}
\cohead{II. §. 12. 4. Verdeutlichung und Verstärkung.}
\subsection*{4. Verdeutlichung und Verstärkung.}\phantomsection\label{III.II.II.12.4}
Das etymologische Bedürfniss ist ein Bedürfniss nach Deutlichkeit: nicht nur das Ganze der Rede, sondern auch jeder ihrer formenden Theile soll dem Redenden wie dem Hörenden verständlich sein. Er wird dies um so mehr sein, je leichter sich die Theile unterscheiden lassen, und diese werden sich um so leichter unterscheiden lassen, je regelmässiger sie in gleicher Anwendung wiederkehren.

Allein damit ist noch nicht Alles erreicht, was der Deutlichkeitstrieb verlangen kann. Jene Theile, sie seien Wortformen oder Form\-\fed{{\textbar}238{\textbar}}\phantomsection\label{fp.238}wörter haben mehr oder minder weite, abstracte Bedeutungen, man denke an unsere Casus, Tempora, Präpositionen und Conjunctionen. Verständlich mögen sie schon sein und sind es wohl auch regelmässig im Zusammenhange der Rede und durch diesen. Allein den höchsten Grad der Verständlichkeit erreicht nur das Anschauliche, und dies muss möglichst concret, sinnfällig sein.

\sed{\so{Wiederholungen}, mögen sie wörtlich, mögen sie mit anderen Worten, tautologisch, geschehen, gehören zu den gewöhnlichsten Mitteln der Verdeutlichung und Bekräftigung. Oft sind sie geradezu bildlich im sinnlichsten Sinne des Wortes: das Wiederholte soll als ein Mehrfaches gedacht werden, – als mehrmals vorhanden – Plural –, oder als mehrmals geschehend, – Iterativum, oder bei Mehrzahl der Täter einer Handlung. Man wiederholt auch wohl, was auf’s erste Mal nicht gelingt; und so kann der Sinn in’s Umgekehrte umschlagen, es kann ein nur versuchsweises, erst anfangendes oder schwächliches Thun gemeint sein. So angewandt ist die volle oder theilweise Doppelung eines der verbreitetsten Werkzeuge der Sprachformung.}

\sed{Anschaulich und eindringlich wirkt auch der Pleonasmus, der zweimal dasselbe mit verschiedenen Worten sagt: hell und klar, spritzen und sprudeln, blinken und funkeln, Kummer und Sorgen u.~s.~w. Es ist zunächst ein freies rhetorisches Mittel; die Geschichte lehrt aber, dass solche Synonymverbindungen mit der Zeit in den gewöhnlichen Hausrath der Sprachen übergehen, die einfachen Ausdrücke geradezu verdrängen können. So ist es vielfach im Neuchinesischen geschehen, das schon beinahe aus einer einsylbigen Sprache zu einer zweisylbigen geworden ist. Gelänge es, die sogenannten zweisylbigen Wurzeln mancher Sprachfamilien, z.~B. der malaischen und der semitischen, in einsylbige Elemente zu zerlegen, so würde man voraussichtlich auch auf viele solche Synonymcomposita stossen. Wo die Laute sich verschleifen, die Homo}{\textbar}{\textbar}240{\textbar}{\textbar}\phantomsection\label{sp.240}\sed{phonen sich mehren, da drängt eben das Sprachbedürfniss zu umständlicheren Ausdrucksformen.}

\sed{Oft mag eine Wiederholung zugleich eine Art Apposition darstellen. So wenn wir sagen: „im Hause drin“, – „der Lump, der!“ Und dieser Fall scheint mir besonders wichtig. In unsern Beispielen kehrten die Formwörter wieder: beide Male standen sie erst an ihrer gewöhnlichen Stelle, vor, – dann hinter dem Worte, umschlossen es also. Dies kann zur Regel werden und ist es in einem mir bekannten Falle wirklich geworden. Die Berbersprachen lassen ihre von Substantiven abgeleiteten Feminina im Singular mit dem Geschlechtszeichen \textit{t}, \textit{θ} an- und auslauten: \textit{amγar}, Greis: kabylisch \textit{θamγarθ}, turareg \textit{tamγart}, Greisin; dagegen haben andere Feminina oft nur vorgefügtes \textit{t}: \textit{tarula}, Flucht, von \textit{irwal}, \textit{rwl}, fliehen, – als wäre das Präfix das ursprünglichere. Die verwandten Sprachen verhalten sich hierin verschieden. Das altägyptische suffigirt: \textit{suten}, König, \textit{sutenīt}, Königin, – ganz wie die semitischen Sprachen, z.~B. arabisch ibn\textsuperscript{un}, Sohn: bint\textsuperscript{un}, Tochter. Ebenso das Bedscha, während z.~B. das Bilin präfigirt. Ähnlich verhält es sich mit den Personalaffixen der hamitischen Conjugationen: in einigen Sprachen werden sie vor-, in anderen nachgefügt, und die Berber haben auch hier wieder ein gemischtes System, z.~B. kabylisch lautet das Perfectum von \textit{urar}, spielen: Sing. 1. \textit{urar-eγ}, 2. \textit{θ-urar-ed}, 3 masc. \textit{i-urar}, 3 fem. \textit{th-urar}; Pl. 1. \textit{n-urar}, 2 masc. \textit{θ-urar-em}, 2 fem. \textit{θ-urar-emθ}, 3 masc. \textit{urar-en}, 3 fem. \textit{urar-ent}. Eine seltsame Wandelung hat sich im Ägyptischen vollzogen. Das hatte ursprünglich nur Suffixe; im Koptischen aber gewann eine neue, präfigirende Conjugation die Oberhand. Ein solcher Platzwechsel der Formativa setzt immer voraus, dass diese im Sprachbewusstsein noch sehr selbständig dastehen. Dass sie aber längere Zeit hindurch eine freie Stellung, bald vor bald hinter dem Stoffworte oder Wortstamme, gehabt hätten, ist weder nach der Natur der Sache noch nach der Analogie anderer Sprachen anzunehmen. Sitzt nun das Afformativ sozusagen rittlings auf seinem Träger, so steht es ihm frei, mit welchem Fusse es absteigen, ob es für die Dauer auf der linken oder auf der rechten Seite Platz nehmen will. Alte Traditionen binden es nicht mehr, und der Pleonasmus ist zur Kraftvergeudung geworden mit dem Augenblicke, wo er zur Regel wurde, und somit die Kraft, die ihn auszeichnete, eingebüsst hatte. Was ich früher, S.~214 Umladung der Formativa genannt habe, kann und wird sich am Leichtesten auf dem Umwege über den Pleonasmus vollziehen. Dass dieser Umweg unvermeidlich sei, möchte man gern behaupten, wenn nicht unsere Wissenschaft jeden vorlauten Apriorismus über kurz oder lang mit bösen Gegeninstanzen abstrafte. Genug schon, dass uns die Berber zeigen, dass es auch auf dieser Route eine wohnliche Station giebt.}

Es ist leicht einzusehen, dass der Drang nach Veranschaulichung je nach der Volksart verschiedene Stärke zeigt und verschiedene Richtungen einschlägt. \sed{{\textbar}{\textbar}241{\textbar}{\textbar}}\phantomsection\label{sp.241} Hier sind es die Dinge selbst, ihre Zahl, Grösse, Güte u.~s.~w., dort sind es die räumlichen oder zeitlichen Verhältnisse, Umstände gewisser Art, – dort wieder die Verhältnisse der Gedanken (Sätze) untereinander oder die Stimmungen und Meinungen des Redenden, die nach Ausdruck drängen; und immer wird dieser Ausdruck zunächst sehr drastisch sein. Nun findet das als wirksam Erprobte leicht Anklang. Was erst neu und selten war, wird dann alltäglich, und damit verliert es an Kraft, verblasst, rückt schliesslich wohl gar in die Reihe jener abstracten Bestandtheile der Rede, die es hatte verbessernd und verstärkend ergänzen sollen, und die es am Ende wohl gar verdrängte und ersetzte. Es ist wie im Staatsdienste: es wird angestellt, befördert, auf Wartegeld gesetzt, schliesslich wohl ganz pensionirt; und draussen vor den Pforten harrt eine Schaar Bewerber. Uns Deutschen ist längst der Instrumentalcasus abhanden gekommen. Zwischen dem Subjecte und dem Objecte der Handlung steht das Werkzeug: der Weg vom Handelnden zum Behandelten führt \so{durch} dieses, daher das instrumentale „durch“. Das Werkzeug steht zwischen jenen Beiden in der Mitte, vermittelt zwischen ihnen: daher die Präpositionen \so{mit}, \so{mittels}, \so{vermittelst}. Nun ist \so{durch} ursprünglich örtlich; und \so{mit} hat auch comitative Bedeutung angenommen, weil das Werkzeug sich dem Thäter helfend zugesellt. Darum sind beide Wörter nicht mehr anschaulich genug, und neue Ausdrücke werden eingeführt, um die Art der Instrumentalität näher zu bezeichnen: \so{kraft}, \so{schuld}, \so{dank}, \so{laut}, \so{besage}, \so{ausweislich}, \so{vermöge}, \so{anlässlich}, \so{gemäss}, \so{zufolge} u.~s.~w. Die sind anschaulich, weil ihre Etymologie sofort einleuchtet und noch nicht durch einen erweiternden Sprachgebrauch im Bewusstsein getrübt ist. Wo die Grundbedeutung sich verdunkelt, da liegt es nahe, dass völlige Verschiebungen eintreten, und diese werden wohl meist in vergeistigender Richtung geschehen, z.~B. von der Zeit auf die Ursache, nach dem Satze: \fed{{\textbar}239{\textbar}}\phantomsection\label{fp.239} Post hoc, ergo propter hoc: französisch \textit{puisque} – lateinisch \textit{postquam}; – oder nach den gleichzeitigen Nebenumständen: deutsch \so{weil}, das nur noch in Mundarten die Bedeutung „während“ behalten hat.

Was von den Formwörtern gilt, das gilt, soweit wir die Geschichte verfolgen können, auch von den Wortformen. Wo deren neue geschaffen wurden, da waren sie periphrastisch (umschreibend), frischere neue Farben deckten die verblichenen alten. Die Geschichte unseres Sprachstammes bietet dafür Beispiele in reicher Auswahl: die periphrastischen Futura des Sanskrit, des Lateinischen (\textit{ama-bo} = lieben werde-ich, \textit{–fuo}), der neuromanischen Sprachen, das germanische Imperfectum der schwachen Verba (gotisch \textit{habai-da} = haben that-ich), das nordische Medio-Passivum auf \textit{–st}, älter \textit{sk} = \textit{sik}, sich, endlich die armenischen und neuindischen Casusformen (vergl. über Letztere \textsc{Hoernle}, A Comparative Grammar of the Gaudian Languages, pg. 224 flg.).

Nun ist bei alledem zweierlei möglich: entweder das Alte wird durch das \sed{{\textbar}{\textbar}242{\textbar}{\textbar}}\phantomsection\label{sp.242} Neue bis zur Spurlosigkeit verdrängt, oder es führt daneben noch ein mehr oder minder verkümmertes Dasein, – rückt auf den Altentheil. Im ersteren Falle tritt ein Zustand relativer Formlosigkeit ein. Sage ich z.~B.: „laut Bericht vom ...“, so tragen die beiden ersten Wörter keinerlei Beziehungszeichen. So werden in den neuromanischen Sprachen (mit Ausnahme des Rumänischen) die Casuspräpositionen \textit{de} und \textit{ad}, im Italienischen zudem \textit{da} = \textit{de-ab}, gleich allen übrigen dem nach Casus unveränderlichen Substantivum vorgefügt. Ähnliches gilt von \textit{ama-bo}, \textit{habe-bo}, \textit{ama-bam}, \textit{habe-bam}, von \textit{habaida} und vermuthlich auch von jenen noch älteren Formen der Conjugation und Declination, deren Etymologien wir nicht mehr kennen. Dagegen steht \textit{je dirai} vermöge seines Infinitivzeichens (\textit{dire-ai}) auf der Stufe der sogenannten uneigentlichen Composita nach Art von \textit{paterfamilias}, \textit{patrisfamilias}. Ähnlich das altnordische Passivum: (\textit{ek}, \textit{thú}, \textit{hann}) \textit{kallast}, (\textit{ver}) \textit{kollumst}, von \textit{kalla}, rufen. Dabei mag es denn geschehen, dass gerade in der Umkapselung die alten Formen sich lautlich am besten erhalten, wenn auch wie Mumien, ohne eigenes Leben, der Gegenwart entfremdet.

Wir sprachen bisher nur von grammatischen Formen im weitesten Sinne des Wortes, weil hier die Wandelungen am Auffälligsten sind. Aber auch der Sprachstoff kann durch verdeutlichende und verstärkende Beigaben verändert werden. Sprach der Lateiner nicht mehr von Augen, sondern von Äuglein, \textit{oculis}, so reden Italiener, Spanier und Franzosen \fed{{\textbar}240{\textbar}}\phantomsection\label{fp.240} nicht mehr von Ohren, sondern von Öhrchen, \textit{orecchie}, \textit{orejas}, \textit{oreilles}. Im Neuchinesischen wird der Tiger (\textit{hù}) und die Ratte (\textit{šǜ}), der alte Tiger, die alte Ratte, \textit{laò-hù}, \textit{laò-šǜ}, genannt. Hier ist, dank jenem Deutlichkeitstriebe und schuld des Lautverschliffes, die Zweisylbigkeit der Ausdrücke fast grundsätzlich an Stelle der früheren Einsylbigkeit getreten. Bald werden Sylben verdoppelt, bald zwei Sinnverwandte, einander verstärkend und erläuternd, vereinigt: \textit{mîng-pek}, licht und weiss = klar, – bald treten an die Substantiva oder Verba, scheinbar diese classificirend, gewisse Appositionen, wie \textit{t’eû}, Kopf, an die Namen runder Gegenstände: \textit{žit-t’eû}, Sonnenkopf = Sonne; – bald endlich: wie beim Tiger und der Ratte, werden gewisse beschreibende Attribute beigefügt. Die Verdeutlichung aber, die durch solche und andere Mittel erzielt wird, ist nicht nur materiell, sondern zugleich auch formell: konnte im Altchinesischen das einsylbige Wort jetzt substantivisch, jetzt adjectivisch, verbal oder adverbial gebraucht werden, so sind jene Zweisylbler in Rücksicht auf den Redetheil bestimmt. Es hat viel Einleuchtendes, die zweisylbigen Wurzeln oder Wortstämme der malaischen und semitischen Sprachen auf ähnliche Vorgänge zurückzuführen; \textsc{Pott} freilich hat mit seinen Versuchen, indogermanische Wurzeln zu zerlegen, noch wenig Anklang gefunden.

Man wird nicht fehlgehen, wenn man dem Triebe nach Verdeutlichung und Verstärkung des Ausdruckes einen Hauptantheil an den Wandelungen der Sprache zumisst, und zwar in der doppelten Hinsicht auf die äussere Gestalt und auf den \sed{{\textbar}{\textbar}243{\textbar}{\textbar}}\phantomsection\label{sp.243} Inhaltswerth. Die Neigung ist ja allverbreitet, und zumal Überschwänglichkeiten finden leicht Anklang. Wir Deutschen ersetzen oft recht zur Unzeit das Adverb „sehr“ durch „riesig, ungeheuer, schrecklich, fürchterlich“ u.~s.~w., freuen uns erschrecklich und finden einen Menschen riesig nett. Gerade die vornehmere Gesellschaft geberdet sich in ihrer Umgangssprache bedenklich nervös, vergeht vor Sehnsucht, stirbt vor Langerweile, amüsirt sich rasend, ist wüthend, wenn ihr etwas zuwiderläuft, und behauptet zu fliegen, wenn sie eine beschleunigte Gangart annimmt. Das sind Gallicismen, die vorläufig noch unser gesunder Stil ablehnt. Im Französischen aber sind solche Ausdrücke geradezu geboten und natürlich durch den gemeinen Gebrauch entwerthet und entkräftigt. Hier hat ein wahrer Bedeutungswandel platzgegriffen: bei \textit{enchanté}, \textit{charmé}, \textit{désolé}, \textit{adorer} u.~s.~w. dürfte kaum das lebhafte Temperament des Franzosen noch den vollen ursprüng\-\fed{{\textbar}241{\textbar}}\phantomsection\label{fp.241}lichen Sinn empfinden; – bei \textit{gêne} denkt jedenfalls Niemand mehr an Hölle, was es ursprünglich bedeutete.

Scheinbar einen schwierigeren Stand als jene Kraftausdrücke haben die bloss verdeutlichenden Zusätze; denn sie sind weitläufig, darum nicht bequem. Dafür kommt ihnen aber ein Anderes zu Statten: der Lautverschliff, dem mehr oder weniger, schneller oder langsamer, jede Sprache unterliegt, und der mit der Zeit, mit Überhandnehmen der Homophonen, zu Zweideutigkeiten führen würde. Und dann: wie oft ist das scheinbar Weitläufige doch für den Erfolg das Einfachste und Kürzeste, wie kann ein kleines Einschiebsel in der Rede umständliche Auseinandersetzungen ersparen. Hätte der Lateiner nicht neben \textit{facere} noch \textit{perficere}, \textit{efficere}, \textit{conficere}, so müsste er sich entweder mit einem sehr unbestimmten Ausdrucke begnügen, oder mit weitläufigen Umschreibungen behelfen. Was Wunder, dass er mit der Zeit die Composita dem Verbum simplex vorzog? Die classische Sprache der Homophonen ist die chinesische in ihrem jetzigen Zustande. Die Neigung aber, die Rede durch verdeutlichende Zusätze zu beleben, äussert sich schon in den ältesten, fast viertausendjährigen Schriftdenkmälern, zu einer Zeit, wo sicher die Homophonen noch selten, die einsylbigen Wörter in der Regel für sich schon genügend deutlich waren. Diese Neigung nahm je länger je mehr überhand, vom Lautverfalle begünstigt, vielleicht ihrerseits ihn begünstigend. Denn es ist, als wahrten die Sprachen in dem Zeit- und Kraftaufwande, den sie dem Ausdrucke eines Gedankens widmen, ein gewisses Mass, wobei Verweichlichung der Laute als Entschädigung für die Vermehrung der Sylben gelten \update{dürfte.}{dürfte,} \sed{oder wohl auch umgekehrt eine Ersparniss an Sylben durch zungenbrecherische Consonantenhäufungen erkauft wird, sodass Mund und Lungen einander in der Arbeit ablösen.}

Vielleicht auch, – denn hier sind wir nun wieder auf’s Speculiren angewiesen, – eröffnet sich uns an dieser Stelle ein weiterer Ausblick, in unermessbare Fernen der Vergangenheit und Zukunft. Doch \update{davon [\textit{\mbox{in den} Berichti\-gungen, S.~502}: davon,]}{davon,} von der stätigen Umlaufsbewegung der Sprachen, wollen wir erst später reden.

\sed{{\textbar}{\textbar}244{\textbar}{\textbar}}\phantomsection\label{sp.244}

\pdfbookmark[2]{II. §. 12. 5. Ironie und rhetorische Frage.}{III.II.II.12.5}
\cohead{\edins{II. §. 12. 5. Ironie und rhetorische Frage.}}
\subsection*{5. Ironie und rhetorische Frage.}\phantomsection\label{III.II.II.12.5}
Ironie und rhetorische Frage sind, streng genommen, Redeformen, die auf Verstellung beruhen. Die Ironie stellt sich, als dächte der Redner das Falsche, das Entgegengesetzte; die rhetorische Frage thut, als wüsste er nicht was er weiss, und was klar zu Tage liegt. Beide überlassen es dem Gegner, sich selbst das Richtige zu sagen, und Beide sind wohl allerwärts als rhetorische Mittel beliebt.

\fed{{\textbar}242{\textbar}}\phantomsection\label{fp.242}

Es ist nicht nur denkbar, sondern sogar nachweisbar, dass diese Arten der Rede auf Sinn und Anwendung der Wörter dauernden Einfluss üben und in einzelnen ihrer gebräuchlichsten Äusserungen die Phraseologie und Grammatik der Sprachen umgestalten. Eine Schöne war noch vor hundert Jahren soviel, wie ein schönes Frauenzimmer. Sagt man heutzutage von Einer, sie sei eine Schöne, so enthält dies einen Tadel ihres Charakters. Man braucht nur: „ein guter Mensch“ mit besonderer Dehnung des \textit{u} zu sagen, um einen dummen Menschen zu bezeichnen. In früherer Zeit, als man die Strafen hartherziger betrachtete, als jetzt, hatte die Schadenfreude ihren reichlichen Antheil am ironischen Sprachgebrauche. Zum Strange Verurtheilte nannte man in den Niederlanden „Halsleidende“, Schläge wurden als Geldstücke (Batzen, Schillinge) „aufgezählt“. Es waren grausame Euphemismen. So wurde der mörderische Trank zart als Gabe, „Gift“, oder einfach als Trank, potio, poison, die Kugel des Schützen als blaue Bohne, die Folter als scharfe Frage bezeichnet. \update{Corriger la fortune}{„Corriger la fortune“} heisst „falsch spielen“, und \retro{die es}{dies} thuen, nennt man in Paris Griechen, \update{des Grecs.}{„des Grecs“.} \textsc{Carl Abel} geht gewiss zu weit, wenn er den „Gegensinn der Urwörter“ als Regel hinstellt, und dem Lautwesen gegenüber verfährt er willkürlicher, als es eine beweiskräftige Forschung verträgt. Von Hause aus können die Wörter nicht wohl These und Antithese in sich vereinigt haben. Die Sprache wäre sonst entweder fortwährender mimischer Nachhülfe bedürftig, oder ganz unverständlich, mithin untauglich gewesen. In der Regel trägt das Wort nicht den Gegensinn in sich, sondern er wird für den Augenblick hineingelegt, die eigentliche, ursprüngliche Bedeutung aber recht wohl als solche im Gedächtnisse behalten. Nur ausnahmsweise wird auf die Dauer Sinn und Gegensinn demselben Lautkörper als gleichberechtigt anhaften oder der Gegensinn schliesslich die ursprüngliche Bedeutung ganz verdrängen. So ist der Ausdruck „sauberes Bürschchen“ ungefähr gleichbedeutend geworden mit Taugenichts.

Die rhetorische Frage hat sich in \update{manchen}{gewissen} Sprachen besondere Formen geschaffen; so im Lateinischen \textit{num}, \textit{nonne}, \textit{quidni}, im Chinesischen \update{\textit{k’ì}.}{\textit{k’ì}} \sed{= wäre denn...? und \textit{hop} (zusammengezogen aus \textit{hô-put}, was ... nicht?) = aufforderndem: warum willst du nicht ...?} Vielleicht ist es ihr zuzuschreiben, dass in manchen Sprachfamilien, z.~B. in der semitischen, die Frag- und Verneinungswörter lautverwandt scheinen. \sed{So auch im Koreanischen: \textit{mu-}, \textit{mu’es} = was? \textit{mos} = nicht.}

\sed{{\textbar}{\textbar}245{\textbar}{\textbar}}\phantomsection\label{sp.245} \fed{{\textbar}243{\textbar}}\phantomsection\label{fp.243}

\pdfbookmark[2]{II. §. 12. 6. Sitte und Satzung.}{III.II.II.12.6}
\cohead{II. §. 12. 6. Sitte und Satzung.}
\subsection*{6. Sitte und Satzung.}\phantomsection\label{III.II.II.12.6}
Sitte und die Eitelkeit, für etwas Anderes gelten zu wollen, als man ist, mögen wohl auch die Lautbildung beeinflussen. Es wird dann eine gezierte Aussprache angenommen, die bald zur zweiten Natur und dann dem folgenden Geschlechte völlig heimisch wird. Der allmähliche Verderb der Volksmundarten ist zum Theile mit hierauf zurückzuführen. Wir werden hernach sehen, wie im Verkehre ganz unwillkürlich die Sprache oder Mundart des Einen von jener des Anderen beeinflusst wird; hier galt es nur darauf hinzuweisen, dass das Gleiche auch mit Wissen und Willen geschehen und dadurch natürlich beschleunigt und verstärkt werden kann. 

\largerpage[1]Die Hauptwirkung von Sitte und Satzung glaube ich wo anders zu finden. Sie äussert sich meiner Meinung nach im Gebrauche, in der Bedeutung, der Abschaffung \update{alter}{alter,} oder der Einführung neuer Ausdrücke.

Am Handgreiflichsten liegen die Dinge da, wo ausdrückliche Ge- oder Verbote den Sprachgebrauch regeln: in Recht, Religion und Cultus mit ihren vorgeschriebenen feierlichen Formeln und technischen Bezeichnungen. Da hat man Acht, dass man das richtige Wort zur rechten Zeit gebrauche, und dass man den verhängnissvollen Ausspruch nicht leichthin thue. Bei den Römern knüpften sich an die Frage und Antwort: „Spondesne ...? Spondeo“ die strengen Rechtswirkungen der Stipulation. Andere sinnverwandte Erklärungen: Promittisne? Promitto; Dabisne? Dabo u.~dgl. erlangten erst mit der Zeit annähernd gleiche Kraft. Das Wort „taufen“ wird wohl noch halb scherzhaft in seinem ursprünglichen profanen oder einem verwandten Sinne gebraucht: den Wein taufen = ihn verwässern. Weil aber mit der kirchlichen Taufe die Namengebung verbunden ist, so wird das Wort auch ganz ohne Arg in der Bedeutung „Namen geben“ angewandt, und man sagt: Der Erfinder hat sein Werk so und so getauft. – Mit mehr Scheu behandeln wir das Wort „Abendmahl“, das jetzt ausschliesslich das christliche Sacrament bezeichnet. Mittagsmahl, Früh- und Nachtmahl sind profane Wörter. Aber es ist mit dem Worte Mahl wie mit der griechischen Wurzel ϝιδ (ἰδεῖν, οἶδα): sie gehen nicht in gleicher Bedeutung durch alle Tempora, – man muss Abendmahlzeit oder Abendessen sagen, wie man ἑώρακα sagen muss.

Eine religiöse Scheu verbietet es auch den Polynesiern, gewisse \fed{{\textbar}244{\textbar}}\phantomsection\label{fp.244} Wörter auszusprechen, Namen hervorragender lebender oder verstorbener Personen. Es gehört dies in das vielbesprochene \update{Tabuwesen.}{Tabuwesen,} \sed{das unter anderen Namen auch sonst in der ostindischen Inselwelt verbreitet ist: als Pali bei den Dajaks von Borneo, als Posan in der Minahasa von Celebes, als Fady (lautgesetzlich = Pali) bei den Madegassen, als Saali bei den Galela und anderen Völkern Halmaheras u.~s.~w. (Vgl. \textsc{H. Kern} in Bijdr. v. de Taal-, Land- en Volkenk. van N. J.} {\textbar}{\textbar}246{\textbar}{\textbar}\phantomsection\label{sp.246} \sed{V, \textsc{viii}, S.~120–128)}. Dass man verstorbene Angehörige nicht bei ihren Namen nennt, jedes Wort vermeidet, das an diese Namen erinnert, kommt wohl auch sonst vor. Natürlich muss dann aber für den verpönten Ausdruck ein Ersatz gesucht oder geschaffen werden.

Weit verbreitet ist die Sitte, den Ausdruck je nach dem Range des Redenden und Angeredeten zu wählen. So in unserm amtlichen Geschäftsstile. Da wird nach Oben angezeigt, berichtet, gemeldet, und dies geschieht pflichtschuldigst, gehorsamst, ganz gehorsamst, ehrerbietigst, unterthänigst, allerunterthänigst.\linebreak Gleichstehenden wird stattdessen ergebenst mitgetheilt, und Untergebene werden benachrichtigt, verständigt, oder es wird ihnen eröffnet, zu wissen gethan, kundgegeben u.~s.~w. Die Javanen unterscheiden zwischen einer vornehm-respect\-vollen Sprache, \textit{kråmå}, einer mittleren, \textit{mådyå}, und einer gemeinen, \textit{ngoko}. Ähnlich die Malaien, die Tibetaner und viele Völker Hinterindiens und des ostindischen Archipels. \sed{Oft mögen es Fremdwörter sein, die für ehrerbietiger gelten, als die einheimischen Ausdrücke, zuweilen sind es wohl auch zierliche Umschreibungen; meist aber ist das für gemein geltende Wort das ursprünglichere. So im Lepcha \textit{lōm}, Strasse, gegenüber \textit{ta-mo}, \textit{a-mik}, Auge, gegenüber \textit{čan}.} Die japanische Höflichkeit verbietet, von vornehmen Leuten das einfache Activum zu gebrauchen, als müssten sie sich selber bemühen. Entweder also wählt man das Causativum, als wären sie mittelbare, befehlende Urheber, – oder das Passivum, als geschähe die Sache von selbst. Wahrscheinlich unerreicht stehen in dieser Hinsicht die Koreaner da, die durch die Verbalform ausdrücken, ob der Höhere zum Niederen, der Niedere zum Höheren, oder einer zu Seinesgleichen, ob der Höhere vom Niederen, der Niedere vom Höheren, oder Einer von Seinesgleichen rede, und ob dies verhältnissmässig ehrerbietig, geringschätzig oder gleichgültig geschehe. Es wären dies eigentlich 3 × 3 × 3 = siebenundzwanzig Modi. Diese Rechnung wird indessen kaum zutreffen; denn einerseits scheint es nicht an feineren Abschattungen zu fehlen, und andrerseits dürften manche Posten zusammenfallen, z.~B. wenn im ehrenden Sinne der Höhere den Niederen, oder in geringschätziger Absicht der Niedere den Höheren wie Seinesgleichen behandelt. \sed{Auch in der baskischen Conjugation wuchern die Höflichkeitsformen.}

\sed{Die Sitte, der angeredeten Person den Vortritt zu lassen, hat in den Algonkinsprachen den Bau des incorporirenden Verbums ganz seltsam beeinflusst. Als Beispiel diene das Kri. Da bezeichnet \textit{ni} die 1. Pers. Sing. und, in Verbindung mit dem Suffixe \textit{-ān} (\textit{-nān}), die 1. Pers. Plur. exclusiv (also 1. + 3. Person); \textit{ki} bedeutet die 2. Pers. Sing., ferner in Verbindung mit dem Suffixe \textit{-now} (\textit{-nānow}) die 1. Pers. Plur. inclusive (also die 1. + 2. Person), in Verbindung mit \textit{-waw} endlich die 2. Pers. Plur. Das entsprechende Pron. 3. Pers. \textit{o} (\textit{ot}, \textit{wi}) erscheint nur in bestimmten Fällen. Jene \textit{ni} und \textit{ki} können nun so\-}{\textbar}{\textbar}247{\textbar}{\textbar}\phantomsection\label{sp.247}\sed{wohl subjective als auch objective Bedeutung haben, und hierbei findet eine Rangfolge statt: die zweite Person hat den Vorzug vor der ersten, beide haben ihn vor der dritten. Man sagt also:}

\begin{tabbing}
~~~~\=\sed{\textit{ni}} \= \sed{\textit{tapwehtowaw}}\=\sed{, ich glaube ihm;} \\
\>\> \sed{\textit{ni tapwehtak}, mir glaubt er;} \\
\>\sed{\textit{ki tapwehtowaw}, du glaubst ihm;} \\
\>\> \sed{\textit{ki tapwehtak}, dir glaubt er;} \\
\>\sed{\textit{ni tapwehtowanān}, wir glauben ihm;} \\
\>\> \sed{\textit{ni tapwehtakunan}, uns glaubt er;} \\
\>\sed{\textit{ki tapwehtowanānow}, wir (1 + 2) glauben ihm;} \\
\>\> \sed{\textit{ki tapwehtakunānow}, uns glaubt er;} \\
\>\sed{\textit{ki tapwehtowaw}, ihr glaubt ihm;} \\
\>\> \sed{\textit{ki tapwehtakuwaw}, euch glaubt er;} \\
\>\sed{\textit{ni tapwehtowawok}, ich glaube ihnen;} \\
\>\> \sed{\textit{ni tapwehtakwok}, mir glauben sie;} \\
\>\sed{\textit{ni tapwehtowanānăk}, wir (1 + 3) glauben ihnen;} \\
\>\> \sed{\textit{ni tapwehtakunanăk}, uns glauben sie u.~s.~w.} \\
\>\> \> \sed{Ferner:} \\
\>\sed{\textit{ki tapwehtowin}, du glaubst mir;} \\
\>\> \sed{\textit{ki tapwehtowitin}, dir glaube ich;} \\
\>\sed{\textit{ki tapwehtowinān}, du glaubst uns;} \\
\>\> \sed{\textit{ki tapwehtowitinān}, dir glauben wir u.~s.~w.} \\
\>\> \> \sed{Endlich:} \\
\>\sed{\textit{tapwehtew}, er glaubt ihm oder ihnen;} \\
\>\> \sed{\textit{tapwehtak}, er oder sie glauben ihm;} \\
\>\sed{\textit{tapwehtewok}, sie glauben ihm oder ihnen;} \\
\>\> \sed{\textit{tapwehtakwok}, er oder sie glauben ihnen.}
\end{tabbing}

\sed{Natürlich haben zumal die Pronomina der ersten und zweiten Person unter der Standessitte zu leiden. Das Du wird durch andere Fürwörter ersetzt: Ihr, Er, Sie, italienisch ella, lei; ehrende Titel: Monsieur, Madame, Euer Gnaden u.~dgl. treten an seine Stelle. Statt des Ich gebraucht man bescheidene Ausdrücke: Ihr Diener, meine Wenigkeit. Ehrende oder herabwürdigende Adjectiva vertreten die Possessivpronomina. So im Neuchinesischen: \textit{tsién mîng}, der geringe Name = mein Name, \textit{kuéi tí}, das geschätzte Land = Ihre Heimath. Oder es kann auch das gewählte Stoffwort ohne Weiteres anzeigen, welche Person gemeint ist. Man denke, um ein paar drastische Beispiele zu haben, statt: ich esse, du issest, er isst würde gesagt: „fressen, speisen, essen“, – statt: mein, dein, sein Haus: „Hütte, Palast, Haus“. Nicht viel anders geht es in solchen Sprachen zu, die der persönlichen Ausdrucksweise abhold sind. Und nun mag es wohl geschehen, dass die höflichen Wendungen gedankenlos überall ange\-}{\textbar}{\textbar}248{\textbar}{\textbar}\phantomsection\label{sp.248}\sed{wendet werden, und darüber die ursprünglichen Pronomina ganz verschwinden, dass dann jene höflichen Ersatzmittel, ihrerseits auf die Stufe gemeiner Fürwörter herabgesunken, neuen Surrogaten weichen müssen. Es zeigt sich schon hier etwas von dem, was wir bald unter dem Namen des Spirallaufes als allgemeines Princip der Sprachgeschichte betrachten werden.}

Wo es das Keuschheitsgefühl verbietet, mit Personen des anderen Geschlechts leichthin über gewisse Dinge zu reden, da kann es wohl zu einer Dreiheit der gleichbedeutenden Ausdrücke kommen: Männer und Weiber haben je unter sich besondere Wörter im Gebrauche, die sie dem anderen Geschlechte verheimlichen und im Nothfalle durch gewähltere Ausdrücke ersetzen. Das Beispiel der Insel-Caraiben zeigt aber \fed{{\textbar}245{\textbar}}\phantomsection\label{fp.245} auch, dass eine Doppelwirthschaft der Männer- und Weibersprache auf ganz anderer Grundlage entstehen kann. Vor Jahrhunderten haben Galibi-Krieger die von arawakischen Völkern bewohnten Inseln erobert, die männlichen Einwohner niedergemacht, die Weiber zu Gattinnen genommen. Was nun ursprünglich Nothstand war, das erhielt sich in der Folge als Sitte; was der Weibersprache eigenthümlich ist, das ist fast durchweg arawakisch; was der Männersprache \update{angehört}{angehört,} dagegen galibi-caraibisch; nur in vereinzelten Fällen scheinen die Rollen \retro{vertauscht}{vertaucht [\textit{in den Berichtigungen, S.~502}: vertauscht]} zu sein. (Vergl. \textsc{Lucien Adam}, Du parler des hommes et du parler des femmes dans la langue caraïbe. Paris 1879.)

\sed{Beispiele von Weibersprachen finden sich auch sonst, und überall wird etwas von geschlechtlicher Scheu bei ihrer Entstehung oder Erhaltung im Spiele sein. So verändern die Grönländerinnen auslautende \textit{k} und \textit{t} in die entsprechenden Nasale \textit{ng} und \textit{n} (\textsc{P. Egede}, Gramm. p. 4. \textsc{O. Fabricius}, Gramm. p. 10. \textsc{S.~Kleinschmidt}, Gramm. S.~5). Bei den Chiquitos in der Provinz Santa Cruz besteht gleichfalls ein scharfer Unterschied zwischen der Sprache der Männer und der Weiber. Letztere haben sich für gewisse Begriffe besonderer Wörter zu bedienen; in anderen Fällen kürzen sie die Wörter um den Anlaut, die Suffixe um die letzte Sylbe oder gebrauchen auch grundverschiedene Formen (\textsc{L. Adam} et \textsc{V. Henry} Arte y Vocab. p. 4–8).}

Wo die Keuschheit die Sprache von der Stelle befördert, da leistet ihr Gegentheil \update{Relaidienst. [\textit{in den Berichtigungen, S.~502}: Relaisdienst.]}{Relaisdienst.} Die gute Sitte will, dass gewisse Dinge nicht bei den ihnen eigenen, sondern umschreibend, andeutend, bei entlehnten Namen genannt werden. Diese Euphemismen klingen harmlos und wollen es sein. Jetzt bemächtigt sich ihrer die Zote, treibt Muthwillen mit dem Doppelsinn, deflorirt sie am Ende und macht sie ebenso anrüchig, wie jene Wörter, die sie mit Ehren ersetzen sollten. Nun ist wieder die Prüderie an der Reihe, Neues muss erfunden, wieder ein jungfräuliches Wort auf den bedenklichen Posten geschoben werden, – ein neues Opfer den losen Mäulern. Die Sache ist einleuchtend, und Beispiele sind Jedem zur Hand. Je zimpferlicher ein Volk in solchen Dingen ist, desto mehr Wörter setzt es auf den Index prohibitorum. In manchen \sed{{\textbar}{\textbar}249{\textbar}{\textbar}}\phantomsection\label{sp.249} Ländern, zumal in England, dem classischen Lande der Anständigkeit, kann sich der Fremde mit der Wahl seiner Ausdrücke gar nicht genug in Acht nehmen. Denn die Gegensätze wecken einander, hinter der Prüderie lauert die \update{Zote,}{Lüsternheit,} und das arglos hingeworfene Wort erregt in den empfindsamen Hörern unsaubere Gedanken.

Es ist gewiss ein altes, vielleicht ein ewiges Schauspiel in der Geschichte der Sprachen, dass Sittlichkeit und Frivolität im Ringkampfe sich und mit sich die Wörter und ihre Bedeutungen weiter wälzen. Kaum irgendwo dürften unsere indogermanischen Sprachen reicher an grundverschiedenen Ausdrücken sein, als in den hier einschlagenden Theilen ihres Wortschatzes. Es wäre kein Wunder, wenn an diesen Stellen das Wörterbuch unserer Ursprache \fed{für alle Zeiten} unausgefüllt bliebe.

Allerdings kennen wir auch entgegenwirkende Mächte. Muthwillen und Rohheit finden Gelegenheit sich auszusprechen, in Schlupfwinkeln \fed{{\textbar}246{\textbar}}\phantomsection\label{fp.246} oder frech auf offener Gasse. Die halten sich dann an das Derbste, und das wird in der Regel das erreichbar Älteste sein. Und so geschieht es, dass manche verpönte Wörter sehr weit, vielleicht über ihr ursprüngliches Vaterland hinaus verbreitet sind. Häfen, Märkte, Herbergen u.~s.~w. befördern ja auch diese Art der Ansteckung.

Seltsam äussert sich die geschlechtliche Ehrbarkeit in der Syntax kolarischer Sprachen. Im Munda-Kolh muss man von einer verheirateten Frau im Dual reden, als könnte man sie sich nicht allein denken; im Santal bestehen ähnliche, nur noch verzwicktere Vorschriften.

Wo das Bedürfniss zu fluchen durch die Scheu vor den heiligen oder gefürchteten Namen gehemmt wird, da nimmt es zu allerhand Wortverdrehungen und anderen harmlosen Ersatzmitteln seine Zuflucht: Potzblitz! Allmächtiger Strohsack! Schwerebrett! Parbleu! u.~s.~w. Hier wird nun die Partie in zwei Zügen zu Ende gespielt; denn diese Euphemismen sind auf alle Zeiten vor religiöser Weihe sicher. Höchstens vermeidet der Vornehme die Flüche des gemeinen Mannes und erfindet statt ihrer für seinen Bedarf neue.

Hier zeigt sich also eine neue Triebfeder: das Bedürfniss der höheren Classen, vor den unteren etwas voraus zu haben. Und dies Bedürfniss wird überall da wirken, wo etwas wie ein aristokratisches Gefühl vorhanden ist. Schon im engen Zusammenschlusse der Kaste muss ja \retro{die}{nie} Sprache ein eigenes Gepräge annehmen. Aber auch ohnedem ist es natürlich, dass man sich denen auch in der Rede überlegen zeigen will, denen man sich sonst überlegen dünkt. So verwelschten unsere Vorfahren ihr Deutsch. So vermeiden wir noch heute gewisse Redensarten, die wir gewöhnt sind von den Leuten im Kittel zu hören, – zum Theile höfliche Redensarten von bester Herkunft; unsere Grossväter haben sie gebraucht, aber ihre Diener haben sie ihnen abgelernt, und nun sind sie uns verleidet. Man sagt dann: „Das Wort hat einen Stich in’s Kleinbürgerliche, ça sent le petit bourgeois“, und die Unwissendsten sind hier wie immer \sed{{\textbar}{\textbar}250{\textbar}{\textbar}}\phantomsection\label{sp.250} am Absprechendsten. Seit die Zeiten der Kleiderordnungen vorüber sind, verbietet nichts mehr den geringen Leuten, es den Vornehmen nachzumachen. Nur nachkommen können sie ihnen nicht; denn wessen sie sich einmal bemächtigt haben, das gilt in den oberen Kreisen für entwerthet. Seit die Bauern ihre Kinder auf die Namen Arno, Alma, Ida taufen, sind die Hänse, Greten und Ilsen hoffähig geworden. Seit der Kleinstädter sagt: „Mit Vergunst, mit Verlaub,“ hat die vornehmere \fed{{\textbar}247{\textbar}}\phantomsection\label{fp.247} Gesellschaft gelernt, weniger umständlich zu sein. So sind mit der Zeit die Anreden Er und Ihr, die Titel Madame, Mademoiselle, Fräulein im Werthe gesunken, die französischen Anreden theils schon \retro{verpönt,}{verpönt [\textit{am Zeilenende}]} theils auf dem besten Wege, es zu werden. Dafür haben die Gebildeten gelernt, viel gutes \update{Deutsches}{Deutsch} wieder aufzunehmen, was der Dünkel ihrer Vorfahren in’s alte Gerümpel geworfen hatte. Es ist wie mit den alten Schränken und Truhen, die wir aus dem Staube der Bodenkammern hervorholen und in unsern Prunkzimmern aufstellen. Die Zeit wird kommen, wo auch sie wieder in’s Dunkel wandern; denn in der Mode hat sich der Europäer ein Perpetuum mobile geschaffen. Wir stehen hier wieder vor dem Schauspiele eines ewig wirbelnden Umlaufes.

\pdfbookmark[2]{II. §. 13a. Nach- und Neuschöpfung von Wurzeln und Wortstämmen.}{III.II.II.13a}
\cohead{II. §. 13a. Nach- und Neuschöpfung von Wurzeln und Wortstämmen.}
\subsection*{\sed{§. 13 a.}}\phantomsection\label{III.II.II.13a}
\subsection*{\sed{Nach- und Neuschöpfung von Wurzeln und Wortstämmen.}}
\sed{Bei der Durchsicht sprachvergleichender Wörterbücher stösst man immer auf Wortstämme und Wurzeln, die nur ein begrenztes Verbreitungsgebiet haben, nur einem Zweige des Sprachstammes, vielleicht gar innerhalb dieses Zweiges nur einer einzelnen Sprache angehören. Woher diese? Es sind verschiedene Fälle denkbar.}

\sed{1. Die Wurzeln oder Stämme waren schon in der gemeinsamen Ursprache vorhanden, sind aber bei deren Verzweigung überall, den einen Zweig abgerechnet, verloren gegangen. Dies ist möglich, es kann das eine oder andere Mal der Fall sein. Dass es aber allgemein gälte, ist nicht anzunehmen. Denn dann würde man den Ursprachen einen Reichthum, den Töchtersprachen eine Verarmung zutrauen, gegen die in der Regel die Vermuthung stritte. Allerdings ist auf niederen Culturstufen, in engbegrenzten Sprachgemeinden, eine ganz andere Freiheit der Neuschöpfung zu erwarten, als da, wo der gefestigte Brauch einer zahlreichen Gesammtheit der Willkür des Einzelnen Schranken setzt und die neuen Gebilde durch das Sieb des Herkömmlichen sichtet. Und zweitens liebt es gerade der Ungebildete, jeder ihm naheliegenden Vorstellung einen besonderen, eigenen Namen zu geben, statt sie umschreibend und beschreibend einer Classe einzureihen. Es mögen Augenblicksgeschöpfe sein, im Nu erfunden, ausgesprochen, vom Hörer verstanden und wieder vergessen. Aber woher dann Stoff und Kraft zur Schöpfung?}

{\textbar}{\textbar}251{\textbar}{\textbar}\phantomsection\label{sp.251}

\sed{2. Denkbar ist es auch, dass das Gesuchte sich bei weiterem Nachforschen doch auch noch anderwärts finden würde, sei es dass es selbst oder die Lautgesetze., die die Vergleichung ermöglichen müssten, noch nicht entdeckt sind. Dann liegt der Fehler am Stande unseres Wissens.}

\sed{3. Öfter mögen alte Entlehnungen, vielleicht von ganz unbekannten Sprachen, oder Dialektmischungen vorliegen. Etruskisches mag sich in’s Latein, Baskisches in’s Französische oder Spanische, Finnisches in’s Germanische oder Slavische eingeschlichen haben. Manche Wurzeldoubletten des Sanskrit entstammen den Prâkrit-Dialekten, z.~B. \textit{majj}, \textit{muj} neben \textit{m\textsubring{r}j} = wischen, \textit{mulgere}.}

\sed{4. Auch von ganz willkürlichen Worterfindungen weiss die Geschichte zu erzählen. Das bekannteste Beispiel ist „Gas“, dessen Namen die Sprachen Europas einem holländischen Chemiker, \textsc{van Helmont} und seinem berühmten Ortus medicinae (1658) verdanken.}

\sed{5. Doch Ähnliches kann wohl auch ganz naiv geschehen. Wir sind daran gewöhnt, die Töne und Geräusche, die wir vernehmen, onomatopoetisch in die Laute unserer Sprache zu übertragen, und binden uns darin keineswegs immer an das Überkommen. Fast immer jedoch wird dabei das Vorbild stilisirend umgewandelt; denn die Gehörseindrücke, die wir von der Aussenwelt empfangen, sind unendlich mannigfaltig, jene dagegen, die die menschliche Sprache hervorbringt, sind verhältnissmässig sehr beschränkt. In dem, was ich Stilisirung nannte, scheint nun wirklich ein gewisser Stil zu herrschen, der sich nicht an die Grenzen der Sprachfamilien bindet. So kann es kommen, dass derselbe Naturlaut in zwei nahe verwandten Sprachen zwei ganz verschiedene Abbildungen erfährt und wiederum in zwei ganz entlegenen Sprachen mit den gleichen Mitteln dargestellt wird. Dem deutschen kling-klang gleicht das mandschuische \textit{kiling-kalang}; lateinisch \textit{tinnire}, \textit{tintinnabulum} sind ebenso onomatopoetisch, aber doch ganz verschieden.}

\sed{6. Wichtiger als alle solche Neuschöpfungen scheinen mir die \so{Nachschöpfungen} zu sein. Ich muss hier an das erinnern, was ich über die Analogie im Allgemeinen, über das lautsymbolische Gefühl und über das etymologische Bedürfniss gesagt habe. Der Mensch will das, was er ausspricht, auch bis in die Theile hinein verstehen, er glaubt es so zu verstehen, und er wird nicht leicht darauf verzichten, diese Theile, seien sie nun geschichtlich begründet oder nur eingebildet, zu neuen Gebilden zusammenzufügen oder bedeutsam abzuändern. Muster schweben ihm vor, denen er unbewusst nachschafft; und wenn er damit der Sympathie seiner Sprachgenossen begegnet, so kann sein Erzeugniss Gemeingut werden. Wir haben in den indogermanischen Sprachen manche Gruppen sinn- und lautähnlicher Wurzeln, und noch viel mehr solcher Gruppen finden sich in dem ungleich grösseren semitischen Wurzelschatze. Man hat dabei an vorgeschichtliche Zusammensetzungen gedacht, und die mögen ja} {\textbar}{\textbar}252{\textbar}{\textbar}\phantomsection\label{sp.252} \sed{wohl stellenweise vorliegen. Aber dass sie dies waren, konnte längst vergessen sein, und dann blieb nur noch das Gefühl, dass gewisse lautähnliche Zuthaten oder Veränderungen in der Articulation den Sinn in gewissser Weise beeinflussten: die Muster waren da und luden von selbst zur Nachbildung ein. Ich brauche kaum zu wiederholen, dass solchen Einladungen der naive Geist viel leichter folgt, als der gebildete, und dass in kleineren Gemeinwesen Neuerungen leichter An- und Aufnahme finden, als in grösseren. Was ich von Veränderungen in der Articulation, das heisst vom Ersatze eines Lautes durch einen verwandten, gesagt habe, dürfte zumal von den semitischen Sprachen gelten, von ihren vorderen und hinteren, härteren und weicheren Gutturalen und Dentalen und anderen Lauten, die scheinbar regellos wechseln, z.~B. hebräisch \textit{qāzaz}, \textit{qāzāh} = schneiden; \textit{qāzar} = abschneiden; \textit{qāzab}, \textit{χāzab} = beschneiden; arabisch \textit{faşşa}, \textit{faşala} = trennen; \textit{da}{\ain}\textit{asa}, \textit{da}{\ain}\textit{aqa}, \textit{da}{\ain}\textit{asaqa} = treten, stampfen. Eine neue Theorie hiermit aufzustellen bin ich weder gewillt noch berufen; es genügt mir, wenn ich eine jener vielen Möglichkeiten nachgewiesen habe, mit denen der geschichtliche Sprachvergleicher rechnen muss. Die Indogermanisten, die ihre Lautgesetze immer schärfer ausfeilen und mit gerechter Befriedigung die Nachwirkungen der ältesten lautlichen Erscheinungen in den jüngsten und kleinsten Dialekten nachweisen, – die mögen dagegen einwenden: „Non liquet, soweit sind wir noch nicht, und folglich sind die Anderen erst recht noch nicht so weit.“ Die Frage aber, inwieweit jene Wurzelgruppen aus Agglutination und Composition, inwieweit sie aus anderen Vorgängen zu erklären seien, müssen sie eben um ihrer strengen Methode willen unbeantwortet lassen; widerlegen können sie die Möglichkeit lautsymbolischer Nachschöpfungen nicht. Auch ist ihnen die Sache keineswegs neu. \textsc{Pott}, \textsc{Curtius}, \textsc{Benfey} u.~A. haben längst ihren Scharfsinn an jenen Wurzelgruppen geübt, und \textsc{Curtius’} Ansicht von dem, was er Wurzeldeterminativa nennt, dürfte sich insoweit mit der meinigen berühren. Auch hat schon \textsc{Bopp} Zusammenstellungen von Sanskritwurzeln gewagt, wie: \textit{jarc}, \textit{carc}, \textit{j’arc}, \textit{jarc’}, \textit{jarj}, \textit{jarts}, \textit{b’arts}, \textit{garj}, \textit{tarj} = drohen. In solchen Fällen wird man weder an Dialektmischung noch an eigentliche Onomatopöien, sondern nur an eine zugleich freie und in ihrer Freiheit bedeutsame Articulation denken können. Und schliesslich ist die Lautsymbolik doch nur eine Unterart jener Analogie, zu der auch die gewissenhaftesten Lautvergleicher, und gerade sie erst recht, zuweilen ihre Zuflucht nehmen müssen, „in der höchsten, schrecklichsten Noth“.}

\begin{styleAnmerk}
\sed{Anmerkung. Im folgenden Buche („Die Ausspracheweise oder Stimmungsmimik“) werde ich nochmals auf den Gegenstand zurückkommen, und einige weitere Beispiele beibringen.}
\end{styleAnmerk}

\sed{{\textbar}{\textbar}253{\textbar}{\textbar}}\phantomsection\label{sp.253}

\clearpage\pdfbookmark[2]{II. §. 13b. Schwund alter u. Entstehen neuer grammat. Kategorien.}{III.II.II.13b}
\cohead{II. §. 13b. Schwund alter u. Entstehen neuer grammat. Kategorien.}
\subsection*{§. \sed{13b.}}\phantomsection\label{III.II.II.13b}
\update{13.}{}
\subsection*{Schwund alter und Entstehen neuer grammatischer Kategorien.}
Die sprachgeschichtlichen Erfahrungen, von denen ich im Folgenden einige Beispiele geben will, sind besonders schwer zu erklären. Grammatische Kategorien sind Denkgewohnheiten. Diese müssen, so sollte man meinen, durch die Rede immer von Neuem geweckt, sich von Geschlecht zu Geschlechte forterben, eine Art eisernes Inventar, das nicht leicht zu mehren, noch weniger leicht zu mindern wäre. Man sollte meinen, der Denkapparat der Väter müsse in der Regel auch den Kindern genügen, die ja von Jenen gelernt und geerbt haben; was aber den Ahnen unentbehrlich war, dessen könnten sich die Enkel, so sie nicht geistig herabgekommen sind, erst recht nicht entrathen. Die Erfahrung belehrt uns eines Anderen.

Die Romanen haben feinsinnig dem Plusquamperfectum das Passé antérieur (j’avais donné – j’eus donné) entgegengesetzt, und von Norddeutschen kann man ein Imperfectum futuri hören: „er wurde gehen“. In beiden Fällen hatte das Anschaulichkeitsbedürfniss leichtes Spiel: es brauchte nur zur nächstliegenden Analogie zu greifen, um die periphrastischen Formen um eine zu vermehren.

Umgekehrt ist uns Germanen der Unterschied zwischen Imperfectum und Aorist abhanden gekommen. Der Verlust mag sehr harmlos sein: schwer erklärlich ist er doch. Imperfectum und Aorist (aor. II) waren ja vormals formverwandt, bis auf die Behandlung des Verbalstammes identisch: \textit{a-bhava-t}, \textit{a-bhū-t}. Die Möglichkeit war also gegeben, dass \fed{{\textbar}248{\textbar}}\phantomsection\label{fp.248} damals schon beide Formen zusammenflossen. Die Germanen aber haben \update{Beide}{beide} schon in vorgotischer Zeit verloren, die Kategorien vermischt und dann bei einem Theile der Verba durch das alte Perfectum, bei einem anderen durch die Neubildung \update{mittels}{mittelst} \textit{–da} ersetzt. In beiden Fällen scheint bei ihnen die Vorstellung des perfectisch Abgeschlossenen vorgewogen zu haben, – ganz wie bei unseren oberdeutschen Landsleuten, die das neue periphrastische Perfectum mit „haben“ und „sein“ als erzählendes Tempus gebrauchen. Aber es hilft nichts, von einer gewissen Abstumpfung des Tempussinnes können wir unsre Vorfahren nicht freisprechen.

Ähnliches muss schon in indogermanischer Vorzeit mit den Casus geschehen sein. Schon zur Zeit der Sprachentrennung wimmelte es von Synonymformen, die sich declinationsweise \update{vertheilten.}{vertheilen.} Das waren natürlich noch früher ebensoviele sinnverschiedene Casuszeichen gewesen, deren Bedeutungsunterschiede man vergessen hatte. Dies wiederholte sich dann nach der Sprachentrennung noch öfter: sinn- oder lautähnliche Casus flossen ineinander, der Ablativ nach der einen Richtung mit dem Dative, nach der anderen mit dem Genitive, der Instrumentalis und Locativus mit dem Ablative oder Dative u.~s.~w. So hat \sed{{\textbar}{\textbar}254{\textbar}{\textbar}}\phantomsection\label{sp.254} das Sanskrit noch acht, das Lateinische \update{nur}{noch} sechs oder sieben, das Griechische nur fünf, das Gotische und Altnordische gar nur vier Casus.

Wie die Sprachen zur Kategorie des Dualis kommen, ist wohl einzusehen. Schon die Natur mit ihren paarweisen Dingen, zumal den betreffenden Körpertheilen, ist hierin Lehrmeisterin. Man begreift nun auch weiter, dass dieser Numerus sich gern auf jene natürlichen Zweiheiten zurückzieht, wie dies in den slavischen Sprachen zu beobachten ist. Und endlich begreift man, dass das Seltene mit der Zeit ganz schwindet, durch das Gewöhnlichere ersetzt wird: der Dualis mit durch den Pluralis, – eine \update{Neigung}{Neigung,} die wohl im ganzen indogermanischen Sprachstamme und auch im semitischen zu Tage tritt.

Dass Formdoubletten zu Bedeutungsunterscheidungen führen können, beweist das Deutsche mit dem durativen „wurde“ neben dem momentanen „ward“, mit den Collectivpluralen „Orte, Worte, Mannen, Lande, Bande“ neben den individualisirenden Pluralen „Örter, Wörter, Männer, Länder, Bänder“. \retro{Augenscheinlich}{Augen\-blicklich} liegen hier vertrocknete Keime zu einer weitergehenden Bereicherung der Grammatik vor. Ob sie weiter wuchern oder verkümmern sollten, hing von zwei Dingen ab: von ihrer eigenen \fed{{\textbar}249{\textbar}}\phantomsection\label{fp.249} Tüchtigkeit oder Untüchtigkeit als Typen zu weiteren Analogien, und von der Macht des Analogietriebes in uns, – von der Güte der Keime und der Empfänglichkeit des Bodens. In anderen Sprachen mögen solche zufällige Doubletten sehr fruchtbar geworden sein.

Das grammatische Geschlecht ist, wo es nicht auf dem natürlichen beruht, ein Luxus. Kein Wunder also, wenn es einschrumpft und wohl gar schwindet. Die romanischen Völker haben das Neutrum fast gänzlich mit dem formähnlichen Masculinum zusammengeworfen, die Engländer nur noch in den Fürwörtern Erinnerungen an das Geschlecht, aber nicht an das grammatische, sondern an das logische, gewahrt. Die zweigeschlechtigen hamito-semitischen Sprachen haben \update{hierin}{hiermit} mehr Beharrungsvermögen bewiesen.

Um so interessanter ist es, dass wir in der Geschichte der slavischen Sprachen das Erwachen einer vierten Geschlechtskategorie beobachten können: \textsc{Leskien} (Handbuch der altbulgarischen [altkirchenslavischen] Sprache, 2. Aufl. \update{§.}{§} 36) sagt: „Nach einer syntaktischen Eigenthümlichkeit des Slavischen, die im Altbulgarischen nicht völlig ausgebildet ist, kann beim Masculinum, wenn es ein belebtes Wesen bezeichnet, der Accusativus Singularis durch den Genitivus Singularis vertreten werden.“ Also nur in der Einzahl, da wo das Individuum am stärksten zur Geltung kommt. Nun ist in den slavischen Sprachen der Genitiv als partitivus auch Objectscasus; im Accusativus gilt das ganze Object als erfasst, im Genitivus nur der Theil. Vielleicht liegt hierin die psychologische Erklärung: das belebte Individuum soll niemals gänzlich und schlechthin passiv erscheinen. Vielleicht erklärt sich hieraus jene syntaktische Eigenthüm\-\sed{{\textbar}{\textbar}255{\textbar}{\textbar}}\phantomsection\label{sp.255}lichkeit des Spanischen, dass in gewissen Fällen das directe Object, wenn es ein belebtes ist, im Dativ zu stehen hat, gleichsam als verhalte es sich nicht ganz leidend, sondern irgendwie zurückwirkend.

\begin{styleAnmerk}
Anmerkung. \textsc{Brugmann}, Das grammatische Geschlecht in den indogermanischen Sprachen (\textsc{Techmers} Ztschr. IV, S.~100 flg.), nimmt an, dass die indogermanischen Femininformen erst durch eine Analogiewirkung gewisser so auslautender Substantiva für weibliche Wesen zur eigentlichen Geschlechtsbedeutung gelangt, und dass dann durch Analogie der Bedeutung weitere Classen von Substantiven als Feminina behandelt worden seien, – nach den Abstractis auf –ā auch die übrigen Abstracta u.~s.~w.
\end{styleAnmerk}

\fed{{\textbar}250{\textbar}}\phantomsection\label{fp.250}

\pdfbookmark[2]{II. §. 14. Rückblick. Der Spirallauf der Sprachgeschichte etc.}{III.II.II.14}
\cohead{II. §. 14. Rückblick. Der Spirallauf der Sprachgeschichte etc.}
\subsection*{§. 14.}\phantomsection\label{III.II.II.14}
\subsection*{Rückblick. Der Spirallauf der Sprachgeschichte, die Agglutinationstheorie.}
Die Sprache ist nicht fix und fertig, nicht mit Schild und Speer, wie eine Athene, nicht ausgerüstet mit einem Vorrathe von Wortformen und Formenwörtern dem Haupte des Menschen entsprungen, sondern allmählich geworden und weiter werdend. Was heute Affixe sind, das waren einst selbständige Wörter, die nachmals durch mechanische und seelische Vorgänge in dienende Stellung hinabgedrückt wurden. Dies anzunehmen nöthigt uns die Analogie alles Geschehens, nöthigt uns zumal auch die Analogie alles dessen, was wir in der Sprachgeschichte mit einiger Sicherheit verfolgen können.

Von der Beschaffenheit der ältesten Lautkörper (Wörter) der menschlichen Sprache können wir uns nur eine unvollkommene Vorstellung machen. Früher nahm man wohl an, sie müssten allesammt einsylbig und unveränderlich gewesen sein, man führte die monosyllabisch isolirenden Sprachen des östlichen Asiens, die chinesische, siamesische, annamitische an, behauptete wohl gar, die wären auf jener Urstufe sitzen geblieben. Das Letztere darf heute als widerlegt gelten; eher mag man annehmen, es seien jene Sprachen besonders weit fortgeschrittene. Aber auch den Glauben an die durchgängige Einsylbigkeit und Unabänderlichkeit jener Urwörter kann ich nicht theilen. Der Urmensch wird wohl der Wachtel ihr dreisylbiges Pikderik, dem Hahne sein viersylbiges Kikeriki nachgemacht haben, und reduplicirt hat er ganz gewiss: das hat ihm gleichfalls die Aussenwelt beigebracht. Sie hat ihn auch gelehrt, dass entferntere Geräusche dumpfer klingen, als nahe, Geräusche von grösseren Körpern dumpfer, als solche von kleinen; und so sind piff – paff – puff, bim – baum, ritsch – ratsch, tik – tak, tippen – tappen, scharren – schurren u.~s.~w. Gruppen, die ihresgleichen schon in der Ursprache haben mussten. Endlich mussten auch die Erregungen des eigenen Gemüths den Urmenschen veranlassen, Dasselbe bald in diesem, bald in jenem Tone, jetzt lauter, jetzt leiser, jetzt schärfer, jetzt weicher auszusprechen, und der Hörer musste alle diese Abschattungen mit empfinden. Alles dies leuchtet ein, – man kann es an jedem Kinde und an jedem \sed{{\textbar}{\textbar}256{\textbar}{\textbar}}\phantomsection\label{sp.256} in geistiger Kindheit verbliebenen Erwachsenen beobachten. Und nun wird man auch nicht mehr glauben, dass die indo\-\fed{{\textbar}251{\textbar}}\phantomsection\label{fp.251}germanischen Wurzeln auch nur als Typen jenen Urwörtern vergleichbar wären. Doch das ist auch nebensächlich: im Wesentlichen dürfte die sogenannte \so{Agglutinationstheorie}, wie sie heute wohl von allen Indogermanisten angenommen ist, unumstösslich und gemeingültig sein; alle Afformativen waren ursprünglich selbständige Wörter.

Nun bewegt sich die Geschichte der Sprachen in der Diagonale zweier Kräfte: des Bequemlichkeitstriebes, der zur Abnutzung der Laute führt, und des Deutlichkeitstriebes, der jene Abnutzung nicht zur Zerstörung der Sprache ausarten lässt. Die Affixe verschleifen sich, verschwinden am Ende spurlos; ihre Functionen aber oder ähnliche bleiben und drängen wieder nach Ausdruck. Diesen Ausdruck erhalten sie, nach der Methode der isolirenden Sprachen, durch Wortstellung oder \fed{durch} verdeutlichende Wörter. Letztere unterliegen wiederum mit der Zeit dem \update{Aggluti\-nations\-processe,}{Aggluti\-nations\-prozesse,} dem Verschliffe und Schwunde, und derweile bereitet sich für das Verderbende neuer Einsatz vor: periphrastische Ausdrücke werden bevorzugt; mögen sie syntaktische Gefüge oder wahre Composita sein (englisch: \textit{I shall see}, – lateinisch \textit{videbo} = \textit{vide-fuo}): immer gilt das Gleiche: die Entwickelungslinie krümmt sich zurück nach der Seite der Isolation, nicht in die alte Bahn, sondern in eine annähernd parallele. Darum vergleiche ich sie der Spirale.

Zu dieser Theorie sind gewiss schon viele Andere vor mir gelangt, – ich weiss nicht, wer zuerst. Einen hohen Grad innerer Wahrscheinlichkeit dürfte sie für sich haben, und auch die unserer Beobachtung zugänglichen Thatsachen scheinen sie zu stützen, zumal, – immer \update{soweit}{so weit} wir sie verfolgen können, – die Geschichte des indogermanischen und des indochinesischen Sprachstammes.

Alle Hauptvertreter der indogermanischen Sprachenfamilie, etwa das Keltische, Armenische und Albanesische ausgenommen, erscheinen bei ihrem ersten geschichtlichen Auftreten in einem Zustande der Agglutination, den man als Flexion bezeichnet hat, zum Unterschiede von jenen reineren Typen des anfügenden (agglutinirenden) Baues, die noch einer zugleich freieren und regelmässigeren Formenbildung geniessen. Erblicken wir in dieser den Höhepunkt der Agglutination, so gewinnt freilich die sogenannte Flexion unseres Sprachstammes eine andere, noch wörtlichere Bedeutung: die Agglutination selbst beugt und neigt sich abwärts, strebt der Isolation zu, periphrastische Formen nehmen überhand und erleiden ihrerseits langsamer oder schneller, mitunter wohl \fed{{\textbar}252{\textbar}}\phantomsection\label{fp.252} überaus schnell, das Schicksal der Agglutination. Es ist, als ob manche Sprachen, wie die neuindischen, rasch an der kritischen Periode der Isolation vorbeischlüpften, um in die Bahn einer regelrechten Agglutination einzulenken, während andere sich in dem neuen Zustande immer heimischer machen, – ich denke an die neuromanischen Sprachen und das Englische, das in der That dem rein \update{isolirenden}{iso\-{\textbar}{\textbar}257{\textbar}{\textbar}\phantomsection\label{sp.257}lierenden} Systeme zuzueilen scheint. \sed{Und wagen wir es einen Augenblick, uns in ihre vermeintliche Wurzelperiode hineinzuträumen: wer steht uns dafür, dass dies die erste, dass es nicht etwa die vierte, oder siebente oder zwanzigste ihres Vorlebens war? Dass sie, um zum Bilde von der Spirale zurückzukehren, nicht damals schon so und soviele Umläufe hinter sich hatte. Was wissen wir von dem Alter des Menschengeschlechtes, was von der Langsamkeit oder Geschwindigkeit jener vorgeschichtlichen Wandelungen im Leben der Sprache?}

Könnten wir die indochinesischen Sprachen an der Hand schriftlicher Denkmäler so weit zurückverfolgen, wie die arischen, griechischen, italischen und germanischen, so wäre das Bild, das wir gewönnen, wohl noch bunter und lehrreicher. Bunt genug ist es zwar schon jetzt, denn es zeigt uns Sprachen auf dem Höhepunkte der Agglutination, – so die der Kirânti, der Kuki (Luschai), Naga, Katschari, Manipuri u.~A., – dann das Tibetische mit seinen wunderlich wandelbaren Verbalstämmen, ihm nahe verwandt, doch schon der isolirenden Art näher stehend, das Barmanische und Arakanische, – dann fast ganz rein isolirend das Altchinesische und die Sprachen der Thai-Gruppe: Siamesisch, Lao, Schan, Khamti, Ahom, Aitom und Miaotse, – dazwischen unzählige Mittelformen, – endlich das Neuchinesische, das eben von der Isolation zur Agglutination hinüberschreitet. Allerdings kann das Chinesische sich der ältesten Urkunden rühmen, und diese bezeugen meiner Meinung nach unbestreitbar, dass die Sprache vor viertausend Jahren einsylbig und isolirend, wennschon mit Spuren eines älteren agglutinativen, vielleicht flexivischen Zustandes behaftet war.\footnote{Den Monosyllabismus des ältesten Chinesisch haben allerdings in neuerer Zeit einige Forscher angezweifelt. Ihre Gründe jedoch, auf die ich hier nicht näher eingehen darf, können mich nicht überzeugen.} Darum nannte ich die damalige Stufe der Isolation eine tertiäre. Wiederum giebt es unter jenen agglutinirenden Typen noch solche, die kaum erst der Isolirung und Composition entwachsen scheinen, während andere bei reinerer Agglutination den Eindruck höherer Alterthümlichkeit machen, als das Tibetische und Altchinesische. Es ist, als hätte man hier reichliche drei Viertel der Spirale vor Augen.

Bei den malaisch-polynesischen Sprachen ist es mir noch zweifelhaft, welche Stufe die alterthümlichere sei, ob jene fast isolirende der eigentlich polynesischen oder die einer hoch entwickelten Agglutination, auf der das Tagalische mit seinen Schwestern steht.

\fed{{\textbar}253{\textbar}}\phantomsection\label{fp.253}

Wie sich scheinbar die Extreme, Isolation und einverleibender Polysynthetismus, berühren können, zeigen die Sprachen der Ureinwohner Amerikas. Da haben wir Sprachen von fast isolirendem Baue, wie das Othomi, und dann wieder die verschiedensten Stufen der Agglutination. Wo aber, wie in \textsc{Stoll}’s und \textsc{Seler}’s Untersuchungen über die Mayasprachen, die Wissenschaft ihr Secirmesser mit Erfolg eingesetzt hat, da erscheinen die polysynthetischen Gebilde \sed{{\textbar}{\textbar}258{\textbar}{\textbar}}\phantomsection\label{sp.258} als Sätze oder zusammengesetzte Satztheile einer isolirenden Sprache, die durch Worteinheit für das Gehör verbunden, allenfalls auch durch Lautverschliff und Sandhi getrübt sind. Auf amerikanischem Boden wird die sprachgeschichtliche Forschung einige ihrer ergiebigsten Minen finden, und die meisten sind eben erst angebohrt.

\pdfbookmark[2]{II. §. 15. Hemmende und beschleunigende Kräfte.}{III.II.II.15}
\cohead{\fed{II. §. 15. Hemmende und beschleunigende Kräfte.}}
\subsection*{§. 15.}\phantomsection\label{III.II.II.15}
\subsection*{Hemmende und beschleunigende Kräfte.}
Wir sahen, die Sprachen machen es auf ihrem Entwickelungswege ungefähr wie die Reisenden, die jetzt weite Strecken im Eilzuge durchfliegen, jetzt ein kleines Ländchen gemächlich zu Fusse durchschlendern. Nur fehlen bekanntlich auf dieser Wanderung die Rasttage, denn auch das langsamste Werden ist ein ununterbrochenes Fortschreiten.

Jetzt fragt es sich: Was verlangsamt die Bewegung, was beschleunigt sie? \sed{Wir müssen uns daran erinnern, dass jede Neuerung ursprünglich ein Fehler ist. Das Gefühl, vermöge dessen wir das Fehlerhafte empfinden und verwerfen, wollen wir das \so{sprachliche Gewissen} nennen. Wodurch wird dies geweckt oder wachsam gehalten, wodurch wird es abgestumpft?}

In der Regel leben in einem Volke drei bis vier Generationen gleichzeitig. Die ganz alten haben nicht mehr viel zu sagen, weil sie nicht mehr arbeiten, mag auch eine schöne Pietät sie wie Halbgötter verehren. Die ganz Jungen haben noch nicht viel zu sagen, weil sie noch nicht mit arbeiten, mag auch Elternliebe ihrem Gelalle mit Entzücken lauschen. Die Gegenwart gehört denen, auf denen die nationale Arbeit ruht, – die führen recht eigentlich das Wort. Jeder verdankt seine Sprache dem vorigen Geschlechte, von dem er sie gelernt hat. Und Jeder übt sie redend und hörend im Umgange mit allen Altersstufen, zumeist wahrscheinlich mit seinen Alters- und Berufsgenossen. Er übt sie redend, das heisst: er theilt Anderen von seiner Individualsprache mit. Er übt sie hörend, das heisst: er empfängt von den Individualsprachen Anderer. Denken wir uns nun die Sprache im Stillleben der Vereinzelung, so stellt schon jede Altersstufe eine wenn auch noch so kleine Stufe der Sprachentwickelung dar, und der sprachliche Austausch \fed{{\textbar}254{\textbar}}\phantomsection\label{fp.254} zwischen Älteren und Jüngeren ist ein Austausch zwischen verschiedenen Sprachstufen. Austausch aber heisst Ausgleichung. Der Jüngere empfängt vom Älteren: damit gehen in conservativer Richtung Theile der älteren Sprache in die seinige über. Der Ältere empfängt vom Jüngeren: damit nimmt er vorwärtsschreitend Theile der jüngeren Sprache in die seinige auf. So ist Beides, Fortschritt und Hemmung, Sache des Verkehres.

\sed{{\textbar}{\textbar}259{\textbar}{\textbar}}\phantomsection\label{sp.259}

\pdfbookmark[2]{II. §. 16. Einfluss des Verkehres, Sprachmischung.}{III.II.II.16}
\cohead{II. §. 16. Einfluss des Verkehres, Sprachmischung.}
\subsection*{§. 16.}\phantomsection\label{III.II.II.16}
\subsection*{Einfluss des Verkehres, Sprachmischung.}
\subsection*{Einleitung.}
Die Sprache dient in erster Linie dem Zwecke des Verkehres in Rede und Gegenrede: man will verstanden werden, wenn man spricht, und was man \update{hört}{hört,} will man verstehen. Dies gegenseitige Verständniss ist von einer Menge individueller Umstände abhängig, und diese Umstände werden die Art der Rede beeinflussen. Der Anzuredende ist weit von mir entfernt oder leidet an Schwerhörigkeit: darum spreche ich mit erhobener Stimme und möglichst deutlicher Lautbildung. Er sei ein Dummkopf oder ein Ungebildeter, so werde ich mich zu ihm herabstimmen und meine Rede seinem Verständnisse anpassen. Er gehöre zu meinen nächsten Vertrauten: ein kurz andeutendes Wort, halb flüsternd \update{gesprochen,}{gesprochen} genügt: er weiss was ich sagen will. Er sei ein Ausländer, meiner Muttersprache unkundig, und ich versuche in der seinen mit ihm zu reden; nun merkt er, wie ich es mir sauer werden lasse, und kommt mir \update{in}{mit} seiner Sprache gefällig entgegen, redet langsam und deutlich in gewählt einfachen Sätzen, mischt wohl auch soviel Deutsch in seine Rede, als er irgend zusammenbringen kann.

Das Alles sind alltägliche Vorgänge, man möchte um Entschuldigung bitten, dass man sie erwähnt; aber unter allen Mächten, die das Leben der Sprache beherrschen, sind die kleinen Alltäglichkeiten die wirksamsten. Verkehren heisst einander mittheilen: das und das habe ich; das und das empfange ich hinzu, nehme es auf in mein Vermögen. Ein solches Vermögen ist auch die Einzelsprache, die ich besitze. Nicht immer mindert es sich, nein, meist bleibt es unverringert durch das, was ich Anderen mittheile. Allein wir sahen schon, dass ich auch genöthigt sein kann, dem Verständigungszwecke wahre Opfer zu bringen, die viel\-\fed{{\textbar}255{\textbar}}\phantomsection\label{fp.255}leicht doch mit der Zeit auf Kosten meines Vermögens gehen können. Freilich die Kräfte und die erzielten Wirkungen sind von sehr verschiedener Stärke, und was derart alltäglich und überall geschieht, ist im Einzelnen so winzig, dass man es nicht wahrnimmt; der ganze Vorgang wird leicht übersehen oder doch unterschätzt. Grund genug, beim Gröbsten, Handgreiflichsten anzufangen, beim Ringkampfe verschiedener Sprachen miteinander.

Man darf ohne Übertreibung sagen: \update{ein}{Ein} solcher Ringkampf findet überall statt, wo verschiedensprachige Menschen miteinander verkehren. Mag der eine oder der andere Theil so friedwillig und schmiegsam sein, wie er wolle: immer wird ihn die heimische Sprache im Gebrauche der fremden beeinflussen. Mag der eine oder der andere Theil noch so zäh an der Eigenart seiner Muttersprache festhalten wollen: immer wird die Gewohnheit, Fremdartiges zu hören, auch seine eigenen Sprachgewohnheiten ändern. Die Macht der Berührung \sed{{\textbar}{\textbar}260{\textbar}{\textbar}}\phantomsection\label{sp.260} äussert sich aber da am \update{stärksten,}{Stärksten,} wo wir uns selbst am \update{unbewuss\-testen}{Unbewuss\-testen} äussern, also im Gebrauche der Muttersprache. Und jene Macht ist um so beträchtlicher, je häufiger und dauernder sie wirken darf, in unserem Falle also da, wo verschiedensprachige Bevölkerungen durcheinander gemischt denselben Ort bewohnen und auf regelmässigen Verkehr untereinander angewiesen sind, wo etwa der Knecht oder Hörige anderen Volksstammes ist, als der Herr. Die Geschichte Europas wimmelt von Beispielen dieser Art, und wo immer kriegerische Völker die gefangenen Feinde als Sklaven mit sich heimschleppen oder sich als Herren im Lande der Unterjochten niederlassen, führen sie ihrer Sprache neue Keime zu. Wir haben es recht eigentlich mit pathologischen Vorgängen zu thun, mit Ansteckungen.

\sed{Ganz unempfänglich für solche Ansteckungen ist wohl keine Sprache. Aber die eine leistet kräftigeren Widerstand als die andere. Was man nun \so{Pflege} oder \so{Verwahrlosung einer Sprache} nennt, beruht wesentlich auf dem Verhalten dem Fremden gegenüber. Und fremd in diesem Verbande ist auch, was einem andern Dialekte angehört. Es ist für die Geschichte einer Sprache wichtig, ob eine bestimmte Mundart in der Rede und den Schriften der höheren Classen die Alleinherrschaft führt, wie etwa die castilianische in Spanien, oder ob alle Gaue des Landes gleich frei über das nationale Gemeingut verfügen, – jetzt Flicken beitragend, jetzt Fetzen abreissend. Es ist wichtig, ob ein starkes oder erstarkendes Nationalgefühl sich des Ausländischen puristisch erwehrt, oder ob die Vornehmthuerei der tonangebenden Stände sich in fremden Wort- und Phrasenflittern gefällt. Das merkwürdigste Beispiel in dieser Hinsicht bietet, soweit mir bekannt ist, Korea. Hier ist geradezu das Chinesische die Schriftsprache der oberen Classen, und ganz wie in Japan, hat auch die Umgangssprache unzählige überflüssige Fremdwörter in sich aufgenommen. Allein die Japaner schätzen trotzdem das Altheimische, wenn auch mehr so, wie man Alterthümer eines Museums schätzt. Die Koreaner dagegen behandeln ihre Muttersprache geradezu mit Missachtung; was sie in ihr schreiben, ist für Weiber und Ungebildete bestimmt, und Jeder schreibt nach Gutdünken. Es giebt keine feste Orthographie, in manchen Stücken keine feste Grammatik; die Bücher wimmeln von Provinzialismen und Individualismen. Hilft anderwärts eine gepflegte Literatur und die Buchstabenklauberei der Philologen einem erschlaffenden Sprachgefühle nach: so scheint hier das vaterlandslose Treiben der Schriftsteller und Gelehrten das Möglichste gethan zu haben, um bei dem Volke den Sinn für das Sprachrichtige vollends abzustumpfen.}

\sed{Wir werden es im Folgenden immer wieder sehen: alle Sprachgeschichte ist zugleich Sprachmischung. Fassen wir das Wort im weitesten Sinne, so gilt es auch da, wo Heimisches zu Heimischem gemischt wird, wo ein Ortsnachbar dem Anderen von seinem Sprachgute mittheilt. Diese Art der Mischung wird in der} {\textbar}{\textbar}261{\textbar}{\textbar}\phantomsection\label{sp.261} \sed{Regel erhaltend, oft wahrhaft bereichernd wirken; denn in der Regel spricht Jeder seine Muttersprache richtig, und was er ihr neu hinzuschafft, wird aus ihrem Stoffe und nach ihren Gesetzen geschaffen sein. Allein für die folgenden Untersuchungen ist es wichtiger zu sehen, wie auch das Fremde und Falsche nachwirken kann.}

\pdfbookmark[2]{II. §. 17. Aussterben der Sprachen.}{III.II.II.17}
\cohead{II. §. 17. Aussterben der Sprachen.}
\subsection*{§. 17.}\phantomsection\label{III.II.II.17}
\subsection*{Aussterben der Sprachen.}
Soweit uns die Geschichte Auskunft giebt, hat sich die Zahl der Sprachen nicht vermehrt, sondern beträchtlich vermindert. Freilich, wie wenig Auskunft giebt uns die Geschichte! Da gerade, wo wir ihrer diesmal am Meisten bedürften, in jenen wimmelnden Sprachennestern des Himâlaya und seiner östlichen Ausläufer bis hinab in die hinterindische Halbinsel oder in den sprachen- und völkerreichen melanesischen Archipelen, lässt sie uns schnöde im Stich; nur von Europa und einem Stücke \fed{{\textbar}256{\textbar}}\phantomsection\label{fp.256} des westlichen Asiens können wir mit Sicherheit reden, und hier trifft es zu: sehr viele alte Sprachen sind verklungen, nur wenige neue durch Dialektspaltung entstanden.

Wir haben an einer früheren Stelle, S.~\update{155,}{146,} flüchtig Umschau gehalten über eine Reihe Sprachen der alten Welt, die in geschichtlicher Zeit verklungen sind. Jetzt brauchen wir die Blicke nicht weiter schweifen zu lassen, nicht dorthin, wo der Europäer an den Opfern seiner Cultur und seiner Laster die Nothtaufe vollzieht, auch nicht zu den Ainos und so manchen anderen Todescandidaten. Überall dasselbe Recht des Stärkeren: die kleinen Völker und ihre Sprachen erhalten sich nur solange, als ihnen ihr Einsiedlerthum abseits vom grossen Völkerverkehre unangefochten bleibt.

Die Lebenskraft der Sprachen hängt ab von der Lebenskraft der Völker. Beide sind zuweilen zähe, so schwächlich sie scheinen: das Gefühl des Zusammenhaltes leistet fremder Übermacht nachhaltigen Widerstand. Wo ein Nachbar dem anderen täglich die heimischen Laute \update{in’s}{ins} Ohr und Gedächtniss ruft, da haften sie, und das Zerstörungswerk hat nur langsamen Fortgang, wie die Schwindsucht in einem wohlgenährten Körper. Unsere Zeit weiss noch von manchen kleinen Völkern, die sich plötzlich zu \update{mächtigem,}{mächtigem} nationalem Selbstbewusstsein aufgerafft haben, und auch sie setzen ihr Bestes an die Pflege der Muttersprache. Wenn in späteren Jahrhunderten eine Geschichte der europäischen Sprachen geschrieben wird, so scheint es, es sei darin dem Verkünder des Nationalitätsprinzips, \textsc{Napoleon} III., ein hervorragender Platz gewiss. Mit Recht gelten uns Volksthum und Sprache für unzertrennlich: die Nationen bewahren Eins mit dem Anderen, geben Eins mit dem Anderen auf.

\begin{sloppypar}Die Erhaltung Beider setzt aber entweder die Isolirung von fremden Einflüssen oder die örtliche Vereinigung einer genügend grossen Sprachgenossen\-\sed{{\textbar}{\textbar}262{\textbar}{\textbar}}\phantomsection\label{sp.262}schaft voraus. Daher entspringen die Annexionsgelüste und Vergewaltigungen solcher Völker gegen ihre Nachbarn einem sehr richtigen Gefühle, – und einem ebenso richtigen entspringt es, wenn mächtige Staaten auf die Lossonderungsgelüste ihrer Provinzen mit brutalen Sprachmandaten antworten. Nicht nur die Kleinen haben aus dem Nationalitätsprinzip ihre Lehre gezogen.\end{sloppypar}

Einfacher gestalten sich die Dinge dort, wo sich der Einzelne von seiner Sprachgenossenschaft losgelöst hat, wie soviele unsrer auswandernden Landsleute im englisch redenden Amerika. Da klingt wohl die \fed{{\textbar}257{\textbar}}\phantomsection\label{fp.257} Muttersprache noch eine Weile im häuslichen Kreise fort; bald aber mag es ihr ergehen, wie des verstorbenen Grossvaters Rocke: die Enkel haben ihn noch mit wehmüthiger Pietät aufgehoben, die Urenkel ergötzen sich eine Weile an dem altmodischen Schnitte und werfen ihn dann als unbrauchbar weg. Gewiss, es ist zu beklagen, dass gerade wir Deutschen unsre Muttersprache so leichten Kaufes daran geben. Ob man aber Recht hat, dies allein aus Mangel an Nationalgefühl zu erklären und zu erwarten, nun müsse es damit besser werden? Wo man unter gleich wirksamen Mitteln die Wahl hat, pflegt man das \update{bequemste}{Bequemste} zu bevorzugen, und das ist unsere reiche, tiefe Muttersprache wahrhaftig nicht. Die ist so schwierig, dass wir selber es empfinden, sobald wir uns etwa im Englischen oder Französischen leidlich heimisch gemacht haben. Und wer die gätlichere fremde Sprache hört und redet, sobald er den Fuss auf die Strasse setzt, der wird sie, wenn er nicht sehr Acht hat, bald genug auch in’s Haus tragen. Es ist eben eine der vielen Bequemlichkeitsfragen, die im Leben der Sprachen eine so wichtige Rolle spielen.

\pdfbookmark[2]{II. §. 18. Entlehnungen.}{III.II.II.18}
\cohead{II. §. 18. Entlehnungen.}
\subsection*{§. 18.}\phantomsection\label{III.II.II.18}
\subsection*{Entlehnungen.}
\subsection*{Lehnwörter.}
Der Kampf der Sprachen, er geschehe noch so friedlich, ist immer ein Kampf der Nationalitäten: mit seiner Sprache giebt ein Volk sich selbst auf; was ihm auch sonst an provinziellen Eigenthümlichkeiten verbleiben möge, es ist hinfort doch nur eine neue Varietät der Nation, der es sich einverleibt. Allein Sprachen und Völker kennen neben dem Kampfe auch den freundschaftlichen Tauschverkehr. Waaren, Begriffe, ganze Weltanschauungen wandern von Lande zu Lande, mit ihnen meist ihre Namen. Den Griechen und Römern verdanken wir Neu-Europäer die Mehrzahl unsrer wissenschaftlichen Ausdrücke, und wo wir diese in puristischem Streben durch heimische Wortgebilde ersetzt haben, da pflegt es zu \update{gehen}{gehen,} wie so oft mit Büchern aus zweiter Hand: der Name des Vorbesitzers ist durchstrichen, aber noch zu lesen, oder er ist herausgeschnitten, und die verklebte Lücke verräth, dass er vormals da war. Die Holländer haben mit vielem Geschicke eine Menge Fremdwörter aus Elementen \sed{{\textbar}{\textbar}263{\textbar}{\textbar}}\phantomsection\label{sp.263} ihrer eigenen Sprache nachgebildet: \textit{voorvoegsel}, \textit{achtervoegsel}, \fed{{\textbar}258{\textbar}}\phantomsection\label{fp.258} \textit{invoegsel} für Prä-, Sub- und Infix, \textit{onderwerp} für Subject, \textit{voorwerp} für Object, \textit{bijwoord} für Adverb u.~s.~w. Das Gleiche haben die Russen gethan, und bei Beiden \update{haben diese Wörter sich}{haben sich diese Wörter} ganz anders eingelebt, als etwa die entsprechenden Verdeutschungen, die Mancher erst in’s Lateinische zurückübersetzen muss, um sie zu verstehen. Doch gleichviel, Nachbildungen sind es hüben wie drüben, also doch auch Entlehnungen. \textit{Convincere} war ursprünglich ein Ausdruck des Rechtsverfahrens: man besiegt den Angeklagten oder Processgegner, indem man ihm seine Schuld, den Ungrund seiner Ausflüchte oder Ansprüche beweist. „Taces? convincam, si negas!“ ruft Cicero dem Catilina zu. Der Beweis geschieht natürlich oft durch Zeugen; die können einander widersprechen, und dann kommt es darauf an, wer die meisten und besten Zeugen aufzuführen hat. Im alten deutschen Rechtsstreite konnte man den Gegner „überzeugen“ oder durch Eideshelfer „überschwören“. Es lag nahe, dem Worte \textit{convincere} eine erweiterte Bedeutung beizulegen, und das ist denn auch bei den romanischen Völkern schon frühe geschehen: man gebraucht es, wo man Einen durch Gründe dahin bringt, eine Äusserung zurückzunehmen oder eine Meinung zu ändern; und wenn man sich nun selbst eines Irrthums bewusst geworden war, so durfte man das Wort auch reflexiv anwenden. Nun verblich das etymologische Bewusstsein; an den Sieg, der einen Kampf voraussetzt, dachte man nicht mehr, sondern nur an das Endergebniss, an die erlangte Gewissheit, die ja auch nur eine längst gehegte Vermuthung bestätigen konnte. Jetzt durfte man also auch sagen: „Je l’avais toujours pensé; aujourd’hui j’en suis convaincu.“ Wir Deutschen hatten uns lange mit nüchternen Ausdrücken beholfen: „ich bin es gewiss, ich zweifle nicht mehr“ u.~dgl. \sed{\textsc{Reimarus} sagt schon: „nach meiner Überführung“ = nach meiner wohlbegründeten, unerschütterlichen Meinung. Dies war aus dem Gerichtsleben übertragen, liess aber noch dahin gestellt, wodurch die Gewissheit erlangt worden war. Andere, – ich weiss nicht, ob damals schon oder später, – wendeten stattdessen den noch uneigentlicheren Ausdruck „Überzeugung“ an, der nun in diesem Sinne ganz gäng und gäbe geworden ist.} \fed{Da kam das wunderliche Deutschthum des Turnvaters Jahn mit den Kraftmenschen der Hasenhaide. Denen war ein ehrliches deutsches Gewissen nicht genug, und ein gründliches Forschen, das zur Gewissheit führt, war ihnen wohl zu viel. Wie es aber in der Welt werden müsse, wussten sie alle, und das nannten sie ihre Überzeugung, nicht ahnend, dass sie bei dieser Taufe die „verfluchten Wälschen“ zu Gevatter geladen hatten. Weil das Wort im processualen Gebrauche so leidlich dasselbe besagte, wie \textit{convictio}, so wurde beider Schicksal aneinander gekettet,} \retro{und heutzutage}{Heutzutage} redet alle Welt mehr von moralischer Überzeugung als von juristischer. \fed{(Vergl. \textsc{v. Treitschke}, Deutsche Gesch. im 19. Jahrh. II, S.~390–391.)}

Das Erzählte ist ein vereinzeltes Beispiel eines sehr verbreiteten \fed{{\textbar}259{\textbar}}\phantomsection\label{fp.259} Vorganges; seltsam ist dabei höchstens das, dass man für einen längst vorhandenen Begriff ein längst vorhandenes, wenig passendes Wort anwandte. Wo aber neu eingeführte Begriffe halbwegs passende heimische Ausdrücke vorfinden, da geschieht es wohl, dass Beide sich miteinander vermählen, und das alte muttersprachliche Wort eine ganz bestimmte technische Bedeutung annimmt. Ich erinnere nur an Ausdrücke der Psychologie und Logik, wie Vorstellung, Begriff, Urtheil, Schluss, Gegenstand, Eigenschaft, Umstand, Merkmal. Die Gattung A \so{begreift in sich} die Arten a, b, c, d ... Die stimmen in den und den Merkmalen über\-\sed{{\textbar}{\textbar}264{\textbar}{\textbar}}\phantomsection\label{sp.264}ein, und diese Merkmale bilden ihren Begriff, – \so{ich begreife unter} dem Namen A \update{Alles}{Alles,} was diese Merkmale hat; noch einen Schritt weiter in meiner Erkenntniss, und \so{ich begreife das Ding selbst}. Es ist damit fast ganz wie mit dem \textit{comprehendere} der Romanen, und dieses wird die Führerrolle gespielt haben; unser Verdienst war es nur, dass wir den Begriff im wissenschaftlichen Sinne der Reihe einzufügen wussten.

Es ist ein anziehendes Capitel der Culturgeschichte, wie die Ausdrücke für Natur- und Geisteserzeugnisse gewandert sind und oft weite Strecken zurückgelegt haben. Die griechischen νόμος, πιττάκιον, διφθέρα, das persische \textit{tuman} = zehntausend finden sich bei den Mandschu wieder: \textit{nomun}, \textit{bitχe}, \textit{debtelin}, \update{\textit{tumen}}{\textit{tumen},} – bei den Mongolen wenigstens \textit{nom} und \update{\textit{bičik}; \textit{defter}}{\textit{bičik}. \textit{Defter}} haben die Araber und Perser, \textit{piṭaka} die Inder und ihre Culturverwandten angenommen. So ist denn \textit{defter} (oder \textit{daftar}) mit dem Islâm zu den Malaien gelangt. Doch das sind einzelne Versprengte. Interessanter sind die Entlehnungen da, wo ein dauernder Verkehr der Nationen zum Austausche von Sprachgut geführt hat. Was hat jeder Theil vom anderen entnommen? Gewiss vor Allem das, wofür ihm eine einheimische Bezeichnung fehlte, weil sein Sinn bisher nicht auf die Sache gerichtet war. Und wie hat er dann das fremde Wort gedeutet? Unser harmloses \so{trinken} hat bei den Franzosen die Bedeutung des Zutrinkens und Anstossens mit den Bechern angenommen (\textit{trinquer}), und aus dem Zurufe beim Zutrinken: „Bring’ Dir’s!“ ist bei ihnen und den Italienern \textit{brindisi}, der Trinkspruch, geworden. Das sind ältere Entlehnungen; im Trinken sind wir aber bis auf den heutigen Tag die Lehrmeister der Franzosen, sie sagen \textit{une choppe}, und nun, da sie an unserm Biere Gefallen finden, \textit{un bock}. Der Trunk (\textit{sup}) vor der Mahlzeit, der in Schweden jetzt noch üblich ist, hat der Suppe ihren Namen gegeben. So sind denn auch manche Ausdrücke des Rechts- und Kriegslebens von uns zu \fed{{\textbar}260{\textbar}}\phantomsection\label{fp.260} den Romanen gelangt, – man schlage nur ein italienisches Wörterbuch unter \textit{Gua}, \textit{Gue}, \textit{Gui} auf. Garde, Bivouac, Garantie sind uns geläufige Ausdrücke; das Gefühl sagt uns, dass es Fremdwörter sind, es sagt uns aber nicht, dass sie doch deutschen Ursprung und nur ausländisches Gepräge haben, – das lehrt uns erst die wissenschaftliche Untersuchung. In Angelegenheiten der verfeinerten Geselligkeit und ihrer Erfordernisse reden wir noch immer gern die Sprache unsrer Lehrmeister, der Franzosen; die beherrschen das amüsante und elegante Leben, wie die Engländer den Sport. Die Sprache des Rennplatzes, auch wo sie deutsch sein will, wimmelt von Anglicismen. Eitles Bemühen, sich der fremden Sprache zu erwehren, wenn man einmal dem fremden Begriffe Bürgerrecht ertheilt hat. Unsere Sprache sei noch so reich: wo soll sie die Wörter hernehmen für alle die zarten Abschattungen der Begriffe, die unser vielgestaltiges Culturleben mit sich bringt? Galant besagt mehr als höflich, weniger als ritterlich, ein Cavalier ist etwas Anderes als ein Edelmann im deutschen \sed{{\textbar}{\textbar}265{\textbar}{\textbar}}\phantomsection\label{sp.265} Sinne, und \textit{gentleman} ist vollends unübersetzbar. Jedes Volk hat Begriffe und Anschauungen, die nur ihm eigen sind, und dafür manchmal recht bezeichnende Lücken. Der Chinese verbindet in einem Worte, \textit{šîng}, die Begriffe der Wahrheit und der sittlichen Freiheit, und die Ehre fasst er mit der Rechtlichkeit in dem Worte \textit{ngí} zusammen. Soll man ihn darum loben oder tadeln? Schlimmer ist es doch, wenn ein slavisches Volk das lateinische Wort \textit{honor} anwenden muss für die prunklose Ehre, \largerpage[1]die im Bewusstsein und der Anerkennung der Rechtschaffenheit beruht.

Oft lassen untergegangene Sprachen ihren Siegerinnen Lehnwörter als schwache Vermächtnisse zurück. Nicht nur Orts- und Personennamen, wie die slavischen bei uns, die keltischen und germanischen in Frankreich, erinnern an die Sprachen der früheren Insassen: auch \update{Appellativa}{Appellative} können sich erhalten, zumal wenn sie Dinge bezeichnen, die zum Eigensten der Landschaft gehören. \textit{Clan}, \textit{loch}, \textit{glen} sagt man noch in den längst germanisirten Theilen Schottlands; weithin durch Deutschland ist für eine Art Eierkuchen der \update{slavischen}{slavische} Name \so{Plinse} gebräuchlich. \sed{Für slavisch gelten auch die Namen einiger Vögel: Zeisig, Stieglitz, Kiebitz, ferner die Wörter: Peitsche, Petschaft, Jauche, Gurke, Kummet.} Ein Dorf in der Güldenen Aue heisst zum Andenken an seine früheren Bewohner Windehausen. Dort wird, – wurde wenigstens noch in den fünfziger Jahren, – ein roh geschnitztes Madonnenbild mit dem Christus aufbewahrt, das die Leute den Pomai-Bog nannten, nicht ahnend, dass sie damit den wendischen Gruss: Helf Gott! aussprachen. In den deutsch\-\fed{{\textbar}261{\textbar}}\phantomsection\label{fp.261}redenden Dörfern Obersachsens hört man noch zuweilen die Schenke slavisch: den Kretzscham, und den Schenkwirth den Kretzschmar nennen. Slavischen Ursprungs ist wohl auch der Ruf Hösse, Hösse! womit man die \update{Gänse}{Gänse,} \sed{und Kutsch! womit man die Schweine} lockt. \sed{Von uraltem Verkehr zwischen den Finnen und Nordgermanen zeugen Entlehnungen auf beiden Seiten, auf finnischer z.~B. \textit{kulti} = Gold, \textit{pelto} = Feld, \textit{tuoli} = Stuhl, \textit{tupa} = Stube, \textit{tori} = Markt, schwedisch \textit{torg}, – auf germanischer: schwedisch \textit{poike}, finnisch \textit{poika} = Knabe, \textit{piga}, dänisch \textit{pige} = \textit{piika}, Magd. Auch der skandinavische Name des Fuchses, \textit{refr}, \corr{1901}{schwedisch}{schwedich} \textit{räf}, dänisch \textit{räv} scheint dem finnischen \textit{repo} entlehnt zu sein.}

Offenbar liegt es der sprachgeschichtlichen Forschung ob, die Fremdlinge als solche zu erkennen und ihr Vaterland, wohl auch den Weg ihrer Wanderung festzustellen. Das Erstere ist in den meisten Fällen nicht allzuschwer: die Laute selbst, die Etymologie, wohl auch die Unerfindbarkeit einer solchen in der Muttersprache, lassen den Eindringling erkennen, und man braucht nicht allemal lange bei den Nachbarn Umfrage zu halten, um zu wissen, woher er gekommen ist. Die Wörter des kaufmännischen Rechnungs- und Wechselwesens (Conto, Saldo, Giro, Tratte u.~s.~w.) tragen noch ihr volles heimisches Gepräge; sie verhalten sich wie Colonisten, die nicht daran denken, Bürgerrecht zu er\-\sed{{\textbar}{\textbar}266{\textbar}{\textbar}}\phantomsection\label{sp.266}werben. Ebenso jene anderen Fremdwörter, deren heimischer Aussprache sich der deutsche Mund anbequemt, so gut er kann: Cinquecento, Renaissance, Sport, Whist und allerhand Namen für Speisen, Getränke und Kleidungsstücke. Andere Fremdwörter, meist von höherem Culturwerthe, erkennt man an ihren lateinischen und griechischen Prae- und Suffixen; und was im Spanischen mit \textit{al-} anfängt, ist ohne Weiteres maurischer Herkunft verdächtig. Die Japaner haben das Lautwesen ihres chinesischen Lehngutes erbarmungslos verunstaltet; man braucht aber nicht Chinesisch und sehr wenig Japanisch zu verstehen, um instinctiv die Mehrzahl jener fremden Bestandtheile als solche herauszufühlen. Ähnlich ist es im Koreanischen.

Gründlicher hat sich die Einbürgerung der zugewanderten Wörter bei den Malaien und ihren Stammesvettern vollzogen; dort werden die Fremdlinge ganz wie die Einheimischen behandelt. Von sanskrit \textit{kathâ}, Erzählung, leitet der Malaie ab: \textit{meṅāta}, reden, \textit{perkatāen}, Rede, von \textit{râǰa}, König: \textit{karaǰāan}, Königreich; die Tagalen von Luzon leiten mit Hülfe ihres Infixes \corr{1891 und 1901}{–\textit{um}–}{– \textit{um} –} vom arabischen \textit{surat}: \textit{sumulat}, schreiben, ab. Was bei uns vereinzelt in Gebilden wie „bekritteln“ vorkommt, ist dort ganz allgemein, und auch dem heimischen Lautwesen muss sich das Fremde fügen. Nur wer die Heimathsprachen der Lehnwörter kennt, vermag diese auszusondern, und auch er kann sich täuschen. Wie z.~B., wenn zufällig ein urmalaischer Wortstamm \textit{kāta} vorläge, der mit jenem indischen Klang und Sinn gemein hätte? Solche Irrlichter flackern aller\-\fed{{\textbar}262{\textbar}}\phantomsection\label{fp.262}wärts. Der Mandschu nennt die Hose \textit{hosihon}, die Katze \textit{kesike}. In den malaischen Sprachen heisst \textit{duwa}: zwei, \textit{tolu}, \textit{toru} drei; unser grosser \textsc{Bopp} (Über die Verwandtschaft der malaisch-polynesischen Sprachen mit den indisch-europäischen, Berlin 1842) ist dem trügerischen Scheine zu seinem Schaden gefolgt, hat sogar Urverwandtschaft der malaischen Sprachen mit den arisch-indischen nachzuweisen versucht. Was dieser Verwandtschaft entgegensteht, ist nicht hier zu erörtern. Aber auch eine Entlehnung wäre höchst unwahrscheinlich; denn auch die übrigen Zahlwörter stimmen in den malaisch-polynesischen Sprachen sehr gut\footnote{Nur die Namen für 7, 8, 9 im eigentlichen Malaischen machen davon eine Ausnahme, stimmen aber auch nicht zu den indogermanischen, sondern sind einheimische Neuschöpfungen.}, mit den indogermanischen aber gar nicht überein.

Die Finnen, seit vorgeschichtlicher Zeit mit ihren germanischen Nachbarn in innigem Verkehre, \update{haben}{haben,} \sed{wie schon angedeutet,} von diesen eine Menge Culturwörter entlehnt und ihrem Lautwesen angepasst. Demzufolge vertreten hier die Tenues unsre Fricativen, \textit{k} steht statt \textit{h}, \textit{p} statt \textit{f}, während doch deutsches \textit{h} und \textit{f} ihrerseits an Stelle des ursprünglichen \textit{k} und \textit{p} getreten sind. Man hat vorschnell geschlossen, jene Entlehnungen müssten geschehen sein, ehe sich diese Lautverschiebung vollzogen hatte: die finnische Lautgestaltung konnte \sed{{\textbar}{\textbar}267{\textbar}{\textbar}}\phantomsection\label{sp.267} so wie so nicht anders ausfallen. \sed{Andere Male können alte Entlehnungen für die Sprach- und Culturgeschichte recht wichtig werden. Die Wörter \so{Kirsche}, \so{Kiste}, \so{Kerker}, \so{Kicher}(-erbse), \so{Pfirsich}, \so{Pflaume} und noch manche andere beweisen durch ihr Lautwesen, dass sie schon vor Eintritt der althochdeutschen Lautverschiebung aus dem Lateinischen in die germanischen Sprachen herüber genommen worden sind, – jene, die das \textit{c} vor \textit{e} und \textit{i} durch \textit{k} wiedergeben, bezeugen zudem die damalige Aussprache des lateinischen Lautes.}

Doppelformen, Wortpaare mit verwandten Lauten und gleicher oder ähnlicher Bedeutung, sind zumal da häufig, wo Entlehnung von benachbarten Dialekten oder nahe verwandten Sprachen stattgefunden hat. Von solchen Doubletten hat die für \update{ächt}{echt} zu gelten, deren Lautform den Gesetzen der betreffenden Sprache gemäss ist. So bekunden sich \so{Born}, \so{Schacht}, \so{ducken} neben den hochdeutschen \so{Brunnen}, \so{Schaft}, \so{tauchen} als niederdeutsche Eindringlinge. Bekanntlich ist das französische besonders reich an solchen Beispielen: \textit{cause}, \textit{cavalier}, \textit{escalade}, \textit{potion}, \textit{captif} neben \textit{chose}, \textit{chevalier}, \textit{échelle}, \textit{poison}, \textit{chétif}. In China haben seit Jahrtausenden die Dialekte in Austauschverkehr gestanden, und dort ist denn auch jenes Doublettenwesen wuchernd gediehen, – der Sprache zur Bereicherung, den Sprachvergleichern zur Plage.

Interessant sind die Namen unserer wichtigsten Culturpflanzen. Viele derselben stimmen in den europäischen Sprachen lautgesetzlich so überein, dass sie, nach ihrer Form zu schliessen, sehr wohl ursprüng\-\fed{{\textbar}263{\textbar}}\phantomsection\label{fp.263}liches Gemeingut der europäisch-indogermanischen Völker sein könnten. Dagegen aber erheben Geschichte, Archäologie und Pflanzengeographie so gewichtigen Einspruch, dass man diesmal annehmen muss, die Nordvölker haben jene Pflanzen und ihre Namen von ihren reicheren südlichen Stammesvettern empfangen, ehe die Spaltung des Lautwesens erfolgt war. In solchen Fällen hat der Sprachforscher mit geborgten Werkzeugen zu arbeiten. \sed{\so{Hanf} passt lautlich zu κάναβις, allein das Wort scheint keine indogermanische Etymologie zu haben, ebensowenig wie das gleichbedeutende arabische \textit{qinnabun}, \textit{qunnabun} eine semitische. Wir müssen annehmen, dass unsere Vorfahren das Wort sehr frühe, jedenfalls vor dem Eintritte der germanischen Lautverschiebung, von einem stammfremden Volke entlehnt haben, – ob von den Semiten, oder aus der gleichen Quelle wie diese, wird schwer zu ermitteln sein.}

Mit der lautlichen Anpassung ist es nicht immer gethan. Witz oder Unwissenheit haben manchmal die Lehnwörter durch ähnlich klingende, leidlich bezeichnende heimische Wortgebilde ersetzt: \textit{arcoballista} durch \so{Armbrust}, \textit{rondel} durch \so{Rundtheil}, \textit{planchette} durch \so{Blankscheit}, \textit{radical} durch \retro{rattenkahl,}{rattenkahl.} \fed{das maurisch-spanische \textit{alberga} durch \so{Herberge}.} \update{[\textit{in den Berichtigungen, S.~502}: streiche das Beispiel: Herberge.]}{} \sed{Aus \textit{biscuit} wurde, mit Verwendung des einheimischen Präfixes be-, im Holländischen \textit{beschuit}, im westphälischen Dialekte \so{Beschütchen}; \textit{asparagus} oder \textit{asperges} hat der englische Volksmund in \textit{sparrow-grass}, Sperlingsgras, um\-}{\textbar}{\textbar}268{\textbar}{\textbar}\phantomsection\label{sp.268}\sed{gewandelt.} Der Wessier, \textit{fers}, des persischen Schachspiels wurde im mittelalterlichen Französisch \textit{fierce}, \textit{fierge}, später \textit{la vierge} und dann \textit{la dame} genannt.

\sed{Zum Schlusse wollen wir versuchen, die Fälle, in denen eine Entlehnung zu vermuthen ist, übersichtlich zu ordnen. Wir sahen, die Gründe können sprachlicher, aber auch sachlicher, culturgeschichtlicher oder geographischer Natur sein.}

\sed{I. Die \so{sprachlichen} sind entweder phonetische oder etymologische.}
\sed{
\begin{compactenum}
\item Die \so{phonetischen} können entweder
\begin{compactenum}[a.]
\item das \so{Lautwesen an sich} betreffen, z.~B. englisches \textit{w} in Whist,\\
– oder
\item in den Gesetzen der \so{Lautverschiebung} beruhen, indem ein Wort dem entsprechenden einer anderen Sprache ähnlicher klingt, als es die regelmässige Lautvertretung zulassen würde.
\end{compactenum}
\item Die \so{Etymologie} ist entweder
\begin{compactenum}[a.]
\item in der heimischen Sprachfamilie überhaupt nicht zu entdecken\\
oder
\item wohl klar, aber dem Sinne nach nicht befriedigend. So meist bei den Volksetymologien, die freilich auch Alteinheimisches verunstalten können.
\end{compactenum}
\end{compactenum}
}

\sed{II. Die \so{sachlichen} Erwägungen laufen alle auf die eine Frage hinaus: Ist es anzunehmen, dass dem Volke die betreffende Vorstellung von Hause aus geläufig war? Und wenn dies zu verneinen ist: bleibt nicht die andere Möglichkeit, dass doch ein altheimisches Wort die Vertretung des neuen, fremden Begriffes übernommen hat? Man denke an das Wort \so{Taufe}, \so{beichten} und an jene Menge internationaler Culturbedürfnisse und Begriffe, für die wir, und mehr noch manche andere Völker, einheimische Namen gebildet haben, wo also die Entlehnung nicht dem Wortkörper sondern dem Sinne gilt, und höchstens Nachbildungen vorliegen.}

\sed{Übrigens giebt es kaum ein bequemeres Findelhaus für die unbequemen Kinder einer Phonetik, die auf ihren tadellosen Ruf hält. Und das eben macht die Sache bedenklich: man wittert Entlehnungen, wo man gar keinen Grund zur Entlehnung einsehen kann. Im Deutschen scheint das dialektische \so{Wepse} älter zu sein, als das schriftübliche \so{Wespe}. Letzteres, – also wohl auch die Engländer ihr \textit{wasp}, – sollen wir als eine Nachbildung des lateinischen \textit{vespa} anerkennen, – und nun hätten es wieder die Franzosen mit \corr{1901}{ihrem}{ihren} \textit{guêpe} uns nachgemacht? Warum und wozu? Ist da nicht die erklärende Hypothese ebenso überraschend, wie die zu erklärende Thatsache?}

\sed{{\textbar}{\textbar}269{\textbar}{\textbar}}\phantomsection\label{sp.269}

\pdfbookmark[2]{II. §. 19. Beeinflussung des Lautwesens durch Nachbarsprachen und -Dialekte.}{III.II.II.19}
\cohead{II. §. 19. Beeinflussing des Lautwesens durch Nachbarsprachen und -Dialekte.}
\subsection*{§. 19.}\phantomsection\label{III.II.II.19}
\subsection*{Beeinflussung des Lautwesens durch Nachbarsprachen und –Dialekte.}
Jede Sprache und jede Mundart erfordert und erzeugt gewisse Gewohnheiten der Sprach- und Stimmorgane, die ihren richtigen Klang bedingen, und die man sich aneignen muss, um das fremde Idiom zwanglos richtig zu sprechen. Diese Aneignung geschieht unfehlbar und unwillkürlich in zweisprachigen Landschaften, z.~B. in den Ostseeprovinzen, wo der deutsche Herr mit seinen Bauern und Dienstboten Lettisch oder Esthnisch reden muss. Diese Sprachen haben ihn von Kindheit an umklungen, sein Organ mit gebildet, und das seit Reihen von Geschlechtern. Es konnte nicht ausbleiben, dass die Eigenart der Lautbildung und des Tonfalles allgemach aus den Sprachen der Ureinwohner in die der Ansiedler überging. \update{Mouilli\-rungen}{Mouilli\-rungen,} wie sie das Deutsch der Kurländer besitzt, kennt sonst kein deutscher Dialekt.

Die slavischen Sprachen haben das dicke \textit{ł} und das dumpfe Ы mit den mongolischen und türkischen gemeinsam, vermuthlich eine Nachwirkung alter Völkermischung. Die von Deutschen umringten Czechen dagegen haben beide Laute aufgegeben. Umgekehrt besitzen unter allen Zigeunern nur die russischen das dicke \textit{ł}. \sed{Derselbe Laut war auch den Franzosen und den ihnen benachbarten Niederländern gemein und hat sich in beiden Sprachen zu \textit{u} vocalisirt. Im Spanisch-Baskischen hat \textit{j} den spanischen Laut \textit{ch} in „machen“, beibehalten, während es im Französisch-Baskischen theils den ursprünglichen Laut eines consonantischen \textit{i} behalten, theils den Wandel zu \textit{ž} mitgemacht hat.} Die indische Zweitheilung zwischen Cerebralen und Dentalen hat man längst dem Einflusse der drâvidischen Ureinwohner zugeschrieben; und von den Hottentotten und \fed{{\textbar}264{\textbar}}\phantomsection\label{fp.264} Buschmännern haben die benachbarten Kaffern und Zulus, sie allein unter den Bantuvölkern, die Schnalzlaute angenommen. \sed{Die Schwaben und Pfälzer in unserm Westen haben nasalirte Vocale, wie die Franzosen; und diese und ein Theil der Norditaliener haben die Laute \textit{ö} und \textit{ü} wie die ihnen benachbarten Deutschen. Das Lautwesen der malaischen Tscham (\textit{cham}) im Südosten der transgangetischen Halbinsel erinnert auffallend an das der benachbarten Môn-Annam-Sprachen. Das kann nicht blosser Zufall sein; es kehrt zu oft und an zu verschiedenen Stellen wieder. Die Nachbarn beeinflussen einander, sei es innerhalb derselben Sprache, von Mundart zu Mundart, sei es über die Grenzen der Sprachen und Sprachsippen hinaus.}

Gerade in dieser Hinsicht kann man oft bei Einzelnen ganz wunderliche mundartliche Schichtungen beobachten. Ein mecklenburgischer Handwerker hatte viele Jahre lang in Württemberg \update{gearbeitet}{gearbeitet,} und sich schliesslich in Sachsen niedergelassen. In der Laut- und Tonbildung seiner Sprache war nun sehr deut\-\sed{{\textbar}{\textbar}270{\textbar}{\textbar}}\phantomsection\label{sp.270}lich die mecklenburgische Grundlage, darüber eine sehr starke schwäbische, endlich eine dünnere obersächsische Schicht zu unterscheiden.

\sed{Ein merkwürdiges Beispiel vom Gegentheile, von zähem Beibehalten vererbter Lautgewohnheiten dürften die Juden bieten. Ihr eigenthümlich gelispeltes \textit{s} und schnarchendes \textit{r}, ihr breitzüngiges \textit{sch}, ihr Verweilen auf der anlautenden Liquida („Mmann, Nname“ u.~s.~w.), wohl auch den Tonfall ihrer Stimme, können sie nicht in Europa und von den Indogermanen erworben haben; das müssen Erbstücke aus Palästina sein.}

\sed{Ganz wunderlich ist, was einer unserer Colonialbeamten, Herr Regierungs-Assessor \textsc{Aug. Köhler}, aus Deutsch-Südwest-Afrika berichtet. Dort wohnen bantuische Damaras neben Nama-Hottentotten. Viele der Ersteren können die hottentottischen Schnalzlaute nicht aussprechen, lassen sie daher einfach weg. Aber auch manche Namas haben die Fähigkeit, den cerebralen Schnalzer ihrer Muttersprache zu bilden, verloren und ersetzen ihn nun durch ein ähnlich klingendes Schnippen mit dem Mittelfinger gegen den Daumen. Der Fall ist der einzige mir bekannte, wohl auch der einzig mögliche, wo andere als die natürlichen Sprachorgane mit verwendet werden, um die akustische Wirkung der Rede zu vervollständigen.}

\pdfbookmark[2]{II. §. 20. Entlehnte Redeweisen.}{III.II.II.20}
\cohead{II. §. 20. Entlehnte Redeweisen.}
\subsection*{§. 20.}\phantomsection\label{III.II.II.20}
\subsection*{Entlehnte \sed{Redeweisen, Einführung fremder grammatischer und stilistischer Formen.}}
\update{Redeweisen.}{}
Wer viel mit und in einer fremden Sprache verkehrt, kommt leicht dahin, dass er unversehens syntaktische und stilistische Eigenthümlichkeiten von ihr in seine Muttersprache hinüberträgt. Erst erscheinen Germanismen in unserm Französisch oder Englisch, dann, wenn wir eine Weile im Auslande gelebt haben, Gallicismen oder Anglicismen in unserm Deutsch. Die fremde Sprache hat uns an neue Denkformen und Ideenassociationen gewöhnt, die bleiben uns nun wohl für’s Leben und drängen nach Formung. Und ihre Form ist schon vorgebildet, wir brauchen sie nur in der Muttersprache nachzuahmen. Dass wir dieser damit Zwang anthuen, fällt uns kaum ein, uns ist, als hätten wir nie anders denken und sprechen lernen. Nun braucht das Neue nur nicht gar zu sprachwidrig, und wo möglich treffend und bequem zu sein, so kann es Anklang finden und von Anderen wiederholt werden. Da widersetzt sich aber die zähe Macht der Gewohnheit, und der Eindringling hat einen um so schwereren Stand, je grösser und fester jene Macht ist. Die Tausende deutscher Miethssoldaten und Matrosen, die aus englischen und niederländischen Diensten heimkehren, werden selbst am Plattdeutschen nur wenige dauernde Änderungen geschaffen haben, denn das Übergewicht der Daheimgebliebenen war zu stark. Dagegen wimmelt das Wendische der Ober- und Niederlausitz von deutschen Redewendungen, die die ausgewanderten slavischen Dienstboten und Arbeiter mit nach Hause bringen. Es \sed{{\textbar}{\textbar}271{\textbar}{\textbar}}\phantomsection\label{sp.271} lohnte sich der Mühe zu untersuchen, welche Neuerungen das Schwedische den Soldaten Gustav Adolfs verdanke.

Doch wirksamer noch scheint mir eine andere Art der syntaktisch-\fed{{\textbar}265{\textbar}}\phantomsection\label{fp.265}stilis\-tischen Aneignung. Diejenige meine ich, die sich bewusst oder unbewusst zu vollziehen pflegt, wo immer das Geistesleben eines Volkes durch fremde Literaturen befruchtet wird. Es waren wenige und zunächst nur enge Kreise, von denen die Gesittungen der alten Welt ihren Ausgang genommen haben. Die chinesische Cultur, zu Anfang der Geschichte auf ein Gebiet von der ungefähren Grösse des heutigen Deutschland beschränkt, hat sich über ein Drittheil der Menschheit verbreitet; die Mandschu, Japaner, Koreaner und Annamiten schöpfen ihre Bildung aus der Litteratur des Mittelreiches, durchwirken ihre Rede mit chinesischen Fremdwörtern, ahmen wohl auch chinesische Stil- und Satzformen in ihren Muttersprachen nach. Die Japaner wenigstens haben dem seltsam eintönigen kettenförmigen Periodenbaue, der ihre alte Sprache auszeichnete, je länger je mehr entsagt. \sed{In den uralaltaischen Sprachen steht regelmässig das Verbum am Ende des Satzes, und das Object geht ihm voraus. Vereinzelte Ausnahmen kommen allerdings selbst im Jakutischen vor, das doch wohl nur von stamm- und geistesverwandten Nachbarsprachen Beimischungen empfangen haben wird. Wenn aber im Suomi das Object seinen Platz hinter dem Verbum hat, wenn ebenda, allem uralaltaischen Brauche entgegen, das attributive Adjectivum mit seinem Substantivum in Numerus und Casus congruirt: so ist dies offenbar indogermanischem Einflusse zuzuschreiben.} Dank dem Buddhismus stehen im Norden die Tibetaner und Mongolen, im Süden viele Völker Hinterindiens und der malaischen Inselwelt unter indischem Einflusse. Davon zeugen bei ihnen allen so und soviele Wörter der Sanskrit- und Palisprache, die in ihren Sprachen Aufnahme gefunden, bei den Tibetern zudem wunderliche, nach indischem Muster gemodelte Composita und Constructionen. Wo der Islâm herrscht, da haben arabische Wörter in Masse Einzug gehalten, zuweilen ihre Formenlehre und Syntax für sich behauptend; Osmanli-Türkisch, Persisch, Hindustanisch, wie sie heute gesprochen und geschrieben werden, kann man nicht \update{verstehen,}{verstehen} ohne eine Vorschule im Arabischen. An der griechisch-römischen Prosa hat sich, glücklich nachbildend, die Syntax der neueuropäischen Sprachen erzogen, bis ihr Frankreich ein neues, gefälligeres Muster lieferte. Dort hatte sich in einer verfeinerten leichtlebigen Geselligkeit eine Umgangssprache von prickelnder Anmuth herangebildet, ganz frei und doch ganz geschult, so recht von vornehmer Art. Wer dieser Sprache mächtig war, der brauchte nur zu schreiben, wie er zu reden pflegte, so war er des Beifalles sicher. Schon früher mochten Deutsche und Andere frischweg geschrieben haben, wie es ihnen von der Leber ging. Obenan ist \textsc{Luther} zu nennen, der mit Recht als Schöpfer der neuhochdeutschen Schriftsprache gilt. Vorbild ist er aber doch nur durch diejenigen Schriften ge\-\sed{{\textbar}{\textbar}272{\textbar}{\textbar}}\phantomsection\label{sp.272}worden, in denen er der überströmenden Kraft seiner derb leidenschaftlichen Natur Zwang anthun musste; seine sprudelnde Frische war nicht nachahmbar, seine hainbüchene Grobheit nicht einmal nachahmenswerth. Der Stil der Beamten und Gelehrten ist noch lange, vielfach bis auf den heutigen Tag \fed{{\textbar}266{\textbar}}\phantomsection\label{fp.266} dem griechisch-römischen Muster mit seinen in breitem Strome \update{dahin fliessenden,}{dahinfliessenden,} sorgsam aneinander gereihten Perioden gefolgt, und dieser Anregung zum Periodenbau verdanken wir doch eigentlich die grössten Vorzüge unserer Syntax und den Geschmack für eine edele Form der prosaischen Rede. Allein \update{beides}{Beides} zu vereinigen, die edele Haltung mit der leichten Beweglichkeit, das haben wir und wohl auch andere Völker erst bei den Franzosen gelernt. Denen danken wir es zumal, dass uns jene zwiebelförmig ineinandergeschachtelten Satzformen nicht mehr behagen, die unserer Sprache doch so natürlich sind. Wahre Gallicismen haben sich aber \update{auch dabei}{dabei auch} mit eingeschlichen, Redewendungen, die oft den wissenschaftlichen Kenner mehr befremden als andere Leute. Den neuromanischen Sprachen sind gewisse \retro{participiale}{participialen} und gerundiale Constructionen gemeinsam, die das Lateinische nicht kennt, die aber im Holländischen, Englischen und den neuscandinavischen Sprachen in überraschender Einhelligkeit wiederkehren. Ich weiss nicht, welcher Theil hier der entlehnende war; jedenfalls sind unserm jetzigen Deutsch diese Satzformen fremd. Dem berufsmässigen Übersetzer aber können sie doch geläufig werden, und nun fliessen sie unversehens aus der Feder, und Tausende von Lesern lernen sich mit ihnen aussöhnen. Es lässt mehr apart als garstig: „Den Boten abgefertigt, kehrte er in die Gesellschaft zurück.“ Ein anderer ganz eingebürgerter Gallicismus ist es, wenn wir mit \so{um} zu verknüpfen, was nur zeitlich, nicht absichtlich aufeinanderfolgt. „Ein Brief wurde ihm gebracht, welchen er las, um sodann zum Spiele zurückzukehren.“ Das ist gut französisch und immer noch erträglicher als das schlechte Deutsch: „welchen er las \so{und} sodann zum Spiele zurückkehrte.“ \largerpage[2]\sed{Wir haben unsern romanischen Fremdwörtern Mischgebilde wie „stolzieren, schnabelieren, hofieren, Grobian, Läuteration“ (im älteren sächsischen Processrechte) nachgeschaffen. Das sind Ausnahmen, die immer absonderlich lassen. Die Engländer aber nehmen Wörter wie „\textit{whimsical}, \textit{truism}, \textit{eatables}“ arglos hin, und ich habe gelesen, wie arme Geschöpfe, die sich Fusstritte gefallen lassen müssen, frischweg als „\textit{kickable}“ bezeichnet wurden. In dieser immergrünen, zeugungskräftigen Agglutination scheint das Fremde längst Vollbürgerrecht erlangt zu haben.}

Bei uns wie bei den Römern wurde doch das griechische Reis auf einen verwandten Stamm gepfropft. Die Ägypter (Kopten) aber und die Äthiopier hatten ganz andere Schwierigkeiten zu überwinden, als sie begannen griechisch-christliche Werke in ihre so völlig artverschiedenen Sprachen zu übersetzen. Die Syntax der Ge\texttt{Ꜣ}ez-Literatur verleugnet gänzlich die semitische Art, und die Kopten haben es nicht für Raub geachtet, ihre Bücher mit griechischen Hülfs\-\sed{{\textbar}{\textbar}273{\textbar}{\textbar}}\phantomsection\label{sp.273}wör\-tern zu spicken. Ähnlich haben es die Syrjänen, ein Volk finnischen Stammes, mit russischen Partikeln gemacht. \sed{Wenn sich die Magyaren einen Artikel, \textit{a}, \textit{az}, geschaffen haben, so wird dies indogermanischem Einflusse zugeschrieben. Unerklärt, und doch schwerlich ein Zufall ist es, dass im südöstlichen Europa drei benachbarte und doch sonst so verschiedene Sprachen suffigirte Artikel haben: Das Rumänische, das Albanesische und das Bulgarische, die einzige slavische Sprache, die überhaupt einen Artikel besitzt. Das Annamitische, das Cochinchinesische (Khmêr) und das Peguanische (Talaing, Môn) sind, wie man jetzt weiss, den reich agglutinirenden kolarischen Sprachen verwandt, aber, gleich ihren indochinesischen Nachbarn der siamesischen und barmanischen Familie, einsylbig und isolirend. So ergiebt sich die wunderbare Thatsache, dass eine so seltene Form des Sprachbaues auf einem verhältnissmässig engen geographischen Gebiete bei Sprachen zweier ganz verschiedener Stämme wiederkehrt.}

\fed{{\textbar}267{\textbar}}\phantomsection\label{fp.267}

\pdfbookmark[2]{II. §. 21. Sprachmischung innerhalb der Muttersprache.}{III.II.II.21}
\cohead{II. §. 21. Sprachmischung innerhalb der Muttersprache.}
\subsection*{§. 21.}\phantomsection\label{III.II.II.21}
\subsection*{Sprachmischung innerhalb der Muttersprache.}
So widersinnig es klingen mag, so wahr ist es doch, dass alle Sprachgeschichte von beständiger Sprachmischung begleitet ist. Jeder Mensch hat seine eigene Sprache, die von der eines jeden anderen in gewissen Punkten verschieden ist. Ich rede nun nicht von den Associationen des im einzelnen Menschen schon vorhandenen Sprachgutes, nicht von dem Analogiegefühle, das seine Rede beherrscht, sondern von dem Einflusse, den \update{Alles}{Alles,} was er hört oder liest, auf sein Sprachvermögen ausübt.

Wo immer es sich um \update{ächte}{echte} Entwickelung, also um allmähliches Werden handelt, da haben wir mit kleinsten Kräften und kleinsten Wirkungen zu rechnen; – wer weiss, ob nicht die vielen kleinen Posten eine grosse Summe ergeben werden? Was den Menschen erzieht, sind die Eindrücke, die er \update{empfängt,}{empfängt;} und man weiss, dass oft sehr geringfügige Ereignisse einen sehr nachhaltigen Eindruck verursachen. Man weiss auch, dass kein Eindruck, er sei noch so leicht, ganz ohne Nachwirkung bleibt, – etwas hat er allemal an uns verändert. So lassen denn auch alle sprachlichen Eindrücke, die wir empfangen, ihre Nachwirkungen in unsrer Seele zurück und werden seiner Zeit, früher oder später, in unsern Sprachäusserungen \update{nachwirken.}{durchschimmern.} Wir hören oder lesen einen uns neuen Ausdruck oder eine Redewendung, vernehmen eine Mundart, die uns noch nicht geläufig ist, so sind mehrere Möglichkeiten gegeben. Das Neue fällt uns auf, wir merken es uns, das heisst, wir vermehren damit unsern sprachlichen Besitz. Oder wir gleiten scheinbar darüber hinweg, verwerfen es wohl gar; später kehrt es wieder: nun ist es schon nicht mehr ganz neu, findet den Boden schon besser vorbereitet, um Wurzel zu fassen, und bei öfterer \sed{{\textbar}{\textbar}274{\textbar}{\textbar}}\phantomsection\label{sp.274} Wiederholung nistet es sich denn doch ein. Schon das \update{aber ist}{ist aber} eine Veränderung der potenziell in uns lebenden Sprache (des Sprachvermögens), dass wir für etwas Neues um ein Weniges empfänglicher gemacht werden. So schleichen sich oft fremde Provinzialismen oder dialektische Lautbildungsweisen in unsere Rede ein, und wenn wir uns des Fremden erwehren wollen, so begeben wir uns unter die Macht des Gegensatzes, lassen von diesem unsere Rede mitbestimmen und carikiren sie schliesslich wohl selbst, indem wir ihre \fed{{\textbar}268{\textbar}}\phantomsection\label{fp.268} Eigenart übertreiben. Solche Fälle gehören zu den selteneren, dafür sind sie aber auch um so wahrnehmbarer. Man beobachte nur einen unserer Angehörigen, der von einer längeren Reise zurückgekehrt ist: ganz so, wie er sie auf den Weg mitgenommen hat, bringt er seine Muttersprache nicht zurück.

Dies von den neuen, fremden Eindrücken. Aber auch die gewohnten wirken durch ihre Wiederholung. Sie festigen sich je mehr und mehr und können am Ende andere, minder gewohnte verdrängen. Jede Sprache hat Synonymausdrücke, die sich in ihren Bedeutungen nur schwach gegeneinander abschattiren, und die wenigsten Menschen beherrschen diese Synonymik in dem Umfange, dass sie nicht gewisse Ausdrücke gewohnheitsmässig bevorzugten. Die Conjunctionen \update{aber,}{„aber,} allein, jedoch, \update{indessen}{indessen“} wenden die Meisten fast unterschiedslos an, wechseln höchstens unter ihnen ab eben um der Abwechselung willen. Eine Art unbewusster Vorliebe aber für die eine oder die andere hat wohl Jeder, und kaum Einer wird allen Vieren gleiche Gerechtigkeit widerfahren lassen. Nun verkehre ich mit einem Menschen oder lese einen Schriftsteller, der fortwährend das mir wenig geläufige „jedoch“ anwendet. Das mag mich anfangs stören, nachher aber wird es mir gewohnt, und schliesslich wende ich es selbst an, natürlich auf Kosten der Aber, Allein und Indessen, mit denen ich mich früher beholfen hatte. So können beliebte Schriftsteller innerhalb ihrer Leserwelt wahre sprachliche Epidemien verbreiten, man hat es nur nicht Acht.

\sed{Wenn wir von dialektischen Spaltungen oder von aufeinanderfolgenden Phasen der Sprachgeschichte reden, so müssen wir immer daran denken, dass es sich doch regelmässig nur um allmähliche Abschattungen handelt. Jeder hat seine Individualsprache, Jeder seinen sprachlichen Verkehrskreis, dem er mittheilt, von dem er empfängt. Die Kreise seiner Genossen schneiden sich mit dem seinigen und mit den Kreisen Anderer, die ihm fremd sind. Und so geht es weiter, von A durch B, C u.~s.~w. bis Z, bis zu einer Entfernung, in der die Gemeinschaft der Mundart oder gar der Sprache mit A aufgehört hat. An einer früheren Stelle (S.~56 flg.) sahen wir, dass die Sprachgemeinschaft so weit reicht, wie die Möglichkeit des sprachlichen Verkehres. Jetzt sehen wir, dass dieser Verkehr die Sprachgemeinschaft bedingt und erhält, und dass Letztere da aufhört, wo der Mittelglieder zu viele, oder wo sie ausgefallen sind.}

Es leuchtet ein, dass unsere eigenen Sprachäusserungen nicht mindere \sed{{\textbar}{\textbar}275{\textbar}{\textbar}}\phantomsection\label{sp.275} Wirkungen auf uns ausüben, als die anderer Leute; denn wir hören und lesen ja \update{auch}{auch,} was wir selbst reden und schreiben. Mehr noch, wer nur einigermassen gesprächig ist, der hört nicht leicht einen Anderen öfter reden als sich selbst, – und wie oft reden wir in uns, wenn wir uns die eigenen Gedanken in Worten vergegenständlichen! So nun entstehen wohl am \update{häufigsten}{Häufigsten} die Manieren im guten und üblen Sinne: wozu wir neigen, das wird uns gewohnt, verdrängt Anderes und artet am Ende in Einseitigkeit und Sonderlingsthum aus. Dann aber sind es auch wieder die ausgeprägtesten Individualitäten, die am Mächtigsten ihre Umgebung beeinflussen; und grosse Männer, die nach Zeit und Raum weithin wirken, sind meist solche Individualitäten. Man mag die Sprache eines \textsc{Hegel} schmähen soviel man will, leugnen kann man nicht, dass auch sie in ihren Wirkungen fortlebt bis auf den heutigen Tag. Man möge \textsc{Richard} \fed{{\textbar}269{\textbar}}\phantomsection\label{fp.269} \textsc{Wagner}’s Sprachspielereien noch so sehr geisseln, – ihren Nachhall wird man so bald nicht zum Schweigen bringen. Man lache über \textsc{Victor Hugo}’s pointirtes Gethue, man verbrenne das letzte Exemplar seiner Schriften: die Kinder seines Geistes kann man vertilgen; aber die Enkel, die Schaaren Jener, die sich an ihm gebildet haben, werden weiter von ihm zeugen. Das Gesetz von der Erhaltung der Kraft gilt hier wie allerwärts.

Will man die Nachwirkungen solcher leisester Anstösse recht ermessen, so wird man wohl auch das Geistesleben der Schlafenden mit in Rechnung ziehen müssen. Der Geist ruht ja nicht; er arbeitet fort, auch wenn ihm die Sinne keine Empfindungen zutragen. Er verarbeitet empfangene Eindrücke, und von diesen beschäftigen ihn natürlich die lebhaftesten am meisten. Den lebhaftesten Eindruck macht aber auf uns das, was uns neu oder ungewohnt ist, auch das sprachlich Ungewohnte. Ich habe es oft erlebt: wenn ich am Tage fleissig in einer fremden Sprache gelesen oder gesprochen \update{habe, [\textit{\mbox{in den} Berichtigungen, S.~502}: hatte,]}{hatte;} so träumte ich des Nachts in ihr, las weiter in dem interessanten Buche, plauderte, – wer weiss wie fehlerhaft, jedenfalls aber mit erstaunlicher \update{Geläufigkeit}{Geläufigkeit,} in der fremden Sprache. So in den Fällen, wo ich nach dem Erwachen von dem Treiben meines Geistes während des Schlafes noch eine Erinnerung hatte, wo ich also sagen konnte: \update{das}{Das} und das habe ich geträumt. Offenbar thuen wir aber das Gleiche auch ohne dass wir am Morgen noch davon wissen. Und doch muss die Nachwirkung auf unseren Geist \update{bleiben:}{bleiben,} wir haben in der That in der fremden Sprache weiter gearbeitet, also weiter gelernt, wir haben den Fremden in seiner eigenthümlichen Weise reden hören, soviel mehr als vorher im wachen Zustande. Und wenn hernach die Erinnerung nicht vor die Seele treten will, so ist doch, uns unbewusst, eine Neuerung in uns eingetreten, die nicht ohne Folgen bleiben kann. Wer daran zweifelt, der denke an das oft gehörte Wort: „Ich will es beschlafen, frage morgen wieder!“ Mit dem Räthsel sind wir zu Bette gegangen, mit der Lösung wachen wir auf.

\sed{{\textbar}{\textbar}276{\textbar}{\textbar}}\phantomsection\label{sp.276}

\largerpage[1]Unter den Wirkungen jener Mischungen innerhalb der Muttersprache scheint mir aber die die wichtigste zu sein, die ich als \so{Abstumpfung des sprachlichen Gewissens} bezeichnen möchte. \sed{Es ist bezeichnend und zugleich natürlich, dass in sprachlichen Dingen Niemand empfindlicher und unduldsamer ist, als wer immer nur die eine, heimische Mundart gehört hat.} Wo \sed{aber} in einer rasch durcheinanderwogenden Bevölkerung die Vertreter verschiedener Mundarten als Gleichberechtigte miteinander zu verkehren pflegen, da führt natürlich die heimische Mundart nicht mehr das ungestörte Sillleben, \fed{{\textbar}270{\textbar}}\phantomsection\label{fp.270} in dem allein sie gedeiht. Die Sprachweisen der Zu- und Durchwandernden werden uns fast ebenso heimisch, wie die unserer Nachbarn. Verschiedenes gilt uns für gleich richtig; \sed{die Kreise der Laute erweitern sich, und die Articulation wird unsicher; Synonymformen wie „ward“ und „wurde“, „frug“ und „fragte“ werden gleichgültig hingenommen, bald unterschiedslos gebraucht;} am Ende wissen wir \update{gar nicht}{garnicht} mehr, was bodenwüchsig ist, und was eingeschleppt, und wenden Fremdes und Einheimisches unterschiedslos durcheinander an. So entsteht der Unfug gleichwerthiger Doubletten, eine sprachliche Anarchie, die den Forscher zur Verzweiflung bringen kann; – das ägyptische Vulgärarabisch soll nach \textsc{Spitta Bey}’s Zeugnisse ein geradezu tolles Beispiel hierfür abgeben. Es muss eine Zeit der Ruhe kommen, oder es muss bereits eine classische Sprache vorhanden sein, wenn solche Zuchtlosigkeit nicht gar verderblich werden soll. Die Neigung, überflüssige Doppelformen zu beseitigen, wird immer bestehen; aber während noch die Gäste im Saale ein- und ausgehen, fegt man nicht die Dielen. \sed{Und dann ist es doch immer ein Zufall, welche von den gleichwerthigen Formen schliesslich zur Alleinherrschaft gelangt; hier wird dieser und dort jener Dialekt obsiegen, und so wird die Verwirrung im Lautwesen verewigt; es bleiben Nachwirkungen von Ursachen, denen man nicht mehr auf die Spur kommen kann.} Die vergleichende Indogermanistik hat bekanntlich mit solchen mundartlichen Doubletten harte Auseinandersetzungen zu bestehen.

\sed{Es übt aber jene Abstumpfung des sprachlichen Gewissens nach einer anderen Richtung hin eine sehr heilsame Wirkung. Jenen, die anders reden, als wir, gestehen wir stillschweigend zu, dass sie auch richtig reden. Wir erkennen die fremden Eigenthümlichkeiten, die individuellen und die mundartlichen, als berechtigt an, das heisst, wir erkennen an, dass sie in unserer Sprachgemeinde und Gemeinsprache statthaft sind. Damit erweitern sich die sprachgemeindlichen Grenzen, über die Mundarten hinweg, als über ihnen stehend, gilt die gemeinsame \so{Volkssprache}, und innerhalb dieser kann ein besonderer Dialekt zur \so{Schriftsprache} erhoben werden. So duldet das Hochdeutsch der Gebildeten eine Menge berechtigte mundartliche Eigenthümlichkeiten; das Schriftdeutsch aber, das ihm als Muster dient, ist aus dem niederösterreichischen Dialekte erwachsen und dann wieder, zumal in Obersachsen, im alten Meissner Lande, und} {\textbar}{\textbar}277{\textbar}{\textbar}\phantomsection\label{sp.277} \sed{nach dem dortigen Brauche entwickelt worden. Die Mischungen sind indess damit noch nicht abgeschlossen; denn so ganz lassen sich unsere Schriftsteller den Born der heimischen Mundart nicht verstopfen. Jede Schriftsprache einer grossen Nation ist dialektischer Mischungen verdächtig, – ihr zum Gewinne, den Sprachistorikern zum Verdrusse.}

\pdfbookmark[2]{II. §. 22. Einfluss der Kindersprache.}{III.II.II.22}
\cohead{II. §. 22. Einfluss der Kindersprache.}
\subsection*{§. 22.}\phantomsection\label{III.II.II.22}
\subsection*{Einfluss der Kindersprache.}
Dass die Kinder anders sprechen, als die Erwachsenen, beruht bekanntlich auf verschiedenen Ursachen.

Vor Allem auf dem ungeübten Sprachorgan. Die Gutturale und Zischlaute verursachen fast allen Kindern von Anfang an Schwierigkeiten, und das japanische Kind verwandelt das \textit{k} in ein \textit{t}, ganz wie es das europäische thut.

Zweitens bemeistert das kindliche Denkvermögen nicht mit einem Male alle Schwierigkeiten der Muttersprache; die unregelmässigen Formen werden durch regelmässigere, dem Kinde geläufigere ersetzt; \so{gebringt} und \so{gesingt} wird gesagt statt \so{gebracht} und \so{gesungen}.

Drittens übt wohl auch die tändelnd kosende Sprache der Erwachsenen ihren Einfluss auf die Redegewohnheiten des Kleinen. Man redet zu ihm von seinen Händchen, Füsschen, Öhrchen, und nun gebraucht es die Diminutive auch am unrechten Orte, nennt jede Hand ein Händchen u.~s.~w. 

Wer recht kinderlieb ist, der findet eine Wonne darin, sich den kleinen, schwachen Wesen zu fügen. Man unterwirft sich wohl auch einmal den Sprachgewohnheiten des Kindes, lallt mit, wenn man mit ihm \fed{{\textbar}271{\textbar}}\phantomsection\label{fp.271} redet, wohl gar wenn man zu seinen Angehörigen von ihm redet. Offenbar kann dies schliesslich die Sprache der Erwachsenen dauernd beeinflussen. So erkläre ich mir das Überhandnehmen der Diminutiva in den Sprachen der kinderfreundlichen Slaven und in einigen deutschen Dialekten, z.~B. dem ostpreussischen. Andere Male haben die kindlichen Lautverdrehungen Aufnahme gefunden, zumal bei Eigennamen wie italienisch \textit{Peppo} = \textit{Giuseppe}, \textit{Dick} = \textit{Richard}, \textit{Bob} = \textit{Robert}, \corr{1891 und 1901}{\textit{Peggy}}{\textit{Paggie}} = \textit{Margaret}. Auch Thiernamen, Namen von Spielzeugen (Puppe, \textit{joujou}) und Bezeichnungen von Dingen, über die man nur in der Kinderstube unbefangen redet, mögen der Kindersprache entlehnt sein.

Diese ist nun wohl sehr individuell; kaum zwei Geschwister, wenn sie nicht fast gleichalterige Gespielen sind, reden im gleichen Gelalle. Aber gewisse Eigenthümlichkeiten sind doch fast allverbreitet, weil sie eben sehr natürlich sind. So die Vermeidung schwieriger Consonantenverbindungen, die Angleichung von An- und Auslaut, die Neigung zu Doppelungen und wiederum zu Kürzungen. Und auf Seite der Eltern ist es natürlich, dass man die Kleinen mit \sed{{\textbar}{\textbar}278{\textbar}{\textbar}}\phantomsection\label{sp.278} den Namen nennt, die sie sich selber geben. Und darein kann wirklich Methode kommen. So sind unsere deutschen Kosenamen auf \textit{a} und \textit{o} (Arno, Bodo, Bertha, Frida u.~s.~w.), so die reduplicirten französischen (\textit{Dodore}, \textit{Lolotte}, \textit{Fifine}, \textit{Nénette} u.~s.~w.) nach einheitlich festen Prinzipien gebaut.

Eine Gewohnheit festigt sich um so leichter, je öfter sie geübt wird. Wo die Kinder im Hauswesen eine grosse Rolle spielen, da wird auch ihre Sprache nicht ganz ohne Einfluss bleiben. Und manchmal ist es, als griffen die Grossen den Kleinen vor, als könnten sie es nicht erwarten, bis sie die Kindersprache aus Kindermunde hörten. Das leise Gezwitscher mancher Vögel beim Nestbaue hat wohl in der Menschenwelt Seinesgleichen. Irre ich nicht, so ist es weit verbreitet, dass Liebende bei ihrem Gekose in die Kindersprache verfallen. Ich weiss nicht, soll es eine Erinnerung an die eigene Kinderzeit, soll es ein heimliches Versprechen sein, dass man sich gegenseitig hegen wolle wie ein geliebtes Kind, oder ist es eine unbewusste Ahnung dessen, was man dereinst gemeinsam lieben und hegen will? Es ist Sache der Sitte, ob sich dies Treiben auf trauliche Stunden unter vier Augen beschränkt oder sich weiter hinaus wagt.

Auf alle Fälle haben wir hier wieder einen jener irrationalen Factoren, die die Geschichte der Sprache beeinflussen, die Gleichmässigkeit \fed{{\textbar}272{\textbar}}\phantomsection\label{fp.272} ihrer Entwickelung durchbrechen können. Es sind ja eigentlich auch Naturlaute, die nachgeahmt werden; aber diese Laute gehören schon einer menschlichen Sprache an, und so hatten wir es hier mit einer Art der Sprachmischung zu thun.

\pdfbookmark[2]{II. §. 23. Eigentliche Mischsprachen.}{III.II.II.23}
\cohead{II. §. 23. Eigentliche Mischsprachen.}
\subsection*{§. 23.}\phantomsection\label{III.II.II.23}
\subsection*{Eigentliche Mischsprachen.}
Zwischen jenen vereinzelten Einwirkungen einer fremden Sprache auf die eigene: der Aufnahme von Fremdwörtern, der Nachbildung von Zusammensetzungen, Redensarten und syntaktischen Formen, der Einführung ausländischer Hülfswörter und Formativa einerseits und andererseits dem gänzlichen Aufgeben der Muttersprache zu Gunsten der fremden, liegt mitteninne eine Reihe unzähliger Möglichkeiten, die wohl alle in der Sprachgeschichte zu Thatsachen geworden sind. Die Mischung braucht ja keine gleichtheilige zu sein, ist es gewiss nur in den \update{aller\-seltsamsten [\textit{\mbox{in den} Berichtigungen, S.~502}: allerseltensten]}{allerseltensten} Fällen, war es vielleicht überall nur während einer kurzen Übergangszeit: schliesslich wird sich doch das Zünglein der Waage nach links oder rechts geneigt, das Heimische oder der Eindringling das Übergewicht gewonnen haben.

\largerpage[-1]\sed{Je verwandtschaftlich näher nun die sich mischenden Sprachen stehen, desto dornenvoller wird die sprachgeschichtliche Arbeit. Da erscheinen Wörter, die sich schlechterdings nicht in’s Lautsystem fügen wollen, und die gleichwohl unbestreitbar von der Sprache als Vollbürger behandelt werden. Mit unsern nieder\-}{\textbar}{\textbar}279{\textbar}{\textbar}\phantomsection\label{sp.279}\sed{deutschen Eindringlingen haben wir noch leichtes Spiel: über ihre Herkunft können wir in ihrer Heimath Auskunft erholen. Beim Lateinischen aber haben wir mit halbverschollenen Italersprachen zu rechnen; wir wissen das zehnte Mal nicht, wohin wir die Fremdlinge heimschicken sollen, und sind froh, wenn wir sie über die Grenze geschoben haben. Dagegen ist nichts einzuwenden, vorausgesetzt, dass es wirklich Fremdlinge waren. Allein unsere Unduldsamkeit in phonetischen Dingen kann uns wohl auch zu weit führen, wenn wir einem Heischesatze zuliebe die entgegenstehenden Thatsachen hinwegdecretiren, oder vorschnell fremde Mächte da annehmen, wo möglicherweise unerkannte heimische Lautgesetze gewaltet haben. Jetzt findet sich die Indogermanistik immer mehr darein, dialektische Nebenformen anzunehmen. Das böse Zahlwort Sechs: sanskrit \textit{šaš}, altbaktrisch \textit{χšwaš}, griechisch ἕξ, ϝέξ, lateinisch \textit{sex}, albanesisch \textit{gjašte} u.~s.~w., scheint überall verwandt und doch nicht auf eine einheitliche Urform zurückführbar. \textsc{Brugmann}, Grundriss der vergl. Gramm. II, \textsc{ii}, §. 170, nimmt deren drei an: \textit{ueks} (\textit{weqs}), \textit{sueks} (\textit{sweqs}) und \textit{seks} (\textit{seqs}, – \textit{q} = velarem \textit{k}, \textit{w} = halbvocalischem \textit{u}). Auch die schwierigen Doubletten mit Wechsel von Tenuis und Media, I, § 469, 7, 8, und gewiss noch manches Andere, was der Einreihung unter die Lautgesetze widersteht, dürfte, wo nicht aus unsicherer Articulation, doch aus mundartlichen Verschiedenheiten und Mischungen innerhalb der Ursprache herzuleiten sein.}

Dass sich eine Sprache während des ganzen Verlaufes ihrer Geschichte von fremden Einflüssen völlig frei gehalten hätte, ist von vornherein nicht zu vermuthen; in diesem weitesten Sinne mag ja jede Sprache für gemischt gelten. Die Genealogie hält sich nun an den Satz: Denominatio fit a potiori, ordnet eine jede Sprache derjenigen Familie zu, der sie der Hauptsache nach zugehört, und ist damit bis in die neueste Zeit gut gefahren. Mit den Erkenntnissen mehren sich aber auch die Probleme; und eben jetzt arbeitet unsere Forschung auf Gebieten, wo mit den Begriffen der vollbürtigen Verwandtschaften, der Entlehnung und Nachbildung nicht mehr Haus zu halten ist. Immer mehr wird sie auch daran denken müssen, dass durch annähernd gleichtheilige Mischungen stammverschiedener Sprachen neue, bastardische Gebilde entstehen konnten, halbbürtige Geschwister zweier Familien. Es war ein hohes Verdienst \textsc{Lucien Adam}’s und \textsc{Hugo Schuchardt}’s, dass sie jene verachteten Creolensprachen zergliedernd auf ihre Herkunft untersuchten. Es war auch sehr verständig, dass sie unter allen Blendlingssprachen gerade diese zuerst unter’s Messer nahmen, – die offenkundigsten, die einfachsten und ärm\-\fed{{\textbar}273{\textbar}}\phantomsection\label{fp.273}lichsten, die jüngsten, darum die, deren Bestandtheile noch am \update{deutlichsten}{Deutlichsten} das Gepräge ihres Ursprunges tragen.

Schon jedoch waren der Wissenschaft schwierigere Räthsel gestellt. In der südöstlichen Inselwelt leben zwischen braunen, mehr oder weniger mongoloiden Malaio-Polynesiern negerartige Menschen mit schwärzlicher Haut und krausem \sed{{\textbar}{\textbar}280{\textbar}{\textbar}}\phantomsection\label{sp.280} Haare. Die haben theils ganze Eilande und Archipele inne, wie Neu-Guinea und die melanesischen Gruppen, theils wohnen sie, scheinbar als zurückgedrängte Autochthonen, im Innern \update{malaischer}{Malaischer} Gebiete, – so auf Sumatra und den Philippinen. In körperlicher Hinsicht ähneln sie am Meisten den wilden Andamanenvölkern im Nordwesten, den Australnegern im Südosten. Sie zerfallen in unzählige, sprachlich verschiedene Völkerschaften, und ihre Sprachen, soweit sie damals zugänglich waren, hat zuerst mein seliger Vater vergleichend untersucht. Dabei ergab sich, wie er ausgesprochen hat, die interessante Thatsache, dass sie zu einem Stamme gehören, und dass sie mit den (malaio)-polynesischen Sprachen „mehr gemein haben, als aus einer blossen Entlehnung der einen von den anderen hervorgehen kann“. Mit anderen Worten: es ergab sich eine wahre Verwandtschaft (H. C. v. d. \textsc{Gabelentz}, die \update{melane\-sichen}{melane\-sischen} Sprachen. I. u. II. Abh. der K. S.~Ges. der Wissensch. III, S.~1–266, VII, 1–186. Vergl. dazu \so{meine} und A. B. \textsc{Meyer}’s Beiträge zur Kenntniss der melanesischen, mikronesischen und papuanischen Sprachen, das. XIX, S.~375–542 und R. H. \textsc{Codrington}, The Melanesian Languages, Oxford 1885). \retro{Jetzt}{Jezt} fragte es sich: Welcher Art ist die Verwandtschaft, ist sie vollbürtig oder halbbürtig? Im ersteren Falle war anzunehmen, dass der eine Theil die Sprache des anderen unter Aufgebung der eigenen angenommen habe. Ob aber der Braune vom Schwarzen, wie \textsc{Codrington} vermuthet, oder nicht umgekehrt der minder gesittete Schwarze vom rührigen, handeltreibenden Braunen, wie es mir glaubhafter scheinen würde? Ist \update{aber}{dagegen} die Verwandtschaft halbbürtig, so liegt Mischung vor, und zwar, wie die Thatsachen ergeben, in den verschiedenen Sprachen nach sehr verschiedenen Verhältnissen. Das Fidschi ist fast ganz polynesisch; die Sprachen der philippinischen Negritos (Zambales, Mariveles) scheinen sich eng an jene der benachbarten braunen Völker anzuschliessen; im Maré (Nengone) und Lifu scheinen die malaisch-polynesischen Elemente stark zurückzutreten; im Mafoor von Neu-Guinea überwiegen sie; in den Sprachen der Maclayküste sind sie nur in dürftiger Zahl zu erkennen. Hier war der fremde Einfluss schwächer, \fed{{\textbar}274{\textbar}}\phantomsection\label{fp.274} dort stärker. Doch dies ist das Wenigste. Liegt Mischung vor, welches ist das andere Mischungselement, das nicht malaio-polynesische?

Da drängte sich nun ein neues Problem dazwischen. Wie steht es mit den Australnegern? Sind sie wirklich sprachlich und anthropologisch so vereinsamt, ein „Saatwurf des Schöpfers“ für sich, wie man wohl gemeint hatte? Und wenn nicht: wer sind ihre nächsten Verwandten? Erwiesen sich als solche die Tasmanier, denen die Engländer unlängst die letzten Ehren erwiesen hatten, so war wenig gewonnen. Wo hätten diese ἔσχατοι ἀνδρῶν sonst hingehören sollen? Ähnlich war es mit den Papuas von Neu-Guinea, von denen und deren Sprachen man ohnehin noch sehr wenig wusste. Eine Anfrage bei den Melanesiern schien wenig zu versprechen. Deren Sprachen, soweit man sie \sed{{\textbar}{\textbar}281{\textbar}{\textbar}}\phantomsection\label{sp.281} kannte, waren doch in der inneren wie der äusseren Form zu verschieden von den australischen. Und dann wären doch auf alle Fälle eher die Insulaner wie Ausläufer der Festlandsbewohner erschienen, als umgekehrt. Mit Recht schaute \textsc{Bleek} hinüber nach dem grossen asiatischen Continente; aber sein Versuch, die australischen Sprachen mit den drâvidischen zu verbinden, hat gerechten Widerspruch erweckt. Die Methode hätte wohl erfordert, dass man zuerst die Sprachen der Australneger unter sich grammatisch und lexikalisch verglich, um das ihnen ursprünglich Gemeinsame festzustellen und damit dann weiter zu \update{hantieren.}{hantiren.} Allein die Hauptsache, die weithin innerhalb der Familie bestehende Gleichheit in Stoff und Form, lag ohnehin zu Tage, man brauchte nur hinzusehen; und Entdeckungen sind oft Geschenke des Zufalls, der sich an die zünftige Geschäftsordnung nicht bindet. Mich hatte ein Zufall erst zu den australischen und bald darauf zu den kolarischen Sprachen Vorderindiens geführt, und da stiessen mir Übereinstimmungen auf, die jedem Anderen in meiner Lage auch eingeleuchtet hätten, Gleichheiten in den Fürwörtern, den ersten Zahlwörtern, in Numerus- und Casusformen und in einem Theile der Substantiva. \sed{(Allgem. Encyklop. der Künste und Wissenschaften: „Kolarische Sprachen“, II. Sect., Bd. XXXVIII, \corr{1901}{S.}{S} 108.)} Hatte ich in der nancowry-nicobarischen Sprache reichliche malaische Elemente nachgewiesen, so gelang es \textsc{Ernst Kuhn}, in dieser und in den Sprachen der hinterindischen Môn-Annam-Familie kolarische Verwandtschaftsmerkmale aufzuzeigen; und nun hatte ich meinerseits ziemlich leichtes Spiel, den Wortschatz der neuguineischen Maclayküste mit in die Vergleichung zu ziehen. Das war doch derweile eine Etappe auf der Strasse von Indien nach Australien. Fortan darf man also wohl von einer \so{kolaro-australischen Sprachenfamilie} \fed{{\textbar}275{\textbar}}\phantomsection\label{fp.275} reden und von vornherein annehmen, dass zwischen den beiden geographischen Endpunkten Mischlinge sitzen. Was zunächst noch fehlt: der inductive Nachweis kolaro-australischer Bestandtheile in den melanesischen Sprachen, scheint eben geliefert werden zu sollen.

Ob uns aber nicht neue Überraschungen bevorstehen? Jetzt arbeiten neben den Holländern Deutsche und Engländer um die Wette an der Erforschung Neu-Guineas. Ein Heftchen: „British New Guinea Vocabularies“, London \update{(1888)}{(1888),} liegt vor mir. Ich erwartete, es sollte zu dem, was ich an den Sprachen der Maclay-Küste beobachtet hatte, neue \update{Bestäti\-gungen}{Bestäti\-gung} bringen, und finde mich enttäuscht. Sollte dort noch eine dritte Völkerschicht zu Tage treten? Oder wären nur die Zeichen der Urgemeinschaft das eine Mal ärger verwischt, als das \update{andere?}{andere Mal?} Wir dürfen unseren Enkeln nicht vorgreifen, wir stehen ja noch in den ärmlichsten \retro{Anfängen:}{Anfängen;} ein oder zwei Dutzend Sprachen und Mundarten, vielleicht von Hunderten; denn jenes grosse Eiland scheint ein wimmelndes Nest vielzungiger Völkerschaften zu sein. Und von jenen Sprachen besitzen wir vorläufig fast nichts als magere Vocabularien, die keinen Blick in die Tiefe gestatten.

\sed{{\textbar}{\textbar}282{\textbar}{\textbar}}\phantomsection\label{sp.282}

Gross und kühn gedacht ist jene Hypothese, die \textsc{Lepsius} in der Einleitung zu seiner Nubischen Grammatik über die genealogischen Verhältnisse der Sprachen Afrikas aufgestellt hat. Er theilt den grossen Continent in vier Zonen: die nördlichste ist die semitische, jetzt hamito-semitische, folgende Gruppen in sich begreifend: I. die ägyptische (Altägyptisch und Koptisch); II. die \retro{libysche,}{lybische,} der er ausser den Berbersprachen auch noch das Hausa zuzählt; III. die kuschitische: Bedscha, Schoho, Falascha, Agau, Galla, Dankali und Somali. Spätere Eindringlinge sind \retro{semitisch:}{semitisch;} IV. abessinisch: Ge\texttt{Ꜣ}ez, Tigre, Amharisch, \retro{Harari,}{Harari} und \mbox{V. Arabisch}.

\largerpage[1]Die zweite Zone theilt er nach geographischen Gründen in drei Gruppen: \linebreak\mbox{I. die westliche:} Efik, Ibo, Yoruba, Ewe, Akra, Tschi (Odschi), Kru (Grebo, \retro{Gedebo),}{Gedobo),} Mande (Mandingo, Soso, Bambara, Vei), Temne, Bullom, Wolof; II. die mittlere: Pul, Sonrhai, Kanuri, Teda, Logone, Wandala, Bagrima, Maba, Kondschara, Umale; III. die östliche: Dinka, Schilluck, \update{Bogo,}{Bongo,} Bari, Oigob, Nuba, Barea.

Die dritte, breiteste Zone füllen die Bantusprachen.

Die vierte, südlichste endlich nehmen die Hottentotten und Buschmänner ein.

\fed{{\textbar}276{\textbar}}\phantomsection\label{fp.276}

\textsc{Lepsius} nimmt nun an, es sei in den afrikanischen Sprachen der Wortschatz \fed{ganz} ausserordentlich wandelbar, daher mit der lexikalischen Vergleichung nicht weit zu kommen. Entscheidend müsse der Sprachbau sein, der sich spurenweise selbst in etwaigen Mischlingen erhalten werde. In Rücksicht hierauf charakterisirt er zunächst die Bantusprachen im Gegensatze zu den hamitischen, indem er eine Reihe kennzeichnender Merkmale aufstellt. Auf diese Merkmale hin untersucht er die Sprachen der zweiten Zone und glaubt in diesen ebensoviele nach Art und Grad verschiedene bantu-hamitische Mischlinge zu entdecken. Der Bantu-Sprachtypus sei der urafrikanische, wie er denn noch heute von der zahlreichsten Völker- und Sprachfamilie vertreten wird. In vorgeschichtlicher Zeit seien von Osten her Hamiten eingewandert, die hätten, südwärts vordringend, zwischen sich und die reinen Bantu jene Mischlingszone gesetzt.

Woher nun die vierte Zone? \textsc{Lepsius}, wie schon mancher Andere vor ihm, versucht, sie sprachlich mit der ersten zu verknüpfen. In der Vorzeit hätten die Hamiten, die Urafrikaner zeitweilig abdrängend, sich weiter südwärts erstreckt; dann sei eine Rückströmung der Ureinwohner erfolgt, die die südlichsten Hamiten von ihren Stammverwandten losgerissen, vielleicht noch weiter südwärts getrieben habe. Das grammatische Geschlecht des Hottentottischen sei noch ein Überbleibsel jenes Hamitismus.

Dies das Wesentlichste. Auf die anthropo-physiologischen Schwierigkeiten und die Art, wie der Verfasser sie zu überwinden sucht, will ich nicht eingehen. Die Schnalzlaute der Hottentotten- und Buschmannsprachen erklärt er (S.~LXVII) für einen „charakteristischen Ausdruck sprachlicher Indolenz und Verkommenheit“. Das grammatische Geschlecht aber stellt er (S.~XXVI) hoch \sed{{\textbar}{\textbar}283{\textbar}{\textbar}}\phantomsection\label{sp.283} als ein Zeichen höherer Sittlichkeit und eines reineren Familienlebens. Ihm und der Geschichte von Sem, Ham und Japhet zuliebe sollen wir Japhetiten Sem’s und Ham’s Enkel als unsere Vettern anerkennen (S.~XXIII flg.). Es wäre dies nebensächlich, wenn es nicht zeigte, wie starkes Gewicht dieser Forscher auf gewisse Merkmale der inneren Sprachform legte. Dass die uralaltaischen Sprachen in anderen Punkten, zumal in den Grundsätzen des Wort- und Satzbaues, den indogermanischen weit näher stehen, als die hamito-semitischen, scheint er zu übersehen oder zu unterschätzen. Die lexicalische Vergleichung aber ist Niemandem zu erlassen, der Sprachverwandtschaften \fed{{\textbar}277{\textbar}}\phantomsection\label{fp.277} behaupten will. Die Sprachen der gelben Südafrikaner jedoch zeigen nicht einmal in den Für- und Zahlwörtern hamito-semitische Anklänge, – bantuische freilich auch nicht. Dagegen hat \textsc{A. de }\retro{\textsc{Gregorio}}{\textsc{Grecorio}} (Cenni di glottologia bantu, Torino 1882) mit leichter Mühe in den Sprachen der nordwestlich und westlich vom Golfe von Guinea wohnenden Völker unverkennbare Spuren einer bantuischen Verwandtschaft nachgewiesen.

\sed{Wo sich ein kleines, thatkräftiges Volk durch Einverleibung anderer zu Macht und Grösse erhebt, da scheint die Sprache besonders schnell abgenutzt zu werden. Die markigen Züge des Angelsächsischen sind in wenigen Jahrhunderten zum Englischen verblichen. Jene Tungusenstämme, die im Mandschuvolke vereinigt China erobert haben, sprechen eine weit verschliffenere Sprache als ihre in Horden lebenden, halbwilden Stammesvettern. Und zwei der ältesten Cultursprachen, die chinesische und die ägyptische, tragen schon in ihren frühesten Denkmälern ein weit verwischteres, moderneres Gepräge, als ihre jüngeren Verwandten. So muss es der Geschichtsforscher an zwei verschiedenen Stellen erleben, dass das Alter der ältesten Urkunden zu der Alterthümlichkeit der Sprachen fast im umgekehrten Verhältnisse steht.}

\pdfbookmark[2]{II. §. 24. Dialektforschung.}{III.II.II.24}
\cohead{II. §. 24. Dialektforschung.}
\subsection*{§. 24.}\phantomsection\label{III.II.II.24}
\subsection*{Dialektforschung.}
\largerpage[1]Die Arbeit der historischen Sprachforschung ist recht eigentlich mikroskopisch, sollte es wenigstens sein. Ob es gelte, den Wandelungen eines Einzellautes oder des ganzen Lautsystems, den Veränderungen in den Bedeutungen der Wörter oder im Gebrauche grammatischer Formen nachzuspüren: immer sollten ihr die Vergleichsobjecte um so willkommener sein, je näher sie einander selbst und den beobachtenden Augen des Forschers liegen, – vorausgesetzt nur, dass sie noch unterscheidbar sind. Nur die Schranke, die die menschliche Schwäche setzt, darf hier gelten; denn wo es sich um die Beobachtung eines Werdens handelt, da steht der Werth der Beobachtung im umgekehrten Verhältnisse zur Grösse der beobachteten Abstände. Geflissentlich treibe ich auch diesmal wieder die Sache auf die Spitze. Ich setze den Fall, in einem ein\-\sed{{\textbar}{\textbar}284{\textbar}{\textbar}}\phantomsection\label{sp.284}samen Gebirgsdorfe leben Grossvater, Vater und Sohn; alle Drei haben nie die Heimath verlassen. Und nun \retro{unternähme}{unternähne} es Jemand, die Individualsprachen der Drei auf ihre feinsten Unterschiede hin zu untersuchen, mit photographischer Schärfe schriftlich zu fixiren. Und dann nach ein paar Jahren wiederholte er dieselbe Arbeit, wiese nun in einer jeden der drei Individualsprachen die kleinen Veränderungen nach, wie er vorher ihre kleinen Verschiedenheiten untereinander aufgezeigt hatte: – wäre die Arbeit menschenmöglich, so glaube ich, ihre Ergebnisse wären nicht hoch genug zu veranschlagen; der scheinbare Sonderling, der sie durchzuführen wüsste, hätte ohne Weiteres Sitz und Stimme da, wo es sich um die grundsätzlichsten Streitfragen der Sprachgeschichte handelt.

Und so gar überspannt, wie sie scheint, ist die Sache doch nicht. Auch wo Schulzwang und alle die von fremdher kommenden sprachlichen Ansteckungen keinen Einfluss üben, kann man in einzelnen Fällen wahrnehmen, wie sich die Sprache von Geschlechte zu Geschlechte ändert. \fed{{\textbar}278{\textbar}}\phantomsection\label{fp.278} Die Bauern der Altenburger Gegend haben eine Art Modalpartikel \update{\so{meech}}{„\so{meech}“} (ursprünglich wohl: meene ich, meine ich), etwa soviel besagend wie: „relata refero“: „Er kann (oder könnte) meech heute nicht kommen“ (– er oder die Seinen haben es gesagt). Zu Anfang dieses Jahrhunderts lebte noch ein Mann, der stattdessen „\so{minch}“ sagte, und deshalb den Spitznamen Minch trug. In der gleichen Bedeutung wurde damals wohl noch ab und zu \update{\so{halch}}{„\so{halch}“} (= halte ich) gesagt, das nun schon längst ausser Gebrauch gesetzt ist. Dort waren noch vor Kurzem und sind vielleicht noch jetzt die alterthümlichen Wörter \so{nunmehro} und \so{hinfüro} in Allerwelts Munde: es müsste sich nachweisen lassen, wie sie allgemach verdrängt werden und verschwinden. Das Wort \so{itzunder} hört man in Obersachsen noch ab und zu von alten Leuten, die es in der Anwendung gar wohl von \so{itzt} zu unterscheiden wissen. Die jüngeren Geschlechter aber, wenigstens in manchen Gegenden, haben das \update{breite}{breite,} alterthümliche Wort aus ihrer Rede verbannt. In diesem alten, verkehrsreichen Culturlande, in Landschaften von kaum einer Quadratmeile Umfang, wo wir Fernerstehende schwerlich eine leise mundartliche Verschiedenheit entdecken würden, kommt es vor, dass ein Bauer die Heimath des anderen, der kaum eine Meile weit von ihm zu Hause ist, mit annähernder Sicherheit aus der oder jener sprachlichen Eigenthümlichkeit erkennt. Deucht es uns, als sprächen die jungen Leute noch ganz wie ihre Väter, so können uns diese eines Besseren belehren: die heutige Sprache braucht insoweit nicht der Schriftsprache ähnlicher geworden zu sein, aber sie ist anders als jene, die die Alten von ihren Eltern gehört haben, und die ihnen noch in den Ohren klingt. Hier ist die Enge des Gesichtskreises ein wahrer Vorzug, denn auf ihr beruht die Schärfe des Unterscheidungsvermögens. Wie klein mag nun das \update{Gebiet}{Gebiet,} und wie gering die Seelenzahl unserer indo\-\sed{{\textbar}{\textbar}285{\textbar}{\textbar}}\phantomsection\label{sp.285}germanischen Altvordern gewesen sein, ehe sie mundartliche Verschiedenheiten untereinander wahrnahmen? Und war es erst einmal so weit gekommen, so war schon der erste Anstoss zu particularistischer Sonderung gegeben, wenn nicht gemeinsame Bedürfnisse, etwa feindliche Nachbarn, dafür sorgten, dass das Gefühl der Stammesgemeinschaft noch eine Weile lebendig blieb. War aber das der Fall, so werden Zwischenheirathen von Stamm zu Stamm dialektische Mischungen mit all ihren unberechenbaren Launen herbeigeführt haben; die Articulation wurde unsicher, zweierlei Aussprachen desselben Wortes liess man als gleich richtig gelten, bis sie sich etwa in ihren Bedeu\-\fed{{\textbar}279{\textbar}}\phantomsection\label{fp.279}tungen differenzirten und nun zu zwei anerkannt verschiedenen Wörtern wurden. Das mittelhochdeutsche \textit{joh} = doch hat sich, wie früher erwähnt, in mitteldeutschen Dialekten erhalten: da antwortet man auf eine verneinende Frage oder Behauptung nicht mit \textit{ja}, sondern mit \textit{jō}. Diesem entspricht in der Mundart des weimarischen \update{Voigtlandes}{Vogtlandes} lautgesetzlich \textit{gu}; aber dies wird nur als Modalpartikel gleich dem schriftdeutschen \so{ja}, \so{doch} gebraucht, – in der Antwort heisst es, ganz wie in den Nachbardialekten, \textit{χ’ō}, und auf affirmative Fragen \textit{χ’å} = \textit{ja}. A sagt: \update{Du}{„Du} warst \textit{gu} nicht \update{dabei.}{dabei.“} B erwidert: \update{\textit{χ’ō},}{„\textit{χ’ō},} ich war \update{dabei!}{dabei!“} Einen Mann namens Jakob konnte man von seinen Nachbarn bald \textit{X’åkŭp} bald \textit{Gäkop} rufen hören. Letzteres war das lautgesetzlich Erforderte, Ersteres mochte in dem \textit{χ’å} der Antwort und etwa noch in der schulüblichen Aussprache des biblischen Namens seine Stütze finden, jedenfalls aber galt Beides für gleich richtig, wenn auch jeder Dorfbewohner für seine Person die eine oder andere Form bevorzugen mochte.\footnote{Mein Vater hat dies auf seinem Gute Lemnitz bei Triptis beobachtet und mich seiner Zeit darauf aufmerksam gemacht.}

Was uns heute als Thatsache entgegentritt, dessengleichen war offenbar auch in urgeschichtlicher Zeit möglich; und so haben sich denn neuerdings auch die Indogermanisten darein gefunden, diesen unbequemsten aller Factoren mit in Rechnung zu ziehen. Es erinnert an jene geordneten Rückzugsbewegungen, die nach militärischen Ehrbegriffen an Ruhme dem Siege nahe kommen. Überall ausser im Altindischen hatten sie es nicht mit Beobachtungen eines wohlgeübten Gehöres, sondern mit mehr oder weniger mangelhaften Lautaufzeichnungen zu thun; die \update{nutzten}{nutzen} sie aus mit aller Pedanterie einer unbeugsamen Methode. Und nun war es schliesslich eine That der Selbstüberwindung, wenn sie sich durch jene Methode selbst zur Anerkennung ursprünglicher dialektischer Doppelformen bewegen liessen. In der Wissenschaft bestimmt sich oft der Werth eines Erwerbes nach der Art, wie er erlangt worden.

Man ist sehr weit geschweift, und das Gute lag sehr nahe. Alle die sprachbildenden Kräfte, die unsre heutige Indogermanistik entdeckt zu haben \sed{{\textbar}{\textbar}286{\textbar}{\textbar}}\phantomsection\label{sp.286} glaubt und nun so geist- und kunstvoll zusammenwirken lässt: die Feinheiten der Articulation und die Folgerichtigkeit der Lautentwickelung, die Analogie- und Neubildungen, die Satzphonetik und die Einflüsse der Betonung, die Entlehnungen und Mischungen, – sie alle \fed{{\textbar}280{\textbar}}\phantomsection\label{fp.280} müssten, nur noch viel lebensfrischer, in jedem beliebigen Dialekte zu beobachten sein. Um diese Kräfte aber, um die Grundsätze der geschichtlichen Sprachvergleichung, handelt es sich doch in erster Reihe. Denn ob zehn einander gleich nahestehende indogermanische Dialekte heute gesprochen werden, oder vor etlichen tausend Jahren gesprochen worden sind, ist für den sprachwissenschaftlichen Werth dieser Dialekte unerheblich, weil der indogermanische Sprachbau doch im Wesentlichen überall der gleiche geblieben ist. Nur das archäologische Interesse macht hier einen Werthunterschied: man möchte gar zu gerne wissen, wie und wovon unsere halbwilden Altvordern geredet haben.

\sed{Den archäologischen Werth der Dialektforschung hat man längst schätzen gelernt. Vieles Alte, was sich in der Sprache der Literatur und der Gebildeten verwaschen und verschliffen hat, lebt, mehr oder minder rein erhalten, auf den Dörfern weiter. Ich erinnere an eines der bekanntesten Beispiele, an die \textit{ei} und \textit{au} unserer Schriftsprache. Die meisten Mundarten unterscheiden noch scharf, ob das \textit{ei} aus \textit{î} oder \textit{ai}, das \textit{au} aus \textit{û} oder \textit{ou} entstanden ist und sagen z.~B. „eens, zwee, drei, ein Haus und ein Boom“, plattdeutsch „een, twee, drie, ’n Hus un ’n Boom“. Uns erklärt sich das leicht aus den älteren Phasen und den Schwestersprachen, die wir kennen. Wo aber beide unerreichbar sind, in den isolirten Sprachen, kann jede dialektische Abschattung unschätzbaren Erkenntnisswerth erlangen.}

Hier bietet sich nun ein Arbeitsfeld, das der Freund \update{unsrer}{unserer} Wissenschaft mit den spärlichsten Werkzeugen bebauen kann. Es kommt aber auch darauf an, dass sich recht Viele an der Arbeit betheiligen, und dazu sollten nicht nur wir Sprachforscher, sondern die Regierungen selbst auffordern und anregen.

In der That sind hier Sprach- und Volksgeschichte innig miteinander verquickt, und die letztere steht ja überall unter besonderem staatlichen Schutze. Neben den Trachten und Sitten, vielleicht mehr noch als Beide, müssen die Mundarten von der Herkunft der Gaubewohner zeugen. Seit dem zwölften Jahrhundert sitzen deutsche Colonisten vom Niederrhein und Flandern in Siebenbürgen, und heute noch redet man in der Bistritzer Gegend fast genau so, wie in Luxemburg. Die Sprache hat sich in den siebenhundert Jahren verändert, – das ist \update{zweifellos. Aber}{zweifellos; aber} hüben und drüben sind die Veränderungen in gleicher Richtung und fast in gleichem Schritte geschehen. Jene Siebenbürger haben sich in einer Art stolzer Vereinsamung \update{gehalten. Welche}{gehalten; welche} Stürme auch über das Land gehen mochten: die Sprache genoss eines fast ungestörten Still\-\sed{{\textbar}{\textbar}287{\textbar}{\textbar}}\phantomsection\label{sp.287}lebens. Wie steht es nun mit der Mundart jener Salzburger Ansiedler in Litauen, deren Vorfahren erst vor anderthalbhundert Jahren ihr Bergland verlassen haben? Sie sind von Anfang an hineingezogen worden in das Leben des norddeutschen Staates; weder Stolz noch Furcht oder Hass konnte sie hindern, sich ihren neuen Landsleuten auch gesellig anzuschliessen: hier müsste also ein rascheres Hinschwinden des Dialektes zu beobachten sein. Ferner: um den alten Limes sorabicus herum sind weithin deutsche und slavische Ortsnamen gemischt. Manche der ersteren zeugen noch von der Herkunft ihrer ersten Bewohner: Sachsen, Bayern, Franken, Schwaben, Vlämingen. Die Germanisirung der dazwischen \fed{{\textbar}281{\textbar}}\phantomsection\label{fp.281} wohnenden Sorben und Wenden muss fast überall ziemlich rasch von statten gegangen sein; nur stellenweise noch hat der Bauer Erinnerungen an die alte Stammverschiedenheit gewahrt und betrachtet wohl, wie man es im Königreiche Sachsen hören kann, das Beiwort „wendisch“ oder „überelbisch“ als eine Art Schimpf. Anderwärts sind die Gegensätze völlig geschwunden, die Volksstämme gemischt, und der stolze Dorf- und Kirchspielparticularismus fragt nicht mehr nach den ehemaligen Grenzen des Volksthumes. Ganz spurlos kann aber doch die slavische Sprache nicht verklungen sein. Es waren doch so und soviele slavische Köpfe und Münder, die sich das Deutsche aneigneten, das heisst zunächst, es sich kopf- und mundgerecht machten. Damit war ein neuer Dialekt geschaffen, natürlich ein fehlerhafter. Der mochte sich im Laufe der Zeit unter dem Einflusse der deutschen Nachbarn bessern. Doch das setzte einen fortgesetzten Verkehr voraus, das heisst einen Austausch, das heisst also auch umgekehrt eine gewisse Einwirkung der slavischen Radebrecherei auf die reindeutsche Sprache der Colonisten. Diese waren ja selbst in vielen Gegenden dialektisch gemischt, stammten aus verschiedenen Gauen Hoch- und Niederdeutschlands. Aber sollte nicht doch am Ende eine vergleichende Untersuchung jener Ostgrenzmundarten gewisse gemeinsame Merkmale aufzeigen, die nur slavischen Einflüssen zuzuschreiben wären? Die Untersuchung wäre lohnend, auch wenn ihr Ergebniss verneinend ausfiele; denn auch dann wäre sie für die „Principien der Sprachgeschichte“ verwerthbar.

Die beschreibenden und erzählenden Wissenschaften verdanken den gebildeten Bewohnern des platten Landes sehr viel. Heimathssinn und Liebe zur Heimathskunde brauchen hier nicht erst geweckt zu werden. Gelänge es aber, für die Dialektkunde Freiwillige zu werben, so zu sagen ein Netz von Beobachtungsstätten über das Land zu ziehen: so wäre viel gewonnen. Die Beobachtung zu organisiren, die Beobachter zu schulen und zu ermuntern, sollte nicht schwer fallen; vor den Opfern und Kosten braucht Keinem zu bangen, – eher vielleicht vor der Weisheit derer, die die Kosten bewilligen und die Opfer bringen sollen.

\sed{{\textbar}{\textbar}288{\textbar}{\textbar}}\phantomsection\label{sp.288}

\pdfbookmark[2]{II. §. 25. Standessprachen.}{III.II.II.25}
\cohead{\fed{II. §. 2\edins{5}. Standessprachen.}}
\subsection*{§. 25.}\phantomsection\label{III.II.II.25}
\subsection*{Standessprachen.}
Die Cultursprachen beweisen, dass die Sprachen sich nicht nur nach Raum und Zeit, sondern auch nach den Volksclassen spalten. Gesell\-\fed{{\textbar}282{\textbar}}\phantomsection\label{fp.282}schaftliche Stellung und Berufsart bringen es mit sich, dass sich die Bevölkerung in verschiedene engere und weitere Kreise zusammenschliesst, deren Angehörige vorzugsweise untereinander verkehren, mithin nach dem Gesetze von der Sprachmischung mehr Anregung voneinander, als von auswärts empfangen. Dabei können doch diese Anregungen in anderem Sinne völlig fremde sein. Ein grösserer Bau, Bestellung und Ernte auf landwirthschaftlichen Grundstücken ruft massenhaften Zuzug an fremden Arbeitern herbei. Meist kommen diese schaarenweise aus demselben Lande: Italiener zu Strassen- und Eisenbahnbauten, Schweden und Polen auf die norddeutschen Güter, Westfalen, s.~g. Hollandsgänger, nach Holland; Österreich, zumal wohl Böhmen, liefert Kellner und Musicanten, die Schweiz Bäcker in die Städte und Milchwirthe auf’s Land u.~s.~w. So empfangen verschiedene einheimische Berufsclassen von verschiedenen Gegenden des Auslandes her sprachliche Einflüsse. Die deutsche Gaunersprache hat von dem Hebräisch der polnischen und rheinischen Juden, die spanische (Germania) von dem Deutsch der Landsknechte einen Theil ihres Wortschatzes entlehnt.

Aber auch ohnedem mussten sie sich sprachlich sondern. Vom Unterschiede zwischen Bauern und Städtern sehe ich ab; er ist ja doch auch örtlich, und es ist natürlich, dass der Kleinbürger einer grösseren Stadt mehr von der Sprache der Gebildeten annimmt, als jener eines Landstädtchens oder ein Bauer. Auf das kommt es hier an, was sich im Schoosse des eigenen Berufs- und Gesellschaftskreises herausbilden muss. Offenbar hängt die Entwickelung unserer Sprache mit davon ab, worüber und in welchen Gedankenreihen wir am Meisten nachdenken. Das zeigt sich selbst \update{in}{an} den gebildeten Classen, z.~B. in den häufigen „eventuell“ und den Conjunctiven Imperfecti der Juristen, deren Wissenschaft ja ein grosser Conditionalis ist. Schärfer aber tritt es bei denen hervor, deren Gesichtskreis beschränkter ist. Hier bewegt sich Gedanke und Rede in sehr engen Kreisen, der Wortschatz ist einseitig entwickelt, der Geist wie der Körper zu gewissen Übungen ebenso geschickt, wie zu anderen ungefüge und schwach. Und wie tief \update{wirkt der Beruf auf}{beeinflusst der Beruf} das Temperament. Unser verbesserter Schulunterricht und eine billige Presse wirken ja auch hier bis zu einem gewissen Grade ausgleichend. Und doch nicht ganz. Denn auch von der zweiten Natur, der angewöhnten, gilt das „usque recurret“.

Mit Recht hat sich die Forschung auch mit diesen Standessprachen \fed{{\textbar}283{\textbar}}\phantomsection\label{fp.283} befasst, wenn auch die Schriftsteller nur selten Sprachforscher von Fache sein mochten. \sed{{\textbar}{\textbar}289{\textbar}{\textbar}}\phantomsection\label{sp.289} Begreiflich auch, dass sie sich den grellsten Formen mit Vorliebe zuwendete, den Ausdrücken der Jäger, Bergleute, Schiffer, Studenten und zumal dem polizeilich wichtigen Gaunerjargon. Zwei Fragen scheinen mir nun auch für die Sprachgeschichte im weiteren Sinne erheblich: erstens die Herkunft jener Ausdrücke und zweitens ihre etwaige Aufnahme in den Wortschatz der allgemeinen Sprache. Letzteres liegt offenbar um so näher, je gebildeter die betreffenden Stände sind, und je mehr sie sich an der Literatur betheiligen, oder je mehr Interesse auch die übrige Bevölkerung an dem Treiben der betreffenden Classe nimmt. So sind Studentenausdrücke wie Philister, Kneipe, Paukerei, einpauken, \update{anpumpen}{anpumpen,} in’s sprachliche Gemeingut übergegangen; und kohlen, schmusen, pleite gehen, Einem einen Zinken stechen, vertuschen, Moos, Penne, Kassiber, zum Theil Wörter hebräischer Herkunft, verdanken wir der Gaunersprache. Was der soldatischen Redeweise angehört, empfindet man im Vaterlande der allgemeinen Wehrpflicht längst nicht mehr als fremd, und in den Hafenstädten hat man sich seit Jahrhunderten an die Sprache der Seeleute gewöhnt. Auch der Süden unseres Vaterlandes wird sich diesen Einflüssen auf die Dauer nicht ganz erwehren können. Dafür schleppen die Norddeutschen aus den österreichischen Gasthäusern, den bairischen Brauereien, den Senn- und Jagdhütten der Alpen manches Wort mit nach Hause, das ihre Vorfahren nicht verstanden hätten. Die Sprache des Kaufmanns verräth bekanntlich noch heute die italienische Schule, und das Niederdeutsch der Harzer Bergleute hat nicht nur im übrigen Deutschland, sondern auch in Schweden Nachhall gefunden.

\pdfbookmark[2]{II. §. 26. Zusatz I. Anregungen zu sprachgeschichtlichen Untersuchungen. Irrlichter.}{III.II.II.26}
\cohead{II. §. 26. Zusatz I. Anregungen zu sprachgeschichtlichen Untersuchungen. Irrlichter.}
\subsection*{§. 26.}\phantomsection\label{III.II.II.26}
\subsection*{Zusatz I.}
\subsection*{Anregungen zu sprachgeschichtlichen Untersuchungen. Irrlichter.}
Wo in vergleichenden Forschungen die Ähnlichkeiten nicht haufenweise zu Tage liegen, ist man immer zunächst auf Einfälle angewiesen. Die können glücklich oder unglücklich sein, sich im weiteren Verfolg bewähren oder als nichtig erweisen, und auch im letzteren Falle verdienen sie noch lange nicht allemal Tadel. Dem Forscher, der in neue Gebiete vordringen möchte, geht es wohl wie der Spinne, die es dem Winde überlassen muss, wohin er ihren Faden wehen wird, – und darin \fed{{\textbar}284{\textbar}}\phantomsection\label{fp.284} gleichen jene Einfälle den Winden, dass man nicht weiss, von wannen sie kommen und wohin sie gehen. Ist aber einmal die Fadenbrücke gebaut, dann mag man auch versuchen, ob sich nicht ein dauerhaftes Netz daran spinnen lässt. Ich glaube dem Leser zu dienen, wenn ich ihm ein paar solcher Einfälle mittheile, von denen ich selbst noch nicht weiss, ob sie sich bewähren mögen.

1. Die bauliche Ähnlichkeit des Japanischen mit dem Mandschu, – natür- \sed{{\textbar}{\textbar}290{\textbar}{\textbar}}\phantomsection\label{sp.290} lich auch mit dessen Verwandten, – fällt Jedem ohne Weiteres auf. Die geistige Verwandtschaft ist unleugbar; es fragt sich nur, ob auch eine leibliche Verwandtschaft bestehe? Nun heisst mandschu \textit{ji} (spr. \textit{dsi}), japanisch \textit{ki}, \update{kommen,}{kommen;} davon der Imperativ mandschu \textit{ju} (spr. \textit{dschu}), \update{jap.}{japanisch} \textit{ko}, – Beides unregelmässig und auffallend parallel laufend. Ich folge der Spur und finde:

\begin{center}
\begin{tabular}{ l@{\hskip 48pt} l }
mandschu & japanisch \\
\textit{jui}, plur. \textit{ju-se} & \textit{ko} = Kind \\
\textit{je}, imperat. \textit{jefu} & \textit{kuf} (–\textit{i}, –\textit{u}) = essen.
\end{tabular}
\end{center}

\noindent Weiter denke ich daran, dass die mandschuischen Zahlwörter denen der übrigen ural-altaischen Sprachen verwandt sind, und dass das Wort für Zwei mandschu \textit{juwe}, in anderen tungusischen Sprachen \textit{jūr}, mongolisch aber \textit{koyar} lautet. Es fragt sich, wie weit man damit kommt, wenn man mandschuisches \textit{j} gleich japanischem und mongolischem \textit{k} ansetzt, – denn bewiesen ist noch gar nichts.

2. Dass das Chinesische vor Alters, gleich anderen Sprachen seiner Familie, die Auslaute \textit{l} und \textit{r} gehabt habe, ist anzunehmen. Auch das ist zu vermuthen, dass aus der Sprache des alten Culturvolkes schon sehr frühe Lehnwörter in die Sprachen seiner Nachbarn übergegangen \update{seien;}{seien.} Nun heisst\update{\textit{ssï}}{}\update{\textit{tsieu}}{}

\begin{center}
\begin{tabular}{ l@{\hskip 48pt} l }
chinesisch & koreanisch \\
\textit{mà} & \textit{măl}, Pferd \\
\sed{\textit{ss\={ï}}} & \textit{sil}, Seide \\
\sed{\textit{tsieù}} & \textit{siul}, Wein;
\end{tabular}
\end{center}

\noindent ferner mandschu: \textit{morin}, Pferd, \textit{sirge}, Seide, aber \textit{nure}, Wein. Mandschu \textit{ninggun} = sechs aber entspricht dem mongolischen \textit{dsirgugan}; also wäre es denkbar, dass das anlautende \textit{n} von \textit{nure} gleichfalls einem älteren \textit{ds} oder ähnlichen Laute entstammte. Wäre dem so, dann liesse sich auch \fed{{\textbar}285{\textbar}}\phantomsection\label{fp.285} weiter mandschu: \textit{niyalma} (spr. \update{\textit{\textsuperscript{n}alma}) [\textit{\mbox{in den} Berichtigungen, S.~502}: ñalma)]}{\textit{ñalma})} = Mensch, mit koreanisch \textit{salăm} vergleichen.

3. Vielleicht ist überhaupt ursprüngliches anlautendes \textit{n} im Mandschu verloren gegangen. Das Genitivsuffix hat neben der älteren, wahrscheinlich durch Inlautsgesetze erhaltenen Form \textit{ni} die jüngere \textit{–i}. Dann entspräche:

\begin{center}
\begin{tabular}{ l@{\hskip 48pt} l }
mandschu & japanisch \\
\textit{omi} & \textit{nomi} = trinken \\
\textit{akô} (\textit{a-kô}) & \textit{naku} = nicht seiend \\
\textit{ai} & \textit{nani} = was?
\end{tabular}
\end{center}

4. An japanisch \textit{mi} = Leib, selbst, Person, erinnert mandschu \textit{beye}, das ganz das Gleiche bedeutet, vielleicht auch \textit{bi} ich, \textit{mi-ni} mein. Dann passten zusammen

\sed{{\textbar}{\textbar}291{\textbar}{\textbar}}\phantomsection\label{sp.291}

\begin{center}
\begin{tabular}{ l@{\hskip 48pt} l }
mandschu & japanisch \\
\textit{bedere} & \textit{modor-} = zurückkehren \\
\textit{ba} = Ort & \textit{ma} = Raum.
\end{tabular}
\end{center}

\noindent Dagegen haben statt der mandschuischen Accusativpartikel \textit{be} andere tungusische Sprachen \textit{wä}, wozu sich das japanische \textit{wo} besser schickt.

Dies Alles nimmt sich nun recht einladend aus, man möchte den Spuren weiter nachgehen, glaubt schon des Erfolges halbwegs gewiss zu sein. Aber wie luftig ist das Gespinst, wenn man es näher betrachtet! Das Koreanische mit den uralaltaischen Sprachen zu vergleichen, ist an sich schon gewagt. Gewisse syntaktische Ähnlichkeiten bestehen wohl; die theilen aber auch andere Sprachfamilien von suffigirendem Baue. Weder die Pronomina noch die Zahlwörter stimmen zusammen; ebenso wenig die Casussuffixe.\footnote{ Eher liessen sich die koreanischen Wörter für 1, 2, 3, 4 mit den entsprechenden des \textit{Aino} vergleichen: 1 \textit{hăna} : \textit{šine}, 2 \textit{tul} : \textit{tu}, 3 \textit{seis} : \textit{re}, 4 \textit{neis} : \textit{ine}. Doch auch das beweist nichts.} Das Koreanische besitzt, im Gegensatze zu den uralaltaischen Sprachen, ein Nominativzeichen. Die Verschiedenheiten der Declinationen und Conjugationen beruhen in den uralaltaischen Sprachen wesentlich auf dem Vocalismus (dem Harmoniegesetze), im Koreanischen auf dem Stammauslaute, der sehr oft consonantisch ist. Wo die Verwandtschaften nicht nahe sind, da ist mit Einzel\-\fed{{\textbar}286{\textbar}}\phantomsection\label{fp.286}sprachen hüben und drüben gar nichts anzufangen, da muss man statt des Mandschu den ganzen uralaltaischen Sprachstamm, statt des Chinesischen die grosse indochinesische Familie oder doch ihre wichtigsten, alterthümlichsten Vertreterinnen in’s Gefecht führen; und lautlich so verschliffene Sprachen wie die chinesische und japanische in ihren uns bekannten Zuständen müssen erst auf ihre älteren Lautformen zurückgebracht werden. Mandschu \textit{ai} = \update{was}{was,} mit japanisch \textit{nani} zu vergleichen wird man Bedenken tragen, wenn man das entsprechende \textit{χai} in den nächstverwandten tungusischen Dialekten kennt, das offenbar alterthümlicher ist. (Vergl. auch \textit{orin}, \textit{χoren} = zwanzig). Soviele Vorarbeiten sind nöthig, ehe man mit gesunden Sinnen und gutem Gewissen derartigen Einfällen nachgehen darf. Hätte ich übrigens, statt mich auf Ostasien zu beschränken, meine Fäden nach Afrika oder Amerika hinüberspinnen wollen, so würde ich wahrscheinlich nicht weniger Stoff zu Combinationen gefunden haben, nur wäre dann das Wagniss augenfälliger gewesen, und man hätte entweder an meiner Gutgläubigkeit oder an meinem gesunden Verstande zweifeln dürfen.

5. Wie es Einem mit solchen Combinationen ergehen kann, wenn man sich einmal da in’s Spiel mischen will, wo man nur Dilettant ist, habe ich selbst erfahren. Ich verglich englisch \textit{to speak} mit deutsch sprechen, englisch \textit{to spit} mit deutsch spritzen (neben spützen), dann deutsch schlitzen mit lateinisch \textit{scin}\sed{{\textbar}{\textbar}292{\textbar}{\textbar}}\phantomsection\label{sp.292}\textit{dere}, schlürfen (neben schürfen) mit \textit{serpere}, schlucken (neben saugen) mit \textit{sugere}, schlafen (neben schwedisch \textit{sofva}) mit \retro{\textit{sopor}}{\textit{sopar}} (sanskrit \update{${\surd}$\textit{svap})}{${\surd}$\textit{svap}),} Schlamm mit englisch \textit{swamp}, Fleder (-wisch, neben Feder) mit sanskrit \textit{patra} = Flügel, fliehen mit \textit{fugere}, fliessen mit \textit{fundere}. Darauf hin glaubte ich einem germanischen Infixe \textit{–l–}, \textit{–r–} auf der Spur zu sein. Wie mich aber Indogermanisten belehrt haben, hält von diesen Vergleichen ein Theil vor der Lautgeschichte nicht Stich, und die übrigen scheinen nichts zu beweisen. Prä-Infixe, das heisst solche, die auf den Anlaut \update{folgen}{folgen,} statt vor dem Auslaute eingeschoben zu werden, sind ohnehin in suffigirenden Sprachen nicht zu vermuthen, und unorganische Einschiebsel würden eine lautgesetzliche oder psychologische Erklärung erfordern, die erst noch zu suchen wäre. \sed{Allerdings werden dergleichen wohl auch von Germanisten angenommen. \so{Reiher}, altsächsisch \textit{hreiera}, wird mit \so{Heiger}, \so{Thräne}, mittelhochdeutsch auch \textit{traher}, mit \so{Zähre}, \so{Rasen}, im älteren Niederdeutsch \textit{wrase}, mit \so{Wasen}, \so{Wechsel} mit angelsächsischem \textit{wrixl} verbunden; für \so{Pumpe} sagt man in Obersachsen \so{Plumpe}. Ähnlich mag es sich mit \so{Guffe} und \so{Glufe} = Stecknadel verhalten. Unorganische Einschiebsel sind wohl in Fremdwörtern am Erklärlichsten. So z.~B. ist der Name \so{Pfister} (lateinisch \textit{pistor}) in Obersachsen zu \so{Pfnister}, \so{Pflister} geworden. Woher aber der ganz ungebräuchliche Anlaut \textit{pfn}?}

Eine ähnliche Enttäuschung wurde mir ein andermal, als ich in den Wörtern \textit{carmen}, \textit{germen}, \textit{terminus} die Wurzeln \textit{can} = singen, \textit{gen} = zeugen und \textit{ten} = dehnen oder \update{halten}{halten,} zu entdecken meinte. Der Sinn \fed{{\textbar}287{\textbar}}\phantomsection\label{fp.287} passte, dieselbe Lauterscheinung kehrte dreimal wieder, Alles schien in bester Ordnung, – nur die sprachgeschichtlichen Thatsachen, von denen ich nichts wusste, sprachen ein entschiedenes Nein.

\sed{Ob nun nicht doch unsere heutigen Sprachenvergleicher im Verneinen und Bezweifeln manchmal zu weit gehen? Was nicht nach erkannten Lautgesetzen zusammenstimmt, soll nicht zusammengehören. So wimmeln denn die etymologischen Wörterbücher von Wörtern und Wurzeln, die nur einer engeren Familie angehören sollen, die also entweder überall sonst verloren gegangen oder an einer Stelle neu geschaffen sein müssen. Im ersteren Falle wäre die Ursprache erstaunlich reich gewesen. Entlehnungen aus Nachbarsprachen anderen Stammes sind wohl möglich; ebenso heimische Neubildungen aus überkommenen Wurzeln. Woher aber die neuen Wurzeln, die doch wohl nur zum kleinsten Theile spontane, onomatopoetische Erzeugnisse sein können? Mir scheint, wo sich die Wurzeln verwandter Sprachen einigermassen in Klang und Sinn ähneln, sollte man immer lieber einen unerklärlichen Lautwandel, das Walten eines noch unerkannten Lautgesetzes oder unsichere Articulation, als eine noch unerklärlichere Neuschöpfung vermuthen. Ähnliches dürfte da gelten, wo man um des Lautwesens willen eine Entlehnung annimmt, ohne zu sagen, auf welchem Wege} {\textbar}{\textbar}293{\textbar}{\textbar}\phantomsection\label{sp.293} \sed{eine solche möglich war, wenn man z.~B., wie ich es irgendwo gelesen, einem schweizerdeutschen Worte niederländische Herkunft ansinnt, ohne dass auch nur versuchsweise angedeutet wäre, warum die Schweizer gerade in diesem Falle, und warum sie gerade bei den Niederländern geborgt hätten.}

\pdfbookmark[2]{II. §. 27. Sprachvergleichung und Urgeschichte.}{III.II.II.27}
\cohead{II. §. 27. Sprachvergleichung und Urgeschichte.}
\subsection*{§. 27.}\phantomsection\label{III.II.II.27}
\subsection*{Zusatz II.}
\subsection*{Sprachvergleichung und Urgeschichte.}
Eigentlich thut die historisch-genealogische Sprachvergleichung mit jedem ihrer Schritte einen Blick in die Vorgeschichte der Völker. Freilich einen Blick durch den Schleier und manchmal einen Blick in den Nebel.

Wissen wir, dass Völker sprachverwandt sind, so glauben wir zunächst, dass sie stammverwandt seien, das heisst, dass es eine Zeit gab, wo sie zusammen ein Volk bildeten. Aber wie haben sich in geschichtlicher Zeit die Völker- und Sprachgrenzen durcheinander geschoben, wie mögen sie sich schon in vorgeschichtlicher Zeit gekreuzt haben. Die Bürger des Negerstaates \update{Hayti}{Haiti} sprechen französisch, und auf der iberischen Halbinsel reden die Nachkommen von Basken, Germanen und Mauren eine romanische Sprache. Es giebt keinen indogermanischen Völkertypus, so wenig wie es einen ural-altaischen giebt, und die Anthropologen streiten, woher das Blond der Germanen und Kelten und jenes der Finnen?

Aus charakteristischen Gleichheiten des Sprachbaues schliessen wir auf eine ausgeprägte Eigenart der Denkgewohnheiten bei dem Urvolke. Manchmal, wie bei den amerikanischen Jägervölkern, den Uralaltaiern, Malaien, Semiten, Bantunegern und Australiern, scheint es, als könnten wir hierin gar nicht irren, weil die Geistesart der Rasse scharf ausgeprägt, und ihre Lebensweise sich vermuthlich seit Jahrtausenden \update{gleich geblieben}{gleichgeblieben} ist. Andere Male, vorab bei uns Indogermanen selbst, haben sich die Volksarten so gespalten, dass das Altgemeinsame nur fleckenweise durch eine dicke Schicht des Neuen und Verschiedenen hindurchblickt. Und wieder andere Male, bei den Indochinesen, zeigt sogar der Sprachbau eine \update{Mannich\-faltigkeit,}{Mannig\-faltigkeit,} die im günstigsten Falle nur mühsam, vielleicht auch niemals die ursprüngliche Form wird erkennen lassen.

\largerpage[-1]

Besonders wichtig scheinen die lexikalischen Übereinstimmungen. \fed{{\textbar}288{\textbar}}\phantomsection\label{fp.288} Man sollte meinen, was Alle gleich benennen, müssen auch Alle gleichmässig gekannt haben, der gemeinsame Wortschatz stelle den gemeinsamen Vorrath an Vorstellungen, mithin den Urzustand der Gesittung dar. So müsste sich aus \textsc{Pott}’s oder \textsc{Fick}’s indogermanischen Wörterbüchern das geistige Inventar unserer Ureltern herauslesen lassen, und aus diesem sich dann der Schluss ergeben auf ihre Heimath, ihre Lebensweise und Gemüthsart, ihre sittlichen und religiösen \sed{{\textbar}{\textbar}294{\textbar}{\textbar}}\phantomsection\label{sp.294} Vorstellungen, kurz auf ihre ganze Gesittung. Nichts anziehender, als solchen  Gedanken zu folgen und auf dem Gefährt der Sprachwissenschaft die Reise in’s Nebelheim der Urgeschichte zu wagen. Man glaubt so sicher zu fahren und so klar zu sehen.

Das erste umfassende Werk dieser Art waren wohl A. \textsc{Pictet}’s Origines indoeuropéennes (Paris 1859). Daneben ist zu nennen: \textsc{Adalbert Kuhn}’s Aufsatz: Zur ältesten Geschichte der indogermanischen Völker (in \textsc{Weber}’s Indischen Studien, Bd. V), dann A. \textsc{Fick}’s Buch Die ehemalige Spracheinheit der Indogermanen Europas (Göttingen 1873), endlich O. \textsc{Schrader}’s Sprachvergleichung und Urgeschichte (Jena 1883, 2. Aufl. 1889), ein besonders anregendes Buch. Auch \textsc{Max Müller}’s vielgelesene Arbeiten zur vergleichenden Religionsgeschichte und gewiss noch viele andere, mir unbekannt gebliebene Versuche, Theile unserer Vorgeschichte durch Sprachenvergleichung zu erschliessen, gehören hierher. Eine jüngst erschienene Abhandlung von \textsc{B. Delbrück}: Die indogermanischen Verwandtschaftsnamen, ein Beitrag zur vergleichenden Alterthumskunde (Leipzig, Abh. der K. Sächs. Ges. d. Wiss. 1889) wendet sich gegen zwei Seiten: einmal gegen gewisse rosenfarbene Malereien, \retro{denen}{deren} zufolge das Familienleben unserer Urahnen von fast idealer Gemüthstiefe gewesen wäre; und dann gegen jene Anderen, die dem Kindheitsalter unsres Geschlechtes ein nahezu viehmässiges Herdenleben zuschreiben, Matriarchat, Kindergemeinschaft und wohl noch Ärgeres. \sed{\textsc{Joh. Schmidt}: Die Urheimath der Indogermanen und das europäische Zahlsystem (Abh. d. K. Pr. Akad. d. Wiss. 1890) zeigt, wie die Sexagesimalrechnung unserer Vorfahren auf uralte Culturbeziehungen zu den Sumero-Akadern deute. Die Literatur auf diesem Gebiete wächst rasch an, und ich will nicht versuchen, sie zu verzeichnen, da ich sie zu würdigen nicht competent bin. Auf Eins aber möchte ich hier nochmals hinweisen: nicht die Wanderungen an sich, sondern der Verkehr, die Mischungen wirken am Meisten dahin, dass die Sprachen entstellt werden. Auf Island, in der Abgeschiedenheit, hat sich das Skandinavische alterthümlicher erhalten, als in seiner Urheimath.} – Einen Versuch, den Gesittungszustand des finnisch-ugrischen Urvolkes nach sprachlichen Zeugnissen zu schildern, hat \textsc{O. Donner} gemacht.

\largerpage[-1]
Immer gelten doch solche Untersuchungen in erster Reihe der äusseren Lage der Urahnen: wo wohnten sie? was trieben sie? wovon lebten sie? wie war ihr Gemeinwesen geordnet? Tiefer in’s innere Leben dringen schon die religionsgeschichtlichen Fragen, – vielleicht noch tiefer, weil noch mehr in’s Elementare, jene, die die Etymologie zu beantworten sucht, und jene weiteren, die die Sprachen familienweise auf \fed{{\textbar}289{\textbar}}\phantomsection\label{fp.289} ihren geistigen Gehalt und Werth prüfen. Diese letzteren gehören aber der allgemeinen Sprachwissenschaft an.

\sed{{\textbar}{\textbar}295{\textbar}{\textbar}}\phantomsection\label{sp.295}

\pdfbookmark[2]{II. §. 28. Zusatz III. Die Wurzeln.}{III.II.II.28}
\cohead{II. §. 28. Die Wurzeln.}
\subsection*{§. 28.}\phantomsection\label{III.II.II.28}
\subsection*{Zusatz III.}
\subsection*{Die Wurzeln.}
Was versteht man unter Wurzeln? Soviel ich sehe, wird die Frage verschieden beantwortet, jede dieser Antworten hat ihre gewisse Berechtigung, der Ausdruck ist also mehrdeutig und solange er dies ist, unbrauchbar.

I. Die vorsichtigste und wie mir scheint noch immer zutreffendste allgemeine Definition dürfte diese sein: Wurzeln sind die letzten erkennbaren bedeutsamen Lautbestandtheile der Wörter. Damit ist freilich nicht die Sache beschrieben, sondern nur ein Recept zu ihrer Darstellung gegeben: Zerlege die Wörter etymologisch, sieh zu, wie weit Du damit kommst, \fed{und} wenn Du glaubst an’s Ende des Zerlegens gelangt zu sein, so magst Du die gewonnenen Elemente Wurzeln nennen!

II. Dies klingt zunächst wie schnöder Hohn. A ist zaghaft in der Analyse, möchte um keinen Preis etwas unlösbar Verbundenes zerreissen und zählt seine Wurzeln nach Tausenden auf. B geht kühner zu Werke, scheidet und zerschneidet den Stoff in immer kleinere Stückchen, gelangt am Ende zu einigen Dutzenden einfachster Sylben, denen entsprechend vage Bedeutungen beigelegt werden, und nennt das sein Wurzelverzeichniss. Unter den Indogermanisten sind wohl die Göttinger \textsc{Th. Benfey} und \textsc{Fick} hierin am weitesten gegangen. Man sieht, Wurzel ist ein sehr subjectiver Begriff.

III. Und doch scheint gerade diese Subjectivität im Grunde berechtigt. Stellten wir uns zunächst auf den einzelsprachlichen Standpunkt, so erkannten wir, dass dem sprachschaffenden Geiste ein etymologisches Bedürfniss innewohnt, welches, je nach der Sprach- und Volksart, hier lebhafter dort mit mehr Zurückhaltung verlangt, dass in der Rede auch die Theile der Theile bedeutsam seien. Wir erkannten weiter, dass sich hierzu ein lautsymbolisches Gefühl gesellt, dem es behagt, mit ähnlichen Lauten ähnliche Vorstellungen verbunden zu sehen. Setzen wir nun an Stelle des Erkennbaren und Erkannten das vom Sprachgefühle Empfundene, \fed{{\textbar}290{\textbar}}\phantomsection\label{fp.290} so müssen wir zugeben, dass in diesem Sinne jede Sprache in jeder ihrer geschichtlichen Phasen Wurzeln hat.

IV. Freilich: was für Wurzeln? Welches sind die Wurzeln von fliessen, floss, flössen, Fluss, flüssig, – von binden, Band, Bänder, Bund, bündig, – von fallen, fiel, Fälle, – von sterben, \sed{stirb,} starb, gestorben, stürbe? Hier sind es doch, – immer im Sinne der Einzelsprache, – nur die gemeinsamen Consonanten. Und in diesem Verstande redet man mit Recht von dreiconsonantigen semitischen Wurzeln. Wo aber der Vocalismus im Sprachgefühle eine so gewaltige Rolle spielt, wie bei den Semiten, da könnten und sollten wir \sed{{\textbar}{\textbar}296{\textbar}{\textbar}}\phantomsection\label{sp.296} folgerichtig noch von einer zweiten Art Wurzeln, etwa den thematischen, reden und z.~B. \textit{qātil}, tödtend, bezeichnen als eine Durchdringung der Wurzeln \textit{qtl} = tödten und /\textit{ā}/\textit{i}/ = \textit{agens}, \textit{actor}.

\sed{V. An dieser Stelle bietet sich ein weiterer Ausblick. Bei den Wurzeln kann das Sprachgefühl mit sehr vagen Bildern fürlieb nehmen; es verbindet „\so{denk}-en, ge-\so{dach}-t, – ge-\so{schah}, Ge-\so{schich}-te“ u.~s.~w. Wäre es nicht möglich, dass es sich anderwärts mit den Wortbildern ganz ähnlich verhielte? Die baskischen Dialekte weisen für den Begriff „weich, zart“ folgende Wörter auf: labourdinisch: \textit{malgu}, \textit{malba}, \textit{mirigosa}, – niedernavarresisch: \textit{malxo}, \textit{mardo}, \textit{merzil}, \textit{merda}, – guipuzcoanisch: \textit{malso}; ferner für „Erdbeere“ guipuzc. und labourd. \textit{marrubi}, labourd. überdies \textit{marabio}, niedernav. \textit{mahuri}, biscaisch: \textit{malluki}. Diese Beispiele \corr{1901}{liessen}{liesen} sich fast in’s Endlose vermehren, und dann würde es sich ergeben, dass hier von festen Lautvertretungen keine Rede ist, sondern nur von unbestimmten Lautbildern und einer entsprechend unsicheren Articulation. Die mag sich nachmals mehr und mehr gefestigt haben; aber „quod ab initio viciosum est, facto posteriori sanari nequit“, – zur Gewinnung eines gesetzlichen Lautwandels war es zu spät.}

\sed{VI.} Was von der jeweiligen Einzelsprache,\update{V.}{ }das gilt natürlich auch von jener Einzelsprache, die wir die Stammes-Ursprache nennen. In diesem Verstande, aber auch nur in diesem, darf von den Wurzeln eines Sprachstammes geredet werden. Wie diese nun lautlich und graphisch darzustellen seien, ob nach indischer Art nur je in einer Form, ob nach neuerer Art in Stufenreihen, das mag Zunftfrage bleiben.

\sed{VII.} Die Agglutinationstheorie stellt den Satz auf: Alles was jetzt Affix ist,\update{VI.}{ }war früher ein selbständiges Wort. Damit weist sie in eine Zeit zurück, wo es noch gar keine Affixe gab, wo die Wörter entweder völlig unveränderlich waren, wie dies in den jetzigen isolirenden Sprachen die Regel bildet, oder nur bedeutsame, vielleicht lautsymbolische, innere Wandlungen erlitten. In beiden Fällen muss sie erklären: in Urzeiten waren Wurzel und Wort Eins, oder richtiger: was später als Wurzel galt, war in der Ursprache ein Wort. Manche wollen nur das geformte Wort ein Wort, das ungeformte eine Wurzel nennen und bezeichnen demgemäss z.~B. das Chinesische als eine Wurzelsprache. Ich kann das nicht billigen. Hält man sich an das Bildliche des Ausdruckes, so verbindet man mit der Vorstellung einer Wurzel nicht die eines ungeformten, formlosen Dinges, sondern die eines unter der Oberfläche befindlichen, dem zu Tage tretenden organischen Gebilde zu Grunde liegenden. So bezeichnet \textsc{Pott} (citirt von \textsc{Delbrück}, Einl. in das Sprachstudium, 2. Aufl. S.~74) die Wurzel als „die ... Einheit genetisch zusammengehöriger Wörter und Formen, welche dem Sprachbildner bei deren Schöpfung in der Seele als Prototyp vorschwebte“. \sed{Und ähnlich sagt \textsc{Misteli} (Charakteristik der haupts. Typen des Sprachbaus,} {\textbar}{\textbar}297{\textbar}{\textbar}\phantomsection\label{sp.297} \sed{S.~498): „Denn das gehört zum Begriffe der Wurzel, dass sie als Einheitspunkt eines Kreises von Bildungen auch dem ungrammatischen Bewusstsein vorschwebe oder vorschwebte, – vor Jahrhunderten oder Jahrtausenden“. Den „Sprachbildner“ hat er weggelassen; jeder frei Redende ist ja ein Sprachbildner.}

\fed{{\textbar}291{\textbar}}\phantomsection\label{fp.291}

\update{VII.}{VIII.} Ist dem so, dann haben wir zwischen \update{ächten,}{echten,} apriorischen Wurzeln und, so wunderlich der Ausdruck klingen mag, aposteriorischen Wurzeln, das heisst solchen, die je und je das Sprachgefühl der Völker als letzte bedeutsame Elemente der Wörter empfindet, zu unterscheiden. Geht man nur bis auf die indogermanische Ursprache in dem Zustande vor ihrer Spaltung zurück, so kann nur von solchen aposteriorischen Wurzeln die Rede sein. Meint man aber, in noch fernere Tiefen, in die Geheimnisse der isolirenden Ur-Ursprache schauen zu können, so mag man versuchen, wieweit sich auf aposteriorischem Wege apriorische Wurzeln darstellen lassen. Die Erkenntnissmittel der genealogisch-historischen Forschung versagen hier vorläufig ihren Dienst. Entweder muss die Verwandtschaft eines anderen Sprachstammes mit dem unseren nachgewiesen werden, um auf dem Wege weiterer vergleichender Analyse tiefer vorzudringen; oder man ist auf Analogieschlüsse angewiesen, die über die Grenzen der einzelnen Sprachfamilien hinaus in der allgemeinen Sprachwissenschaft ihre Anhaltspunkte suchen müssen.

\pdfbookmark[2]{II. §. 29. Zusatz IV. Laut- und Sachvorstellung.}{III.II.II.29}
\cohead{II. §. 29. Zusatz IV. Laut- und Sachvorstellung.}
\subsection*{\sed{§. 29.}}\phantomsection\label{III.II.II.29}
\subsection*{\sed{Zusatz IV.}}
\subsection*{\sed{Laut- und Sachvorstellung.}}
\sed{Wurzeln im Sinne des jeweiligen Sprachbewusstseins sind mehr oder minder bestimmte Lautvorstellungen (Typen), mit denen sich mehr oder minder bestimmte Sachvorstellungen verknüpfen. Bis hierher gilt das Gleiche auch vom Worte, vom Satze, von der ganzen Sprache. Was die Wurzel unterschied, war, dass sie das letzte Ergebniss der Analyse, für das Sprachgefühl also nicht weiter analysirbar ist. Für jetzt aber kommt es nicht mehr hierauf an, sondern auf das Gemeinsame, dass sowohl das dem Geiste vorschwebende Lautbild als auch die ihm anhaftende sachliche Vorstellung mehr oder weniger scharf umschrieben, vielleicht sehr vag sein kann. Und hier ist es nun weiter wichtig, dass die Sachvorstellung im Sprachgefühle weit mächtiger wirkt, als die Lautvorstellung. Auch der ganz Ungebildete verbindet „dachte, gedacht“ mit „denken“, nicht etwa mit „Dach“; „denke – dachte – gedacht“ bilden auch für ihn eine Kette, wie „sage – sagte – gesagt“, – eine Kette, das heisst ein fest Verbundenes.}

\sed{Hüben und drüben, bei den Lauten und den Bedeutungen, sind die} \corr{1901}{Er-}{Er scheinungen} {\textbar}{\textbar}298{\textbar}{\textbar}\phantomsection\label{sp.298} {{\color{lsMidBlue}schei\-nungen}} \sed{und deren Wirkungen einander sehr ähnlich: beide, die Laute und die Bedeutungen, können sich verschieben, wenn und soweit sie Kreisen gleichen, innerhalb derer der Willkür des Redenden ein Spielraum gestattet ist. Wir sahen, wie der Sprachgebrauch zunächst innerhalb dieser Kreise den Schwerpunkt und dann die Kreise selbst verrücken, wie er sie verengern oder erweitern kann. Dagegen ist es natürlich durchaus nicht nothwendig, dass die Verschiebungen der Laute und Bedeutungen gleichen Schritt halten, geschweige denn, dass eine Art mystischer Parallelismus zwischen ihnen walten müsste. Französisch \textit{cinq} und russisch \textit{pjatj} sind gleichen Ursprungs, haben nicht einen Laut mehr gemein, aber genau die gleiche Bedeutung. Umgekehrt sind die französischen Wörter \textit{cinq} = \textit{quinque}, \textit{sein} = \corr{1901}{\textit{sinus}, \textit{seing}}{\textit{seinus}, \textit{sing}} = \textit{signum}, \textit{sain} = \textit{sanus}, \textit{saint} = \textit{sanctus} nachgerade in der Aussprache zusammengeflossen, ohne sich darum in ihrer Bedeutung irgendwie einander genähert zu haben.}

\sed{In den meisten uns bekannten Sprachfamilien sind ziemlich feste Lautverschie\-bungs- und Lautvertretungsgesetze nachweisbar, die Articulation ist nicht nur jetzt sicher, sondern muss es auch schon früher, zur Zeit der Sprachentrennung gewesen sein. Mit anderen Worten: die Sprache unterschied so und soviele Laute, je mit so und so weitem oder engem Spielraum für die Art der Hervorbringung, doch immer so, dass kein Kreis den andern schnitt; es gab Grenzen zwischen \textit{q}, \textit{k} und \textit{g}, zwischen \textit{t} und \textit{d} u.~s.~w., wahrscheinlich sogar leere Zwischenräume zwischen diesen Grenzen, mochten die Grenzen noch so geräumig sein, mochten sie etwa die Tenuis mit der Media zugleich umschliessen. Immer waren den Sprachorganen nur gewisse Bewegungen geläufig, andere, auch die möglichen Mittelstufen, blieben ihnen jeweilig ungewohnt, waren dem Gehöre unerhört.}

\sed{Gerade in den beschränktesten Verhältnissen ist das Sprachgefühl am Empfindlichsten; da erklärt es: Wer es nicht ganz so macht, wie es bei uns gemacht wird, der macht es falsch! Hier herrschen die allerschärfsten Lautvorstellungen. Aber man mache Kerenz zur Bahnstation, zur Garnison, zum Fabrikorte, gewöhne die Einwohner an’s Wandern, so wird sich ihr Sprachgefühl zugleich erweitern und abstumpfen, neben dem Heimischen, bald vielleicht statt des Heimischen, wird allerhand Fremdes als richtig hingenommen und geübt.}

\sed{In der That giebt es schon ohnehin Fälle genug, wo auch die Beschränktesten, die Empfindlichsten und Anspruchsvollsten sich in solchen Dingen einen Abstrich gefallen lassen müssen. Wir hören draussen auf der Strasse ein schreiendes Gezänk, verstehen jedes Wort, das dabei fällt, sind aber beim besten Willen nicht im Stande, genau das nachzusprechen oder schriftlich darzustellen, was unser Ohr wirklich vernommen hat. Wahrscheinlich waren es nur die Vocale, der Rhythmus, das Steigen und Fallen der Stimmen, – vielleicht ein paar consonantische Dauerlaute, – alles Übrige haben wir selbst hinzugethan.} {\textbar}{\textbar}299{\textbar}{\textbar}\phantomsection\label{sp.299} \sed{Ein Alphabet, das solche Gehörseindrücke richtig wiedergeben sollte, würde seltsame Consonantenbilder enthalten. Ich will probehalber eins entwerfen: 1 = \textit{k}, \textit{t} oder \textit{p}; 2 = \textit{k} oder \textit{g}; 3 = \textit{t} oder \textit{d}; 4 = \textit{p} oder \textit{b}; 5 = \textit{d} oder \textit{n}; 6 = \textit{b} oder \textit{m}; 7 = \textit{m} oder \textit{n}; 8 = \textit{l} odr \textit{r}; 9 = \textit{s}, \textit{sch} oder \textit{f} u.~s.~w. Mit anderen Worten: jedes Zeichen würde einen Spielraum umschreiben, der wieder in andere Kreise hineinragte.}

\sed{Der Versuch selbst aber wäre eine Spielerei, wenn nicht die Sprachwissenschaft ihrerseits, wenigstens in einem Bezirke ihres weiten Gebietes, geradezu die Spielraumtheorie zu verlangen schiene. Als ich das Baskische mit den Berbersprachen verglich, hatte ich zunächst das lautliche Verhalten der baskischen Dialekte untereinander festzustellen. Hier wechselten nun mehr oder weniger oft}

\sed{
1. Die Tenues \textit{k}, \textit{t}, \textit{p} je mit ihren Mediis \textit{g}, \textit{d}, \textit{b};

2. Die Mediae \textit{g}, \textit{d}, \textit{b} untereinander;

3. ebenso, doch selten, die Tenues \textit{k}, \textit{t}, \textit{p};

4. sehr oft die Gutturale \textit{k} und \textit{g} mit den Zischlauten \textit{ch} (=\textit{tš}), \textit{z} (=\textit{ç}), \textit{ts}, \textit{tz};

5. sehr oft die Zischlaute untereinander;

6. \textit{g} und \textit{h}, \textit{p} und \textit{f};

7. die Liquidae \textit{r}, \textit{l}, \textit{n}, \textit{ñ};

8. viermal \textit{g} und \textit{r}, \textit{h} und \textit{l}, \textit{d} und \textit{l}, einmal \textit{p} und \textit{l}, und zweimal \textit{b} und \textit{l}.
}
\noindent \sed{Dabei verhielten sich nur in den wenigsten Fällen die Dialekte gegeneinander consequent. Ich gebe hier zur Probe die Statistik von sieben Lautvertretungen innerhalb der vier Dialekte: Guipuzcoanisch (\textit{\uline{g}}.), biscaisch (\textit{\uline{b}}.), labourdinisch (\textit{\uline{l}}.) und niedernavarresisch (\textit{\uline{bn}}.).}

\begin{table}[h]
\centering
\sed{
\tabcolsep=0.2cm
\begin{tabular}{ c || c | c || c | c || c | c || c | c || c | c || c | c || c | c } 
& \multicolumn{2}{c||}{ I.} & \multicolumn{2}{c||}{ II.} & \multicolumn{2}{c||}{ III.} & \multicolumn{2}{c||}{ IV.} & \multicolumn{2}{c||}{ V.} & \multicolumn{2}{c||}{ VI.} & \multicolumn{2}{c}{ VII.} \\
% \lsptoprule
\hline 
\hline
& \itshape u & \itshape i & \itshape k & \itshape g & \itshape p & \itshape b & \itshape g & \itshape z & \itshape z & \itshape s & \itshape g & \itshape b & \itshape d & \itshape r \\
% \lsptoprule
\hline 
\hline
\itshape \uline{g}. & 6 & 11 & 6 & 8 & 5 & 4 & 2 & 1 & 5 & 1 & 5 & 1 & 3 & 2 \\
\itshape \uline{b}. & 8 & 4 & 7 & 7 & 4 & 2 & 3 & 2 & 3 & 4 & 2 & 1 & 2 & 3 \\
\itshape \uline{l}. & 6 & 7 & 10 & 9 & 3 & 5 & 4 & 3 & 7 & 10 &  & 5 & 3 & 3 \\
\itshape \uline{bn}. & 5 & 8 & 5 & 5 &  & 2 & 3 & 3 & 13 & 6 & 3 & 2 & 1 & 2 \\
\end{tabular}
}
\end{table}

\largerpage
\sed{Merkwürdig nun, das Kabylische weist ganz entsprechende Lautschwankungen auf, sogar den Wechsel zwischen Labial und Zitterlaut: dreimal \textit{b} und \textit{r}, viermal \textit{b} und \textit{l}. Natürlich finden sich analoge Lautvertretungen zwischen dem Kabylischen und dem Baskischen; die Übereinstimmungen mussten sehr zahlreich und sehr einleuchtend sein, um überhaupt Beweiskraft zu haben. Aber zu diesen Übereinstimmungen gehört eben auch die auf beiden Seiten herr\-}{\textbar}{\textbar}300{\textbar}{\textbar}\phantomsection\label{sp.300}\sed{schende Verwilderung. Und durchwandern wir weiterhin die übrigen hamitischen und dann noch weiter die semitischen Sprachen, so stossen wir überall auf ähnliche Erscheinungen. Es mag uns noch so hart ankommen: hier müssen wir annehmen, es haben in der Urzeit die Lautbilder nur in unsicheren Umrissen dem Sprachgefühle vorgeschwebt, Laune, Stimmung, Rhetorik, Lautmalerei habe innerhalb der weit gezogenen Grenzen, bald hier- bald dorthin gegriffen, immer sicher, doch verstanden zu werden, und nebenbei durch die jeweils gewählte Lautfärbung noch besondere Eindrücke erweckend. Massenhafte Völkermischungen mochten zuerst das Lautwesen durcheinander geschüttelt haben: – um die festen Lautgesetze war es damit geschehen. Nun aber hatten die Sprachen die Wahl: sollte die Confusion und Doublettenwirtschaft fortdauern, sollte die Freiheit sinnig verwerthet, oder sollten in der weiteren Sonderentwickelung neue feste Formen geschaffen werden? Die blieben aber dann immer Kinder der Revolution, und hatte vor der Verwirrung einmal ein festeres Lautwesen geherrscht, so ist in ihnen nichts mehr davon zu bemerken, die neue Ordnung konnte die alte nicht voll ersetzen. Ob solche festere Zustände im Vorleben als nothwendig vorauszusetzen, ob sie in der Sonderentwickelung mit Sicherheit zu erwarten sind, mag ich nicht a priori entscheiden. Jedenfalls müssen wir die Thatsache anerkennen, dass in gewissen Sprachen und in gewissen Phasen ihres Lebens Zustände möglich sind, wo die Wörter ebenso unsicher Lautbilder darstellen, wie sonst etwa die Wurzeln, und wo die Grenzen der von der Sprache anerkannten Einzellaute ebenso durcheinanderlaufen, wie bei den meisten Wörtern die Grenzen der Bedeutungen.}

\largerpage
\sed{Einem solchen Wirrsale gegenüber versagt freilich die alterprobte Methode der phonetischen Sprachvergleichung ihren Dienst. Die Wissenschaft steht hier vor einer schweren Wahl. Entweder sie schlägt den Thatsachen einfach in’s Gesicht, erklärt: „das ist unmöglich, ist Unsinn, denn es ist unverständlich“. So sagt der „gelehrte Herr“ in Goethe’s Faust, der ein sehr feiger und sehr beschränkter Herr ist; höchstens wählt er andere Worte: „unvereinbar mit den bewährtesten Gesetzen unserer Wissenschaft,“ – oder so ähnlich. Besser schon, wenn er erklärt: „Daran wage ich mich nicht, denn hier kann ich mit meinen Mitteln nichts ausrichten.“ Ultra posse nemo obligatur, – genug, wenn er anerkennt, dass die Schranken seines Könnens nur seine Schranken sind. Ein Dritter, Kühnerer, geht vielleicht der Sache auf dem ihm geläufigen Wege zu Leibe, nimmt soviele Urlaute, das heisst verschiedene Laute der Ursprache an, als er Lautentwickelungen sieht, gelangt mit der Zeit zu \textit{t}\textsuperscript{15}, \textit{d}\textsuperscript{21} und \textit{k}\textsuperscript{37} und verzichtet auf Weiteres. Ein Anderer mag versuchen, sich neue Werkzeuge zu schmieden, womit er den neuen Stoff bearbeitet; und wer die Sprachwissenschaft soweit erstrecken will, wie menschliche Sprachen reichen, der muss den Versuch wagen.}

{\textbar}{\textbar}301{\textbar}{\textbar}\phantomsection\label{sp.301}

\sed{Neue Werkzeuge, das heisst neue Kategorien. Und das ist in unserem Falle entsetzlich schwierig: man möchte mit dem Wischer zeichnen, mit dem Vertreiberpinsel malen, zu jedem Laute, jedem Worte, das man niederschreibt, möchte man beifügen: „oder so ähnlich“. Auf Baskisch heisst ein kleiner Hügel: \textit{muru}, \textit{murru}, \textit{mora}, \textit{murko}, \textit{burko}, \textit{morroko}, \textit{mulko}, \textit{mulho}, \textit{muillo}, \textit{mulzo}, \textit{muno}, \textit{munho}, oder so ähnlich; zart, weich heisst \textit{malgu}, \textit{malba}, \textit{malso}, \textit{malšo}, \textit{mardo}, \textit{merda}, \textit{merzil}, \textit{mirigosa}, oder so ähnlich; die Kabylen nennen einen Kasten oder so etwas Ähnliches \textit{ageruž}, \textit{aγeluz}, \textit{aqarur} (\textit{θaqarurt}), oder so ähnlich; der Tuareg stellt sich unter Wörtern vom ungefähren Klange von \textit{tesokalt}, \textit{tašakalt}, \textit{sukalt}, \textit{asilka}, \textit{aserwi} einen Löffel, unter Wörtern wie \textit{ahenka\uline{d}}, \textit{ašīnke\uline{d}}, \textit{azenkaz}, \textit{enhar}, \textit{agingera} eine Gazelle vor, bei \textit{adekar}, \textit{etkar}, \textit{etkaj}, \textit{iǰǰur} u. dgl. denkt man an zürnen, bei \textit{uhal}, \textit{ušal}, \textit{ošel}, \textit{ašel}, \textit{a\uline{zz}el}, \textit{a\uline{z}el}, \textit{hasar} und Ähnlichem an laufen, bei \textit{agor}, \textit{esar}, \textit{taγeda} u.~s.~w. an einen Speer. Es handelt sich hier nicht um Lautverschiebungen oder Lautvertretungen, sondern um Lautverwischungen und -vermischungen, nicht um Lautgesetze, sondern um Lautmöglichkeiten, deren jede leicht an einer genügenden Zahl anderer Beispiele nachzuweisen wäre. Und darauf kommt es allerdings an; die Thatsachen müssen geradezu zwingend sein, ehe man ihnen die erprobtesten Regeln der Forschung opfern mag. Aber wer vor zwingenden Thatsachen die Augen verschliesst, opfert noch weit mehr, denn er opfert seiner Ängstlichkeit oder Rechthaberei eine Erkenntniss.}

\begin{styleAnmerk}
\sed{Anmerkung. \textsc{Max Müller}’s Theorie von den unsicheren Lautbildern macht eine der seltensten Ausnahmen zur Regel und ist gerade da, wo ihr Erfinder sie am Liebsten anwendet, auf indogermanischem Gebiete, wohl so gut wie überwunden. Gerade unsere Urahnen müssen sehr scharf und gleichmässig articulirt haben; Lautdoubletten werden ihrer Sprache nur in sehr seltenen Fällen zuerkannt. Auch die geschichtlichen Forschungen in anderen Sprachfamilien, der ural-altaischen, drâvidischen, malaio-polynesischen, bantuischen und, soviel sich voraussehen lässt, der indochinesischen, zielen auf immer klarere Gesetze der Lautverschiebung und Lautvertretung. Für solche spricht also überall die erfahrungsmässig begründete Vermuthung, die aber im einzelnen Falle der neuen Erfahrung, der Macht unleugbarer Thatsachen weichen muss.}
\end{styleAnmerk}