\chapter*{\sed{Erstes Buch.}}
\section*{\sed{Allgemeiner Theil.}}
\pdfbookmark[0]{Erstes Buch.}{erstesbuch}
\pdfbookmark[1]{I. Capitel.}{I.Capitel}
\section*{I. Capitel.}
\cehead{{{\large I,}} I. Begriff der Sprachwissenschaft.}
\pdfbookmark[2]{§. 1. Begriff der Sprachwissenschaft}{I.I.1}
\subsection*{Begriff der Sprachwissenschaft.}
\subsection*{§. 1.}\phantomsection\label{I.I.1}
\fed{{\textbar}1{\textbar}}\phantomsection\label{fp.1} \sed{{\textbar}{\textbar}1{\textbar}{\textbar}}\phantomsection\label{sp.1}
Wenn eine Wissenschaft damit beginnt, dass sie sich selbst definirt, so unternimmt sie eine vorläufige Rechtfertigung ihres Bestehens, die unmittelbar die Erhebung gewisser Ansprüche bedeutet. Dies, erklärt sie, ist mein Gebiet; kein Anderer hat es bisher in Besitz genommen, kein Anderer soll es künftig beanspruchen. So verlangt sie nicht Duldung, sondern Anerkennung und übt mit dem Augenblicke, wo sie ihrer selbst bewusst wird, ihren Nachbarinnen gegenüber ein Recht der Ausschliessung. Ein solches Recht setzt eine feste Grenzbestimmung voraus; und diese soll nun unternommen werden mit aller Pedanterie und Umständlichkeit. Wenn irgendwo, so ist hier die Wissenschaft zugleich befugt und genöthigt einen grossen Theil des Publicums rücksichtslos zu langweilen; – den Theil des Publicums meine ich, der bei Begriffsbestimmungen den Ruf zur Sache erhebt, weil er nicht begreift, dass Begriffsbestimmungen zur Sache gehören.

„Sprachwissenschaft ist die Wissenschaft von der Sprache“, – so erklärt die grammatische Definition, die keine sachliche ist noch sein will. Was ist Sprache? das ist die Frage, welche die Sprachwissenschaft zweimal zu beantworten hat, zum ersten Male vorläufig, in der Absicht ihr Gebiet zu umschreiben, zu zeigen, wie dies Gebiet sich gegen andere \update{abgränze:}{abgrenze:} das ist die Wortdefinition, die einem Umrisse gleicht. Die zweite Definition wird sie in ihrem ganzen Verlaufe und Wirken zu liefern haben: den Begriff der Sache, der die Sache erschöpfen soll. Das ist die Sachdefinition, die sich mit dem ausgeführten Bilde vergleichen lässt. Wir haben es vorerst mit der Wortdefinition zu thun. Diese wäre mit wenigen Worten als etwas Feststehendes auszusprechen wie ein Gesetzesparagraph. Wir aber wollen sie von weiteren zu immer engeren \fed{{\textbar}2{\textbar}}\phantomsection\label{fp.2} Kreisen fortschreitend finden; der Erfolg wird zeigen, ob wir recht daran thuen.

\sed{{\textbar}{\textbar}2{\textbar}{\textbar}}\phantomsection\label{sp.2}

\cohead{§. 2. Menschliche Sprache.}
\pdfbookmark[2]{§. 2. Begriff der menschlichen Sprache}{I.I.2}
\subsection*{§. 2.}\phantomsection\label{I.I.2}
\subsection*{Begriff der menschlichen Sprache.}

Gleich beim \update{Beginne}{Beginn} unserer Erörterungen zeigt uns die menschliche Sprache eine Eigenschaft, die uns fernerhin noch öfter beschäftigen wird; sie \so{anthropomorphisirt}, d.~h. sie überträgt menschliches Sein und Thun auf die aussermenschliche Welt. Ich rede, der Andere hört und versteht mich; also ist für ihn die Rede eine Sinneswahrnehmung, die er versteht; und was er versteht, ist nicht die Wahrnehmung allein, sondern der in ihr enthaltene Sinn. Einen Sinn in einer Wahrnehmung finden heisst sie deuten. Nun sucht der denkende Geist jede Wahrnehmung zu deuten, und so redet Alles zu ihm, es mag wollen oder nicht. In diesem Verstande redet man von der Sprache der Natur und lässt die Steine eines alten Gemäuers Geschichte erzählen.

Allein anders wird die Sprache der Natur vom Naturforscher gedeutet, anders vom Naturmenschen; und anders deutet der Alterthumsforscher die Sprache der Steine, anders der Maurer, der sie wegbricht. Und doch haben in beiden Fällen die Zwei genau dasselbe vor Augen gehabt, und beide haben richtig gedeutet. Daraus folgt, dass diese Sprachen mehrdeutig sind. Sprache soll aber eindeutig sein, denn nur das Eindeutige ist verständlich. Und sie muss nicht nur Verständnissgrund des Einen, sondern auch Verständnissmittel des Anderen sein, mithin freiwillige Aeusserung, denn nur wo ein Wille ist, kann von einem Mittel die Rede sein. Mit anderen Worten: Sprache verlangt erst ein Ich und dann ein Du. Da hätten wir zwei Gründe, warum eine Sprache unbeseelter Dinge nur im uneigentlichen Sinne möglich ist; der Vergleich \update{hinkt,}{hinkt} und zwar nicht \update{blos}{bloss} auf einem Beine.

Besser, so scheint es, steht es mit der sogenannten \update{\so{Geberdensprache:}}{\so{Geberdensprache;}} der Eine macht dem Anderen ein sichtbares Zeichen, das dieser so versteht und nur so verstehen kann, wie es von Jenem gemeint ist. Hier haben wir also die zwei Merkmale der Absichtlichkeit und der Eindeutigkeit; aber ein drittes fehlt, das zur Sprache im eigentlichen Sinne nicht minder wesentlich ist: die Stimme, die das Zeichen giebt, das Gehörorgan, das es aufnimmt. Vorhin redeten wir vom Ich und \fed{{\textbar}3{\textbar}}\phantomsection\label{fp.2} Du, jetzt reden wir von meinem Munde und deinem Ohre. Eins aber haben jene sichtbaren Zeichen doch mit der hörbaren Sprache gemein, und insofern vor der Sprache der Natur und der Culturdenkmäler voraus, nämlich die Vergänglichkeit: sie dienen dem Augenblicke und sind nicht mehr, sobald sie gedient haben.

Mit noch mehr Scheine Rechtens redet man von \so{Sprachen der stimmbegabten Thiere}. Hier treffen in der That alle bisher aufgestellten Erfordernisse zu. Das Thier bedient sich seiner Stimme, um sich verständlich zu machen, \sed{{\textbar}{\textbar}3{\textbar}{\textbar}}\phantomsection\label{sp.3} und es wird verstanden, nicht nur von Seinesgleichen, sondern auch von dem beobachtenden Thierfreunde. Käme es nur auf die Lebhaftigkeit des Aus- und Eindruckes an, so wüsste ich nicht, was an den Sprachen des Hundes und der Singvögel zu vermissen wäre: ihre rhetorische Leistungsfähigkeit ist erstaunlich. Gerade diese aber haben sie mit den Gesten und Mienen gemeinsam, deren Bedeutsamkeit und tiefe Wirkung auf das Gemüth wir an den Meistern der Schauspielkunst bewundern. Soweit man aber die Thiersprachen bisher erforscht hat, gleichen sie den Gesten noch in einem anderen, weniger vortheilhaften Stücke: was sie ausdrücken sind Empfindungen oder höchstens Gesammtvorstellungen, nicht in ihre Glieder zerlegte Gedanken. Ein Thier, das Schmerz empfindet, mag in seiner Sprache rufen: \so{Au!} aber ein Gebilde wie unsern Satz: Ich empfinde Schmerz, oder wie das lateinische \update{doleo}{„doleo“} vermag es nicht zu schaffen, es mag wohl auch in seiner Sprache sagen: Burr! oder Plautz! aber es sagt nicht: Wir wollen auffliegen, oder: Da fällt etwas. Hier haben wir eine Fähigkeit, die bisher nur an menschlicher Rede, aber auch an aller menschlichen Rede beobachtet worden ist: die Zerlegung der Vorstellung (Analyse), der der \so{gegliederte Ausdruck des Gedankens} entspricht. Jeder gegliederte Gedankenausdruck ist selbstverständlich ein gewollter und in der Regel eindeutiger. Daher bedürfen wir dieser zwei Merkmale nun nicht länger und fassen unsere Definition dahin zusammen: \so{Menschliche Sprache ist der gegliederte Ausdruck des Gedankens durch Laute}.

Es sei schon hier bemerkt, dass diese Definition ein Mehreres in sich fasst. Zunächst gilt die Sprache als Erscheinung, als jeweiliges Ausdrucksmittel für den jeweiligen Gedanken, d.~h. als \so{Rede}. Zweitens gilt die Sprache als eine einheitliche Gesammtheit solcher Ausdrucksmittel für jeden beliebigen Gedanken. In diesem Sinne reden wir von der Sprache eines Volkes, einer Berufsklasse, eines Schriftstellers u.~s.~w. \fed{{\textbar}4{\textbar}}\phantomsection\label{fp.4} Sprache in diesem Sinne ist nicht sowohl die Gesammtheit aller Reden des Volkes, der Classe oder des Einzelnen, – als vielmehr die Gesammtheit derjenigen Fähigkeiten und Neigungen, welche die Form, derjenigen sachlichen Vorstellungen, welche den Stoff der Rede bestimmen. Endlich, drittens, nennt man die Sprache, ebenso wie das Recht und die Religion, ein Gemeingut der Menschen. Gemeint ist damit das \so{Sprachvermögen}, d.~h. die allen Völkern innewohnende Gabe des Gedankenausdruckes durch Sprache.

Nur mit der menschlichen Sprache hat es unsere Wissenschaft zu thun, und zwar vorwiegend mit der specifisch menschlichen, d.~h. mit der gegliederten, nur nebenher mit denjenigen Lautäusserungen, die dem Menschen mit dem Thier gemein sind.

Wie nun, wenn es gelänge, auch bei dieser oder jener Thierart eine der menschlichen ähnliche, gegliederte Sprache zu entdecken? Der Gedanke scheint vielleicht paradoxer, als er ist. Die Sprache ist ein Erzeugniss der Gesellschaft, \sed{{\textbar}{\textbar}4{\textbar}{\textbar}}\phantomsection\label{sp.4} und gewisse Thiere haben die Gesellschaft höher entwickelt, als viele Menschenvölker. Die Ameisen bauen wunderbar planmässige Ansiedelungen, scheiden sich in Berufsstände; manche von ihnen treiben Viehzucht und Landwirthschaft zur Ernährung ihres Milchviehes. Da redet man von Instinct, setzt ein y für ein x. Dies eine Mal soll die gleiche Wirkung nicht der gleichen Ursache entspringen, das zweckmässige vielseitige Zusammenwirken einer grossen gegliederten Gesellschaft nicht auf einem entsprechenden Geistesverkehre beruhen. Seltsam, die uns körperlich verwandtesten Thiere sind mit Nichten die sprachbegabtesten. Wie nun, wenn jene Ameisen sich zu ihren Verwandten auch im Punkte der Sprache ähnlich verhielten, wie der Mensch zu den anthropoiden Affen? Der Gedanke ist meines Wissens von einem Naturforscher ausgesprochen worden, und die Naturforscher mögen über seine Annehmbarkeit entscheiden. Gesetzt aber, er fände seine Bestätigung: wem fiele die Sprache als Untersuchungsobject zu? Ich meine, nicht dem Sprachforscher, weil es eben keine menschliche Sprache ist, d.~h. keine Menschensprache.

Auf diesen Punkt wollte ich kommen, auf die Gefahr hin, den Weg durch eine Utopie zu nehmen, mit dem vollen Bewusstsein, dass er praktisch noch unerheblich ist, vielleicht nie erheblich wird. In der That ist er für die Stellung der Sprachwissenschaft mit entscheidend. Sprache ist eine \update{Function}{Funktion} –, ihr Vermögen eine Kraft des Menschen. \fed{{\textbar}5{\textbar}}\phantomsection\label{fp.5} Was soll nun entscheiden: die Art der Kraft und ihrer Wirkungen, wie in der Physik, oder das Subject, wie in der Geschichte? Im ersteren Falle wäre die Sprachwissenschaft ich weiss nicht welches Gemisch von vergleichender Physiologie und Psychologie, – im zweiten Falle ist sie ein Theil der grossen Wissenschaft vom Menschen. So eröffnet sich uns an dieser Stelle eine erste Aussicht auf eine viel behandelte Streitfrage.

Wir kehren nun zu unserer Definition der menschlichen Sprache zurück, um zwei in sie aufgenommene Begriffe zu untersuchen, den der Lautsprache und den des Gedankens.

\cohead{§. 3. Lautsprache, Articulation.}
\pdfbookmark[2]{§. 3. Lautsprache, Articulation.}{I.I.3}
\subsection*{§. 3.}\phantomsection\label{I.I.3}
\subsection*{Lautsprache, Articulation.}

Soviel mir bekannt, pflegte man bisher in die Definition der menschlichen Sprache einen Begriff aufzunehmen, der scheinbar in meiner Definition nicht mit enthalten ist: den des \so{articulirten Lautes}. Es fragt sich: Was ist Laut? was ist Articulation?

\textsc{Techmer} geht bei seiner Definition von Letzterer aus: „Articulation sei die schallbildende Abweichung der Sprachorgane von der natürlichen Gleichgewichtslage. Die simultanen Articulationen, die treibenden und hemmenden Kräfte, seien im Kampfe.“ Den Laut definirt er genetisch als „resultirend aus dem \sed{{\textbar}{\textbar}5{\textbar}{\textbar}}\phantomsection\label{sp.5} labilen Gleichgewichte der gleichzeitig wirkenden articulatorischen Kräfte im Kampfe“ (Internationale Ztschr. f. allgem. Sprachw. \update{I,}{I.} S.~109).

Hiernach würde Articulation in den Begriff des Lautes gehören; articulirter Laut wäre ein Pleonasmus, und unarticulirter Laut eine contradictio in adjecto.

Zuvor hatte \textsc{Techmer} die Laute in Schall- und Geräuschlaute getheilt und beide definirt als „solche, kürzere oder längere Zeit sich gleichbleibende Schwingungsweisen in der Aufeinanderfolge der \so{sprachlichen} Schallbewegungen, welche vorwiegend resp. Klang- oder Geräuschcharakter zeigen“ (das. S.~73).

Darnach würde Sprache ein Moment im Begriffe des Lautes und folglich auch im Begriffe der Articulation sein. Und folglich dürfte in die Definition der Sprache nicht das Merkmal der articulirten Laute aufgenommen werden. Doch \update{dies}{dieses} Hinderniss wäre wohl zu beseitigen, wenn man, statt von sprachlichen, etwa von solchen Schallbewegungen redete, \fed{{\textbar}6{\textbar}}\phantomsection\label{fp.6} welche durch die und die Organe hervorgebracht werden und Mischungen von Klängen und Geräuschen darstellen.

Jedenfalls ist somit der Ausdruck Articulation von der Lautphysiologie in Anspruch genommen worden; ihre Berechtigung hierzu brauchen wir nicht zu bestreiten, nicht einmal zu prüfen. Das aber interessirt auch uns, dass nach diesen Definitionen der articulirte Laut nicht mehr als ausschliessliche Eigenschaft der menschlichen Rede gelten darf. Die Lautphysiologie ist ein Theil der Physiologie: sie handelt von gewissen Functionen gewisser Organe des thierischen Körpers. Was die Laute ausdrücken, ob menschliche Gedanken oder thierische Gefühlsregungen, geht den Lautphysiologen als solchen schlechterdings nichts an. Auch kann der Physiolog von seinem Standpunkte aus keinen Unterschied zwischen Menschen und Thier anerkennen, weil er ihn mit seinen Mitteln nicht nachweisen kann; er hat es nur mit dem Körper zu thun, und als körperliches Wesen ist der Mensch eben ein Thier; und wenn das Thier mit den gleichen Organen gleiche akustische Wirkungen erzielt, wie der Mensch, so hat sich der Lautphysiolog dabei zu beruhigen. Ich weiss nicht, ob ein Schaf sein Mää genau mit denselben Stimmorganen hervorbringt, mit denen der Mensch es nachahmt. Gesetzt, dies wäre der Fall, so wüsste ich nicht, warum jenes Mää für weniger articulirt gelten sollte, als etwa unser „mähe“ oder das französische „mais“. Und dasselbe gilt erst recht von mehrsylbigen Rufen der Thiere, wie dem Schreie des Kukuks, dem Kikeriki des Hahnes.

Articulation ist Gliederung, – das besagt der Name. Soll sie aber zu den entscheidenden Merkmalen der menschlichen Sprache gehören, so kann ihr Wesen nicht, oder doch nicht allein in der Art ihrer mechanischen (physiologischen) Hervorbringung und in ihrer akustischen Wirkung bestehen, sondern die Gliederung muss in Rücksicht auf einen Zweck gedacht werden, durch den sie zur \sed{{\textbar}{\textbar}6{\textbar}{\textbar}}\phantomsection\label{sp.6} menschlichen Articulation gestempelt wird. Und diesen Zweck hatten auch Frühere in den Begriff der Articulation aufgenommen (Vgl. \textsc{Techmer}, a.~a.~O. S.~107–108).

Der Zweck der Sprache ist der Ausdruck des Gedankens. Der Gedanke und seine Theile müssen mit einem ausreichenden Grade von Energie in’s Bewusstsein treten, um zum sprachlichen Ausdrucke zu drängen. Energie heisst in diesem Falle soviel als Klarheit. Sich einen Gedanken klar machen heisst ihn zergliedern. Dem Ergebnisse dieser Zergliederung soll der sprachliche Ausdruck entsprechen, mithin muss er \fed{{\textbar}7{\textbar}}\phantomsection\label{fp.7} selbst gegliedert, d.~h. articulirt sein. Ich habe mich in meiner Definition des deutschen Ausdruckes bedient, um den lateinischen den Lautphysiologen unbestritten zu überlassen; und ich habe von einem gegliederten Ausdrucke durch Laute, nicht von einem Ausdrucke durch gegliederte Laute geredet, um anzudeuten, dass die Gliederung eine für den Zweck des sprachlichen Ausdruckes gewollte, nicht blos eine physiologische, nicht die im Wesen des Lautes allein liegende sei.

\cohead{§. 4. Der Gedanke.}
\pdfbookmark[2]{§. 4. Der Gedanke.}{I.I.4}
\subsection*{§. 4.}\phantomsection\label{I.I.4}
\subsection*{Der Gedanke.}

\textsc{Steinthal} sagt in seiner Charakteristik der hauptsächlichsten Typen des Sprachbaues S.~93: „Man sprach unter Menschen von jeher und allüberall; man denkt aber nur seit Sokrates, und nur in dem engen Kreise der Wissenschaft – im strengen Sinn des Denkens.“ Was dieser „strenge Sinn des Denkens“ sei, hat er zuvor durch ein Citat aus \textsc{Lotze} gezeigt, und zwar engt er auch dessen Definition noch ein; denn \textsc{Lotze} kennt neben dem logischen Denken noch einen „psychologischen Gedankenlauf oder ein Denken, welches noch nicht von dem Geiste, dem Logos der Vernunft durchdrungen ist“, während das logische Denken „in einer fortwährend ausgeübten Kritik besteht, die ... der vernünftige Geist dem Vorstellungsmateriale angedeihen lässt“. \textsc{Steinthal} nennt nur dies ein Denken im strengen Sinne des Denkens. Dass er ein solches u. A. dem Pythagoras, den ägyptischen, chaldäischen und chinesischen Astronomen abzusprechen scheint, geht uns hier nichts an. Unter dem „Denken im strengen Sinne des Denkens“ versteht er nun aber wohl nichts \update{anderes,}{Anderes,} als Denken im strengen Sinne des Wortes, m.~a.~W. er sagt: die Wissenschaft dürfe den Ausdruck nur vom bewusst logischen, kritischen Denken gebrauchen. Wenn er gelegentlich vom „Denken in der Sprache“ redet, das „noch kein echt und rein logisches Denken“ sei, so ist das eben ein Zugeständniss, das er dem gemeinen Sprachgebrauche macht. Anderwärts redet er vom „gewöhnlichen Denken, welches am Faden des psychologischen Mechanismus abläuft. Da“, sagt er, „haben nicht \so{wir} gedacht, sondern es ist in uns gedacht worden; unsre Seele war Schauplatz des Denkens. Beim Denken in logischen Formen dagegen waltet eine Thätigkeit des Geistes, die als \sed{{\textbar}{\textbar}7{\textbar}{\textbar}}\phantomsection\label{sp.7} eine wahrhaft subjektive That sich über jenes Schauspiel der Ideenassociation erhebt“.

\fed{{\textbar}8{\textbar}}\phantomsection\label{fp.8}

Natürlich habe ich in meiner Definition das von Steinthal so genannte „gewöhnliche Denken“ gemeint. Dass ich es aber nicht ausdrücklich so bezeichnet habe, bedarf keiner Rechtfertigung. Denn der \retro{Philosoph}{Philisoph} kann und soll mir wohl erklären, was Denken ist, aber er kann mir nicht vorschreiben, dass ich das Wort gegen den allgemeinen Sprachgebrauch nur in einem beliebig eingeengten Sinne anwende. Die Wissenschaft hat wohl ein Interesse daran, die Wörter genau zu definiren; aber sie hat kein Interesse daran, sich in Unverständlichkeit zu hüllen, indem sie die Begriffe wider den Sprachgebrauch \update{begränzt;}{begrenzt;} und die Sprachwissenschaft ist die letzte, die solchen Willkürlichkeiten nachgeben darf.

Soviel zur Wortdefinition, die festzustellen hat, was unter den Begriff der menschlichen Sprache falle, und was nicht. Ob die Sprache eine göttliche oder eine menschliche Schöpfung, ob sie ein Werk oder eine Bethätigung (ἔργον oder ἐνέργεια), ob und in welchem Sinne sie ein Organismus sei, und alles Andere gehört nicht in die Wort-, sondern in die Sachdefinition.

\cehead{{{\large I,}} II. Aufgaben der Sprachwissenschaft.}
\pdfbookmark[1]{II. Capitel}{I.II}
\subsection*{II. Capitel.}
\pdfbookmark[2]{§. 1. Aufgaben der Sprachwissenschaft.}{I.II.1}
\section*{Aufgaben der Sprachwissenschaft.}
\subsection*{§. 1.}\phantomsection\label{I.II.1}

Man kann sich mit fremden Sprachen beschäftigen, um sie praktisch zu verwerthen, sich in ihnen zu unterhalten, in ihnen zu lesen oder zu schreiben. Wir \so{erlernen} da die fremde Sprache ähnlich wie wir als Kinder unsere Muttersprache erlernt haben, nur vielleicht auf anderem Wege. So werden uns in unserer Jugend die altclassischen Sprachen, vielleicht auch durch Lehrer oder Bonnen das Französische oder Englische beigebracht. Die Methode des Unterrichts mag rein \update{praktisch}{praktisch,} oder mehr wissenschaftlich, ihr Erfolg mag eine mässige Fertigkeit oder völlige Meisterschaft in der fremden Sprache sein: einerlei, – solche Spracherlernung ist noch keine sprachwissenschaftliche That. Denn nicht die Methode, auch nicht der erzielte Wissensgewinn, sondern die Betrachtungsweise, der Zweck macht die Wissenschaft. Betrachte ich den Lehrgegenstand als Mittel zu einem ausserhalb seiner liegenden Zwecke, so \fed{{\textbar}9{\textbar}}\phantomsection\label{fp.9} betrachte ich ihn nicht wissenschaftlich, sondern praktisch; dann hat die Sache und ihre Erkenntniss für mich nur insoweit Werth, als Beide jenem Zwecke zu Statten kommen.

Die Sprachwissenschaft bezweckt Erkenntniss der Sprache um ihrer selbst willen. \sed{Ihr Gegenstand ist alle menschliche Sprache, sind also alle menschlichen} {\textbar}{\textbar}8{\textbar}{\textbar}\phantomsection\label{sp.8} \sed{Sprachen, die der Wilden sowohl wie die der gesitteten Völker, die neuen so gut wie die alten, die kleinsten Dialekte nicht weniger, als die grossen Sprachfamilien.} \update{Sie will ihren Gegenstand}{Und sie will diesen ihren Gegenstand} allseitig begreifen, darum hat sie vor Allem zu fragen, welche Seiten er biete? Soviele Seiten, soviele Standpunkte der Betrachtung, soviele Aufgaben der Wissenschaft.

\cohead{§. 2. A. Die Einzelsprachen.}
\pdfbookmark[2]{§. 2. Die Einzelsprachen.}{I.II.2}
\subsection*{§. 2.}\phantomsection\label{I.II.2}
\subsection*{A. Die Einzelsprachen.}

Jede Sprache ist Gemeingut einer grösseren oder kleineren Anzahl Menschen, die wir vorläufig ein Volk nennen wollen, weil in der Regel Sprachgemeinschaft und nationale Gemeinschaft zusammenfallen. Dass diese Regel viele Ausnahmen erleidet, ist bekannt; darum wird der Ausdruck, soweit er unzutreffend ist, keine Missverständnisse verschulden. Passender wäre etwa der Ausdruck Sprachgemeinde, welche naturgemäss auch alle Ausländer in sich begreift, die zu uns in unserer Muttersprache reden. Zu einer solchen Sprachgemeinde gehören nun alle Mitredenden, aber auch nur die Mitredenden, und zu diesen gehören auch Todte. Jedes Volk hat überlieferte Sprichwörter, wohl auch Lieder oder Sagen, die sich in gefestigter Form von Geschlechte zu Geschlechte vererben. Wir lesekundigen und leselustigen Völker stehen überdies durch schriftliche Literaturen unter \update{stäter}{steter} sprachlicher Einwirkung unserer Altvordern. Jetzt dürfte der Ausdruck, \so{dass die ganze Sprache in jedem Augenblicke lebt}, weder überflüssig noch misszuverstehen sein. Was nicht mehr in der Sprache lebt, gehört nicht mehr zu ihr, so wenig wie der ausgefallene Zahn oder das amputirte Bein noch zum Menschen gehört. Dies besagt der Satz in negativer Richtung. In positiver behauptet er aber, dass jede lebende Sprache in jedem Augenblicke etwas Ganzes ist, und dass nur das im Augenblicke Lebende in ihr wirkt.

Es scheint, man könne nicht leicht einen faderen Gemeinplatz \update{aussprechen,}{aussprechen} und doch handelt es sich hier um eine Thatsache, die man oft und leicht verkennt. Man bildet sich nur zu gern ein, man wisse, warum etwas jetzt ist, wenn man weiss, wie es früher gewesen ist, und die einschlagenden Gesetze des Lautwandels kennt. Das ist aber nur insoweit richtig, als diese Gesetze allein die Schicksale der Wörter und Wort\-\fed{{\textbar}10{\textbar}}\phantomsection\label{fp.10}formen bestimmen. Weiss ich z.~B., dass lateinisches f im Spanischen zu h, li vor Vocalen zu j (sprich χ), und die Endung der zweiten Declination im Singular o, im Plural os geworden ist: so ist es mir erklärlich, wie filius zu hijo werden musste. Gesetzt nun, jedes Wort und jede Form der spanischen Sprache wäre auf diese Weise genetisch abgeleitet: wäre damit die spanische Sprache erklärt? Sicherlich nicht. Denn die Sprache ist ebensowenig eine Sammlung von Wörtern und Formen, wie der \sed{{\textbar}{\textbar}9{\textbar}{\textbar}}\phantomsection\label{sp.9} organische Körper eine \update{Sammlnng}{Sammlung} von Gliedern und Organen ist. Beide sind in jeder Phase ihres Lebens (relativ) vollkommene Systeme, nur von sich selbst abhängig; alle ihre Theile stehen in Wechselwirkung und jede ihrer Lebensäusserungen entspringt aus dieser Wechselwirkung. Die Lebensäusserung einer Sprache, richtiger die Sprache selbst, die ja nur eine Lebensäusserung ist, – ist die Rede, die unmittelbar aus der Seele des Menschen fliesst. Wollte man nun sagen: der Spanier spricht so, weil der Römer so gesprochen hat: so hätte das höchstens dann einen Sinn, wenn etwa der Spanier aus dem Lateinischen übersetzte. Nicht Ei, Raupe und Puppe erklären den Flug des Schmetterlings, sondern der Körper des Schmetterlings selbst. Nicht die früheren Phasen einer Sprache erklären die lebendige Rede, sondern die jeweilig im Geiste des Volkes lebende Sprache selbst, mit anderen Worten der \so{Sprachgeist}. Dieser ist recht eigentlich das erste Object der einzelsprachlichen Forschung, und soweit die Philologen ihm nachspüren, sind sie Linguisten, so gut wie die historischen Sprachvergleicher. Inwieweit der \so{Sprachschatz} der ausschliesslich einzelsprachlichen Bearbeitung unterliege, mag später festgestellt werden.

\clearpage\cohead{§. 3. B. Sprachgeschichte, Sprachstämme.}
\pdfbookmark[2]{§. 3. Sprachgeschichte, Sprachstämme.}{I.II.3}
\subsection*{§. 3.}\phantomsection\label{I.II.3}
\subsection*{B. Sprachgeschichte, Sprachstämme.}

Leben ist ununterbrochenes Werden, d.~h. Sichverändern. Heute sind wir nicht mehr wie wir gestern waren, und morgen wird unsere Muttersprache anders sein, als sie heute ist. Diese Veränderungen zu verfolgen, ihre Gesetze zu entdecken ist Aufgabe der Sprachgeschichte.

Die Völker, wenn sie nicht in ihrer Entwickelung gehindert werden, wachsen an Kopfzahl und breiten sich über weitere Gebiete der Erde aus. Die Nachkommen derselben Vorfahren werden, räumlich getrennt, einander fremd, die Sprachen spalten sich in Dialekte, die Völker in \fed{{\textbar}11{\textbar}}\phantomsection\label{fp.11} Stämme. Geschichtliche Mächte können dieser Spaltung Einhalt gebieten; thuen sie es nicht, so erweitern sich die Abstände je länger je mehr, aus den Stämmen eines Volkes werden verschiedene Völker, und aus den Dialekten einer Sprache verschiedene Sprachen. So reden wir von \so{Sprachfamilien} und \so{Sprachstämmen}, von \so{Tochter-} und \so{Schwestersprachen}, kurz von verschiedengradigen \so{Verwandtschaften}. Diese Ausdrücke sind längst in der Wissenschaft eingebürgert und völlig unverfänglich; denn Niemand wird vergessen, dass die Genealogie der Sprachen nicht Reihen verschiedener Individuen darstellt, sondern verschiedene \update{Entwicke\-lungs\-phasen}{Entwick\-lungs\-phasen} desselben Individuums.

Es gilt, diese Genealogie und die Veränderungsprozesse festzustellen, d.~h. nachzuweisen, welche Sprachen untereinander verwandt sind, und wie sie sich verwandtschaftlich \update{zueinander}{zu einander} verhalten und im Laufe der Zeit gestaltet haben. \sed{{\textbar}{\textbar}10{\textbar}{\textbar}}\phantomsection\label{sp.10} Dies kann nur durch eine wissenschaftliche Vergleichung geschehen; darum wird die genealogisch-historische Sprachwissenschaft die vergleichende κατ’ ἐξοχήν genannt, während doch in der That alle Linguistik Erfahrungswissenschaft, alle Erfahrungswissenschaft überwiegend inductiv, und alle Induction überwiegend vergleichend ist.

\sed{Die sprachgeschichtliche Forschung hat auf dem Gebiete der indogermanischen Sprachfamilie zwar nicht ihre ersten, aber jedenfalls ihre bedeutendsten Triumphe gefeiert, und gerade in Deutschland widmet sich noch jetzt die Mehrzahl der historischen Sprachforscher mehr oder minder einseitig dem Studium des eigenen Sprachstammes. Es ist das erklärlich und doch zu beklagen. Anderwärts gibt es noch so viel zu thun, und die Arbeit ist wahrhaftig nicht weniger lohnend. Neue Aufgaben und neue Erkenntnisse würden sich bieten, und man würde nicht immer und immer wieder die halb abgegraste Flur abweiden, während rings umher ein jungfräulicher Boden grünt. Doch das ist das Wenigste; es ist Sache der Liebhaberei, und der Wissenschaft kommt es zu Gute, wenn erst einmal ein kleines Theilgebiet bis in’s Kleinste durchgearbeitet wird. Nur soll man den Theil nicht für das Ganze ausgeben, soll nicht wähnen, es sei mit der geschichtlichen Vergleichung indogermanischer Laute, Wörter und Wortformen das Ganze oder auch nur das Beste der sprachwissenschaftlichen Arbeit gethan. Unter den drei Betrachtungsweisen, die unsere Wissenschaft verlangt, ist die historische nur eine. Unter den Dingen, die es an einer Sprache zu betrachten gilt, mögen Laut-, Wort- und Formenlehre knapp die Hälfte ausmachen. Und von den Sprachen unserer Erde stellen die indogermanischen nur einen kleinen Bruch\-theil dar, – von den Formen des menschlichen Sprachbaues vielleicht einen noch kleineren. Glaubt man an ihnen allein ermessen zu können, was in der Sprachenwelt möglich, was unmöglich oder nothwendig sei, so kann man zu Fehlschlüssen kommen, die um nichts verständiger sind, als der: der Fisch hat keine Lunge, folglich kann er nicht athmen. Die Zoologie darf nicht beim Kaninchen des Physiologen Halt machen, und die Linguistik nicht bei der indogermanischen Sprachfamilie; Beide haben nach Allgemeinheit zu streben, vor verfrühten Verallgemeinerungen sich zu hüten.}

\cohead{§. 4. C. Sprachvermögen.}
\pdfbookmark[2]{§. 4. Das Sprachvermögen; die allgemeine Sprachwissenschaft.}{I.II.4}
\subsection*{§. 4.}\phantomsection\label{I.II.4}
\subsection*{C. Das Sprachvermögen; die allgemeine Sprachwissenschaft.}

Eben jenes, dass alle Sprachwissenschaft Erfahrungswissenschaft \update{sei,}{ist,} hat man zuweilen \update{verkannt.}{erkannt.} Ausgerüstet mit einem unzureichenden Vorrathe thatsächlicher Kenntnisse, hat man sich eingebildet, deductiv, aus dem Wesen der Sache, schlussfolgern zu können, was Alles für die menschliche Sprache nothwendig, was in ihr möglich sei. So entstanden die s.~g. \so{allgemeinen oder philoso\-}\sed{{\textbar}{\textbar}11{\textbar}{\textbar}}\phantomsection\label{sp.11}\so{phischen Grammatiken}, meist Kinder unseres philosophischen Zeitalters, schöne Kinder zum Theil, aber nicht lebensfähige. Heute dürfen wir sie zu den Todten rechnen, brauchen also nicht mehr gegen sie zu kämpfen. Ihre Mittel und Wege waren verfehlt, aber ihre Ziele waren und bleiben berechtigt und verdienen gerade in unserer Zeit vertheidigt zu werden. Wir leben in einem Zeitalter der Entdeckungen, und wer an den neuen Errungenschaften seiner Wissenschaft theilnehmen will und kann, sitzt stets vor wohlbesetzter Tafel; die Bedienung ist fast zu prompt: noch hat er den Fisch auf der \fed{{\textbar}12{\textbar}}\phantomsection\label{fp.12} Gabel, da wird auch schon der Braten aufgetragen! Nicht Hunger hat er zu fürchten, sondern Übersättigung. Solche Zeiten sind für die Gewinnung allgemeiner Anschauungen nicht günstig: man fragt einander öfter: Hast Du davon erfahren? als man fragt: Was denkst Du darüber, wie fügst Du es in’s Ganze ein?

Dazu kommt ein Zweites: Wir Linguisten waren nachgerade drauf und dran, den Satz: \update{denominatio fit a potiori}{„Denominatio fit a potiori“} umzukehren. Die Meisten von uns haben ihre Arbeit auf die Erforschung der einen oder anderen Sprachfamilie beschränkt, und die genealogisch-historische Schule hat so glänzende Fortschritte zu verzeichnen, dass ihr eine gewisse Selbstgenügsamkeit nicht zu verdenken ist. Nichts lag näher, als zu sagen: Der Fortschritt der Sprachwissenschaft ist ganz und ausschliesslich bei dieser Schule; die draussen mögen sich Philologen, Sprachphilosophen, wohl auch Sprachkundige, Polyglotten nennen, oder wie es ihnen sonst beliebt, sie sollen nur nicht sich für Linguisten, nicht ihre Sache für Sprachwissenschaft ausgeben. Wer so spricht, der verwechselt den kleinen Acker, den er pflügt, mit der Flur einer grossen Gemeinde und urtheilt, um mich eines \retro{chinesischen}{chinesichen} Vergleiches zu bedienen, wie einer, der im Brunnen sitzt und behauptet, der Himmel sei klein. Die Wissenschaft kann nur \update{zuviel}{zu viel} erzählen von Meistern in der Beschränkung, die am Ende Meister wurden in der Beschränktheit, fleissigen Männern, die vor lauter Fleisse nie zu den Fenstern ihrer Werkstätten hinausschauten.

\sed{Will man die allgemeine Sprachwissenschaft als Sprachphilosophie bezeichnen, so kommt es darauf an, was man sich unter Philosophie denkt. Hiesse philosophiren soviel, wie: die Welt der Thatsachen durch apriorische Speculationen aufbauen wollen, so stünden wir wieder auf dem alten Flecke, und ich wüsste dann nicht, \corr{1901}{was}{wass} wir den Verfassern der „allgemeinen Grammatiken“ vorzuwerfen hätten. Sagt man gleichungsweise: Die Philosophie verhält sich zu den Einzelwissenschaften so, wie diese sich zu ihren Forschungsgebieten verhalten: so liefe die Sache auf Methodologie und allenfalls auf eine Geschichte der Sprachwissenschaft hinaus, und am Ende hätte der Sprachphilosoph mit den Sprachforschern so viel zu thun, dass ihm keine Zeit mehr übrig bliebe, sich mit den Sprachen selbst zu beschäftigen. Weist man drittens der Philosophie die Aufgabe zu, in der Mannigfaltigkeit der Erfahrungswelt die Einheit der obersten Prinzipien} {\textbar}{\textbar}12{\textbar}{\textbar}\phantomsection\label{sp.12} \sed{darzuthun, so scheint es, als würde dem Sprachphilosophen allein das Unmögliche zugemuthet, die gesammte Sprachwissenschaft zu beherrschen, oder als wäre ihm ein Freibrief zur leichtfertigsten Halbwisserei ausgestellt. Mit anderen Worten: nach der ersten Auffassung wäre die Aufgabe unsinnig, nach der zweiten sehr unerquicklich, nach der dritten entweder undurchführbar oder geradezu unverantwortlich.}

\sed{So schlimm stehen die Dinge nun doch nicht. Aprioristisch denken muss Jeder, der es unternimmt, der Wissenschaft ideale Ziele zu setzen. Jeder, der nicht die Arbeiten seiner Vorgänger von vorn an wiederholen will, muss aus der Vorgeschichte seiner Wissenschaft die ihm nöthigen Lehren ziehen. Und dass ein flüchtig dilettantisches Herumschnüffeln und Nippen da nichts gilt, wo den Thatsachen ihr Höchstes und Tiefstes abgewonnen werden soll, das wird jeder vorlaute Prinzipienentdecker über kurz oder lang zu seinem Schaden an sich selbst erfahren. Der Sprachphilosoph sei vor Allem Sprachforscher und Sprachenkenner, und das in so weitem Umfange und mit so tiefgehender Gründlichkeit, wie nur immer möglich. Alle Sprachfamilien, alle Sprachen kann er nicht mit seinem Wissen beherrschen. Dafür wird er sich möglichst viele und möglichst weit voneinander verschiedene Typen anzueignen suchen, um sich eine Vorstellung zu bilden von der vielgestaltigen Welt, der er ihre Gesetze abfragen soll. Und tüchtig mitgearbeitet muss er haben, ehe er sich über seine Vor- und Mitarbeiter ein Urtheil zutrauen darf.}

Sind die einzelnen Sprachen und Sprachfamilien nothwendige Objekte unsrer Wissenschaft, so ist es nicht minder das menschliche Sprachvermögen, das ihnen allen zu Grunde liegt und nur in der wunderbaren Vielgestaltigkeit seiner Entfaltungen völlig begriffen werden kann.

\pdfbookmark[2]{Rückblick.}{I.II.rueckblick}
\subsection*{\sed{Rückblick.}}\phantomsection\label{I.II.rueckblick}

\sed{Halten wir uns an die drei Bedeutungen des Wortes \so{Sprache}, S.~3–4, so dürfen wir nunmehr sagen: Die einzelsprachliche Forschung erklärt die \so{Rede} aus dem Wesen der Einzelsprache. Die genealogisch historische Forschung erklärt die \so{Einzelsprache}, wie sie sich nach Raum und Zeit gespalten und gewandelt hat. Die allgemeine Sprachwissenschaft endlich will die vielen Sprachen als ebensoviele Erscheinungsformen des \so{einen gemein menschlichen Vermögens}, und somit dieses Vermögen selbst erklären. Solange nun die Sprachgeschichte mehr den Körper der Sprachen, als ihren Geist zum Gegenstande hat, steht sie der allgemeinen Sprachwissenschaft kaum so nahe, wie die Arbeit dessen, der möglichst viele und verschiedene Einzelsprachen zu verstehen trachtet. Jedenfalls hat Keiner mehr Anlass als er, über die verschiedensten und allgemeinsten Probleme der menschlichen Sprache nachzu\-}{\textbar}{\textbar}13{\textbar}{\textbar}\phantomsection\label{sp.13}\sed{denken, und Keinem fliessen die fruchtbaren Anregungen reichlicher zu. Dafür wird er, je mehr er das bunte Reich der Thatsachen durchmessen hat, desto schwerer behaupten, dass hier etwas schlechthin nothwendig oder unmöglich sei.}

\cehead{{{\large I,}} III. Stellung der Sprachwissenschaft.}
\cohead{Anthropologie, Völkerkunde, Geschichte.}
\pdfbookmark[1]{III. Capitel. Stellung der Sprachwissenschaft.}{I.III}
\subsection*{III. Capitel.}\phantomsection\label{I.III}
\section*{Stellung der Sprachwissenschaft.}

Eine Wissenschaft ist berechtigt als eine besondere innerhalb der übrigen aufzutreten, wenn ihr Gegenstand ihr allein eigen ist. Diesem Erfordernisse genügt die Linguistik vollkommen: weder macht sie anderen Wissenschaften ihre Gebiete streitig, noch braucht sie vor etwaigen An\-\fed{{\textbar}13{\textbar}}\phantomsection\label{fp.13}nexionsgelüsten ihrer Nachbarinnen sonderlich auf der Hut zu sein. Darum darf und soll sie aber nicht minder innig mit diesen verkehren, hier entlehnend, dort ausleihend. In der That will und soll ja alle Wissenschaft \so{einem} Endzwecke dienen: der Erkenntniss des Alls; und dies All muss als Eines gedacht werden, damit es erkennbar sei. Erkennen im wissenschaftlichen Sinne heisst auf Gesetze zurückführen; darum muss das All gedacht werden als von einheitlichen, widerspruchslosen Gesetzen beherrscht, damit es erkennbar, d.~h. begreiflich sei. Jede Wissenschaft will und soll Gesetze entdecken, und jede dieser Entdeckungen geht mindestens mittelbar der Gesammtheit aller Wissenschaften zugute. Das hat der denkende Menschengeist wohl von jeher geahnt und unzählige Male voreilig auszunutzen gesucht. \sed{Jene „curiösen Wissenschaften“ früherer Jahrhunderte, alle die verschiedenen Arten der Wahrsagekunst beruhten, ausgesprochener- oder unausgesprochenermassen, auf dem wahren Satze, dass in der Welt Alles mit Allem in nothwendigem Zusammenhange steht.} \update{Wo}{Und wo} immer die Philosophie sich \update{angemaasst}{angemasst} hat, die Welt der Thatsachen aus apriorischen Hirngespinnsten aufzubauen, lag allen diesen Versuchen dieselbe falsche Verwerthung derselben richtigen Erkenntniss zu Grunde. Die Einzelwissenschaften sollten hierin weiser sein, als die Weltweisheit. Fest steht nur, wer seinen Schwerpunkt in sich hat; wer sich an einen anderen lehnen will, der warte ab, bis er ihm nahe genug gerückt ist, und er sehe zu, ob der Andere fest genug steht, dass man sich auf ihn stützen könne. Dagegen hat auch die Sprachwissenschaft zuweilen gefehlt, zu ihrem und Anderer Schaden. Sie muss sich ihres Standpunktes genau bewusst werden, die Richtungen bestimmen, die Entfernungen abmessen gegenüber ihren Genossinnen.

Die Sprache ist Gemeingut des Menschen; nur als solche, d.~h. nur die menschliche Sprache ist Gegenstand der Linguistik. Mithin ist diese letztere ein Bestandtheil der Wissenschaft vom Menschen, also der \so{Anthropologie} im weiteren Sinne des Wortes.

\sed{{\textbar}{\textbar}14{\textbar}{\textbar}}\phantomsection\label{sp.14}

Jede Sprache ist Eigenthum eines Volkes, und wo wir von verwandten Sprachen reden, da reden wir nothwendigerweise von sprachverwandten Völkern. Somit berührt sich die Sprachwissenschaft mit der \so{Ethnographie}, ohne doch in ihr aufzugehen.

Die Sprache eines Volkes ist der unmittelbarste Ausdruck seines Geisteslebens, mithin von diesem, mithin auch von seiner \update{Entwicke\-lung}{Entwick\-lung} abhängig. Somit ist sie ein Stück der Volksgeschichte, und die Linguistik, sofern sie die Einzelsprachen und ihre Schicksale zum Gegenstande hat, ist eine \so{historische} Wissenschaft.

In Wahrheit würde sie nichts weiter als eine historische Wissen\-\fed{{\textbar}14{\textbar}}\phantomsection\label{fp.14}schaft sein, wenn sie ihren Gegenstand nicht weiter und tiefer fassen wollte. Nur würde sie damit aufhören \so{eine} Wissenschaft zu \update{sein:}{sein;} es gäbe dann nur noch eine Indogermanistik, eine Altaistik, eine Hamito-Semitistik u.~s.~w., nur Linguistiken, keine Linguistik. Nun aber ist die gemeinsame Grundlage aller menschlichen Sprachen das menschliche Sprachvermögen, und dieses fällt von selbst in das Untersuchungsgebiet des Sprachforschers. Handelt es sich um die Fähigkeit des Menschen zur Hervorbringung der Sprachlaute, so ist dies die Aufgabe der \so{Lautphysiologie}, die ein Zweig der \so{Naturwissenschaft} ist. Handelt es sich um das Vermögen des Menschen, seine Gedanken zu gliedern, so ist die Untersuchung \so{psychologisch}. Handelt es sich um die Aufgaben, welche der Sprache gestellt werden als einem Ausdrucke der Begriffs- und Gedankenverbindungen, so liefern \so{Logik} und \so{Metaphysik} die Antwort; und somit liegen die wichtigsten Bestandtheile des Sprachvermögens im Gebiete der \so{Philosophie}. Aber wohl gemerkt: nur die Aufgaben sind a priori gestellt, nicht die Lösungen gegeben, nur die Denkformen enthält die Logik, nicht auch die möglichen Ausdrucksformen. Auch jenen Denkformen gegenüber verhalten sich die Sprachen unendlich verschieden, zuweilen recht unzureichend, fast immer phantastisch, Sinnliches hineinmengend. Und zweitens: Logik und Metaphysik sind es nicht allein, die den Sprachen ihre Aufgaben stellen, sondern das ganze leibliche und seelische Leben des Menschen liefert das Thema, das die Sprache zu bearbeiten hat. \sed{Die Sprache ist eine geistleibliche Function des Menschen wie das Denken eine geistige, das Athmen eine leibliche Function ist. Die Voraussetzungen unseres geistleiblichen Lebens, die inneren Vorgänge, vermöge deren dies Leben sich äussert, gehören nicht zu den Dingen, die die Geschichtswissenschaft zu bearbeiten hat: nicht das sich Gleichbleibende, sondern das sich Verändernde ist ihr Gegenstand. Wer ihr die gesammte Sprachwissenschaft einbezirken will, müsste ihr folgerecht auch gleich die Physiologie und Psychologie mit beipacken. Die hohe oder geringe Begabung, die Erkrankung eines Herrschers sind gewiss oft mächtige Factoren in der Geschichte. Der Historiker aber hat nicht ihre Herkunft, sondern nur ihre Wirkungen zu erklären.}

\sed{{\textbar}{\textbar}15{\textbar}{\textbar}}\phantomsection\label{sp.15}
\cohead{Philosophie, Naturwissenschaften.}

Eine seltsame Einseitigkeit war es, die Sprachwissenschaft den Naturwissenschaften einreihen zu wollen. Einem platten Materialismus, wie er noch vor wenigen Jahrzehnten unreife Köpfe verwirrte, ist freilich nicht einzureden, dass nicht alle Wissenschaft Naturwissenschaft sei; und als nun vollends \textsc{Charles Darwin} mit seiner epochemachenden Theorie hervortrat, da streckte ihm selbst ein ernsthafter Linguist wie \textsc{August Schleicher} die Bruderhand entgegen. Es ist ja wahr, die inductive Methode des Sprachforschers ist mit der des Naturforschers völlig gleich. Aber man nennt den wissenschaftlichen Arbeiter nicht nach dem Werkzeuge, das er führt, sondern nach dem Stoffe, den er bearbeitet, – und der ist wahrlich verschieden genug. Mit den Begriffen der Entwickelung, der Artentheilung u.~s.~w. haben wir Sprachforscher hantiert, lange ehe man etwas von \textsc{Darwin} wusste, und Übergangsformen wussten wir zu Tausenden aufzuweisen, lange vor der Entdeckung des fossilen Hipparion und des Archaeopteryx. Bei den Naturforschern brauchen wir also vor\-\fed{{\textbar}15{\textbar}}\phantomsection\label{fp.15}läufig nicht zu Tische zu gehen, – und ob sie uns je ein Und, ein Oder, einen Conjunctivus Plusquamperfecti oder Ähnliches nachweisen in der Körperwelt, die ihr alleiniges Dominium ist, – das wollen wir erst noch abwarten. Wer freilich in der Sprache nichts Besseres sieht, als todte Lautgebilde, Cadaver, die man auf dem Seciertische zerlegt und zerstückelt, der muss sich wohl zum Anatomen verwandtschaftlich hingezogen fühlen. Aber man kann Jahre im anatomischen Museum und im Seciersaale verbringen, ohne zum Menschenkenner zu werden, und man kann jahrelang Wörter und Wortformen zerlegen, ohne vom Wesen der Sprache eine Ahnung zu erlangen. Die Sprache lebt, und nur im Leben lernt man Lebendes verstehen.

\sed{Doch wir müssen gerecht sein. Jener Materialismus war nicht nur nach den Umständen erklärlich und entschuldbar, sondern er hat auch, wie jede redliche Einseitigkeit, gute Früchte getragen.}

\sed{Erklärlich und entschuldbar war er; er lag, so zu sagen, in der Luft. Vor mehr als zwei Jahrhunderten hatte \textsc{Francis Bacon} die Männer der Wissenschaft ermahnt, sich von den speculativen Hirngespinnsten abzuwenden, den Erfahrungsthatsachen ihre Gesetze abzulauschen. Es war, als hätte er in den Wind geredet. Jetzt endlich kam sein Programm zur Ausführung, und wie häuften sich nun die Entdeckungen und Erfindungen! Während aber die Naturforscher mit Messer und Mikroskop, mit Tigel und Retorte die Körper untersuchten, während die Historiker im Staube der Archive wühlten, baute \textsc{Schelling} das System der Natur, \textsc{Hegel} die Geschichte der Menschheit aus reinen Begriffen auf, und es waren nicht immer die Schlechtesten, die ihnen den lautesten Beifall zuklatschten. Auch unsere Wissenschaft hatte der \corr{1901}{philoso\-phischen}{psiloso\-phischen} Speculation einen willkommenen Spielplatz geboten. J. \textsc{Harris}, Lord \corr{1901}{\textsc{Monboddo,}}{\textsc{Mouboddo,}} der grosse Arabist \textsc{Sylv. De Sacy}, \textsc{J. S.~Vater}, \textsc{A. F. Bernhardi}, \textsc{K. F. Becker} und viele Andere prangten in den} {\textbar}{\textbar}16{\textbar}{\textbar}\phantomsection\label{sp.16} \sed{Bibliotheken mit philosophischen Sprachlehren. Daneben dicke Polyglotten, ungeordnete Anhäufungen, vor denen Einem die Augen übergingen, so recht „non multum, sed multa“. Und als nun \textsc{Humboldt} Ernst damit machte, die Polyglottik philosophisch zu vertiefen und systematisch zu gestalten, die Sprachphilosophie auf polyglotter Basis neu zu begründen: da wollte es das Verhängniss, dass ihm zum Reformator das Äusserlichste fehlte, die Lust und Fähigkeit zu gemeinverständlicher Darstellung. Er sagt einmal: „Habe ich mir eine Idee entwickelt, so ekelt es mich an, sie nun auch einem Andern auszuknäueln“. Wer diesen Ekel nicht überwindet, der taugt freilich nicht zum Werber. Auch setzte das, was er erstrebte, ganz andere Neigungen und Fähigkeiten voraus, als jene scharfsinnigen Vergleichungen von Wörtern, Wortformen und Lauten. Nicht Jeder hat Lust, und die Wenigsten hatten damals die äusseren Mittel, sich literaturlose Sprachen von Menschen aller Zonen und Farben anzueignen. Doch nur, wer sich darin versucht, kann empfinden, was es mit der „Verschiedenheit des menschlichen Sprachbaues“ auf sich hat. Wer sich hingegen der Indogermanistik widmet, der bringt gleich von der Schule her die Kenntniss einiger stammverwandter Sprachen mit, und wenn ihn bis dahin die trockenen Regelsammlungen altmodischer Schulgrammatiken wie eine sinnlose Quälerei verdrossen hatten, so ist ihm jetzt, wo er anfängt, hinter den Regeln die Gesetze zu entdecken, so frei zu Muthe, als wäre er selber der Gesetzgeber. Nun hat er etwas Greifbares in Händen: die Thatsachen selbst müssen ihm Rede stehen, und sie werden seine Frage beantworten.}

\sed{Eben hierin liegt nun das Zweite, die Fruchtbarkeit und verhältnissmässige Sicherheit dieser Forschungsart. In die Philologie spielt doch scheinbar zuviel des Subjectiven hinein. Ich muss mich in meinen Schriftsteller versenken, mich mit ihm eins fühlen, wenn ich die Eigenart seiner Rede erklären will. Und ebenso ist es in der allgemeinen Sprachwissenschaft mit der Geistesart verschiedener Völker, die sich in ihren Sprachen ausprägen soll. Wer in solchen Dingen nicht selbst die Gabe der Congenialität hat, traut sie auch Anderen nicht so leicht zu. Wie fassbar und Allen zugänglich scheint dagegen der Lautkörper, wie klar und unumstösslich das Gesetz, das man seiner Beobachtung abgewonnen hat. Dass die Untersuchung sich immer mehr verfeinerte, dass man mit der Zeit anzweifeln lernte, was anfangs als unbestreitbar galt: das wurde mit Recht als ein Fortschritt begrüsst. Denn der Stoff war doch zäher, als man erst gemeint hatte; und an ihm schärfte sich nun die Methode. Diese, wie sie von den Indogermanisten ausgebildet worden, hat sich bisher überall bewährt, wo es galt, Sprachen einer Familie untereinander zu vergleichen. Und wie gesagt, soweit nach ihr körperliche Erscheinungen aus körperlichen Gesetzen hergeleitet werden, gleicht sie der naturwissenschaftlichen auf’s Haar.}

Die Verwandtschaft der Linguistik mit den Naturwissenschaften liegt aber \sed{{\textbar}{\textbar}17{\textbar}{\textbar}}\phantomsection\label{sp.17} auch sonst im Wesen der Sache. Nichts gleicht einem Organismus mehr, als die menschliche Sprache. Alles in ihr steht in ursächlichem und zwecklichem Zusammenhange; sie hat ihr Formprinzip, darum reden wir von ihrer Morphologie; sie entwickelt sich nach inneren Gesetzen, zuweilen auch nach äusseren Einwirkungen, krankt, altert, stirbt wohl auch: darum dürfen wir von Physiologie, Biologie, Pathologie der Sprache reden; den Kampf um’s Dasein hat auch sie gelegentlich zu bestehen, – jedenfalls bleibt er keinem ihrer Theile erspart –, und wer weiss, ob ihr nicht noch natürliche Zuchtwahl, Mimicry und mehr dergleichen zugesprochen wird? \sed{Dagegen giebt es eine Macht, die der Naturforscher als solcher nie begreift, mit der nur der Historiker zu rechnen versteht: die Macht des Individuums. Der Naturforscher mag die Biologie erforschen: eine Biographie zu schaffen ist nicht seines Amtes; er mag seinen Arm ausstrecken, ob er die Psychologie in sein Bereich herüberziehen könne: das Geistesleben eines Menschen, eines Zeitalters bleibt ihm unerfassbar, unerreichbar.}

Man redet vom \so{Organismus der Sprache} mit vollem Rechte, mindestens ohne Schaden, solange man im Sinne behält, dass die Sprache nicht ein eigenlebiges Wesen, sondern eine Fähigkeit und Function ist der geistleiblichen Natur des Menschen. Erkennt man dies, so werden ganz andere Verwandtschaften auftauchen, ächte, genetisch begründete, nicht blosse Analogien. Die Religionen, das Recht, die Sitten, kurz das ganze Culturleben der Völker ist von denselben Mächten bestimmt, wie ihre Sprachen, sie können von keinen anderen Mächten bestimmt sein. Damit erwachsen ganz andere Räthsel und wir schauen in Tiefen, die lange noch der Ergründung harren werden. Einzelne matte Lichtstrahlen meint man wohl schon jetzt wahrzunehmen, und es dünkt mir wahrscheinlicher, dass man dereinst begreifen lerne, wie aus derselben Wurzel das römische Recht und die lateinische Sprache emporgewachsen ist, als dass der grösste Anatom in dem Hirne des besten Lateiners einen Accusativus cum infinitivo entdecke.

\fed{{\textbar}16{\textbar}}\phantomsection\label{fp.16}

\cehead{{{\large I,}} IV. Anregung zur Sprachwissenschaft.}
\cohead{Aegypter, Assyrer, Chinesen.}
\pdfbookmark[1]{IV. Capitel. Anregung zur Sprachwissenschaft.}{I.IV}
\subsection*{IV. Capitel.}\phantomsection\label{I.IV}
\section*{Anregungen zur Sprachwissenschaft.}

Gilt es den Ursprung jedes wissenschaftlichen Strebens in einem Worte zu begreifen, so wähle ich das Wort \so{Verwunderung}. Zwei Dinge sind nöthig, damit wir uns wundern: eine Wahrnehmung, deren Gründe uns nicht einleuchten, und ein Sinn, der nach diesen Gründen fragt. Diese Frage mag blosser \so{Neugier} entspringen, und sie entspringt in der That \so{nur} dieser, wenn sie nur für etwas Vereinzeltes die Erklärung sucht. Anders jene höhere Neugier, die uns nach den inneren Zusammenhängen unserer Erfahrungswelt forschen lässt. Jeder \sed{{\textbar}{\textbar}18{\textbar}{\textbar}}\phantomsection\label{sp.18} Versuch, die Dinge von einheitlichen Gründen herzuleiten, ist im eigentlichen Sinne wissenschaftlich, er sei noch so verfehlt, noch so roh, noch so „unwissenschaftlich“, wie man zu sagen pflegt; jedes Fragen nach allgemeinen Gründen ist eine Äusserung wissenschaftlichen Interesses. In diesem Verstande bilden auch die Beobachtungen des Jägers über die Lebensgewohnheiten der Thiere und ihre Fährten, die Stern- und Wetterkunde des Schiffers, die Heilkunst des Schäfers oder Hufschmieds niedere Arten der Wissenschaft. Niedere Arten sind es aber, weil das Wissen dabei nur als Mittel gilt, nicht als Zweck. Solange der Mensch unter dem Drucke der Lebenssorgen steht, können seine Interessen keinen höheren Flug nehmen, müssen sie sich auf das beschränken, was zum Lebensunterhalte und leiblichen Genusse dient. Auch die Sprache kann in diesen Bereich fallen. Wo ein Verkehr von Volke zu Volke stattfindet, ist gegenseitige Verständigung nöthig, daher in der Regel Spracherlernung. Ich sage: in der Regel; denn es ist bekannt, welche Dienste die Zeichensprache gerade den bescheidenen Bedürfnissen des internationalen Gedankenaustausches zu leisten vermag.

Aber nicht jedes Lernen ist eine wissenschaftliche Arbeit, und die Spracherlernung ist in den weitaus meisten Fällen nichts weiter als die Aneignung einer Fertigkeit durch Übung. Sie kann wissenschaftlich anregen, sie thut es aber erfahrungsmässig nur bei den Wenigsten. Den Meisten ist und bleibt die Sprache, auch die fremde, ein Werkzeug, dessen Gebrauch man sich einübt, das man gebraucht, wenn man seiner bedarf, um es dann beiseite zu legen, – nicht ein Gegenstand, bei dem \fed{{\textbar}17{\textbar}}\phantomsection\label{fp.17} man betrachtend verweilt, nicht ein Räthsel, das man zu lösen versucht. An dieser Stelle ist es lehrreich, einen Blick auf die Vorgeschichte \update{unsrer}{unserer} Wissenschaft zu werfen. Dabei sehen wir jetzt von den Mythen über den Ursprung der Sprache ab.

\phantomsection\label{I.IV.aegypter}
Dem ältesten Culturvolke gebührt auch das Verdienst der ersten, elementarsten sprachwissenschaftlichen That. Die Zerlegung der Sprache in ihre Einzellaute und die Erfindung von Buchstaben, die diese Laute ausdrücken, verdankt die Welt den \so{Aegyptern}.

\phantomsection\label{I.IV.assyrer}
Den ersten eigentlich grammatischen Versuchen aber begegnen wir bei den \so{Assyrern}, einem Volke semitischer Zunge. Tausende von Ziegeln mit Keilschrift bedeckt, jetzt grösstentheils Eigenthum des British Museum, sind die Überbleibsel ihrer Literatur, deren Entzifferung eine ansehnliche Zahl begeisterter Forscher beschäftigt. Vieles ist noch streitig; Eins aber darf wohl als sicher gelten: die Semiten fanden in Assyrien ein älteres Culturvolk nichtsemitischen Stammes vor, dem sie ihre \update{Schrift}{Schrift,} und wohl auch einen grossen Theil ihrer sonstigen Gesittung entlehnten. Auch in der Sprache dieses älteren Volkes, der Akkader oder Sumerier, sind schriftliche Denkmäler erhalten, und die \update{Assyrier}{Assyrer} haben Sorge getragen, Schlüssel zum Verständnisse dieser fremden Sprache zu \sed{{\textbar}{\textbar}19{\textbar}{\textbar}}\phantomsection\label{sp.19} schaffen: Syllabare, Vocabulare und grammatische Paradigmen mit assyrischen Übersetzungen, die wenigstens eine Art wissenschaftlicher Analyse voraussetzen, aber sichtlich nur dem praktischen Zwecke des Sprachunterrichtes dienten.

\phantomsection\label{I.IV.chinesen}
Die \so{Chinesen} besitzen eine Literatur, deren älteste Denkmäler bis in die letzten Jahrhunderte des dritten Jahrtausends vor unserer Zeitrechnung hineinragen dürften. Von den Wandelungen des Lautwesens zeugt die Wortschrift dieses Volkes nur mittelbar, deutlicher das Reimwesen seiner alten Lieder. Grammatische Anregungen, wenigstens mächtige, sind von einer Sprache mit unveränderlichen Wörtern nicht zu erwarten. Wohl aber zeigten diese Wörter Veränderungen in der Bedeutung und Anwendung, die dem auf alles Geschichtliche gerichteten chinesischen Geiste zu denken gaben: so war denn hier die erste Arbeit die lexikalische, deren Anfänge in die Zeit um 1000 v. Chr. Geb. fallen mögen. Im Jahre 213 v.~u.~Z. erliess der Kaiser Schí-hoâng-tí der \update{Ts’în-Dynastie}{Ts’ìn-Dynastie} das berüchtigte Edikt, wonach alle alten Bücher, ausser den wahrsagerischen, verbrannt werden sollten. Der Befehl wurde sehr streng gehandhabt, und unahnbare Schätze der Literatur mögen damals \fed{{\textbar}18{\textbar}}\phantomsection\label{fp.18} für immer zu Grunde gegangen sein. Zum Glücke dauerte die Barbarei nicht lange, und als nach kaum zwanzig Jahren eine neue Dynastie einsammeln liess, was von den Werken der Vorfahren noch erhalten \update{war\newline[\textit{in den Berichtig\-ungen, S.~502}: war,]}{war,} da stellte sich heraus, dass der \update{Ts’in-Herrscher}{Ts’ìn-Herrscher} mit all seinem Wüthen doch lange nicht sein Ziel erreicht hatte. Die kaiserliche Bibliothek wuchs rasch an, und Gelehrte wurden \update{angestellt}{angestellt,} sie zu ordnen, die Texte zu prüfen, zu säubern, wo nöthig zu ergänzen und auszulegen. Damals begann also die grossartige philologische Arbeit, die bis auf den heutigen Tag eine Menge der besten Köpfe im Mittelreiche beschäftigt. Nächst der Textkritik richtet sie sich auf die Paläographie, den Wort- und Phrasenschatz, die Stilistik und Poetik. Indischem Einflusse war es zu \update{danken,}{verdanken,} dass man später auch den Lautbestand der Sprache untersuchte und systematisch ordnete. Grammatiken haben die Chinesen erst geschaffen, als ihre Beamten die Sprachen fremder Herrschervölker, der Mongolen und der Mandschu, erlernen mussten; für ihre eigene Sprache aber haben sie mit richtigem Verständnisse die Eintheilung in Stoff- und Formwörter, in Verba und Nomina und gewisse syntaktische Kategorien durchgeführt. Ihre sprachphilosophischen Arbeiten sind uns nur in \update{kleinen}{kleineren} Bruchstücken bekannt; sie behandeln die Frage nach dem Ursprunge der Sprache, zum Theile in recht fein- und tiefsinniger Weise. Hân-iü (768–824 n.~Chr.) z.~B. erklärt die Töne in der Natur aus einer Störung des Gleichgewichts, wendet dies auf die Menschen an, auf ihren Gesang, ihr Weinen und ihre Sprache: der Mensch rede, singe, weine, wenn sein Gemüth „nicht sein Gleichgewicht erlange“, und nun zieht er eine hübsche Parallele zwischen der Instrumentation in der Musik und der Stilistik in der Sprache. (Vgl. meine Anfangsgründe der chines. Grammatik S.~115 fg.)

\sed{{\textbar}{\textbar}20{\textbar}{\textbar}}\phantomsection\label{sp.20}
\cohead{Griechen und Römer, Christenthum, Islâm.}\phantomsection\label{I.IV.grci}

Die \so{Griechen} und \so{Römer} waren von Hause aus und im ganzen Verlaufe ihrer Geschichte auf den Verkehr mit Nachbarvölkern hingewiesen; es hätte ihnen also wohl nicht an Anregung gefehlt, sich mit fremden Sprachen zu beschäftigen. Wieviele Griechen mussten Persisch und diese oder jene kleinasiatische Sprache, wieviele Römer Griechisch und dann die dem Latein nächstverwandten italischen Sprachen, wohl auch einen keltischen oder germanischen Dialekt lernen. Xenophon war Feldherr im Dienste des jüngeren Cyrus, schätzte diesen aufrichtig hoch, wählte den älteren Cyrus zum Helden seines Erziehungsromans: er musste Persisch verstehen, die Ähnlichkeit vieler Wörter und der ganzen Conju\-\fed{{\textbar}19{\textbar}}\phantomsection\label{fp.19}gation zwischen dieser Sprache und dem Griechischen konnte ihm gar nicht entgehen, aber darauf auch nur hinzudeuten fällt ihm nicht ein. Den Tod seines Kriegsherrn erzählt er so lebhaft, wie es sein trockener Stil erlaubt, aber dessen letzte Worte, die doch wohl persisch gesprochen waren, theilt er griechisch mit: τὸν ἄνδρα \retro{ὁρῶ.}{ορῶ.} Herodot hat Vorderasien bereist, als Ethnograph bereist, gewiss ähnliche und \retro{wohl umfass\-endere}{wohl\-umfass\-endere} Kenntnisse erworben und Beobachtungen gemacht, wie Xenophon: aber davon zeichnet er nichts auf. Tacitus stellt seinen überfeinerten Landsleuten unsre halbwilden Vorfahren als Sittenmuster hin, aber von ihrer Sprache theilt er nur so nebenher ein paar Vocabeln mit. Man möge einwenden: in allen diesen Fällen habe es sich um Barbaren gehandelt, im Sinne des griechisch-römischen Eigendünkels. Allein die Römer wussten wohl, was sie der griechischen und etruskischen Gesittung zu verdanken hatten; Griechisch wurde von den Gebildeten Roms gelernt; aber dass sie versucht \update{hätten}{hätten,} auch nur die Aneignung dieser Sprache durch ein Lehrbuch zu erleichtern, davon ist nichts bekannt.

Die Griechen, nach Anlage und Neigung mehr Denker und Schöpfer als Sammler und Forscher, begannen bei der Sprachphilosophie: worin und worauf beruht die Richtigkeit des sprachlichen Ausdruckes? wie verhalten sich die Wörter zu den Begriffen, ist die Verbindung Beider durch die Natur, φύσει, oder durch Übereinkunft, θέσει, gegründet? \sed{Schon \textsc{Protagoras} hatte die drei Geschlechter der Hauptwörter, die Zeiten der Verba und die Arten der Sätze unterschieden.} \textsc{Aristoteles}, von der Logik ausgehend, \update{that, soviel wir wissen, die ersten, noch unsicheren}{that weitere, immer noch unsichere} Schritte zur Entdeckung der Redetheile. Erst die Stoiker kamen hierin der Wirklichkeit näher und schritten weiter zur Entdeckung der grammatischen Functionen. In der etymologischen Wortanalyse, wo man sie versuchte, machte man die tollsten Missgriffe; glücklicher war man in der Analyse des Satzes. Logik, Rhetorik, Stilistik und Poetik hatten die Grammatik in ihren Dienst genommen. Die Dialekte waren anfangs in der Literatur gleichberechtigt, das Übergewicht des Attischen war nur thatsächlich durch die geistige Übermacht Athens begründet. Allerwärts jedoch verstand man ohne Schwierigkeit die ionischen und äolischen Dichtungen; ihre Sprache ahmte man gelegentlich \sed{{\textbar}{\textbar}21{\textbar}{\textbar}}\phantomsection\label{sp.21} nach, machte sie aber nicht zum Gegenstande wissenschaftlicher Untersuchung. Erst als die κοινὴ διάλεκτος die anderen Dialekte in eine gewisse Ferne rückte, wurden auch diese in den Bereich philologischer Forschung gezogen.

\phantomsection\label{I.IV.byzanz}
Die Philologie ist recht eigentlich die Wissenschaft der Epigonen, die mit wehmütiger Bewunderung durch die Grabstätten einer erstorbenen \fed{{\textbar}20{\textbar}}\phantomsection\label{fp.20} Cultur wandern. Sie wollen bei den grossen Ahnen lernen, ihre Werke geniessen, daher müssen sie \update{diese}{die} Werke sammeln und untersuchen. So verhielten sich die Alexandriner zu ihren Vorfahren in Hellas und Kleinasien, die Begründer der grössten Bibliothek des occidentalischen Alterthums.

Die Römer hatten, angeregt von den Stoikern, begonnen ihre Sprache grammatisch zu bearbeiten. \textsc{Varro} hatte selbst das Altlateinische und die verwandten italischen Dialekte in den Kreis seiner Untersuchungen gezogen. Was aber für das Griechische Alexandrien, das wurde für das Latein Byzanz: was dort \textsc{Apollonios} \update{\textsc{Dyskolos},}{\textsc{Dyskolos}} und sein Sohn \textsc{Aelius Herodianus}, das leistete hier \textsc{Priscianus}, dessen Institutiones grammaticae Jahrhunderte lang in den Schulen eine unbestrittene Herrschaft geübt haben. Fast ist es, als lernten in der Schule der Verbannung auch die Sprachen Selbsteinkehr zu halten.

\phantomsection\label{I.IV.chistenthum}
Das \so{Christenthum} hatte für die Wissenschaft zunächst den negativen Vortheil, dass es mit dem Begriffe der Barbaren aufräumte. Wenn es statt dessen den Christen und Juden die Heiden gegenüberstellte, so betrachtete es doch diese mit Theilnahme als zu Bekehrende. Und vorzugsweise den Armen predigten seine Sendlinge die Heilsbotschaft, den Armen, das heisst den Ungebildeten, Leuten der verschiedensten Zungen, die alle in ihren Muttersprachen belehrt sein wollten. Viele wichtige Texte verdanken wir den ersten Jahrhunderten unserer Religion, die ältesten und umfassendsten, wohl auch die einzigen Quellen zur Erforschung so mancher alten Sprache, – zunächst aber auch nicht mehr. Was die Christen jener Zeit an philologischer Arbeit geliefert, war entweder Fortsetzung der griechisch-römischen Wissenschaft, oder Bibelforschung. Wichtiger für \update{unsre}{unsere} Wissenschaft war es wohl, dass nunmehr eine Menge Völker der Schreibkunst theilhaftig und in einen gewissen geistigen Verkehr mit Rom gezogen wurden. Das Latein wurde im Westen Kirchensprache, musste erlernt und gelehrt werden, und diesem Umstande verdanken wir einige germanische und keltische Grammatiken und Wörterbücher.

\phantomsection\label{I.IV.islam}
Anders als das Christenthum wirkte der \so{Islâm}. Sein heiliges Buch in andere Sprachen zu übersetzen verbot er geradezu, der Korân muss in der Ursprache gelesen werden. So wurde die Erlernung des Arabischen für \update{Viele}{viele} zur religiösen Pflicht. Gleichzeitig aber erlitt diese reiche und schöne Sprache in dem \update{Maasse,}{Masse,} wie sie sich über die Völker \fed{{\textbar}21{\textbar}}\phantomsection\label{fp.21} verbreitete, vielerlei dialektischen Verderb, der zu wissenschaftlicher Reaction, d.~h. zur grammatischen Feststellung der ächten, rechten Sprache des Propheten herausforderte. Wieviel dabei etwa \sed{{\textbar}{\textbar}22{\textbar}{\textbar}}\phantomsection\label{sp.22} griechischer Anregung zu danken sei, wissen wir nicht; sicher ist, dass wenige Sprachen des Ostens eine sorgfältigere Behandlung erfahren haben als diese. Diese Arbeit begann sehr frühe, schon in den ersten Jahrhunderten nach der Hedschra, und es ist merkwürdig, dass der rühmlichste Antheil dabei den Persern zukommen soll. \sed{Nirgends sonst im Orient ist die Syntax so sorgfältig bearbeitet worden, wie hier.}

\cohead{Inder, Juden, Pâri}\phantomsection\label{I.IV.juden}
Die \so{Juden} scheinen erst nach der Zerstörung ihrer Hauptstadt auf eine philologische Untersuchung ihrer heiligen Schriften verfallen zu sein. Diese Bücher sind bekanntlich in einer vocallosen Schrift verfasst, und ihre richtige Lesung war Sache der Tradition, die nach der Auseinandersprengung des Volkes gefährdet war. Mit den Lauten wäre aber der Sinn verloren gegangen oder entstellt worden, und dem musste vorgebeugt werden. So begann, angeblich um das zweite Jahrhundert unsrer Zeitrechnung, die Thätigkeit der Massoreten und Punctatoren, denen kritischer Feinsinn nachgerühmt wird. Die systematische Arbeit der hebräischen Grammatiker scheint aber erst arabischem Einflusse ihren Ursprung zu verdanken.

\phantomsection\label{I.IV.parsi}
Die \so{Pârsi} waren durch die alten Sprachen ihrer heiligen Schriften, Altbaktrisch und Pehlewi, auf philologische Untersuchungen hingewiesen; was sie darin etwa geleistet, ist aber wohl erst zum kleinsten Theile bekannt.

\phantomsection\label{I.IV.inder}
Geradezu unvergleichlich sind die grammatischen Leistungen der \so{Inder}. Kein Volk des Alterthums mochte zu diesem Zweige des Forschens zugleich glänzender befähigt und mächtiger angeregt sein, als dieses. In seinen Veden besass es einen reichen Schatz von Hymnen, in alter Sprache verfasst, noch immer dem religiösen Cultus dienend, und welch vielgestaltigem, schwierigem Cultus! Dem Worte des Gebetes wurde eine magische Kraft beigelegt, Unheil drohend dem, der es falsch gebrauchte, es auch nur unrichtig aussprach. Und ferner welche Sprache! In der Formen- und Wortbildungslehre zugleich reicher und durchsichtiger entwickelt, als irgendeine ihrer Verwandten, wohllautend wie wenige, für sich schon dem ästhetischen Werthe nach ein Kunstwerk, zu künstlerischer Verwerthung und Gestaltung einladend, in einer reichen poetischen, theologischen und philosophischen Literatur entfaltet und bewährt. Endlich ein Volk, das an Vielseitigkeit, Feinheit, Tiefe \fed{{\textbar}22{\textbar}}\phantomsection\label{fp.22} und freier Voraussetzungslosigkeit des Denkens dem griechischen nahe kommt und nun mit seiner Forschung an eine solche Sprache herantritt. Die Vorgeschichte der indischen Grammatik ist noch lange nicht vollkommen aufgehellt, vielleicht zum Theile auf ewig verdunkelt durch \textsc{Pânini}’s Wunderwerk. Es ist dies die einzige wahrhaft vollständige Grammatik, die eine Sprache aufzuweisen hat, eine der reichsten Sprachen zudem; und sehen wir von den Elementarbüchern ab, so dürfte sie zu gleicher Zeit die kürzeste aller Grammatiken sein; denn man hat ausgerechnet, dass sie in fortlaufendem gewöhnlichem Drucke, in lateinische Buchstaben \sed{{\textbar}{\textbar}23{\textbar}{\textbar}}\phantomsection\label{sp.23} transscribirt, kaum hundert Octavseiten füllen würde. Sie fasst ihren Stoff in etwa viertausend kurzen Regeln zusammen, die in acht Haupttheile geordnet sind. Die Reihenfolge und Vertheilung der Lehrsätze ist aber nicht organisch in unserem Sinne, das Zusammengehörige, z.~B. verschiedene Formen desselben Wortes, muss man oft an den verschiedensten Stellen zusammensuchen, und es kommt vor, dass eine einzige Form durch eine lange Reihe von Regeln und Ausnahmen hindurch Spiessruth\-en laufen muss, ehe sie endlich für den Lernenden feststeht. Was diesem dabei zugemuthet wird, will ich wenigstens annähernd an einem Beispiele aus der deutschen Grammatik veranschaulichen. Bei §. 80 bildet er sich ein, es müsse nach der Analogie von \so{flehte}, \so{wehte} auch heissen: \so{gehte}, \so{stehte}, \so{sehte}; bei §. 100 nach \so{sah}, \so{geschah}, auch: \so{gah}, \so{stah}; bei §. 140 nach \so{stand} auch \so{gand}, bis er endlich in §.~200 die Form \so{ging} lernt. Das sind vier Stadien, man hat aber bei Pânini in einzelnen Fällen mehr als doppelt soviele gezählt. Sein Buch kann nur der gebrauchen, der in jedem Augenblicke alle Lehrsätze im Geiste gegenwärtig hat; kein Wunder, dass es allein an die sechs Jahre fleissigsten Lernens erfordern soll. Ich weiss nicht, ob man in der gleichen Zeit bei gleichem Fleisse mit unseren Hülfsmitteln auch nur im Latein die gleiche Vollkommenheit erreichen würde, wie der Brâhmane unter Pânini’s Leitung im Sanskrit.

Ein Lehrbuch dieser Art ist zunächst ein Kunstwerk, um nicht zu sagen ein Kunststück; aber die wissenschaftliche Arbeit liegt ihm zu Grunde, und auf sie kommt es hier allein an. In der That ist das Gelingen eines solchen Wunderwerkes an mehr als eine Voraussetzung geknüpft: erstens an die vollständige Beherrschung eines ganzen Sprachstoffes, an einen Geist, dem in jedem Augenblicke jede Einzelheit der Sprache gegenwärtig ist, – und wo fände sich ein solcher Geist wieder? \fed{{\textbar}23{\textbar}}\phantomsection\label{fp.23} Zweitens an die schärfste Analyse dieses Stoffes, zumal auch an die genaueste Feststellung aller seiner Lautgesetze. Man sieht, die unvergleichliche Verdichtungs- und Gestaltungsgabe des grossen Inders kommt erst in dritter Reihe. Sie ist das Künstlerische an ihm, aber auch das, was keinem Gelehrten fehlen sollte, der nicht nur Steinbrecher sein will, sondern auch Baumeister. Pânini’s Baustil freilich war so ungefähr der des Reisenécessaires, das in möglichst engem Raume möglichst Vieles vereinigen \update{soll.}{soll, und das Ziel, das er sich gesteckt hatte, war nicht das wissenschaftliche, die Dinge zu erklären, sondern das praktische, vorzuschreiben, wie man die Sprache richtig anwenden solle.} Unser Ideal ist ein anderes, und wenn wir den indischen Meister nicht im Punkte der Vollständigkeit erreichen, so können wir ihn dafür in der organischen Auffassung und Anordnung des Stoffes übertreffen. Für die \retro{Unbequemlich\-keiten}{Unbequem\-lichkeiteu} der Pânini’schen Methode waren übrigens auch die Inder nicht unempfindlich, und manche ihrer jüngeren Lehrbücher sind, anscheinend völlig unabhängig von europäischem Einflusse, nach einem uns genehmeren Schema gearbeitet.

\sed{{\textbar}{\textbar}24{\textbar}{\textbar}}\phantomsection\label{sp.24}

\phantomsection\label{I.IV.japaner}
\largerpage[1]Die \so{Japaner} haben vielleicht auf keinem Gebiete selbständigen geistigen Schaffens glänzendere Erfolge aufzuweisen, als in der Sprachforschung. Fast seit anderthalb Jahrtausenden bildet das Chinesische die Grundlage ihrer humanistischen Bildung. Zu grammatischer Behandlung hat es auch sie nicht angeregt, man müsste denn hierher gewisse Arbeiten über die chinesischen Hülfswörter rechnen; dafür sind die lexikalischen und schriftkundlichen Arbeiten um so bedeutender. Des Confucius Lehre, zu der sich bald die Gebildeten bekannten, schont nicht nur die Pietät, sondern fördert sie sogar; in Japan fand sie eine alte, ihr fremde Cultur vor, deren ehrwürdige Denkmäler, alte mündlich fortgepflanzte Sagen, Lieder und Gebete, zunächst in Schriften niedergelegt und dann bis auf den heutigen Tag geschätzt und durchforscht wurden. Die einheimische Sprache veränderte sich aber fast ebenso rasch, wie das Sächsische in England oder das Altnordische in Dänemark; in der Vergleichung ihrer grossen Vergangenheit mit ihrem jetzigen Zustande, – einem wahren Verfalle –, lag auch hier die stärkste Anregung zur Untersuchung. Andrerseits führte der Buddhismus dem Lande indische Einflüsse zu, deren Tragweite wir noch nicht voll ermessen können. Ihnen schreibe ich u.~A. die rationellere Anordnung des japanischen Syllabares zu, die, wenn man s als Stellvertreter der Palatalen annimmt, der indischen folgt: a, i, u, e, o, k, s, t, n, f, m, y, r, w. Es soll eine Sanskrit-Grammatik in japanischer Sprache geben, und so mag denn der erste Anstoss zur systematischen Bearbeitung der eigenen Sprache von \fed{{\textbar}24{\textbar}}\phantomsection\label{fp.24} fremdher gekommen sein. Immerhin ist dabei nichts von sklavischer Nachahmung zu spüren. Denkt man daran, wie gewaltsam wir Europäer lange Zeit die fremdartigsten Sprachen in das Prokrustesbett der lateinischen Grammatik hineingezwängt haben, so müssen wir die Grammatiker des östlichen Inselreiches um ihres wissenschaftlichen Taktes willen bewundern. In der Etymologie haben sie nur ein wenig toller gehaust als unsere Vorfahren; in der voraussetzungslos sachgemässen Behandlung der Formenlehre und Syntax aber kommen sie dem Ideale \update{ziemlich}{sehr} nahe. Erst in diesem Jahrhunderte haben ein paar japanische Gelehrte verunglückte Versuche gemacht, ihre Grammatik über den europäischen Leisten zu spannen; sie haben heftigen Widerspruch hervorgerufen, und es müsste eine Freude sein, dem Kampfe beizuwohnen.

\cohead{Rückblick; die neuere Zeit.}\phantomsection\label{I.IV.rueckblick}
Jetzt mag es an der Zeit sein, einen Rückblick zu thun. Die sprachphilosophischen Versuche, auf sehr beschränkten sachlichen Kenntnissen beruhend, sind nur für die Geschichte des menschlichen Denkens von bleibendem Werthe. Uns aber interessirte, was in positiver Sprachforschung geleistet worden, und da ergab sich nun folgende Beobachtung: Bei allen Völkern, die einzigen Assyrer ausgenommen, begann die Arbeit bei der eigenen Muttersprache; aber auch diese wurde erst dann untersucht, als sie oder ihre Literatur in Verfall gerathen war. Nun diente die Forschung nicht theoretischen, sondern praktischen, \sed{{\textbar}{\textbar}25{\textbar}{\textbar}}\phantomsection\label{sp.25} reglementären Zwecken: das Alte ist classisch, das Neue ist anders, folglich ist es unclassisch, folglich unrichtig oder unschön. Wie muss man es anfangen, wenn man sich schön und richtig ausdrücken will? Dazwischen zeigt sich wohl hie und dort etwas von der Raritätensucht des Alterthümlers. Die Etymologie und die Sprachvergleichung, wo sie versucht wurden, entbehrten der wissenschaftlichen Grundsätze; Dauerndes wurde nur auf einzelsprachlichem Gebiete geschaffen, und eine Erweiterung unseres einzelsprachlichen Wissens war nöthig, ehe mit Erfolg an eine genealogisch-geschichtliche und an eine allgemeine Sprachwissenschaft gegangen werden konnte. Bisher hatte die Forschung da die reifsten Früchte gezeitigt, wo sie sich auf das Nächstliegende beschränkte; die Zeit sollte kommen, wo die entferntesten Sprachen auf einander Licht warfen.

\phantomsection\label{I.IV.neuerezeit}
Unter den geistigen Erzeugnissen der \so{neuen Zeit} ist die vergleichende und allgemeine Sprachwissenschaft eines der jüngsten: unbewusst aber hat man schon seit der Zeit des Humanismus, der grossen Entdeckungen und der kirchlichen Reformation auf sie vorgearbeitet. \fed{{\textbar}25{\textbar}}\phantomsection\label{fp.25} Das Lateinische wurde gründlicher getrieben mit der Absicht, es in classischer Form zu beherrschen. Das Studium des Griechischen, zunächst durch vertriebene byzantinische Gelehrte den Occidentalen eröffnet, gelangte bald zu allgemeinem Ansehen. Die Bibelforschung zog schnell das Hebräische in ihren Bereich. Christliche Sendboten, anfangs meist katholische, zogen in die neu erschlossenen Länder, lernten ihre Sprachen, schrieben in ihnen und über sie; manche ihrer Bücher sind in unseren Tagen mehr als doppelt mit Golde aufgewogen worden. An Anregungen zum Vergleichen der Sprachen mangelte es den frommen Männern nicht; sie waren aber zumeist allzu befangen in jüdischen \update{Ueber\-lieferungen,}{Über\-lieferungen,} als dass sie der Sache mit wissenschaftlicher Frische hätten zu Leibe gehen können, und, irre ich nicht, so wurde ihnen schliesslich diese Beschäftigung von der Curie untersagt. Liest man ihre Bücher, so wundert man sich wohl zuweilen, wie pedantisch sie den fremdartigen Stoff den Formen der lateinischen Grammatik anzupassen suchen; dann aber ist man auch nicht selten freudig überrascht über ächt wissenschaftliche Ahnungen. Der Italiener \textsc{Fil. Sassetti} war der erste, der es bekannt machte, dass die indische Sprache „viele Dinge mit der italienischen gemein habe“; das war im sechszehnten Jahrhunderte! Zu Anfang des vorigen Jahrhunderts sprach der spanische Dominicaner \textsc{Franc. Varo}, der erste, von dem man eine gedruckte chinesische Grammatik besitzt, den Satz aus: auf drei Dinge komme es im Chinesischen an, auf die Betonung, die Wortstellung und die Phraseologie. Das war weniger schneidig aber erschöpfender als das hundert Jahre später vom Engländer \textsc{Marshman} niedergeschriebene Wort: The whole of chinese grammar depends on position. Der französische Jesuit \textsc{P. Prémare}, wenig jünger als Varo, eiferte schon gegen die Einführung der lateinischen Terminologie in die chinesische Sprachlehre und verlangte unmittelbare Ein\-\sed{{\textbar}{\textbar}26{\textbar}{\textbar}}\phantomsection\label{sp.26}führung des Schülers in den Sprachgeist, – ad legitimum germanumque sinicae loquelae usum et exercitationem. Manche dieser Geistlichen quälen wohl erst sich und ihren Stoff durch alle Kapitel der lateinischen Grammatik hindurch, theilen aber dann in Form eines Anhanges mit, was eigentlich den Geist der Sprache ausmacht. Im Jahre 1798 erschien des österreichischen Carmeliters \textsc{Johann Philipp Wesdin} (Paulinus a S.~Bartholomaeo) Abhandlung De antiquitate et affinitate linguae Zendicae, Samscrdamicae et Germanicae.

Der erste mir bekannte Sprachvergleicher im heutigen Sinne des \fed{{\textbar}26{\textbar}}\phantomsection\label{fp.26} Wortes ist der gelehrte Holländer \textsc{Hadr. Relandus}, der in seinen Dissertationes miscellaneae, Utrecht 1706–1708, die weite Verbreitung des malaischen Sprachstammes, sogar \so{Lautvertretungsgesetze} zwischen Malaisch und Madegassisch nachweist. Im Jahre 1770 erschien des Ungarn \textsc{J.~Sajnovics} Buch: Demonstratio idioma Hungarorum et Lapponum idem esse, ebenfalls mit Lautvergleichungen und in der That den Verwandtschaftsnachweis führend. Der Vater der \so{grammatischen Sprachvergleichung} ist ein anderer Ungar, \textsc{S.~Gyarmathy}, der, wie er sich ausdrückt, die Verwandtschaft seiner Muttersprache mit den Sprachen finnischen Ursprunges grammatisch, durch eine Formenvergleichung, erweist.

\cohead{Das Sanskrit. Die Polyglotten.}\phantomsection\label{I.IV.sanskritstudien}
Uns Westeuropäern musste die Sache der vergleichenden Sprachwissenschaft näher gerückt werden, ehe sie eine bleibende Heimstätte bei uns finden konnte. Sprachvergleichung verlangt etymologische Analyse, und diese will zuerst an einem leicht zerlegbaren Stoffe geübt sein. Ein solcher sind aber unsre westländischen Sprachen keineswegs. An dieser Stelle erklärt sich die leitende Rolle, die dem \so{Sanskrit} in unsrer Wissenschaft zufallen sollte, – dem Sanskrit und der grammatischen Kunst der Inder. Ein Deutscher, der Jesuit \textsc{Hanxleden}, war der Erste, der eine Sanskritgrammatik für Europäer verfasste, ein anderer, der schon genannte \textsc{Wesdin}, der erste, der eine solche herausgab (1790); leider verdunkelte er dabei das Lautwesen der Sprache durch seine unglückliche Transscription, die der tamulischen Aussprache folgte. Dem halfen die Arbeiten der Engländer \textsc{Colebrooke}, \textsc{Carey}, \textsc{Wilkins} und \textsc{Forster} (1805–1810) ab. Werke der indischen Literatur wurden nach Europa gebracht, gedruckt, übersetzt, und die Zeit war da, wo man sie zu geniessen verstand. \sed{Es war die Zeit der Romantiker, und ein Romantiker war der Erste, der uns Deutsche in das jüngst erschlossene Zauberland einführen sollte.} Im Jahre 1808 erschien \textsc{Friedrich Schlegel}’s Buch: \update{Ueber}{Über} die Sprache und Weisheit der Inder, das durch seine Classification der Sprachen epochemachend wurde. Zu den Freunden der indianistischen Studien gehörte nun auch \textsc{Franz Bopp}, der mit 25 Jahren, 1816, das Conjugationssystem der indogermanischen Sprachen (noch mit Ausschluss der slavo-litauischen, keltischen und des Armenischen) veröffentlichte, den würdigen Vorläufer seiner grossartigen Vergleichenden Grammatik (1833 flg.). Gleichzeitig mit ihm hatte der Däne \textsc{Rasmus Christian Rask}, einer der bedeutendsten Sprach\-\sed{{\textbar}{\textbar}27{\textbar}{\textbar}}\phantomsection\label{sp.27}forscher seiner Zeit, von der skandinavischen Familie ausgehend, seine Untersuchungen weiterhin über den indogermanischen Stamm ausgedehnt; in einer 1818 erschie\-\fed{{\textbar}27{\textbar}}\phantomsection\label{fp.27}nenen Preisschrift entwickelt er zuerst in allen wesentlichen Punkten das nachmals an \textsc{Jacob Grimm}’s Namen geknüpfte Lautverschiebungsgesetz. \textsc{Grimm} selbst hat dieses Gesetz in der ersten Auflage seiner Grammatik 1819 noch nicht erwähnt, in der \update{zweiten}{zweiten, 1822,} aber es mustergültig dargestellt. \sed{Im Grunde hatte man es doch den Indern zu danken, dass man von nun an in der Grammatik den lautlichen Erscheinungen ganz anders Rechnung tragen lernte, als vorher.}

Nicht immer gilt der Satz, dass ein grosses Buch ein grosses Unglück sei. Zwei vierbändige Werke, \textsc{Jacob Grimm}’s Deutsche Grammatik und \textsc{Franz Bopp}’s Vergleichende Grammatik, sowie \textsc{A. F. Pott}’s fast gleichzeitig erschienene Etymologische Forschungen, 2 Bände, \sed{1833, 1836,} sind für die gesammte genealogisch-historisch vergleichende Linguistik grundlegend geworden. \update{Dort}{Von \textsc{Bopp}} wurde zum ersten und bisher einzigen Male die ungeheure Arbeit unternommen, das ganze Material einer Sprachfamilie grammatisch zu ordnen und zu beurtheilen; \update{hier}{bei \textsc{Grimm}} war es ein grosser Sprachstamm, der unsrige, dessen Zweige in ihren Übereinstimmungen und Verschiedenheiten einander gegenseitig erklärten und ergänzten. \sed{Der Jüngste der Drei aber fügte den grammatischen Arbeiten seiner Vorgänger ein gross angelegtes lexicalisch vergleichendes Werk hinzu.} Die Gerechtigkeit gebot, dass wir auch Jener gedachten, die zuerst den Weg gewiesen haben, der Pfadfinder, die den Strassenbauern \update{voran\-gehen}{voraus\-gehen} mussten.

\phantomsection\label{I.IV.polyglotten}
Seltsame Dinge aus fernen Ländern, von Reisenden heimgebracht, sind zuerst nur Gegenstände neugierigen Ergötzens; erst später werden sie zu wissenschaftlicher Forschung gesammelt. Bärenzwinger und Menagerien waren die Vorläufer unsrer zoologischer Gärten; Waffen und Ziergeräthe der Wilden wanderten in Prunksäle und Raritätencabinets, ehe sie sich in ethnographischen Museen zusammenfanden. Und mit den Sprachen war es ähnlich: polyglotte Vaterunsersammlungen eröffneten den Reigen: \textsc{Cl. }\textsc{Duret}’s \update{Thrésor}{Trésor} de l’histoire des langues de cest univers, 2. Aufl. Yverdon 1619, ein stattlicher Quartant, war doch nur ein werthloses Sammelsurium. Des grossen \textsc{Leibniz} allseitiger Geist richtete sich auch auf die Probleme der menschlichen Sprachen: die europäischen suchte er verwandtschaftlich zu ordnen, wobei ihm u.~A. die Verwandtschaft des Magyarischen mit dem Suomi, des Türkischen mit dem Mongolischen und Mandschu auffiel; den Ursprung der Sprache, den Werth ihrer Ausdrücke für allgemeine und besondere Begriffe zog er in Betrachtung und verlangte synoptische Wörtersammlungen aus allen Sprachen der Erde. Das sogenannte \retro{Vocabul\-arium}{Vocubal\-raium} Catharinae, 1786–89, ist wohl mit auf seine Anregung entstanden, jedenfalls war es die Verwirklichung eines Leibniz’schen Gedankens. Ein ausserordent\-\fed{{\textbar}28{\textbar}}\phantomsection\label{fp.28}lich begabter und thätiger Linguist war der Spanier \textsc{Lorenzo Hervás}. Als Missionar in dem vielsprachigen Amerika mochte er sein Interesse für die Mannichfaltigkeit der \sed{{\textbar}{\textbar}28{\textbar}{\textbar}}\phantomsection\label{sp.28} menschlichen Zungen gewonnen haben, und dies Interesse fand später, als er in Rom mit Amtsbrüdern aus allen Erdtheilen zusammentraf, reichliche Nahrung. Sein Hauptwerk, der dreibändige Catálogo de las lenguas, 1800–1802, \update{ist,}{ist} wie der weitere Titel besagt, ein in vieler Hinsicht gelungener Versuch, die Völker nach der Verwandtschaft ihrer Sprachen zu classifiziren. Weniger selbständig aber reichhaltiger ist \textsc{Johann Christoph Adelung}’s bekannter Mithridates, fortgesetzt von \textsc{Johann Severin Vater}, 4 Bände, 1806–17, mit grammatischen Abrissen und Vaterunsern, ein Buch, trotz aller Mängel einzig in seiner Art, das als Mittel zur raschen Orientirung erst in neuerer Zeit durch \textsc{Friedrich Müller}’s noch weit umfänglicheren „Grundriss der Sprachwissenschaft“ (Wien 1876 flg.) verdunkelt und mehr als ersetzt worden ist.

\phantomsection\label{I.IV.keilschriften}
\largerpage[-1]\sed{Auch der \so{Entzifferungen} müssen wir hier gedenken. Denn die schweigenden Zeugen einer fernen Vergangenheit, die Urkunden der ältesten Culturstaat\-en, reizten mit Zaubermacht die Neugier, wohl auch den Ehrgeiz, und übten den Scharfsinn. Im Jahre 1814 that \textsc{G. Fr. Grotefend }die ersten erfolgreichen Schritte zur Enträthselung der persepolitanischen Keilschriften. Nach und nach wurden auch mit Hülfe zwei- und mehrsprachiger Inschriften die anderen Keilschriftgattungen und die in ihnen vertretenen Sprachen in den Bereich der Untersuchung gezogen (\textsc{Lassen}, \textsc{Burnouf}, \textsc{Hincks}, \textsc{Rawlinson}, \textsc{J. Oppert} u.~A.). \textsc{J. Fr. Champollion-Fignac} (1824 flg.) aber gebührt der unsterbliche Ruhm, die Hieroglyphen der Aegypter zum Reden gebracht zu haben. So wurden der Philologie neue Bahnen eröffnet, der Sprachforschung neue Quellen zugeführt.}

\cohead{W. v. Humboldt; die Gegenwart.}\phantomsection\label{I.IV.humboldt}
Das philosophische Zeitalter hat seine Neigung zu apriorischen Constructionen auch auf die menschliche Sprache ausgedehnt, aber die hierher gehörigen Versuche, so geistvoll manche von ihnen sind, dürfen wir übergehen. \textsc{Wilhelm von Humboldt} war der Erste, der umfassende Kennerschaft mit philosophischem Tiefsinne in sich vereinigte, zudem ein sehr fruchtbarer Schriftsteller. Die \update{Fragen\newline[\textit{in den Berichtig\-ungen, S.~502}: Frage]}{Frage} nach der Wechselbeziehung zwischen Sprechen und Denken, nach dem Werthe der Sprachen als Werkzeugen der Cultur und Zeugen der Culturbegabung ist von ihm zuerst aufgeworfen und in umfassender Weise erörtert worden. Seine Werke sind so zu sagen der classische Text der allgemeinen Sprachwissenschaft bis auf den heutigen Tag; sie sind \fed{leider} meist weniger klar als tief, beurtheilen mehr, als sie lehren, und setzen Kenntnisse voraus, die schon aus äusseren Gründen den Wenigsten erreichbar sind. \sed{Es ist erstaunlich, wie allseitig dieser Riesengeist seinen Stoff durchdacht hat, allerwärts hin anregend und Wege weisend. Selten sind reiches Wissen und tiefes Denken, Scharfblick für das scheinbar Kleinste und die Fähigkeit, selbst das scheinbar Entlegenste sicher zu combiniren, glücklicher gepaart gewesen. Wer es liebt, in den Grundanschauungen unserer Wissenschaft den Prioritätsansprüchen nachzuforschen, der versäume nicht, auch bei diesem gedankenreichsten unter den Sprachforschern an\-}{\textbar}{\textbar}29{\textbar}{\textbar}\phantomsection\label{sp.29}\sed{zufragen. Es ist wohl nicht allemal leicht zu sagen, dass er gerade den gesuchten Gedanken gehabt habe; es ist noch viel schwerer zu sagen, dass er einen richtigen Gedanken nicht gehabt habe. Sehr oft aber wird man finden, dass was vor seinen Prophetenaugen in voller Klarheit stand, erst lange hernach mühsam wieder entdeckt worden ist. Wenige Schriftsteller verlangen so angestrengtes, beharrliches Studium, wie dieser; wenige aber lohnen es auch in gleichem Masse: und es ist unter \textsc{Steinthal}’s Verdiensten nicht das kleinste, dass er immer und immer wieder das Andenken des Halbvergessenen aufgefrischt hat.} \textsc{Humboldt} verehren, bewundern wird Jeder, der seine Schriften liest, Mancher bewundert ihn auch ohnedem; – ihm nachstreben werden immer nur Wenige, eine Schule aber nach Art der Bopp’schen und Grimm’schen wird sich wohl nie um ihn schaaren. Von ihm wie von \textsc{Pott} gilt es, dass sich Universalität und Genialität nicht schulmässig züchten lassen.

\phantomsection\label{I.IV.indogermanistik} Fast von Anfang an bildeten die Indogermanisten mit ihren Unterabtheilungen: Germanisten, Romanisten, Slavisten u.~s.~w., eine gesonderte Gemeinde für sich, die rasch zu der zahlreichsten anwuchs. Es war dies kein Wunder. Unsern Vorfahren und Seitenverwandten gebührt unser \fed{{\textbar}29{\textbar}}\phantomsection\label{fp.29} nächstes Interesse; die Frage: wie hat unsre Sprache vor Jahrhunderten, vor Jahrtausenden geklungen, die archäologische Frage, übt auf jeden Denkenden ihren Zauber, sie liegt dem Gemüthe, der pietätsvollen, patriotischen Gesinnung, fast ebenso nahe, wie dem forschenden Verstande. Und wie reichlich fliessen gerade uns Indogermanen die Quellen der Vorzeit! Dazu mochte noch Eins kommen: Wörter, Laute und grammatische Formen verschiedener Sprachen kann man vergleichen, ohne der Sprachen selbst mächtig zu sein, man arbeitet eben am todten Körper, hat es, streng genommen, nicht mit der Sprache, sondern mit ihren losgerissenen Theilen zu thun. Dazu bedarf es keines Sprachtalentes, dem ersten Anscheine nach auch keines philologischen Eindringens in fremde Literaturen, am allerwenigsten der philosophischen Vertiefung. In der That schien hier eine Wissenschaft erstanden zu sein, die weder Kennerschaft noch tiefsinnige Spekulation erforderte, sondern nur fleissiges Sammeln und saubere Analyse. \update{Der verdienstvolle \textsc{A. Hovelaque}\newline[\textit{in den Berichtig\-ungen, S.~502}:\newline Hovelacque] erklärt in seinem vielgelesen\-en Buche: la lingui\-stique (2.~Aufl. S.~14) frischweg:}{Der Verfasser eines, wie es scheint, viel gelesenen Buches über Sprachwissenschaft erklärt getrosten Muthes:} „Le linguiste n’a que faire d’être polyglotte, ou, du moins, il n’est point nécessaire qu’il le soit.“ Zum Beweise schreibt er ein Werk in Taschenformat, das hintereinander die isolirenden, agglutinirenden und flectirenden Sprachen der Erde schildert und durch eine Reihe der wunderlichsten Missverständnisse den Verfasser vom Verdachte der Vielsprachigkeit reinigt. Ein anderer hervorragender Sprachforscher hat ein Werk ähnlich umfassenden Inhalts geliefert, worin z.~B. gelehrt wird, die malaischen Sprachen hätten nur vocalisch auslautende Wörter, die drâvidischen seien ausschliesslich präfigirend. Die Wahrheit ist, dass die meisten malaischen Sprachen, wie schon ein Blick auf die \sed{{\textbar}{\textbar}30{\textbar}{\textbar}}\phantomsection\label{sp.30} Landkarte lehrt, auch consonantische Auslaute haben, und dass die drâvidischen rein suffigirend sind. \sed{Wieder ein Anderer legt gleichfalls feierlich „Protest ein gegen die Voraussetzung, als ob der, welcher die Sprache studirt, ein grosser Sprachkundiger sein müsse“, und behauptet wirklich, noch im Jahre 1890, Dinge wie die, dass es im Chinesischen keinen lautlichen Verfall gebe; dass die Sprache des Ulfilas in die Karls des Grossen umgewandelt worden; dass Sprachen sich niemals vermischen; das ginta im lateinischen triginta eine Ableitung und Abkürzung von sanskrit (!) daça oder daçat, zehn ist\footnote{\sed{Erst an viel späteren Stellen zeigt der Verfasser, dass auch er nicht daran denkt, das Althochdeutsche vom Gotischen, das Lateinische vom Sanskrit herzuleiten.}}; dass arische und semitische die einzigen Sprachfamilien sind, welche diesen Namen vollkommen verdienen.} Le linguiste n’a que faire d’être polyglotte! Der dies geflügelte Wort gesprochen, kennt auch eine „gemeinsame Grammatik aller isolirenden, und eine ebensolche aller agglutinirenden Sprachen.“ Ich kenne in der That zwei Mittel, sich diese „gemeinschaftlichen Grammatiken“ ganzer Sprachclassen anzueignen: entweder man lerne von jeder dieser Classen nur eine Sprache, oder, – noch einfacher –, man lerne gar keine!

\phantomsection\label{I.IV.verzweigung} Wir theilten vorhin die Sprachwissenschaft organisch in einzelsprachliche, hist\-orisch-genealogische und allgemeine. Eine andere Dreitheilung hatte sich thatsächlich herausgebildet: erstens die Indogermanistik, \fed{{\textbar}30{\textbar}}\phantomsection\label{fp.30} in sich geschlossen, wenn auch nicht immer unter sich einig, dann die classische Philologie, und endlich, in loseren Gruppen, alle Übrigen: die orientalischen Philologen und die, welche sich auch mit literaturlosen Sprachen beschäftigten. Ganz durchgreifend war natürlich die Scheidung schon von Hause aus nicht: Brücken führten über die Klüfte, die sich allen Anzeichen nach bald von selbst schliessen werden, so sehr die Masse des Stoffes zur Arbeitstheilung nöthigt. Bis zu seinem Tode, 1887, war der Altmeister unter den Indogermanisten \textsc{August Friedrich Pott} in Halle. Der konnte nachgerade von seinen Sprachstudien sagen, sie haben die Reise um die Erde gemacht; seine meisten Gefährten und Nachfolger waren zugleich Sanskritphilologen. \sed{Mein Vater war nicht minder \corr{1901}{vielseitig.}{vielseitg.} Oft verband sich bei ihm das philologische Interesse an fremden Literaturen mit dem allgemein sprachwissenschaftlichen. Am Liebsten wandelte er auf unbetretenen Pfaden, führte der Sprachenkunde neuen Stoff zu; und dabei hat er auch an der Entdeckung und Feststellung mehrerer Sprachstämme unter den ersten Pionieren mitgearbeitet.} \textsc{Georg Curtius} gebührt das hohe Verdienst, die classische Philologie mit der Indogermanistik vermählt zu haben. \sed{Es war kein leichtes Werk. Lange ist es der älteren Schwester hart angekommen, sich von der jüngeren in die Schule nehmen zu lassen.} Jene, die sich der Vergleichung fremder Sprachstämme zuwandten, waren von selbst auf das Vorbild der Indogermanisten hingewiesen. Bei diesen musste sich zudem die Forschung \sed{{\textbar}{\textbar}31{\textbar}{\textbar}}\phantomsection\label{sp.31} der altbaktrischen Religionsurkunden und der \retro{persischen}{persichen} Keilschriften Raths erholen. So waren die Indogermanisten die vielseitig umworbenen; – kein Wunder, dass sie die Macht ihrer Stellung fühlten, wohl auch fühlen liessen. Freilich lag ihre Stärke zum guten \update{Theile}{Theil} in der Einseitigkeit, und eben dieser sind sie sich zu ihrem und Aller Vortheile nun auch bewusst geworden. Schärfer als je zuvor betonen und untersuchen sie jetzt auch die seelischen Kräfte, von denen die Entwickelung der Sprachen abhängt. Dass die Sprache mehr ist, als ein Aggregat von Wurzeln, Stämmen \retro{und}{uud} Formen, dass sie im Satze ihre Einheit hat, begreifen sie jetzt, ziehen die Syntax in den Bereich ihrer vergleichenden Untersuchungen und fragen gelegentlich, um zur Erklärung räthselhafter Erscheinungen zu gelangen, bei Sprachen fremdartigen Baues \update{an:}{an;} die Zeit der gegenseitigen Verständigung ist gekommen.

\begin{styleAnmerk}
\sed{Anmerkung. Vgl. Th. Benfey, Geschichte der Sprachwissenschaft und orientalischen Philologie in Deutschland. München 1869.}
\end{styleAnmerk}

\fed{{\textbar}31{\textbar}}\phantomsection\label{fp.31}

\clearpage\cehead{{{\large I,}} V. Schulung des Sprachforschers.}
\pdfbookmark[1]{V. Capitel.}{I.V}
\section*{V. Capitel.}
\pdfbookmark[2]{§. 1. Schulung des Sprachforschers}{I.V.1}
\section*{Schulung des Sprachforschers.}
\subsection*{§. 1.}\phantomsection\label{I.V.1}

Die Linguistik war bis in die neuere Zeit keine Berufswissenschaft, der man sich widmen konnte um von ihr zu leben, die man auf Hochschulen studirte, und in der man Examina bestand. Sie glich einer Colonie, deren erste Bebauer aus verschiedenen Gebieten zugewandert \update{waren.}{waren,} \sed{und noch heute gereicht ihr solcher Zuzug oft zum Gewinne. Vielleicht heute erst recht; denn wo immer sich Schulen bilden, da liegt auch die Gefahr zünftlerischen Schlendrians und unduldsamer Vereinseitigung nahe. Da ist es denn nur heilsam, wenn fremde, von Traditionen freie Elemente sich als Mitarbeiter unter die Fachleute mischen, Männer, die weiter nichts mitbringen, als positive Sachkenntniss, – ich meine Sprachkenntniss –, einen in scharfer Logik geschulten Verstand und die Gewohnheit um- und vorsichtiger inductiver Methode. –} \update{Den nächst\-gelegenen,}{Dem nächst\-gelegenen Fache,} der Philologie und Orientalistik, gehörten z.~B. die beiden \textsc{Schlegel}, \textsc{Adelung}, \textsc{Vater}, \textsc{Bopp}, \textsc{Klaproth} an; der berühmte Finne \textsc{Alexander Castrén} sowie, wenn ich nicht irre, \textsc{Wilhelm Schott} waren von Hause aus Theologen; \textsc{Wilhelm von Humboldt}, \textsc{Jacob Grimm}, \textsc{Sylvestre de Sacy}, \textsc{Eugen Burnouf} und mein verewigter Vater hatten die Rechte studirt, ebenso von \update{Neuereren}{neueren} z.~B. der vielseitige und scharfsinnige \textsc{Lucien} \update{\textsc{Adam};}{\textsc{Adam}} \sed{und sein College \textsc{Raoul de la Grasserie},} \textsc{Abel} \update{\textsc{Rémusat} und}{\textsc{Remusat},} der grosse Indianist \textsc{Wilson} \sed{und der verdiente Americanist \textsc{Otto Stoll}} Medizin. Auch Offiziere in den Colonialdiensten der \sed{{\textbar}{\textbar}32{\textbar}{\textbar}}\phantomsection\label{sp.32} verschiedenen Staaten haben sich um unsere Wissenschaft hohe Verdienste erworben, nicht nur als Sammler, sondern auch als tüchtige Forscher. Ich will nur einen nennen, den General \textsc{Faidherbe}. Der Zuzug von Auswärts hat fortgedauert bis auf den heutigen Tag, der jungen Wissenschaft war und ist er willkommen, sie ist bisher eine freie Wissenschaft geblieben, – so frei, dass ihr kaum ein allgemeiner Studienplan vorgezeichnet werden kann: Jeder hat es selbst zu erproben, welche Richtung des Forschens seiner Anlage und Neigung am besten entspricht. \sed{Er kann sein Genügen darin finden, eine oder einige Sprachen möglichst allseitig gründlich zu beherrschen: das ist zumal das Ziel der Philologen. Oder er kann eine ganze Sprachfamilie in Rücksicht auf den einen oder anderen Theil ihrer Erscheinungen durch alle Phasen ihrer Entwickelung durch verfolgen, wie es die Indogermanisten, Altaisten u.~s.~w. thuen. Oder endlich mag er sich als Problem das menschliche Sprachvermögen selbst, die Ursachen seiner vielgestaltigen Entfaltung gesetzt haben und zu dem Ende polyglottes Wissen erstreben. Man sieht, die Aufgaben sind verschieden genug.} Allein gewisse Kräfte und Fertigkeiten sind doch gemeinsame Voraussetzung einer jeden Art sprachwissenschaftlicher Arbeit, und von diesen soll hier die Rede sein.

Man redet vom geistigen Auge; man sagt, mit anderen Augen durchwandere der Botaniker, mit anderen der Mineralog die Fluren, mit anderen etwa der Jäger oder der Landgensdarm. Sie alle haben dieselben Dinge vor Augen und doch andere Dinge im Auge. Sie Alle suchen, und das Suchen bedingt und befördert zugleich eine gewisse Vereinseitigung, es schärft und richtet den Blick nach einer einzelnen Art von Gegenständen. Jetzt dürfen wir auch von dem Auge des \fed{{\textbar}32{\textbar}}\phantomsection\label{fp.32} Sprachforschers reden, vom sprachwissenschaftlichen Blicke; und auch dieser kann bis zu einem gewissen Grade anerzogen werden.

Eine recht missliche Seite hat es nun freilich, eine solche Hodegetik zu veröffentlichen. Es kann ja nicht anders sein: der Verfasser empfiehlt vor Allem das, dessen Nutzen er selbst erprobt zu haben meint, oder dessen Mangel ihm bei Anderen besonders störend aufgefallen ist. So scheint er sich stillschweigend in aller Naivität selbst als Muster eines Sprachforschers hinzustellen. \fed{Der Vorwurf einer solchen Lächerlichkeit trifft mich hoffentlich nicht.} \sed{Die Sache ist aber einfach die, dass Jeder das erstrebt, was er für das Richtigste hält, und folglich auch als das Richtigste das bezeichnet, was er selbst erstrebt und nach Kräften verwirklichen möchte.}

\cohead{§. 2. a. Phonetische Schulung.}
\pdfbookmark[2]{§. 2. a. Phonetische Schulung.}{I.V.2}
\subsection*{§. 2.}\phantomsection\label{I.V.2}
\section*{a. Phonetische Schulung.}

Dass der Sprachforscher nicht auch Sprachkünstler zu sein braucht, dass ein Sprachkünstler darum noch lange kein Sprachforscher sein muss, versteht sich von selbst. Und unter den Kunststücken des Sprachkünstlers ist wieder \sed{{\textbar}{\textbar}33{\textbar}{\textbar}}\phantomsection\label{sp.33} das Nachmachen fremder Laute vom linguistischen Standpunkte aus das werthloseste. Es scheint, als wäre von Hause aus jedes normale menschliche Sprachorgan zur Hervorbringung aller möglichen Sprachlaute \update{geschickt;}{geschickt:} erst fortgesetzte einseitige Übung erschwert uns die Bildung fremder Laute. Insofern verhält es sich mit der Phonetik genau, wie mit dem Geiste einer beliebigen Sprache, der am leichtesten da aufgenommen wird, wo er \update{tabula}{Tabula} rasa vorfindet.

Die Erfahrung hat nun bewiesen, dass man Sprachen von Grund aus grammatisch verstehen und sehr richtig beurtheilen kann, ohne von ihren Lauten mehr zu wissen, als dass sie deren \so{ungefähr} so und soviele besitze, die sich \so{ungefähr} so und so \update{zueinander}{zu einander} verhalten. Für die alten Cultursprachen hat man in den verschiedenen Ländern conventionelle Ausspracheweisen eingeführt, wohl wissend, dass man sich damit weit vom ursprünglichen Klange \update{entferne,}{entfernte,} – und doch ohne Nachtheil für die Praxis, wie für die Theorie. Und gesetzt, es gelänge uns, etwa Griechisch genau in den Lauten und dem Tonfalle der Athener perikleischer Zeit auszusprechen: was wäre gross damit gewonnen? Ein Jahrhundert früher oder später hat man in Athen schon etwas anders gesprochen, und ein paar Meilen von Athen wieder anders. Dem Historiker, auch dem Biographen, muthet man nicht zu, dass er uns einen grossen Mann vorführe, „wie er sich räuspert und wie er sich spuckt“; \fed{{\textbar}33{\textbar}}\phantomsection\label{fp.33} und Ähnliches gilt in der Regel von dem Sprachforscher und den kleinen Absonderlichkeiten der Lauterzeugung. \textsc{Winteler}’s vielgerühmtes Werk über die Kerenzer Mundart gilt als Muster sorgsamer Lautbeobachtung; sein wissenschaftlicher Werth beruht aber nicht sowohl in der untersuchten Mundart, als in der Art der Untersuchung und den gewonnenen Ergebnissen; jene Mundart war eben das Kaninchen des Physiologen.

Man irrt, wenn man die Lautphysiologie oder Phonetik, wie man sie heutzutage nennt, als einen Theil der Sprachwissenschaft bezeichnet. Letztere hat es mit den Schallerzeugnissen der menschlichen Sprachorgane nur insoweit zu thun, als sie in den Sprachen thatsächlich Verwendung finden; die Phonetik dagegen hat alle überhaupt möglichen Schalläusserungen jener Organe zu untersuchen, folgerichtig auch die krankhaften und die auf individuellen Fehlern beruhenden, z.~B. die Wirkungen eines Stockschnupfens oder eines fehlenden Zahnes auf die Hervorbringung der Laute. Und dies ist nicht der einzige Unterschied. Die Lautphysiologie hat es mit dem Laute zu thun, wie er jeweilig von und in den Sprachwerkzeugen gebildet und vom \update{Ohre}{Ohr} vernommen wird; in ihrem Sinne bringt also die geringste Änderung in der Stellung und Bewegung der Sprachorgane einen anderen Laut hervor. Das ist von ihrem Standpunkte aus berechtigt und nothwendig. Die Sprache aber, und wäre es die kleinste Mundart, unterscheidet nur eine bestimmte Anzahl von Lauten, die sich zu den lautlichen Einzelerscheinungen verhalten wie Arten zu Individuen, wie \sed{{\textbar}{\textbar}34{\textbar}{\textbar}}\phantomsection\label{sp.34} Kreise zu Punkten; sie zieht die Grenzen weiter oder enger, immer aber duldet sie einen gewissen Spielraum. Nicht Alle, die die Mundart richtig sprechen, sprechen den nämlichen Laut genau auf dieselbe Weise aus, ja man darf zweifeln, ob es der Einzelne immer thue. Es handle sich um das Wort „Thee“. Der Leser frage sich, ob er, so oft er es ausspricht, allemal das Vorderende der Zunge gleich spitz oder breit macht, ob er es allemal genau an der nämlichen Stelle der Zähne oder des Gaumens anlegt, ob er allemal genau einen gleichstarken Luftstrom verwendet u.~s.~w. Hat er gelernt, in solchen Dingen scharf zu beobachten, so wird er wahrscheinlich kleine Schwankungen wahrnehmen. Nun aber weiss er, und bestätigt ihm der Sprachforscher, dass er immer dasselbe Wort, und dass er es immer richtig ausgesprochen hat; der Phonetiker hingegen wird ihm nachweisen, dass es im physiologischen Sinne verschiedene Einzellaute waren. Natürlich gilt dies erst recht, wenn dasselbe Wort von verschiedenen Leuten \fed{{\textbar}34{\textbar}}\phantomsection\label{fp.34} ausgesprochen wird, seien es auch Angehörige derselben Landschaft oder Gemeinde. Das Sprachgefühl, das für uns \update{maass\-gebend}{mass\-gebend} ist, macht da keinen Unterschied, es erkennt jede Art der heimischen Lautbildung für gleich richtig an, weiss aber recht wohl die \so{in seinem Sinne} fremdartige Aussprache zu erkennen.

Bekanntlich verhalten sich die Sprachen in Rücksicht auf die Lautunterscheidung und die Schärfe der Articulation sehr mannichfaltig. Die Polynesier besitzen ausser den fünf Vocalen \textit{a}, \textit{e}, \textit{i}, \textit{o}, \textit{u} nur acht bis zehn unterschiedene Consonanten; die alten Inder erkannten im Sanskrit deren 35 oder (mit \textit{jihvāmūlīya} und \textit{upad‘mānīya}) 37; die Abchasen im Kaukasus kennen deren 49. Und doch sind diese Zahlen, so vielsagend sie scheinen mögen, noch immer das Äusserlichste. Welche Laute besitzt die Mundart? Ich erinnere an die Gutturale, Aspiraten und Fricativen der Semiten, der Kaukasier und vieler amerikanischer Indianerstämme, an die Zischlaute und Jodirungen der Slaven, an die Schnalzer der Hottentotten, Buschmänner und ihrer kaffrischen Nachbarn. Wie häufig oder selten kommen die einzelnen Laute vor? Man denke an die e und n, die im Deutschen, an die i und s, die im Lateinischen vorherrschen, an die statistischen Nachweise in \textsc{Whitney}’s Sanskrit-Grammatik. Ferner: welche Gesetze und Neigungen ergeben sich in Rücksicht auf die Lautvertheilung im Worte? Man denke an die Hiaten in den polynesischen Sprachen, an die Sandhigesetze im Sanskrit, an die Consonantenhäufungen im Alttibetischen, im Georgischen, im Selish (Kallispel) u.~s.~w., an die gefällige Vertheilung der Vocale und Consonanten im Italienischen, im Hausa und in vielen malaischen und kongo-kaffrischen Sprachen, an Sprachen, die nur vocalischen Auslaut dulden, wie das Altslavische und Altjapanische, oder die daneben noch Nasale gestatten, wie die kongo-kaffrischen Sprachen und das Mandschu, – dann wieder an die seltsame Vorliebe der melanesischen Annatom-(Aneiteum-)\update{Insulaner}{Insulaner,} \sed{ja auch der Basken und der Berbervölker,} für \sed{{\textbar}{\textbar}35{\textbar}{\textbar}}\phantomsection\label{sp.35} vocalischen Anlaut. Und wie vielerlei bedingt sonst noch den Klangcharakter einer Sprache oder Mundart! Die ruhige Lippenhaltung des Engländers, die gutturale Lautbildung des Holländers und Schweizers, die Betonung der vorletzten Silbe im Polnischen, Malaischen u.~s.~w., die der ersten im Czechischen und den finnischen Sprachen, die der letzten im Türkischen und Mongolischen, der hüpfende Tonfall der baltischen Deutschen, das „Singen“ der Thüringer, kurz so und soviele Dinge, die wir unter dem französischen Namen accent zusammenzufassen pflegen.

\fed{{\textbar}35{\textbar}}\phantomsection\label{fp.35}

Eine gewisse Ausbildung des Sprach- und Gehörorganes ist wohl für jeden erreichbar und auf alle Fälle jedem Sprachforscher zu empfehlen.

Erstens nämlich hat er es mit fremden Lauten zu thun, die er sich am besten merkt, wenn er sie sich vorstellen und selbst hervorbringen kann. Je mehr Sinne zusammenwirken, desto leichter verrichtet das Gedächtniss seinen Dienst. Die Buchstaben eines fremden Alphabetes, die Häkchen, Pünktchen und Strichelchen unsrer phonetischen Schriftsysteme verwechselt man nur zu leicht, wenn man nicht scharf unterschiedene Gehörvorstellungen mit ihnen verbindet.

Zweitens kann man die mechanischen Vorgänge beim Lautwandel nicht besser verstehen, als wenn man sie selbst darstellen und somit an sich selber erleben kann. \sed{Was die Münder unzähliger Generationen zu Wege gebracht haben, das kann sich, wenn wir verständnissvoll experimentiren, rasch in unseren eigenen Sprachorganen vollziehen; und wo uns die Urkunden gewisse Hauptstationen, oder auch nur die Anfangs- und Endpunkte lautlicher Entwickelungen bezeugen, da tritt uns nun der Hergang in seiner ununterbrochenen Allmählichkeit vor die Sinne.}

Endlich drittens wird das systematische Studium lautphysiologischer Werke dem am leichtesten, der mit einem Vorrathe eigener Erfahrungen an \update{es}{dasselbe} herantritt.

Man bildet sich nur zu leicht ein, zu einer Sprache gehöre nicht viel mehr, als was man schwarz auf weiss auf dem Papiere findet. Nein, Alles gehört zu ihr, was bei der Rede in und mit den Sprachwerkzeugen \update{geschieht,}{geschieht:} Rhythmus und Tonfall (Singen, Eintönigkeit, Breite oder Schärfe) des Vortrages, aber auch die Haltung des Mundes, breit oder spitz gezogene Lippen, vorgeschobener Unterkiefer, schlaffes, verschnupftes Gaumensegel u.~s.~w. Das mag individuell und beim Individuum nur vorübergehend sein, es kann aber auch zur Eigenthümlichkeit der Gaumundart, der Sprache eines ganzen Volkes gehören, und dann gehört es zur Sprachkunst, zur Sprachkunde, zur Sprachlehre.

Es soll an dieser Stelle nicht eine Theorie der Phonetik vorgetragen, sondern nur angedeutet werden, wie jene praktische Aus- und Vorbildung zu erreichen sei. Es handelt sich um eine Gymnastik der Sprachorgane, die mit einer Schärfung des Gefühls- und \update{Gehörs\-sinnes}{Gehör\-sinnes} Hand in Hand gehen wird: wir er\-\sed{{\textbar}{\textbar}36{\textbar}{\textbar}}\phantomsection\label{sp.36}zeugen Laute und beobachten dabei, wie wir die einzelnen Sprachorgane bewegen, was wir dabei in ihnen empfinden, und was die Wirkung auf das Gehör ist. Dabei lernen wir allmählich unterscheiden, was uns anfangs ganz gleich vorkam.

Die Gabe der Nachahmung ist für solche Zwecke recht schätzenswerth, und man sollte sie pflegen, soweit es die gute Sitte erlaubt. Alles was in unsrer Muttersprache als fehlerhaft erscheint, kann in einer \fed{{\textbar}36{\textbar}}\phantomsection\label{fp.36} anderen nothwendig sein: lispelndes Anstossen mit der Zunge, Näseln, Nuscheln, wie es Leute thuen, die mit der dicken Zunge an die Backenzähne anstossen, Muffeln, wie Einer, der redet, während er den Bissen im Munde hat, verschnupfte Lautbildung u.~s.~w. Leute, die einen \update{anderen}{andern} Dialekt reden, Ausländer, die sich vergebens anstrengen die deutschen Laute hervorzubringen, sind gleichfalls gute Modelle. Man lernt bald genug diesen Modellen auf den Mund zu sehen und ahmt dann unwillkürlich ihre Haltung der Lippen und des Unterkiefers nach. Damit ist \update{oft schon}{schon oft} viel gewonnen, z.~B. für die Aussprache des Englischen und mancher deutschen Dialekte. Auch die Stellung der Zähne und die Bewegungen der Zunge muss man zu beobachten und nachzumachen suchen, z.~B. bei den beiden \textit{th} der Engländer und den verschiedenen \textit{r} und Zischlauten. Jedenfalls werden durch die Spielerei die Sprachwerkzeuge geschmeidig, und die Beobachtungsgabe gesteigert. Es kommt ganz von selbst, dass man auch bezeichnende Redewendungen und Gedankenverknüpfungen seines Vorbildes mit nachbildet, und alles das kommt der sprachwissenschaftlichen Befähigung zugute. In keiner Wissenschaft spielt die Reproduction eine wichtigere Rolle, als in der unseren.

Näher schon der eigentlich lautwissenschaftlichen Arbeit liegen freie Versuche, deren einige ich hier beschreiben will.

1. Man versuche denselben Laut mit verschiedener Stellung der Sprachwerkzeuge hervorzubringen: \textit{o} und \textit{u} mit mehr oder minder gerundeten Lippen, die Gutturale \textit{k}, \textit{g}, \textit{ch} (in „machen“), \textit{ṅ} (=ng in „Ding“) soweit hinten in der Kehle als möglich, und allmählich möglichst weit vorwärts weiterschreitend; ebenso die Dentale \textit{t}, \textit{d}, \textit{n} und die Zischlaute \textit{s}, \textit{z} (= weich \textit{s}), \textit{š} (= sch), \textit{ž} (= französisch \textit{j}), \textit{tš}, \textit{dž}, sowie \textit{l}, in zwei Reihen, erstens die Zungenspitze nach dem weichen Gaumen zurückbiegend und dann schrittweise vorwärts bis zwischen die Vorderzähne rücken lassend, – zweitens die Zungenspitze möglichst weit hinten an die Backenzähne anlegend und dann allmählich auf dem seitlichen Wege vorwärtsschiebend; dasselbe wiederhole man mit mehr oder minder zugespitzter oder breiter Zunge; \textit{f} und \textit{w} erst mit beiden Lippen (bilabial), dann durch Berührung der Unterlippe und der Oberzähne (labiodental). \sed{Lässt man beim \textit{l} die Zungenspitze den Gaumen etwas weiter hinten berühren und die Luft zu beiden Seiten der Zunge durchströmen, so gewinnt man einen Laut, der dem \textit{chl} in „weichlich“, dem \textit{lch} \corr{1901}{in „welcher“}{in, ,welcher“} einigermaßen ähnelt, – das welsche \textit{ll}.}

\sed{{\textbar}{\textbar}37{\textbar}{\textbar}}\phantomsection\label{sp.37}

2. Man spreche die Vocalreihen \textit{a}, \textit{ä}, \textit{e}, \textit{i}, – \textit{a}, \retro{\textit{å},}{\textit{a},} \textit{o}, \textit{u}, – \textit{a}, \textit{ö}, \textit{ü}, \textit{i} ohne Absatz wie eine Art langer Diphthongen und beobachte, wie \fed{{\textbar}37{\textbar}}\phantomsection\label{fp.37} sich dabei die Mundstellung allmählich verändert und wie das Ohr dabei eine Reihe unzähliger, winzig verschiedener Klänge vernimmt. \sed{Was man \so{Lautverschiebungen} nennt, sind ursprünglich nichts weiter, als solche Verschiebungen in der Stellung der Sprachorgane, die natürlich die akustischen Wirkungen beeinflussen. In der Sprachgeschichte mögen sie Jahrhunderte gebrauchen, ehe sie wahrnehmbar werden; mittels des Experimentes kann man sie sich in wenigen Minuten vollziehen sehn.}

3. Man bringe die Sprachwerkzeuge in die Lage, die für einen bestimmten Laut passt, und nun bemühe man sich einen beliebigen anderen Laut hervorzubringen, thue z.~B., als ob man \textit{u} sprechen wollte, und versuche dann ein \textit{i} auszusprechen oder umgekehrt, forme die Lippen zum \textit{w} oder \textit{f} und strenge sich dann an, etwas wie ein \textit{s}, \textit{š} oder \textit{h} herauszubringen u.~s.~w. Die verschiedenen \textit{r} sind vielleicht auf diese Weise am leichtesten zu lernen. \sed{Benachbarte Laute, d.~h. solche mit verwandter Mundstellung, gehen vermöge der Lautverschiebung leicht ineinander über. Auf neue, seltsame Laute muss man immer gefasst sein. Ein Labiodental, eine Art \textit{tp} oder \textit{pt}, wobei die Zunge zwischen den Zähnen hindurch die Oberlippe berührt, findet sich in der Tangoa-Sprache von Espiritu Santo, einer Insel der Neuen Hebriden. Bei homerischem πτόλις, πτόλεμος möchte man an etwas Ähnliches denken. (S.~\textsc{H. Ray} in Bydr. tot de Taal-, L- en Volkenk. van Nederl. Indië, V. Volgr. VII. D. pg. 708.)}

4. Man spreche oft und schnell hintereinander einen Consonanten zwischen zwei \textit{i} oder zwei \textit{u} und beobachte, wie der Consonant dabei etwas von der Eigenart jener Vocale annimmt und wohl schliesslich auch für das ungeübte Ohr ein ganz anderer wird. \textit{ĭkĭ} zu \textit{itšĭ}, \textit{ĭtĭ} zu \textit{ĭtsĭ} oder \textit{ĭtšĭ}, \textit{ŭkŭ} zu \textit{ŭkfŭ} oder Ähnlichem. Eine andere, doch verwandte Beobachtung kann man machen, wenn man \textit{a} und \textit{i} mit einem dazwischen gefügten Doppelconsonanten rasch hintereinander ausspricht: \textit{alli}, \textit{anni} werden dann wohl zu \textit{allyi}, \textit{annyi} und schliesslich zu \retro{\textit{aillyi},}{\textit{αillyi},} \textit{ainnyi}, \textit{aiyi}, \textit{ai}, – \textit{ammu} etwa zu \textit{ommu}. \update{Aehnlich}{Ähnlich} mit \textit{a} oder \textit{i} zwischen zwei Labialen: \textit{wap}, \textit{bam}, \textit{wip}, \textit{bim} werden sich bald in etwas wie \textit{wop}, \textit{bom}, \textit{wüp}, \textit{büm} verwandeln. Alles dies erreicht man natürlich nur, wenn man sich gehen lässt, das heisst der Bequemlichkeit nachgiebt. Diese übt aber in den Sprachen eine grosse Macht.

5. Man halte beide Ohren zu, spreche verschiedene Laute und beobachte, bei welchen Lauten die Ohren dröhnen und bei welchen nicht.

6. Lehrreich in ihrer Art sind auch zungenbrechende Sätze wie: „Fischer’s Fritz frisst frische Fische“, – „sechs und sechszig Schock sächsische Schuhzwecken“, – „wenn der Kottbuser Postkutscher mit der Kottbuser Postkutsche nach Putbus fährt, fährt der Putbuser Postkutscher mit der Putbuser Post\-\sed{{\textbar}{\textbar}38{\textbar}{\textbar}}\phantomsection\label{sp.38}kutsche nach Kottbus“, – „in Ulm, um Ulm und um Ulm herum“, – „keine kleinen Kinder können keine kleinen Kirschkerne knacken“ u.~s.~w. Die Fehler, die man dabei im raschen Nachsprechen macht, sind vorbildlich für manche Erscheinungen des geschichtlichen Lautwandels. \sed{In der Übereilung vertauscht die Zunge die Laute, und der Fehler, wenn er bequem ist, kann zur Regel werden. So spanisch olvidar, vergessen, = oblitare, milagro = miraculum, peligro = periculum.}

7. Ähnliche Beobachtungen mechanischer Lautverschiebungen kann man machen, wenn man Wörter mit harten Consonantenverbindungen oder mit Hiaten rasch und wiederholt ausspricht: Ausschnitt wird zu Au\-\fed{{\textbar}38{\textbar}}\phantomsection\label{fp.38}schnitt, entfernt: empfernt\footnote{Vgl. empfinden, empfehlen, empfangen. Woher aber die Inconsequenz?},– etwas: eppas, eppes, haben wir: hammir u.~s.~w. \fed{Ferner: zuerst: zuwerst, zwerst, – beiordnen: beijornn u. s. w.} \update{Unsre}{Unsere} Volksmundarten bieten dessen Beispiele die Hülle und Fülle. \sed{Es sind das Fälle der sogenannten \so{Euphonik}, thatsächlich Äusserungen der Trägheit und Flüchtigkeit; dem Munde wird Arbeit erspart, aber die Lautbilder werden verwischt.}

8. So, durch eine Art Zimmergymnastik vorbereitet, mag man an das Studium eines lautphysiologischen Werkes gehen; man wird es um so leichter verstehen, je mehr Selbsterlebtem man darin begegnet.\footnote{Als erstes Lehrmittel dieser Art dürfte \textsc{Sievers}’ Phonetik zu empfehlen sein wegen der schonenden, allmählichen Art, wie es den Anfänger in die neuen Vorstellungen einführt.}

Mit der Ausbildung der Lautphysiologie ging die Aufstellung verschiedener \so{phonetischer Schriftsysteme} Hand in Hand. Unter diesen hat \textsc{Lepsius}’ s.~g. Standard-Alphabet die weiteste Verbreitung gefunden und darum für den Sprachforscher besonderen praktischen Werth. Die Menge seiner diakritischen Zeichen über und unter der Linie ist beim Lesen und Schreiben \update{einiger\-maassen}{einiger\-massen} störend, und in systematischer Hinsicht ist es durch neuere Versuche weit überholt. Seine Lücken lassen sich aber nach Bedarf durch Hinzufügung neuer Unterscheidungszeichen oder Zeichencombinationen ausfüllen, und so wird man einstweilen gut thun, es für \so{sprachwissenschaftliche} Zwecke, wozu eben die lautphysiologischen \so{nicht} gehören –, beizubehalten. \textsc{F. Techmer}, Phonetik, 2 Bde., Leipzig 1880, hat ein nachmals noch verbessertes Zeichensystem aufgestellt, dessen wissenschaftliche Vorzüge in Fachkreisen wohl anerkannt werden. Die von ihm redigirte Internationale Zeitschrift für allgemeine Sprachwissenschaft bedient sich dieser Schrift, die somit vielleicht bestimmt ist, mit der Zeit die Lepsius’sche zu verdrängen. Wenig bequem zu schreiben ist freilich auch sie, aber reichhaltig, nicht allzuschwer erlernbar und wohl in jeder Druckerei ohne Beschaffung neuer Typen darzustellen.

\so{Zusatz}.\phantomsection\label{I.V.zusatz} Es wäre Pedanterie, jede alphabetlose Sprache ein- für allemal mit irgend einem phonetischen Alphabete schreiben zu wollen. Ein solches dient \sed{{\textbar}{\textbar}39{\textbar}{\textbar}}\phantomsection\label{sp.39} einestheils als eine Art Generalnenner, mit Hülfe dessen sich die Lautwesen verschiedener Sprachen oder Mundarten bequem vergleichen, anderntheils als Mittel, um die Laute einer noch unbekannten Sprache aus dem Munde der Eingeborenen aufzuzeichnen. Handelt es sich um einzelne Sprachen, so vertheilt man die Buchstaben des Alphabetes so, dass sie den Lautwerth möglichst annähernd anzeigen, und ver\-\fed{{\textbar}39{\textbar}}\phantomsection\label{fp.39}mehrt sie nach Bedürfniss durch Hinzufügung diakritischer Zeichen. Bei Sprachen mit eigener Lautschrift substituirt man in gleicher Weise den einheimischen Buchstaben lateinische, mit oder ohne diakritische Zuthaten, sodass die \update{Rück-Umschrei\-bung}{Rückumschrei\-bung} ohne \update{Weiteres}{weiteres} möglich ist. Handelt es sich aber um eine Mundart, die von der einheimischen Rechtschreibung abweicht, so liegt natürlich die Sache so, als wenn keine einheimische Lautschrift vorhanden wäre. Wo endlich schon von Anderen leidlich brauchbare Schreibweisen eingeführt sind, da halte man sich möglichst an diese. Das gilt namentlich oft von den Arbeiten der Missionare.

\cohead{§. 3. b. Psychologische Schulung.}
\pdfbookmark[2]{§. 3. b. Psychologische Schulung.}{I.V.3}
\subsection*{§. 3.}\phantomsection\label{I.V.3}
\subsection*{b. Psychologische Schulung.}

Viel wichtiger, mächtiger als das physische Moment ist das psychische. \sed{Ja, thatsächlich hatten wir es schon vorhin, als wir von jenen Nachlässigkeiten in der Lautbildung sprachen, mit Dingen zu thun, die zur Hälfte seelischen Ursprungs waren. Warum etwas bequem ist, erklärte die Mechanik; warum man aber dieser Bequemlichkeit jetzt nachgiebt, jetzt entsagt, darüber kann nur die Psychologie Aufschluss geben. Ich weiss nicht, ob ich es den angehenden Sprachforschern empfehlen soll, sich lange beim systematischen Studium dieser Wissenschaft aufzuhalten. Ich für meinen Theil bedauere, dass ich für diesen Theil der Philosophie nie viel Ausdauer gehabt und meinen Bedarf an Seelenkunde mehr aus der Praxis des Lebens und aus feinsinnigen Charakterschilderungen bezogen habe, als aus den Theorien fachgelehrter Psychologen. Doch das man individuell sein; Andere haben meines Wissens solchen Studien mehr Genuss und Gewinn zu verdanken gehabt.}

\update{Die Sprache... [\textit{kein neuer Absatz}]}{Die Sprache} ist unmittelbarster Ausfluss der Seele, ihre wichtigsten Erscheinungen können nur aus seelischen Vorgängen erklärt werden. Und wie wunderlich sind oft diese Vorgänge! Unsere anerzogene Schullogik steht ihnen rathlos gegenüber; erklären kann sie die Sprünge des naiven Geistes nicht, höchstens sie beobachten, tadeln und bändigen. Dieser naive Geist nun hat an der Sprachbildung weit mehr Antheil, als der logisch geschulte. Ganz ungeschult, ganz zuchtlos bleibt aber auch er nicht; im fortwährenden Verkehre mit Seinesgleichen hat er gewisse Gewohnheiten angenommen, gewisse Absonderlichkeiten abgestreift: der Geist des Einzelnen musste sich dem Volksgeiste fügen, \sed{{\textbar}{\textbar}40{\textbar}{\textbar}}\phantomsection\label{sp.40} um sich mit ihm zu verständigen. Noch am \update{freiesten}{Freiesten} mögen sich Gedanke und Rede bei begabten Kindern gestalten, und der Genialität mag es gelingen, etwas von jener Freiheit zu behaupten oder zurückzuerobern; die grosse Masse aber schlendert auf dem Wege der Gewohnheit fort. Man kann diesen Weg glatt und fest treten, – das thuen wir Alle. Man kann ihn verengen oder verbreitern, – verlassen aber kann man ihn nicht. Es ist mit ihm wie mit jenen Richtsteigen, jenen wilden Wegen, die quer durch die Wiesen und Wälder führen: seit Jahrhunderten sind sie begangen worden, jeder Fussgänger glaubte in die Stapfen seines Vorgängers zu treten, und doch wich Jeder ein wenig ab, frühere Fussspuren verrasen, neue werden breitgetreten, der Pfad ist nie verlasssen worden, allein er ist heute ganz anders, als er vor Jahrhunderten war.

Wir haben da einen Blick gethan hinüber nach der Sprachgeschichte. Und doch sind wir unsrer Sache näher als es scheint; denn die Gewohn\-\fed{{\textbar}40{\textbar}}\phantomsection\label{fp.40}heit beherrscht die Seele, wie sie von der Seele erzeugt ist. Das Interessante liegt eben darin, dass sie in den verschiedenen Sprachen so verschiedene Wege eingeschlagen hat, scheinbar launenhaft, zufällig, denn die Gründe, warum sie gerade hier so, dort anders verfahren ist, werden meist unenthüllt bleiben. Genug vorerst, wenn wir ihren Launen verständnissvoll entgegenkommen.

Man sagt, jede Sprache, die wir uns aneignen, eröffne uns eine neue Welt. In der That ist es immer die alte Welt, die wir \update{sehen;}{sehen:} aber wir erblicken sie mit anderen Augen, in anderer Beleuchtung. Darum fallen uns jetzt Dinge auf, die wir vorher nicht sahen, entschwinden andere, die wir zu sehen gewohnt waren, unsern Blicken, und die Dinge scheinen sich nach anderen Gesetzen mit anderen Banden zu verknüpfen, als vordem. Darum scheint uns nun die Welt neu. Ein Geograph hat gesagt: \update{terram}{Terram} mente peragro; der Sprachforscher aber darf von sich sagen: \update{mentes}{Mentes} mente peragro. Wem es gegeben ist, sich in Anderer Seelen hinein zu versenken, der mag ähnlich wechselreicher Schauspiele geniessen wie der Erdumsegler; denn jede Seele baut sich ihre eigene Welt auf, eine Welt weit oder eng, geordnet oder wüst, bunt und belebt oder fahl und starr.

Schon das ist interessant, wie der sprachschaffende Geist sich beholfen hat, um den Dingen Namen zu geben. Jede neue Vorstellung setzt ihm eine neue Aufgabe: wie wird er sie lösen? Nur beispielsweise sollen hier einige der sich bietenden Mittel aufgeführt werden.

1. Das Naivste ist es gewiss, wenn das Ding schlechtweg nach einem anderen, ihm ähnlichen benannt wird, – die Ähnlichkeit mag dem gebildeten Geiste schwer genug einleuchten. Der Fleischer nennt den blätterförmigen Magen der Wiederkäuer den \so{Kalender}. Der Vergleich mit einem Buche ist recht treffend, aber es muss nun gerade das Buch sein, worin der gemeine Mann am meisten liest. Schwerer wird es einleuchten, warum die Bergleute \sed{{\textbar}{\textbar}41{\textbar}{\textbar}}\phantomsection\label{sp.41} eine Art Karren den \so{Hund}, die Fuhrleute eine gewisse Vorrichtung an den Lastwagen den \so{Hasen}, Maurer und Zimmerer die vierbeinigen Gestelle \so{Böcke} genannt haben. Kaum besser ist es, wenn die Zeichner ihren Pantographen mit dem \so{Storchschnabel} vergleichen; die Franzosen nennen ihn, den beweglichen, nachbildenden, mit doppelsinnigem Witze den \so{Affen}, \textit{le singe}. Schulmeistern, Censuren ertheilen soll man aber hier sowenig wie anderwärts, wo es sich um freie Erzeugnisse des Mutterwitzes handelt. Das \fed{{\textbar}41{\textbar}}\phantomsection\label{fp.41} jedoch mag hervorgehoben werden, wie gern wir unsre Vergleiche der Thierwelt entlehnen. Von den Schimpf- und Kosenamen will ich schweigen; Erwähnung verdienen aber noch aus der Schaar der Geräthe der \so{Fuchsschwanz} des Tischlers, der \so{Rattenschwanz} des Schlossers, der \so{Schwanenhals} des Fuchsfängers, das \so{Kuhbein} des Soldaten, der \so{Fliegenkopf} und die \so{Gänsefüsschen} des Setzers, der \so{Hahn} am Gewehre, die \so{Eselsohren} in den Büchern, die \so{Ochsenaugen}, \textit{yeux de boeuf}, an der Laterne, und der \so{Esel}, auf dem der Schreiber vor seinem Pulte reitet. Den Leichdorn nennen wir auch \so{Hühnerauge}, \sed{ebenso die Magyaren: \textit{tyúkszem},} der Holländer \so{Elsterauge}, \textit{eksteroog}, der Mandschu \so{Fischauge}, \textit{nimaha} \update{\textit{yasa}.}{\textit{yasa},} \sed{ebenso der Japaner: \corr{1901}{\textit{iwo-no-me},}{iwo-no-me,} dagegen der Lette: \so{Froschauge}, \textit{wardazs}.} Theile des menschlichen oder thierischen Körpers werden gern auf Theile anderer Gegenstände übertragen: der Berg hat \so{Fuss} und \so{Rücken}, in Spanien, wo Glimmerschieferfelsen die Höhen krönen, auch \so{Backenzähne}, \textit{muelas}; der Hobel hat \so{Nase} und \so{Mund}, der Topf \so{Bauch} und \so{Schnauze}, der Hebel \so{Arme}, die Waage eine \so{Zunge}. \so{Augen} zählt man im \update{Karten\-spiele, – auf der Suppe lässt man sie ungezählt.}{Karten\-spiele; der Japaner zählt sie in seinem Brettspiele Go: da sind es die eroberten freien Punkte.} Der Türke nennt den feinsten Tabak, der in die Mitte der Kiste verpackt zu sein pflegt, \textit{gyöbek}, den \so{Nabel}, daher angeblich \textit{Dubec}. – Hier überall zeigt sich eine vorwiegend phantastische Richtung des Geistes, der sich bei der Benennung der Dinge lieber an noch so entfernte Ähnlichkeiten, lieber an den Schein hält, als an das Wesen, und der dann wohl wieder, Mythen schaffend, dem Scheine ein Wesen, dem Namen eine Geschichte andichtet.

\largerpage[1]2. In den bisherigen Beispielen wurde ohne Weiteres eine Vorstellung auf die andere übertragen. Bedächtiger ist man da verfahren, wo man sich eines unterscheidenden Zusatzes bedient hat. Als die Römer den Elephanten kennen lernten, nannten sie ihn den libyschen Ochsen, \textit{bos libycus}. Die Algonkinvölker haben das Pferd den grossen Hund, \textit{mistātim}, getauft, die Chinesen der Hafenplätze in ihrem Pitchen-Englisch das Klavier: den Singsangkasten, \textit{singsongbox}. Deutsche Wörter wie Meerschwein, Heupferd, Seehund, Hirschkäfer, Blumenkelch, Bierhobel, Kielschwein, Kehreule, Wetterhahn, Windhose u.~s.~w. gehören hierher, so verschieden sich sonst die Glieder der Zusammensetzung zu einander verhalten.

Die Namen für nahe Verwandtschaftsgrade und für dienstliche Verhältnisse \sed{{\textbar}{\textbar}42{\textbar}{\textbar}}\phantomsection\label{sp.42} werden von manchen Völkern gern auch auf Unbelebtes übertragen. Wir selbst haben Wörter gebildet wie Vater- und Mutterschraube, \fed{{\textbar}42{\textbar}}\phantomsection\label{fp.42} Schwesterstadt, Stiefelknecht; den Tisch neben dem Bette nennen wir wohl den Kammerdiener, wie der Tischler das Stützgestell neben der Hobelbank den Gehülfen nennt. \textit{Ānaq pānah}, Kind des Bogens, heisst bei den Malaien der Pfeil, \textit{ānaq līdah}, Kind der Zunge, das Zäpfchen. Auch die Siamesen bezeichnen den Pfeil als Kind des Bogens, \textit{luk šōr}; ihren Hauptstrom nennen sie: Mutter des Wassers, \textit{mē nã} (Menam). In neuchinesischen Zusammensetzungen bedeutet \update{\textit{ts\`{ï}},}{\textit{tsï},} Sohn, Kind: das Kleine; es ist eine Art Diminutivsuffix, das oft die einfachen, einsylbigen Substantive verdrängt hat und so selbst nachgerade als blosses Substantivzeichen dient: statt \textit{tšuāng}, Säule, sagt man \update{\textit{tšuāng-ts\`{ï}},}{\textit{tšuāng-tsï},} Säulenkind, Säulchen = Säule. Sinnig nennen die Chinesen den grämlichen, kopfschüttelnden Bären \textit{šān-laò}, den Bergalten, und die neckische, geschwätzige Schwalbe, das Himmelsmädchen, \update{t’iēn-ni\`{ü}.}{\textit{t’iēn-niü}}. So eng verwandt sind Witz und Poesie der Wortschöpfung, – ein fruchtbares Elternpaar!

\sed{Auch hier spielen natürlich die Körpertheile eine wichtige Rolle. Der Berg hat Rücken, Fuss, wohl auch Hörner und Nasen. Was wir Mündung, \textit{–münde} eines Flusses nennen, heisst bei den Romanen kurzweg der Mund. Weit verbreitet ist es, die Quelle als Auge des Wassers zu beschreiben. So thuen es die Semiten, so die Siamesen: \textit{tā nām}, so die Tibeter: \textit{ču-mig}, so die Malaien: \textit{māta āyer}, die Fidschiinsulaner: \textit{mata ni wai}, – und dann wieder in Afrika die Saho: \textit{lai-t inti}, die Nubier: \textit{essi-n míssi}, die \corr{1901}{Temne:}{Japaner} \textit{ra-for ra m’antr} u.~s.~w. Der Vergleich muss doch dem naiven Geist recht nahe liegen. Schön ist der malaische Name der Sonne: \textit{māta hāri}, Auge des Tages.}

3. Daran reihen sich nun vergleichende Redensarten und redensartliche Vergleiche: stumm wie ein Fisch, arm wie eine Kirchenmaus, reissen wie Schafleder, ein Gesicht wie acht Tage Regenwetter, oder als hätten Einem die Hühner die Butter vom Brode \update{gefressen.}{weggefressen.} Bekanntlich sind viele dieser Vergleiche zu festen Zusammensetzungen geworden: blitzblau, kohlraben-pechschwarz, brettnageldumm, \update{Affenliebe;}{Affenliebe,} und ein Wetter, „bei dem man keinen Hund auf die Strasse jagen mag“, hat man ein Hundewetter genannt; doch mag hier der Zusatz Hund blos schimpfende Bedeutung haben, wie ein anderer Thiername in einer noch derberen Bezeichnung schlechten Wetters.

Gebilde wie die eben aufgeführten liegen so zu sagen auf der Oberfläche. Sie sind völlig frei, werden täglich gemacht, und nicht allemal ist es leicht einzusehen, wie sich „Verdienst und Glück verkettet“ haben, um einzelne dieser Schöpfungen in allgemeine Aufnahme zu bringen. Was uns aber hier vor Allem interessirt, ist dies: weil sie frei sind, darum fallen die in ihnen enthaltenen Vergleiche ohne Weiteres auf, – sie liegen eben zu Tage.

4. Jetzt steigen wir eine Schicht tiefer. Die grosse Mehrzahl unsrer Wörter \sed{{\textbar}{\textbar}43{\textbar}{\textbar}}\phantomsection\label{sp.43} für mehr geistige Vorstellungen und Begriffe ist durch Übertragung von Körperlichem gebildet. Im Geiste stelle ich etwas vor mich hin, dass ich es betrachten kann, – ich \so{stelle es mir vor}. Es steht mir, dem Betrachtenden, gegenüber, – es ist mein \so{Gegenstand}. Im Geiste erfasse, beherrsche ich diesen Gegenstand, – ich \so{begreife} ihn. \fed{{\textbar}43{\textbar}}\phantomsection\label{fp.43} Gar nichts Ungewöhnliches sind Sätze wie die folgenden: „A. \so{fuhr auf}, machte dem B. die \so{bittersten Vorwürfe}, \so{widerlegte} seine \so{Einwendungen} mit den \so{gewichtigsten Gründen}“ u.~s.~w. Wenn man, z.~B. in einem Zeitungsartikel, alle bildlichen Ausdrücke dieser Art unterstreicht, so wird man über das Ergebniss erstaunen; denn in der Regel fällt uns hier das Bildliche gar nicht mehr auf: es liegt unter der Oberfläche.

Und nun beachte man, woher Alles die Bilder entlehnt sind. Es handle sich um Gemüthszustände: da leiht die Musik die Stimmung, – das Wetter Wärme, Kälte, Wetterwendigkeit, Schwüle, Trockenheit, – das Licht Klarheit, Rosenfarbe, Betrübniss, – der Stoff Festigkeit, Härte, Schwere und ihr Gegentheil, das Kochen, Gähren, Aufbrausen u.~s.~w., – der Geschmack Bitterkeit, Säure, Herbheit. Seltsam auch, wie die verschiedenartigsten, im Grunde widersprechendsten Prädicate zusammen an eine Deichsel gespannt werden können. Erzählte Jemand, ein Vortrag sei trocken, recht wässerig, dabei durchaus nicht fliessend gewesen: so würde sich Mancher besinnen, ob hier Witz oder Dummheit die Worte zusammengereiht habe, und Mancher würde wohl gar nichts Auffälliges dabei bemerken.

Vom Sprachforscher ist es nun aber zu verlangen, dass er auf alle solche Erscheinungen seine Sinnen schärfe, und dazu bietet ihm die Muttersprache und der tägliche Verkehr Gelegenheit die Hülle und Fülle. Auch geht ihn keineswegs nur das Richtige, Erlaubte an, sondern auch das Fehlerhafte ist für ihn wichtig, die \textit{lapsus linguae}, \textit{calami}, \textit{mentis}, besser: die Denkfehler, auf denen die Rede- und Schreibfehler beruhen.

Im Verhältniss zur Rede geht das Denken so schnell von statten, dass der Gedanke den Worten nicht Schritt auf Schritt folgen mag, sondern im besten Falle um sie herum schweift, wie ein flinker Hund um seinen Herrn: jetzt eine Strecke voraus, jetzt ein Stück zurück, jetzt links oder rechts querfeldein. Nur eben ist die Rede nicht Herrin sondern Dienerin, und es kann ihr geschehen, dass sie durch die Kreuz- und Quersprünge des Gedankens vom Wege abgelenkt wird. So geschieht es denn, dass man aus der Construction fällt, sich verfitzt, vom Gegenstande abkommt, sich wiederholt oder vorgreift. Von Anfang an war der Rede Ziel und Marschroute vorgezeichnet; der Geist aber, der sie umschwärmt, lässt sich nur zu gern in seinem Laufe von Ursachen leiten, statt von Zwecken. Jene Ursachen sind Ideenassociationen: eine Vor\-\fed{{\textbar}44{\textbar}}\phantomsection\label{fp.44}stellung führt zur anderen, und man geräth „vom Hundertsten auf’s \update{Tausendste.“}{Tausendste“.} Gerade hierin äussert sich die geistige Eigenart und die jeweilige Stimmung der Menschen. Unsre Seele gleicht \sed{{\textbar}{\textbar}44{\textbar}{\textbar}}\phantomsection\label{sp.44} nicht ganz dem musikalischen Instrumente, bei dem jede Saite gleichstark mithallt, wenn ein verwandter Ton angeschlagen wird: das Eine findet bei ihr lebhafteren, das Andere matteren Anklang, Manches lässt sie ganz gleichgültig; – dort genügt die leiseste Hindeutung, eine zufällige Berührung, um sie zu erregen, hier verhält sie sich wie ein Tauber, dem man umsonst in die Ohren schreien mag. Es giebt Anziehungskräfte, denen der sich selbst überlassene Geist, sobald er in ihren Bereich kommt, folgen muss, wie der schwimmende Magnet dem Eisen. Auch in der Verkettung der Vorstellungen und Gedanken waltet Sympathie, Apathie, vielleicht auch Antipathie.

Was nun von den einzelnen Menschen, das gilt auch von den Völkern, was von den einzelnen Reden, das gilt auch von den Sprachen: hier sind diese, dort jene Associationen bevorzugt; andere Ideen verknüpft der mythenbildende Arier mit den Erscheinungen der Natur, andere der nüchtern verständige Chinese. Und weiter: was von den Dingen, den Vorstellungen und Begriffen gilt, das gilt auch von ihren sprachlichen Ausdrücken, den Wörtern und Formen.

Auch dem Vergleichbares finden wir in nächster Nähe. Geschlecht, Alter, Beruf des Menschen, die Umgebung, in die er gestellt ist, beeinflussen seine Denkgewohnheiten. Wo der Volkshumor die Berufsarten zur Zielscheibe wählt, da hält er sich mit Vorliebe an ihre geistigen Einseitigkeiten und deren lächerliche Wirkungen. Auf niederer Culturstufe aber vertheilen sich die Berufsarten völkerweise: wir haben Jäger-, Fischer-, Hirtenvölker u.~s.~w. – genug bei ihnen, wenn Priester und Zauberer und allenfalls noch Schmiede oder Töpfer besondere Classen bilden. Bei uns mildern sich doch die Einseitigkeiten durch den Verkehr der verschiedenen Stände und durch die Schulbildung; das fehlt aber dort, und so begreifen wir es, wenn ihre Sprachen den bevorzugten Denkrichtungen gefolgt, das heisst mehr oder weniger einseitig ausgebildet sind.

Ich erwähnte vorhin die Sprachfehler, die wir selbst oft genug in der Übereilung machen, und die wir in den Reden der Kinder und Ungebildeten hören: falsche Constructionen, falsche Formbildungen u.~s.~w. Der Norddeutsche verwechselt Dativ und Accusativ, in Österreich hört \fed{{\textbar}45{\textbar}}\phantomsection\label{fp.45} man: gefurchten, gewunschen, statt gefürchtet, gewünscht, – in Schwaben: gedenkt, statt gedacht, – in Hannover gelegentlich: verstochen, statt versteckt; bei Kindern geschieht dergleichen allerwärts, und überall reden wir von grammatischen Fehlern. Allein jene grammatischen Veränderungen, die keiner Sprache erspart bleiben, was waren sie ursprünglich anderes, als grammatische Fehler? \update{Unsre}{Unsere} Altvordern unterschieden noch scharf zwischen transitiven und intransitiven Verben; wie „gesetzt“ und „gesessen“, so standen bei ihnen einander gegenüber: „gebrannt“ und „gebronnen“, „verderbt“ und „verdorben“. Heute sagt und schreibt man unbedenklich: „er frug“, statt: er fragte. Der aber zuerst so gesagt hat, der hat falsch gesprochen, gerade so falsch, wie ein \update{Kind}{Kind,} das etwa sagt: Ich habe \sed{{\textbar}{\textbar}45{\textbar}{\textbar}}\phantomsection\label{sp.45} die Tasse ausgetrinkt. Man versteht, wie das Kind dazu kommt, die leicht bildbare Form der schwachen Verba über Gebühr auszudehnen; und wiederum versteht man es, warum etwa ein ungebildeter Norddeutscher sagt „gewunken“ statt „gewinkt“. In beiden Fällen hat die Analogie gewirkt. Schwerer begreift man, warum ein Theil der Fehler nachträglich durch den allgemeinen Gebrauch geheiligt wird, der andere nicht. Und doch bietet auch hierfür eine Jedem zugängliche Erfahrung Anhalt. Ich erinnere an den Ausdruck „bei Muttern“, der sich während der Kriegsjahre 1870–1871 durch unser Heer und dann weiter durch Deutschland verbreitete. Bismarck’s „Wurschtigkeit“ ist auf dem besten Wege, in den Wortschatz der Sprache aufgenommen zu werden; man liest es immer wieder in den Zeitungen, – wer weiss, wie bald es hier die entschuldigenden Anführungsstriche ablegen wird? Dort war es eine grosse Zeit, hier war es ein grosser Mann, der das Gassenmässige salonfähig machte. Die Geschichte der geflügelten Worte ist für unsre Wissenschaft gar bedeutsam; ihr Motto könnte das \update{Sprüchwort}{Sprichwort} sein: Kleine Ursachen, grosse Wirkungen. Es sind von grossen Männern grosse Worte gesprochen worden, die nur bei Wenigen Wiederklang gefunden haben; und Plattheiten, Alltäglichkeiten konnten zur Verewigung gelangen.

Je enger, geschlossener ein Kreis ist, desto leichter wird in ihm der Einfall eines Einzelnen zum sprachlichen Gemeingute werden. Familien und Clubs haben ihre stehenden Redensarten und Witze, die Studenten und die verschiedenen Classen der fahrenden Leute, von den Schauspielern, Handlungsreisenden und Handwerksburschen bis hinab zu den Gaunern haben ihre Standessprachen, ihren \textit{slang} oder \textit{argot}. Grosse \fed{{\textbar}46{\textbar}}\phantomsection\label{fp.46} Städte sind fruchtbare Brutstätten neuer Ausdrücke; denn je dichter die Menschen beisammenwohnen, desto mächtiger und schneller wirkt die Ansteckung, auch die geistige.

Das Schaffen führt naturgemäss zum Abschaffen: der neue Ausdruck, wenn er nicht auch einem neuen Begriffe dient, kann den früher üblichen verdrängen. Daneben giebt es aber noch ein wirklich conventionelles Abschaffen ganz anderer Art: ein harmloses Wort wird euphemistisch oder scherzweise statt eines anstössigen gebraucht; alle Welt kennt diesen Gebrauch, und nun gilt auch jenes Wort in der gesitteten Sprache für verpönt. Die englische Zimpferlichkeit hat einen wahren \textit{index verborum prohibitorum} aufgestellt. Nicht viel anders ist es, wenn bei den Polynesiern der Machtspruch eines Häuptlings oder Priesters ein beliebiges Wort ausser Umlauf setzen, und statt dessen ein anderes einführen kann.

Besondere Beachtung verdient die ungekünstelte Sprache des gemeinen Mannes, die, gerade weil sie so unbewusst natürlich aus der Seele hervorbricht, dem Beobachter eine Menge Geheimnisse verräth. Warum drückt sich der Mann jetzt so aus, jetzt anders? Warum sagt er jetzt: ein Haus bewohnen, ein Zimmer \sed{{\textbar}{\textbar}46{\textbar}{\textbar}}\phantomsection\label{sp.46} betreten, einen Baum erklettern, – und jetzt wieder: in einem Hause wohnen, in ein Zimmer treten, auf einen Baum klettern? Was sollen jene \update{ein\-geschalteten}{ein\-geschaltenen} Hülfswörter und Redensarten, die seiner Sprache den Eindruck breiter Gemüthlichkeit, wohl auch träger Unschlüssigkeit verleihen? Wo die Wortstellung Freiheiten gestattet, was bestimmt ihn in seiner Wahl? Das Natürliche ist immer feiner als das Gemachte, die gewachsene Blume auf der Wiese ist feiner als die wächserne im Ladenfenster; und wer seine Muttersprache verständnissvoll handhaben will, der nehme auch einen Lehrkursus beim Kleinbürger, beim Bauern und Tagelöhner. Luther und Lessing sind des Zeugen. \sed{Der Sprachforscher aber muss sich darin üben, jene Feinheiten zu erklären, das heisst ihre Unterschiede in Worten ausdrücken und wo möglich den Zusammenhang zwischen den Ausdrücken und ihren Bedeutungen nachzuweisen. Offenbar fängt er hier mit am Besten bei den Erscheinungen seiner Muttersprache an, die er ja am Genauesten kennen und am Richtigsten beurtheilen wird.}

Bei uns Culturmenschen wirken jeder sprachlichen Neuerung so uns soviele \retro{erhaltende}{erhaltene} Kräfte entgegen: die Literatur, die uns an das Altberechtigte erinnert, Schule, Kirche, Behörden, die gewissen classischen Sprachmustern folgen, endlich ein grosser Theil unsrer Landsleute selbst, der nicht so gutwillig das Altgewohnte für ein Neues hergiebt. Nun denke man sich aber einen kleinen, vereinzelten Stamm Wilder, – man sollte meinen, da müsste die Sprache sich unglaublich schnell verändern. Das mag stellenweise der Fall sein, ist es aber gewiss nicht überall, \fed{{\textbar}47{\textbar}}\phantomsection\label{fp.47} denn hier widerstrebt wohl den Neuerungen ein sehr starkes Beharrungsvermögen, eine \textit{vis inertiae}.

Auch in unserer Wissenschaft gilt das bewährte Wort: Verstehe dich selbst, so verstehst du Andere. Die Erfahrung lehrt aber, dass nichts lebhafter zur Selbstprüfung anregt, als der Umgang mit vielerlei Menschen. Wir beobachten, wie verschieden sie sich unter den gleichen Umständen verhalten, und nun versenken wir uns in ihre Charaktere, versetzen uns in ihre Lagen, lernen deductiv zu beurtheilen, wie ein Jeder behandelt sein will, und inductiv aus seinen Äusserungen auf sein Wesen zu schliessen; indem wir ihn an uns messen, messen wir uns an ihm. Zu Anfang dieses Capitels habe ich an Beispielen gezeigt, aus wie verschiedenen Kreisen wissenschaftlichen Berufes unsere namhaftesten Meister hervorgegangen sind. Wie billig, lieferten dabei die Philologen einige der besten Namen. Einen anderen Theil aber stellen Jene, die von \update{Berufs wegen}{Berufswegen} praktische Menschenkenner sind, und es möchte schon der Mühe lohnen, diese Berufe mit den Richtungen der Sprachwissenschaft zu vergleichen, in denen sie sich besonders hervorgethan. Wissenschaftlicher Psycholog von Fach ist, dass ich wüsste, nur Einer, \update{Heinrich}{\textsc{Hajim}} \textsc{Steinthal}.

\begin{styleAnmerk}
\sed{\so{Anmerkung}. Empfehlung verdient das feinsinnige Buch von \textsc{Ph. Wegener}: Untersuchungen über die Grundfragen des Sprachlebens. Halle 1885.}
\end{styleAnmerk}

\sed{{\textbar}{\textbar}47{\textbar}{\textbar}}\phantomsection\label{sp.47}

\cohead{§. 4. c. Logische Schulung.}
\pdfbookmark[2]{§. 4. c. Logische Schulung.}{I.V.4}
\subsection*{§. 4.}\phantomsection\label{I.V.4}
\subsection*{c. Logische Schulung.}

Dass \update{unsre}{unsere} Wissenschaft wie jede andere einen logisch geübten Geist voraussetzt, braucht kaum erst erwähnt zu werden. Es ist aber doch ein Unterschied zwischen der theoretischen Beschäftigung mit der Logik und ihrer praktischen Verwerthung. Bekanntlich gehören an vielen Hochschulen die Vorlesungen über diese Wissenschaft zu denen, welche mehr belegt als besucht \update{werden,}{werden;} und ein Gelehrter, wenn er nicht gerade seines Zeichens Philosoph ist, mag in seinem Fache sehr Hervorragendes leisten, ohne je ein logisches Colleg gehört oder ein logisches Lehrbuch in der Hand gehabt zu haben. Logisch denkt er darum doch: das beweisen eben seine Leistungen.

Die Sache ist die, dass uns jede Wissenschaft, ja eigentlich das ganze Leben logisch schult, nur freilich meist mehr oder minder einseitig. Der Arzt fragt nach den Ursachen der Störungen und dann nach den Mitteln zur Heilung; der Politiker und Vewaltungsmann operirt so \fed{{\textbar}48{\textbar}}\phantomsection\label{fp.48} ziemlich nach den gleichen Denkgesetzen. Der Jurist legt das Gesetz nach sprachlichen und logischen Grundsätzen aus und subsumirt ihm die Thatsachen. Anders wieder der Mathematiker, Chemiker, Mechaniker oder Physiolog. Wie verhält es sich mit dem Sprachforscher?

Seine Wissenschaft ist überaus vielseitig, auch in Rücksicht auf die logischen Operationen, die in ihr vorherrschen, und eine gewisse Arbeitstheilung ist daher gerade in ihr wohl berechtigt. Der scharfsinnige Etymolog, der Meister im Vergleichen von Lauten und Formen mag ein sehr schlechter Syntaktiker \update{sein,}{sein;} und jener, der es versteht, dem Sprachgebrauche seine letzten Feinheiten abzulauschen, ist darum noch nicht ohne Weiteres befähigt, das Ganze zu einem wissenschaftlichen grammatischen Systeme aufzubauen.

Wir können und dürfen nicht ein Jeder Alles treiben, aber wir müssen streben einander zu verstehen. Darum müssen wir uns gegenseitig in unseren Werkstätten besuchen, die Arbeit des Nachbars beobachten, uns den Mitgenuss an ihren Erzeugnissen sichern. Offenbar setzt dies eine möglichst allseitige logische Bildung voraus, und offenbar ist eine solche am sichersten durch systematisches Studium zu gewinnen.

Man bedenke auch dies: Jedes Wort und jede Form einer Sprache deckt einen bestimmten Vorstellungskreis, der wissenschaftlich beschrieben, dessen Mittelpunkt festgestellt werden will. Wird statt dessen weiter nichts geboten, als eine Reihe möglicher Übersetzungen, so mag dies zwar für den Schriftsteller und für den Leser recht bequem sein, ist aber doch nur ein unwissenschaftlicher Nothbehelf. Der Aufgabe einer zugleich zutreffenden und verständlichen Definition vermag nur ein logisch geschulter Geist gerecht zu werden.

\sed{{\textbar}{\textbar}48{\textbar}{\textbar}}\phantomsection\label{sp.48}

Wäre das Alles, so wäre es schon Gewinnes genug. Allein es ist nur das Wenigste. In der That stellt die Logik Anforderungen nicht nur an den Sprachforscher, sondern auch an die Sprache selbst. So verschieden die Sprachen sind, so giebt es doch allgemeine Denkkategorien, die sie alle ausdrücken müssen, wenn sich auch der Ausdruck zu ihnen verhalten mag, wie etwa die Formen der organischen Natur in ihrer unendlichen Mannichfaltigkeit zu jenen geometrischen Figuren, mit denen wir sie vergleichend beschreiben. Hier ist fachlich logische Schulung nöthig, wäre es auch nur um die Vorurtheile zu überwinden, die wir von der lateinischen Grammatik her an anders geartete Sprachen zu bringen pflegen.

\fed{{\textbar}49{\textbar}}\phantomsection\label{fp.49}

Die logische Arbeit des Sprachforschers ist vorwiegend inductiv. Es gilt den Erscheinungen ihre Gesetze abzulauschen; darum gilt es, die Erscheinungen als Beispiele zu sammeln, die gesammelten zu sichten, das ihnen Gemeinsame zu erkennen, die erlangte Erkenntniss in scharf und klar ausgesprochenen Lehrsätzen zu formuliren, endlich die Lehrsätze zu einem wohlgefügten Lehrgebäude organisch zu vereinigen. Alles dies verlangt einen logisch gebildeten, das Letzterwähnte sogar einen philosophisch beanlagten Kopf. Das Sichten und Wählen der Beispiele aber setzt besonderen Takt, und dieser wieder einige Übung voraus. Denn das Erfahrungsmaterial ist nicht gleichwerthig an Beweiskraft. Ein Gesetz kann die Wirkung des anderen einschränken, verdunkeln, aufheben. Darum sind diejenigen Beispiele vorzuziehen, wo das zu erweisende Gesetz sich am Unzweifelhaftesten äussert, d.~h. wo es sich am Ungestörtesten äussern konnte, am Klarsten äussern musste. Hier zeigt sich der Werth der Antithese. In zwei Fällen, \textit{A} und \textit{B}, stimmt Alles überein bis auf die zwei Punkte \textit{a} und \textit{b}. Worin besteht also der Unterschied? Worin zeigt sich die Übereinstimmung von \textit{A} mit \textit{C}, \textit{D}, \textit{E}, die alle auch \textit{a} aufweisen, und von \textit{B} mit \textit{F}, \textit{G}, \textit{H}, \textit{I}, die alle \textit{b} enthalten? Die Theorie ist so einfach wie nur möglich, aber die praktische Geschicklichkeit will erworben sein.

\cohead{§. 5. d. Allgemein sprachwissenschaftliche Schulung.}
\pdfbookmark[2]{§. 5. d. Allgemein sprachwissenschaftliche Schulung.}{I.V.5}
\subsection*{§. 5.}\phantomsection\label{I.V.5}
\subsection*{d. Allgemein sprachwissenschaftliche Schulung.}

Sprachtalent und Sprachwissenschaft sind sehr verschiedene Dinge. Das Talent, das heisst die Fähigkeit zu rascher und sicherer Erlernung, kann mit dem Triebe und der Fähigkeit zum wissenschaftlichen Begreifen gepaart sein, aber es ist es nicht immer. Wir wissen von Männern, die in Dutzenden von Sprachen mit Leichtigkeit lasen, gewandt componirten, wohl auch, Dank einem feinen Ohre und beweglichen \update{Sprach\-organen}{Sprach\-organen,} meisterlich plauderten, und die doch über alles das, was sie übten, weder sich noch Anderen Rechenschaft zu geben wussten. \sed{Der berühmte Cardinal \textsc{Mezzofanti} war ein solcher Virtuos. Er soll} {\textbar}{\textbar}49{\textbar}{\textbar}\phantomsection\label{sp.49} \sed{schliesslich nahe an sechszig Sprachen geläufig gesprochen haben, war aber doch bescheiden genug, \textsc{Zumpt} um seine lateinische Grammatik zu beneiden: zu dergleichen reichte sein Verstand nicht aus, das fühlte er.} Und umgekehrt hören wir von verdienten Sprachforschern, deren Wissen sich auf ein sehr enges Gebiet beschränkte. Es waren Specialisten, vielleicht Meister in ihrer Specialität; wo sie sich aber aus ihrer Sphäre heraus auf das weite Gebiet der allgemeinen Sprachwissenschaft gewagt haben, da wurden sie im günstigsten Falle Dogmatiker und Schema\-\fed{{\textbar}50{\textbar}}\phantomsection\label{fp.50}tiker nach Art jener älteren Sprachphilosophen, wohl auch Essayisten, die allerhand Lesefrüchte aus zweiter und dritter Hand mehr oder minder geschickt zur Schau ausstellten. Von ihren Büchern gilt, was mir mein verewigter Vater einmal sagte: „Während Du ein solches liesest, kannst Du eine neue Sprache hinzulernen, und davon hast du mehr!“ \sed{Er konnte sich in solchen Dingen wohl ein Urtheil zutrauen. Er hatte in die achtzig Sprachen getrieben, und das hiess bei ihm, wo immer es möglich war, soviel wie erlebt. Daneben war er in jener Literatur wohl belesen und wusste, wohin es führt, wenn man verallgemeinernd von der Sprache redet, ehe man sich in der weiten Sprachenwelt umgeschaut hat.}

In der That müsste es seltsam um unsre Wissenschaft stehen, wenn sie nicht gleich den übrigen in erster Reihe Sachkenntniss voraussetzte. Das wird auch wohl allgemein anerkannt; nur über den Umfang des erforderlichen positiven Wissens bestehen Zweifel. Man verlangt wohl die genauesten Kenntnisse im Bereiche des eigenen engeren Faches, meint aber im Übrigen mit einem flüchtig orientirenden Überblicke genug zu thun, und nun hält man sich gutgläubig an Führer, die wohl mehr flüchtig als orientirend sind. Was man da lernen kann, davon habe ich früher einige Beispiele mitgetheilt (S.~29), die meines Vaters Urtheil bestätigen dürften. \sed{Ich habe mich gleichwohl seitdem etwas weiter in diesem Zweige unserer Literatur umgesehen und viel Geistreiches, Anregendes darin gefunden, – neben Vielem, was geringeres Lob verdient\footnote{\sed{Beiderlei allgemeinen Betrachtungen, zuweilen solchen, die man für Errungenschaften der jüngsten Zeit zu halten gewohnt war, kann man auch in viel älteren Büchern begegnen. Eine Sammlung solcher Aussprüche würde manchen Prioritätszweifel wo nicht lösen, so doch beruhigen.}}. Weittragende, fruchtbare Gedanken erwachsen oft auf einem sehr engen Beobachtungsgebiete. Sich als gemeingültig erweisen können sie aber nur auf einem sehr weiten. Was vor zwei bis drei Menschenaltern die Sprachphilosophen gefehlt haben, sollte uns noch heute als warnendes Beispiel dienen. Nach allgemeinen Grundsätzen haben wir zu streben nach wie vor, und auch dies Buch will für seinen Theil dahin wirken. Aber mehr als je haben wir uns vor verfrühten Verallgemeinerungen zu hüten. Respect vor den Thatsachen, Skepsis den Theorien gegenüber: das scheint mir der beste Wahlspruch einer jungen Wissenschaft.}

\sed{{\textbar}{\textbar}50{\textbar}{\textbar}}\phantomsection\label{sp.50}

Dass in Sachen der allgemeinen Sprachwissenschaft nur der zu urtheilen vermag, der in möglichst vielerlei Formen des menschlichen Sprachbaues einen tieferen Einblick gethan hat, leuchtet ohne Weiteres ein. Erlebt aber will es werden, wie gar oft die entlegensten Sprachen \update{auf einander}{aufeinander} ein unerwartetes Licht werfen. Hier, in der einen, scheint eine wunderliche Ideenassociation zu herrschen. Die ist aber gar zu wunderlich, und der etymologische Thatbestand ist doch nicht sicher genug, um ohne Weiteres zu überzeugen. Nun findet sich Ähnliches in einem anderen Erdtheile, nur liegt es da ganz zweifellos zu Tage; was uns dort ein tollkühner Sprung schien, ist hier ein ganz einfacher Schritt: sollte es für unsere Altvordern mehr gewesen sein? Oder umgekehrt: je ausschliesslicher sich unser Forschen in einem eng \update{begränzten}{begrenzten} Gebiete bewegt, desto selbstverständlicher erscheint uns Alles, was da gewöhnlich ist. Für selbstverständlich halten heisst aber, auf die Frage nach den Gründen verzichten. Jetzt wagen wir uns in ein fernes Gelände, erfahren, wie dort Alles so ganz anders hergeht, und lernen nun erst das Heimische nach seiner Berechtigung fragen. Ich meinestheils habe auch wohl das erlebt, dass ich voreilig nach Analogien aus fremdartigen Sprachen urtheilte, bis mich weiteres Forschen und Nachdenken belehrte, wie verschieden hüben und drüben die Voraussetzungen lagen. Aber in seiner Art war doch auch dies ein Gewinn, – statt einer Wahrheit hatte ich deren zwei gefunden: diejenige, die ich anfänglich suchte, \fed{{\textbar}51{\textbar}}\phantomsection\label{fp.51} und den Grund meines Irrthums. Alles läuft aber doch schliesslich darauf hinaus, dass unser Gesichtskreis erweitert, unser Blick geschärft wird, und das wird jederlei Sprachforschung, auch der beschränktesten, zu Nutze gereichen. Ich muss immer wieder an den Maler erinnern, wie er von Zeit zu Zeit von seinem Bilde ein Stück zurücktritt, um zu sehen, wie es „fernt“, und wie sicher er dann die berichtigenden Striche einträgt.

Eine oder \update{wo möglich}{womöglich} mehrere Sprachen verschiedenen Baues sollte also jeder Sprachforscher im eigenen Interesse treiben. Wie leicht heutzutage die nöthigen Hilfsmittel erlangbar sind, lehrt jeder linguistische Katalog. Nun kann man aber den Einwand hören: Es fehlt mir an Sprachtalent. Als ob es sich mit dieser Begabung verhielte, wie etwa mit jener zur Mathematik, zur Musik oder zur Malerei. Nicht Jeder musicirt oder malt, und es giebt Völker, die nicht die Finger an ihren Händen zu zählen verstehen. Die sprachliche Befähigung dagegen hat Jeder mindestens einmal, an seiner Muttersprache, bewiesen, und wer bei Zeiten dazu angehalten wird, der lernt auch fremde Sprachen hinzu. In diesem weitesten Sinne, der zugleich der zutreffendste sein möchte, besitzt also Jeder Sprachtalent in höherem oder niederem Grade. Ein einmal vorhandenes Talent ist aber immer durch Übung zu steigern: Crescit eundo. Mit jeder neuen Sprache, die ich mir aneigne, erleichtere und beschleunige ich mir jede spätere Spracherlernung.

\sed{{\textbar}{\textbar}51{\textbar}{\textbar}}\phantomsection\label{sp.51}

Am meisten graut uns wohl vor dem Gedächtnisswerke; die indogermanischen und semitischen Paradigmen, die Unregelmässigkeiten in Declination und Conjugation sind uns noch in schmerzlicher Erinnerung. Uns hat der sprachliche Schulunterricht durch so dichtes Dornengestrüpp geführt, dass Manche die Rosen gar nicht bemerkt haben. Solche Rosen, aber auch solche Dornen gehören nun in der übrigen Sprachenwelt zu den Ausnahmen; wer will, mag sie meiden, – es bleibt ihm noch immer Auswahl genug. \textsc{Friedrich Müller}’s hochverdienstlicher Grundriss wird sich auch hier als Mittel zur ersten Orientirung bewähren. Man bedenke aber, dass dies Buch auch den reicheren und schwierigeren Sprachen selten mehr als zwanzig, meist weniger als zehn Seiten widmen konnte, und dass seine grammatischen Skizzen bei aller Knappheit der Darstellung weder bestimmt noch geeignet sind, ein einzelsprachliches Lehrbuch zu ersetzen. Mit einem Umhernippen an verschiedenen Sprachen \fed{{\textbar}52{\textbar}}\phantomsection\label{fp.52} ist es aber nicht gethan; erst muss man sich die eine oder andere recht gründlich, theoretisch und praktisch, aneignen. \sed{Sprachen lassen sich nicht platonisch lieben, man muss mit und in ihnen gelebt haben, ehe man wagen darf sie zu beurtheilen. Nur wo man nahe verwandte Typen genau kennt, mag man sich mit dem Studium eines tüchtigen Lehrbuches begnügen und es einer sicher ahnenden Phantasie überlassen, die Paragraphen zu beleben. Doch müssen dann jene Typen gut gewählt sein: nicht die schwächsten und ärmsten ihrer Art, noch weniger solche, die fremder, sei es auch veredelnder, Beeinflussung verdächtig sind. Handelte es sich um die uralaltaischen Sprachen, so würde ich dem einfachen, fast dürftigen Mandschu noch immer vor dem in indogermanischer Schule erzogenen Suomi (Finnischen) den Vorzug geben, vor Beiden aber z.~B. dem Jakutischen, weil es reicher als das Mandschu und unverfälschter als das Suomi ist.}

Man halte sich also an solche Sprachen, in denen man auch Texte, und wären es nur Stücke der Bibelübersetzung, erlangen kann, und suche recht schnell durch die Theorie hindurch zur Praxis vorzudringen. Greift man zu einer leichteren Sprache, etwa zur malaischen oder auch einer ihrer formenreicheren Schwestern, so wird man erstaunen, wie schnell man sich in einer so neuen Welt einbürgern kann; man geniesst den leicht erworbenen Besitz und strebt bald nach weiterem. \sed{Da muss ich nun aber aus eigener Erfahrung vor einem nahe liegenden Fehler gegen die geistige Diät warnen. Man möchte am liebsten gleich mehrere Sprachen neben und durcheinander treiben. Dadurch erschwert und verzögert man sich die Arbeit, steckt sich beim Fortschreiten selber den Stock zwischen die Beine. Denn jede dritte Sprache, mit der man sich beschäftigt, verlangsamt die Erlernung der anderen; wende ich hintereinander an zwei Sprachen je ein halbes Jahr, so habe ich weit mehr Wissensgewinn davon, als wenn ich mich ein Jahr lang abwechselnd bald der einen, bald der anderen gewidmet hätte. –}

\sed{{\textbar}{\textbar}52{\textbar}{\textbar}}\phantomsection\label{sp.52}

Lehrreich ist es nun aber auch, zu beobachten, wie die Grammatiker mit mehr oder minderem Glück – sehr oft mit minderem – gerungen haben, den fremdartigen Stoff in die Form einer Sprachlehre zu bringen. Es ist eine lange Scala zwischen jenen verzweifelten Versuchen, die Sprache eines Indianer- oder eines Bantuvolkes in das Prokrustesbett der lateinischen Grammatik zu spannen, und etwa \textsc{Böhtlingk}’s Darstellung des Jakutischen, \textsc{Stoll}’s Arbeiten über die Sprachen der Maya-Familie, \textsc{Lucien Adam}’s meisterhaften grammatischen Extracten oder auch \textsc{Schlegel}’s bescheidenem Buche über die Ewe-Sprache. Dem geschichtlichen Studium der chinesischen Grammatiken glaube ich reichlich soviel an sprachphilosophischen Anregungen wie an einzelsprachlichem Wissen zu verdanken.

\largerpage[1]\update{Einige Bekanntschaft}{Möglichste Bekanntschaft} mit der Methode und den hauptsächlichsten Ergebnissen und \update{Streit\-puncten}{Streit\-punkten} der vergleichenden Indogermanistik darf man wohl von jedem Sprachforscher erwarten. Sie ist doppelt nothwendig für den, der selber Sprachvergleichung treiben will, wäre es auch auf noch so entlegenen Gebieten. Wer diese Richtung unserer Wissenschaft bevorzugt, der findet unübersehbaren Stoff zum Arbeiten, weite Strecken, die noch der Urbarmachung harren.

Der überreichen Literatur über allgemeine Sprachwissenschaft Schritt für Schritt zu folgen, ist, wie angedeutet, Niemandem zuzumuthen, am wenigsten vielleicht dem Fachmanne, der zu eigenem Schaffen Zeit und Sammlung braucht. Ich meinestheils habe die namhafteren Werke dieser Art, soweit sie zu meiner Kenntniss kamen, gelesen und werde diejenigen unter ihnen, die mir empfehlenswerth scheinen, ihres Ortes aufführen. Eine vorzugsweise Aufmerksamkeit habe ich allerdings diesem Zweige der Literatur nicht zugewandt, manche, zumal neuere Erscheinungen mögen mir entgangen sein, und so deute man denn mein etwaiges \fed{{\textbar}53{\textbar}}\phantomsection\label{fp.53} Schweigen nicht ohne Weiteres als ein abfälliges Urtheil.\footnote{Manche Bücher dieser Art, denen ich Genuss und Anregung verdanke, habe ich vor Jahren gelesen, ehe ich noch allgemein sprachwissenschaftliche Collectaneen führte. Vor Allen erwähne ich die von \textsc{A. H. Sayce}: Principles of Comparative Philology und Introduction to the Science of Language. Auch der allverbreiteten Vorlesungen von \textsc{Max Müller} und \textsc{Whitney} muss ich an dieser Stelle gedenken. \sed{Noch andere werden später Erwähnung finden.}} Vorschläge zur Anlage einer linguistischen Privatbibliothek werde ich nicht machen. Eine rechte Bücherei trägt immer das Gepräge ihres Schöpfers.

\cohead{Zusatz.}
\pdfbookmark[2]{Zusatz.}{I.V.6}
\section*{Zusatz.}\phantomsection\label{I.V.zusatz2}

Dem Philologen, der die Literaturdenkmäler der Sprachen auslegen, ihnen ihre sprachlichen Feinheiten abgewinnen soll, muss nothwendigerweise eine Gabe zur Verfügung stehen, die ich dramatischen Instinkt nennen möchte: Phantasie und Menschenkenntniss, die sich in die Situationen und die Stimmungen der \sed{{\textbar}{\textbar}53{\textbar}{\textbar}}\phantomsection\label{sp.53} Leute zu versetzen, aus ihnen die Eigenheiten ihrer Rede zu erklären und aus diesen wieder jene zu erschliessen weiss. Das ist eine künstlerische Begabung, die wohl zum Theile angeboren sein muss, zum Theile aber auch sicherlich durch ästhetische Bildung und Umgang mit Menschen, dann auch durch feinsinnige Commentare entwickelt werden kann. \sed{Man thut wohl, sie zu pflegen; das ästhetische Feingefühl, die Schauungskraft der Phantasie, die Vertrautheit mit der Menschennatur und den geschichtlichen, sittlichen und gesellschaftlichen Mächten haben viel Antheil an der richtigen Erklärung sprachlicher Vorgänge.}

In unserer Wissenschaft, und vermuthlich in jeder anderen gilt dies, dass man sich nicht ungestraft vereinseitigt, und dass kein Ab- und Umweg ungelohnt bleibt. Geschichtlichen, länder- und völkerkundlichen, philosophischen, ästhetischen, auch wohl naturwissenschaftlichen Interessen gebe man getrost ihr Recht: man wird erstaunen, wie oft auch hierbei aus den entlegensten Gegenden befruchtende Strahlen in das eigene Forschungsgebiet fallen. Nur wen das Vielerlei zu leichtfertiger Oberflächlichkeit verführt, nur der hat das Recht und die Pflicht, sich \update{zu vereinseitigen.}{ganz in sein Einzelfach zu vergraben.}

\hrulefill

\sed{Wenn im Folgenden erst von der einzelsprachlichen, dann von der sprachgeschichtlichen Forschung gehandelt wird, so heisst das natürlich nicht, dass ich nur die oder jene Einzelsprache, dies oder jenes Stück Sprachgeschichte bearbeiten will, sondern es ist mir um die Prinzipien dieser Zweige unserer Wissenschaft zu thun, wie sie sich aus der Natur der Sprache an sich zu ergeben scheinen. Diese Prinzipien gehören der allgemeinen Sprachwissenschaft an, können nur ihr angehören, aber sie sind da zu entwickeln, wo sie Anwendung erleiden. Hieraus erklärt sich die scheinbare Inconsequenz in der Anlage meines Buches, auf die einige meiner Recensenten aufmerksam gemacht haben. Ein „System der Sprachwissenschaft“ vorzulegen, wie es wohl auch von mir verlangt worden ist, masse ich mir nicht an; ich meine, unsere Wissenschaft ist hierzu noch zu jung, und doch schon zu alt.}

