\chapter*{Zweites Buch.}
\pdfbookmark[0]{Zweites Buch.}{zweitesbuch}
\pdfbookmark[1]{I. Capitel.}{II.I.Capitel}
\section*{Die einzelsprachliche Forschung.}

\section*{I. Capitel.}
\cehead{{{\large II,}} I. Umfang der Einzelsprache.}
\section*{Umfang der Einzelsprache.}

\cohead{Sprache, Dialekte, Unterdialekte}
\subsection*{Sprache, Dialekte, Unterdialekte.}\phantomsection\label{II.I}
\fed{{\textbar}54{\textbar}}\phantomsection\label{fp.54} \sed{{\textbar}{\textbar}54{\textbar}{\textbar}}\phantomsection\label{sp.54}

Es wird zuweilen gefragt: Wie viele Sprachen giebt es auf der Erde? Und dann lautet die Antwort: Ungefähr tausend, oder ungefähr zwölfhundert, oder \update{ungefähr fünfzehnhundert,}{fünfzehnhundert, oder ungefähr zweitausend,} – höhere Zahlen entsinne ich mich nicht gelesen zu haben. In dem Sinne wie die Frage gemeint ist, sind die Antworten richtig, und zwar alle \update{drei}{vier} gleich richtig; sie werden auch schwerlich durch eine genauere und richtigere ersetzt werden, wenn man dereinst alle Völker und ihre Sprachen kennt. Im Königreiche Sachsen zählen wir ohne Weiteres Hochdeutsch und Wendisch auf, und auf der pyrenäischen Halbinsel Spanisch, Catalonisch, Portugiesisch, Baskisch und allenfalls noch Zigeunerisch, in Belgien \retro{Französisch,}{Französich,} Wallonisch und Vlämisch u.~s.~w. Schwieriger wird die Sache anderwärts; immer und immer wieder fragt man, ob Sprache oder Dialekt, ob Haupt- oder Unterdialekt?

Diese Ausdrücke sind allgemein üblich und für die Wissenschaft unentbehrlich. Die Bewohnerschaften zweier Ländergebiete reden einander ähnlich, aber nicht gleich; es gilt mit einem Worte anzugeben, wie weit die Ähnlichkeit, wie weit die Verschiedenheit gehe, – und nun sagt man kurzweg: Es sind verschiedene Sprachen, oder: Es sind verschiedene Dialekte derselben Sprache, oder: Es sind verschiedene Abschattungen (Mundarten) desselben Dialektes, also Unterdialekte. Oft lauten auch die Antworten ungleich: der Eine erkennt nur eine Mehrheit von Dialekten, wo der Andere von ebenso vielen Sprachen redet. So bei den slavischen, \fed{{\textbar}55{\textbar}}\phantomsection\label{fp.55} semitischen, polynesischen und noch vielen anderen Sprachengruppen: Die Thatsachen, welche im einzelnen Falle die Verwandtschaftsnähe bestimmen, sind bekannt, nur die Benennungen sind streitig. Folglich ist die Terminologie unsicher, es giebt noch keine gemeingültigen Definitionen der Begriffe.

\sed{{\textbar}{\textbar}55{\textbar}{\textbar}}\phantomsection\label{sp.55}

Die Wörter selbst aber, und also auch gewisse damit verbundene Anschau\-ungen sind tief in den Sprachgebrauch, ja zum Theil in das Volksleben eingedrungen. Unter einer Sprache denkt man sich das Gemeingut eines Volkes, unter einem Dialekte oder einer Mundart das Gemeingut einer Landschaft, – dies dürfte so etwa der Allerweltsauffassung entsprechen. Schriftdeutsch wird im ganzen Vaterlande geschrieben und gelesen, von den Kanzeln gepredigt, in den Schulen gelehrt: mithin ist es Sprache. Bairisch, Schwäbisch, Pfälzisch u.~s.~w. dagegen sind Dialekte. Diesem Standpunkte wird es schwer zu begreifen, dass Plattdeutsch nicht auch blos ein Dialekt ist; Zeitungen, Behörden, Geistliche und Lehrer, in vielen Städten die meisten Bürger reden ja auch in Niederdeutschland hochdeutsch. Dagegen begreift man ziemlich leicht, dass das Holländische als Sprache dem Hochdeutschen nebengeordnet ist. Dialekt spricht der Mann im Kittel; die Gebildeten, vom Kellner aufwärts, bemühen sich „dialektfrei“ zu sprechen. Im Inlande gesteht man nur den Leuten, deren Rede man gar nicht versteht, eine besondere Sprache zu, so z.~B. unsern Lausitzer Wenden. Diese Art die Dinge zu beurtheilen ist äusserlich, oberflächlich, und muss zu Inconsequenzen führen. Die Geschichte \update{unsrer}{unserer} Tage hat aber bewiesen, dass solche Ansichten auch zu recht ärgerlichen Consequenzen führen können: Weil die Sprache Gemeingut des Volkes ist, so begründet Sprachgemeinschaft das Recht zur politischen Vereinigung, Sprachverschiedenheit das Recht zur Losreissung, – so urtheilt das moderne Nationalitätsprinzip nach dem Wahlspruche: „soweit die x’sche Zunge klingt.“

Es giebt noch eine andere Betrachtungsweise, die noch naiver, noch volksthümlicher, und doch im Grunde die einzig richtige ist: Wen ich verstehe, der redet meine Sprache; wen ich nicht verstehe, der redet eine mir fremde Sprache. So urtheilte jener Tyroler, der vom Berliner sagte: „Der Mann versteht kein Deutsch!“ Hätte er statt dessen gesagt: „Der Mann redet eine andere Sprache als ich,“ so wüsste ich nicht, was die Wissenschaft dagegen einwenden wollte.

Sprache ist Verständigungsmittel, Mittel des Gedankenverkehrs. Ein \fed{{\textbar}56{\textbar}}\phantomsection\label{fp.56} Verkehrsmittel begründet eine Gemeinschaft Aller, die sich seiner bedienen. Man nennt die Sprache die Münze des Gedankenaustausches, und in diesem Sinne mag man die Sprachgemeinschaften mit unsern sogenannten Münzverbänden vergleichen. Nur freilich wird in unserm Falle nicht die Gemeinschaft des Gebrauches durch die Einheit der Münze von vorn herein bestimmt, sondern es wird umgekehrt aus der Gemeinschaft des Gebrauches auf die Einheit der Münze geschlossen.

Es ist in der Wissenschaft nothwendig, die Gedanken auf die Spitze zu treiben, unbeirrt die letzten Folgerungen aus ihnen zu ziehen, selbst wenn diese Folgerungen dem gemeinen Menschenverstande und den überkommenen Meinungen zuwiderlaufen sollten. Setzen wir also folgenden Fall: Zwei Nachbarn desselben \sed{{\textbar}{\textbar}56{\textbar}{\textbar}}\phantomsection\label{sp.56} Ortes, \textit{A} und \textit{B}, haben bisher nur ihre Muttersprache in der heimischen Mundart gehört und gelernt. Nun kommt ein Fremder, \textit{C}, zu ihnen; \textit{A} versteht ihn nicht, \textit{B} aber, der rascheren Verstand und schärferes Gehör hat, versteht ihn und redet mit ihm. In diesem Falle ist zu entscheiden: Es besteht Sprachgemeinschaft zwischen \textit{A} und \textit{B} und zwischen \textit{B} und \textit{C}, aber nicht zwischen \textit{A} und \textit{C}. Nun weiter: \textit{C} verstehe den \textit{A}, von dem er nicht verstanden wird: so fällt \textit{A} in die Sprachgemeinschaft des \textit{C}, aber \textit{C} nicht in die Sprachgemeinschaft des \textit{A}. So zieht sich um jeden Einzelnen ein weiterer oder engerer Kreis der Sprachgemeinschaft, und in diesem Verstande mag es fast ebensoviele Grenzlinien der Sprachgemeinschaft geben, wie es sprechende Menschen giebt.

Ähnlich wie zu den Mitlebenden aus verschiedenen Gegenden verhalten wir uns auch zu den Vorfahren aus verschiedenen Zeiten. Man gebe einigen Kindern Wackernagel’s deutsches Lesebuch in die Hand und beobachte, wie ein Jedes anfängt die Texte ohne sonderliches Besinnen zu verstehen: oder sie zu verstehen, wenn man sie ihm vorliest. Dieses Verständniss wird bei den verschiedenen Kindern an sehr verschiedenen Stellen anfangen, und diese Stellen bezeichnen die Grenze ihrer Sprachgemeinschaft mit den Altvordern. Einzelne unverständliche Wörter und Redewendungen kommen hier ebensowenig in Betracht, wie im vorigen Beispiele die etwaigen fremden Provinzialismen; wir sagen uns doch: der Alte oder der Fremde redet dieselbe Sprache wie wir, er redet sie nur ein wenig anders als wir.

Soviel von den individuellen Sprachgemeinschaften. Das Ergebniss ist ähnlich, wenn man statt der einzelnen Menschen ganze Landschaften \fed{{\textbar}57{\textbar}}\phantomsection\label{fp.57} und den Durchschnitt ihrer jeweiligen Bewohner setzt: um jeden Mittelpunkt ein Kreis, und diese Kreise greifen ineinander, überragen einander. So würde man zu einer Zeichnung kommen, die etwa an das Guillochis auf dem Rücken einer Taschenuhr gemahnte, zu einem unruhigen Bilde, das sich die Wissenschaft zwar vorstellen muss, bei dem sie aber nicht stehen bleiben darf. Wo uns die Augen übergehen, da hat das wissenschaftliche Beobachten ein Ende; nur das Feste, Greifbare ist unsrer Einsicht zugänglich, das Flüssige verlangt ein Gefäss, das heisst eine Grenze, die man ihm setzt. Hier, wie so oft, müssen wir unsre Zuflucht zu dem Satze nehmen: Denominatio fit a potiori.

Wollten wir die Grenzen der einzelnen Sprachgemeinschaften umschreiben, so würde die Zeichnung sehr unregelmässig ausfallen: ein Gebiet, in Deutschland das mittlere, würde den meisten Bezirken gemeinsam sein; dann, weiter nord- und südwärts, würden die Linien sich vielfältig kreuzen, noch weiterhin würden sie an Dichtigkeit abnehmen, stellenweise würden sie zusammenfliessen, andere Kreise, z.~B. die slavischen, ausschliessen; oder sie würden nur schmale neutrale Gebiete aufweisen, z.~B. an der niederdeutschen und an der holländischen Grenze. Eine solche Zeichnung müsste den Umfang des hochdeutschen Sprachgebietes aus \sed{{\textbar}{\textbar}57{\textbar}{\textbar}}\phantomsection\label{sp.57} dem Begriffe der Einzelsprache selbst erweisen, und nun erst wäre zu fragen, kraft welcher Eigenthümlichkeiten das Plattdeutsche und Holländische dem Hochdeutschen gegenüber fremde Spracheinheiten bilden, warum z.~B. der Meissner nicht Plattdeutsch versteht? Der Grund würde nur zum Theile in den lautlichen, grammatischen und lexikalischen Verschiedenheiten der Sprachen zu finden sein; zum anderen Theile wäre er im grösseren oder geringeren Sprachtalente des Meissners zu suchen. Und so kommt auch hier wieder die Subjectivität, – diesmal aber die nationale, – zu dem Rechte, das ihr nun einmal nicht versagt werden kann.

\largerpage[-1]Die Sprachwissenschaft hat es zunächst mit den einfachsten Objecten zu thun, hier also mit den Menschen, wie sie von Haus aus sind, nicht wie sie sich durch Schulunterricht, Reisen, Fremdenverkehr, Militärdienst u.~s.~w. gebildet haben mögen, – mit den Leuten, die nur ihre Muttersprache im heimischen Dialekte reden, und wieder hier nur mit dem Durchschnittsmenschen. Einen solchen werden wir nun weiter den \update{Maassstab}{Massstab} seines Verständnisses anlegen lassen. Jemand aus einer anderen Gegend redet mit ihm; die Beiden verständigen sich miteinander, \fed{{\textbar}58{\textbar}}\phantomsection\label{fp.58} aber nur mühsam, und wo sie sich einmal nicht verstehen, da fühlen sie doch wenigstens die Sprachgemeinschaft durch. Der Sachse sagt Meerrettig, der Österreicher Krän, Jener sagt Bindfaden, dieser Spagat, Keiner weiss, was der Andere meint, aber \update{jeder}{Jeder} fühlt und weiss, dass der Andere deutsch spricht. Wo nun zwischen Sprachgenossen die Arbeit der Verständigung als eine mühsame empfunden wird, da möchte ich von verschiedenen (Haupt-) \so{Dialekten} reden.

\begin{sloppypar}Dialektsgenossen sind also solche, die sich leicht miteinander verständigen. Örtliche Verschiedenheiten in ihrer Rede nenne ich \so{Unterdialekte} oder \so{Mundarten} im engeren Sinne. Recht concret gesprochen: wer einen anderen Unterdialekt redet, dem merkt man eben an seiner Sprache nur an, dass „er nicht von hier ist“, aber man versteht ihn ohne Mühe.\end{sloppypar}

Der \update{Maassstab,}{Massstab,} den ich hier überall angelegt habe, erinnert freilich an das volks\-thümliche \update{Schritt\-maass}{Schritt\-mass} für Wegelängen, das sich nach der Länge der Beine richtet. Allein man zeige mir einen besseren, gemeingültigeren. Lautverschiebungen, grammatische und lexikalische Verschiedenheiten finden sich schon zwischen nahverwandten Mundarten; ihr Mehr oder Weniger begründet die Unmöglichkeit, die grössere oder geringere Schwierigkeit des gegenseitigen Verständnisses. Einen springenden Punkt, wie etwa die Gefrier- und Siedepunkte des Thermometers, wird man auf dieser \update{Scala}{Skala} nirgends entdecken. Alles ist hier Sache des Gefühles; zwischen dem Gefühle des ganz Fremden und des ganz Heimischen liegt eine Reihe unzähliger Möglichkeiten; das vorwiegende Gefühl hat zu entscheiden. Und wo nun ferner die Frage, ob Dialekt oder Sprache, ob Haupt- oder Unterdialekt, Schwierigkeiten macht, da ist sie auch unerheblich. Denn Unterscheidungen wie diese sind nur zur Bequemlichkeit da \sed{{\textbar}{\textbar}58{\textbar}{\textbar}}\phantomsection\label{sp.58} und keiner Ereiferung werth. Hier kommt es darauf an, sie auf ihre Bedeutung \fed{und Berechtigung} zu prüfen und ihnen den Theil wissenschaftlichen Werthes zuzumessen, der ihnen gebührt.

Gerade die Culturvölker nehmen es mit der Verschiedensprachigkeit nicht immer genau; das Gefühl der nationalen Zusammengehörigkeit findet genügenden Anhalt in der fremden Sprache, wenn diese nur der eigenen nahe verwandt, und die Verständigung nach kurzem Verkehre möglich ist. Den oberdeutschen Bauern verbindet mit dem plattdeutschen die gemeinsame Schriftsprache, und dieser gegenüber rücken die heimischen Idiome ohne Weiteres hinab auf die Stufe der \update{patois.}{Patois.} Noch tiefer \fed{{\textbar}59{\textbar}}\phantomsection\label{fp.59} eingewurzelt mag dies Gefühl sprachlicher Zusammengehörigkeit mit allen Bürgern seines grossen Vaterlandes beim Chinesen sein. Er hat vielleicht nur wenige Meilen weit zu reisen, um unter Menschen zu kommen, deren Gespräch er nicht versteht. Aber er fühlt sich in ihrer Mitte unter Landsleuten, und wenn er zum Pinsel greift, um sich verständlich zu machen, so lesen sie seine Schrift in ihrer Aussprache. Nun merkt er, dass es sich eben nur um die Aussprache handelt, nicht um die Sprache selbst. Stellt sich die Wissenschaft auf seinen Standpunkt, so muss sie auch hier von Dialekten reden, wie sie es bisher gethan, und wie es der Chinese selber thut.

Man hat den Satz: „universalia sunt nomina“ auf die Einzelsprachen ausgedehnt und gesagt, diese selbst seien keine Realitäten, sondern nur entweder Aggregate oder mittlere Durchschnitte der Individualsprachen, deren es so viele gebe wie Individuen. Der alte Streit zwischen Realisten und Nominalisten ist meines Wissens noch nicht zum Austrage gelangt, und jedenfalls sind es nicht die Sprachforscher, die ihn zu entscheiden haben. Dass die Sprachgrenzen individuell verschieden sind, haben wir gesehen; dass die Handhabung der Sprache auch unter den Sprachgenossen nicht völlig gleich ist, werden wir weiter sehen. Als gemeinsames Verständigungsmittel aber ist die Einzelsprache wirksam und also doch auch wirklich. Als Äusserung, als Rede, gehört sie dem Einzelnen, als Fähigkeit muss sie Gemeingut sein, sonst taugte sie nicht zum Verkehrsmittel. Dass die Sprache ihrerseits ein Erzeugniss des Verkehrs ist, darauf brauche ich an dieser Stelle nicht weiter einzugehen.

\pdfbookmark[1]{II. Capitel.}{II.II.Capitel}
\cehead{{{\large II,}} II. Aufgabe der einzelsprachl. Forschung.}
\cohead{Wandelung und Stätigkeit.}
\section*{II. Capitel.}
\section*{Die besondere Aufgabe der einzelsprachlichen Forschung.}\phantomsection\label{II.II}

Jede Sprache ist in fortwährendem Werden begriffen; ihre Laute, Wörter, Formen und die Bedeutungen dieser Wörter und Formen verändern sich mit der Zeit, und diese Veränderungen bilden den Inhalt der Sprachgeschichte. Aber nur da, wo eine Literatur vorhanden, ist diese Geschichte unmittelbar zu\sed{{\textbar}{\textbar}59{\textbar}{\textbar}}\phantomsection\label{sp.59}gänglich; wo schriftliche \update{Denkmäler}{Urkunden} \sed{oder wörtlich zuverlässige mündliche Überlieferungen alter Sprachdenkmäler} fehlen, da muss sie soweit möglich auf Umwegen ermittelt werden. Nun sind \fed{{\textbar}60{\textbar}}\phantomsection\label{fp.60} die meisten Sprachen in Dialekte gespalten oder haben Seitenverwandte, – wir wissen, dass Beides dem Wesen nach gleich und nur dem Grade nach verschieden ist: Beides, Dialekte und Schwestersprachen sind die Enden längerer oder kürzerer, stärkerer oder schwächerer Zweige, von denen aus wir durch Rückschlüsse dem gemeinsamen Stamme zustreben dürfen.

\sed{Der Gegenstand der einzelsprachlichen Forschung, die Erscheinung, die sie erklären will, ist, – dies sei nochmals hervorgehoben, – die Sprache als Äusserung, das heisst die \so{Rede}. Wie kommt in der zu bearbeitenden Einzelsprache die Rede zustande, und warum gestaltet sie sich gerade so? Eine Äusserung erklären heisst, die ihr zu Grunde liegenden Kräfte nachweisen. Die Rede ist eine Äusserung des einzelnen Menschen, die sie erzeugende Kraft gehört also zunächst dem Einzelnen an. Aber die Rede will verstanden sein, und sie kann nur verstanden werden, wenn die Kraft, der sie entströmt, auch in dem Hörer wirkt. Diese Kraft, – ein Apparat von Stoffen und Formen, – ist eben die Einzelsprache. Sie richtig beschreiben, heisst ihre Äusserungen erklären. Mehr soll und will die einzelsprachliche Forschung als solche nicht. Man sieht, kein Vorwurf wäre leichtfertiger und gedankenloser, als der: also sei es doch nur eine beschreibende Wissenschaft, also gar keine Wissenschaft, blosse Sprachenkunde, um nichts besser oder schlechter als die Pflanzenkunde des Schülers, der mit der Botanisirtrommel und einem Duodezführer nach Linné’scher Methode die Feldraine abgrast. Ich würde den Einwand nicht erwähnt haben, wenn ich ihn nicht gehört und gelesen hätte; nun muss ich ihn auch beantworten. Beides, jene Sprachenkunde und diese Pflanzenkunde, ist doch sehr verschieden. Was wir auf unseren Fluren pflücken, sind nicht Sprachen, sondern Sprachäusserungen; und was dem „Bestimmen“ der Pflanze entspricht, ist in unserm Falle mit der Übersetzung gethan. Wo aber der Pflanzensammler stillvergnügt seine Beute in die Trockenpresse spannt, da fängt erst unsre beste Arbeit an: da bereiten wir den Boden und streuen den Samen, woraus eine neue, lebendige Flora erwachsen soll. Denn wissenschaftlich beschreiben heisst aufbauen, nachschaffen. Wir lernen und lehren die Rede aufbauen aus ihren Stoffen und nach ihren Gesetzen, nachdem wir diese Stoffe und Gesetze inductiv, aus der Rede, ermittelt haben. Dies ist die Grenze, die wir erreichen müssen, die wir aber nicht überschreiten können, ohne in ein anderes Forschungsgebiet überzutreten.}

Die Erkenntniss der Einzelsprache wird nie vollkommen sein ohne die Kenntniss ihrer Vorgeschichte. In Deutschland hat z.~B. die volksthümliche Sprache vieler Gegenden eine seltsame Form zum Ausdrucke annähernder \update{Maass\-angaben:}{Mass}{\textbar}{\textbar}60{\textbar}{\textbar}\phantomsection\label{sp.60}\sed{angaben:} ein Groschener achte, ein Tager vierzehn, ein Stücker zwanzig, ein Schocker drei, ein Wochener (oder Wocher) fünfe u.~s.~w. Woher dies Suffix –er? Man könnte fast an eine Art pronominalen Genitivus Pluralis denken. Bei Schriftstellern aus der Reformationszeit findet sich aber: ein Jahr oder drei, ein Gülden oder zwanzig und Ähnliches. Fehlten uns diese Quellen, so würde uns ein deutscher Dialekt, der etwa das \so{oder} in solchen Fällen noch volllautig erhalten hätte, den gleichen Dienst leisten; und in Ermangelung eines solchen böte noch immer das Holländische mit seinem: \textit{een dag of} (= oder) \textit{veertien} einen guten Fingerzeig.

Nun frage man aber die Leute, was sie sich bei dem –er denken, wie sie ein Tager vierzehn auf gut Deutsch schreiben würden? Den Wenigsten würde eine Antwort einfallen, Keinem vermuthlich der Gedanke an ein abgekürztes „oder“ kommen. Man versuche dann ihnen zu Hülfe zu kommen, erinnere sie an „ihrer drei, unsrer vier“, dann an die doppelten Pluralformen: Orte, Örter, Worte, Wörter, Lande, Länder, endlich an die Conjunction \update{oder:}{\so{oder:}} so wird wohl kaum Einer das Richtige wählen. Der Zusammenhang dieser Form mit ihrem Ursprunge wäre also dem Sprachbewusstsein des Volkes entschwunden, in diesem Bewusstsein stände entweder die Form vereinzelt da, oder sie hätte einen neuen Verwandtschaftsbund eingegangen, und das ist der Punkt, auf den ich den Leser führen wollte.

\so{Die einzelsprachliche Forschung als solche hat die Sprache nur so, aber auch ganz so zu erklären, wie sie sich jeweilig im Volksgeiste darstellt}. Zieht sie die Vorgeschichte, die Dialekte und stammverwandten Sprachen zu Rathe, so tritt sie auf das genealogisch-historische Gebiet über. Ich wiederhole es: sie muss dies thun, wo immer es möglich ist; aber sie darf nicht vergessen, dass zuweilen \fed{{\textbar}61{\textbar}}\phantomsection\label{fp.61} das Sprachbewusstsein eines Volkes alte Verbindungen löst, um neue anzuknüpfen, und dass diese neuen Verbindungen fortan die allein rechtskräftigen, wirksamen sind.

\sed{Darin liegt nun der besondere Reiz der einzelsprachlichen Forschung, dass sie es immer, auch in ihren scheinbar kleinlichsten Spezialuntersuchungen, mit einem lebendigen, durchgeistigten Ganzen zu thun hat. Die geschichtliche Sprachvergleichung beschäftigt sich ihrem Wesen nach mit mehreren solcher Ganzen auf einmal. Um sie zu vergleichen, muss sie sie zerpflücken, sich an die Theile halten und unter diesen wieder die fassbarsten bevorzugen, die nicht immer die vorzugsweise geistigen sind. Sie muss sich auch an die Sprachen einer einzigen Familie halten und dabei gewärtig sein, immer und überall denselben Gestalten und Charakteren, nur in verschiedenem lautlichen Gewande, vielleicht mit dem einen oder anderen absonderlichen Geräthe ausgerüstet zu begegnen. Dem Einzelsprachforscher dagegen steht die ganze bunte Sprachenwelt offen; er darf sein Zelt überall aufschlagen, wo es ihm gefällt. Halten sich aber} {\textbar}{\textbar}61{\textbar}{\textbar}\phantomsection\label{sp.61} \sed{Wissens- und Wanderlust bei ihm die Waage, so wird er sein Zelt gerade dann abbrechen und weiterrücken, wenn er sich am Orte recht heimisch fühlt; und dann braucht sich nur der Denker zum Kenner zu gesellen, so wird die Polyglottik in die allgemeine Sprachwissenschaft einmünden.}

Unter den literaturlosen Sprachen unsrer Erde giebt es viele, die man isolirte nennt, weil sie noch keiner bekannten Familie eingereiht sind, und manche von ihnen haben ein so begrenztes Verbreitungsgebiet, dass von eigentlicher dialektischer Spaltung nicht die Rede sein kann, – vielleicht ist auch von ihren Dialekten nur einer der Forschung zugänglich, was für die Forschung auf dasselbe hinauskommt. Dieser ist somit jede Möglichkeit genealogisch-historischer Vergleichung von vorn herein abgeschnitten, sie ist einzelsprachlich im ausschliesslichen Sinne des Wortes. \sed{Vergleichend ist sie aber doch auch; es werden eben innerhalb der Einzelperiode einer Einzelsprache die Thatsachen untereinander verglichen, um zu ermitteln, aus welchen Stoffen und nach welchen Gesetzen sich die Rede aufbaut. Die Thatsachen, die sie vergleicht, sind eben gleichzeitig und gleichsprachlich, im Gegensatze zu jenen, mit denen es die historisch-genealogische Forschung zu thun hat, und die entweder zu verschiedenen Zeiten aufeinanderfolgen oder, gleichviel ob neben- oder nacheinander, an verschiedenen Orten auftreten.} Aber die einzelsprachliche Forschungsart muss \sed{\so{für ihren Zweck}} auch auf alle anderen Sprachen angewendet werden, und ihre Methode bildet den nächsten Gegenstand der folgenden Erörterungen.

\clearpage\pdfbookmark[1]{III. Capitel.}{II.III.Capitel}
\cehead{{{\large II,}} III. Sprachkenntniss.}
\cohead{Die Muttersprache.}
\section*{III. Capitel.}
\section*{Sprachkenntniss.}\phantomsection\label{II.III.1}

Die Aufgabe ist, eine Sprache lediglich so zu begreifen, wie sie im Geiste des sie redenden Volkes lebt. Dies Volk handhabt seine Sprache ohne rückwärts, auf ihre Vorgeschichte, oder seitwärts, auf ihre Dialekte und auswärtigen Verwandten zu schauen; alle Faktoren, welche die richtige Handhabung der Sprache bestimmen, liegen lediglich in dieser Sprache selbst, \update{wollen also aus ihr heraus begriffen sein.}{in unbewusst wirkenden Gesetzen (Analogien), oder in unmittelbar durch Überlieferung Gegebenem.} Sie so zu begreifen ist aber nur der fähig, der, wie man zu sagen pflegt, die Sprache kann oder beherrscht. Ehe wir untersuchen, wie dies möglich ist, müssen wir uns einige Thatsachen zum Bewusstsein bringen.

\phantomsection\label{II.III.erlernt} 1. \so{Jede Sprache will erlernt sein}, keine ist uns angeboren, auch nicht unsre Muttersprache. Höchstens mag man vermuthen, dass gleich anderen Geistesanlagen auch die zu einer gewissen Sprachform vererblich ist, dass etwa ein irokesisches Kind, das nach der Geburt zu französischen Pflegeeltern kommt, \sed{{\textbar}{\textbar}62{\textbar}{\textbar}}\phantomsection\label{sp.62} schwerer Französisch lernt, als es bei seinen leiblichen Eltern Irokesisch gelernt haben würde.

\begin{sloppypar}\phantomsection\label{II.III.handhabung} 2. \so{Jeder normal entwickelte Mensch, der die Zeit der Sprach-} \fed{{\textbar}62{\textbar}}\phantomsection\label{fp.62}\so{erlernung hinter sich hat, handhabt seine Muttersprache fehlerlos}, solange sie ihm nicht durch fremde Einflüsse verdorben wird. Wir müssen hier von unsern cultursprachlichen Vorurtheilen gänzlich absehen. Wir pflegen unsre Schriftsprache wie eine Taxuswand: was darüber hinausschiesst, wird mit der Heckenscheere des Schullehrers, des Redacteurs oder des Kritikers erbarmungslos abgeschnitten; und das von Rechtswegen, mag auch der Sprachforscher darüber jammern. Die Einheit der Nation verlangt Einheit der Sprache und erlangt sie auch soweit nöthig. Ganz ist aber auch die Schriftsprache nicht mit dem Uniformiren fertig geworden. Die Berliner Localnachrichten sind mit Berlinismen gewürzt, weiter nordwärts und westwärts fliessen plattdeutsche Ausdrücke in die Zeilen, die bairischen Zeitungscorrespondenten erkennt man an ihrem „dahier“, und in Wiener Blättern empfiehlt sich „weiters“ dem „p.~t. Publicum der unterfertigte bürgerliche Handschuherzeuger“. Die Umgangssprache der meisten Gebildeten, also die Muttersprache ihrer Kinder, ist kein reines Schriftdeutsch, sondern ein Compromiss zwischen diesem und dem heimischen Dialekte. Der Braunschweiger geht „nach dem Kruge“, statt in die Schänke, der Süddeutsche gewöhnt sich schwer an das erzählende Imperfectum; in Tonfall und Lautbildung zeigen sich natürlich die heimischen Eigenthümlichkeiten erst recht. Den Sprachforscher nun darf die Mundart irgendwelcher Bauernschaft nicht weniger aber auch nicht mehr interessiren, als die sogenannte allgemeine Sprache der Gebildeten und ihre mannigfachen Abschattirungen. – Nun gilt es nur noch, den Ausdruck \so{Muttersprache} richtig zu verstehen. Es ist die Sprache und Mundart, die wir als Kinder von den Erwachsenen, die uns umgeben, gehört haben. In den meisten Fällen wird dies nur \so{eine} Mundart einer Sprache sein, und diese, sage ich, handhaben wir richtig. Wo die Dienstboten anders reden als die Eltern, wo diese oder die Erzieher dem Kinde eine „gebildete“ Sprache ankünsteln, da liegen eben die Verhältnisse nicht glatt und einfach, die Spracherlernung wird erschwert, verzögert, aber am Ende bildet sich doch eine fehlerlos richtige Sprache heraus. Fehlerlos richtig meine ich aber im Sinne des Sprachforschers, der in diesem Falle nicht den \update{Maassstab}{Massstab} des Sprachlehrers anlegt. Mein verewigter Vater pflegte wohl scherzweise zu sagen: „Richtig spricht, wer redet wie ihm der Schnabel gewachsen ist.“ Schlimmer steht es allerdings, wenn fremde Beimischungen die Muttersprache trüben, und das ist freilich in unsern Culturstaaten fast das Regelmässige. Der \fed{{\textbar}63{\textbar}}\phantomsection\label{fp.63} Aufenthalt in der Fremde, der Verkehr Ungebildeter mit ihren Vorgesetzten erzeugt unzählige sprachliche Blendlinge, pathologische Erscheinungen, die auch ihr Interesse haben.\end{sloppypar}

\sed{{\textbar}{\textbar}63{\textbar}{\textbar}}\phantomsection\label{sp.63}

\cohead{Allgemeine Grundsätze.}
\phantomsection\label{II.III.unbedacht} 3. \so{Die richtige Handhabung der Muttersprache geschieht unbedacht}, ohne dass der Redende sich von den Sprachgesetzen, die seine Rede bestimmen, Rechenschaft giebt.

\begin{sloppypar}\phantomsection\label{II.III.gesetze} 4. \so{Die Sprachgesetze bilden unter sich ein organisches System, das wir den Sprachgeist nennen}. Der Sprachgeist bestimmt die Art und Weise, wie der Sprachstoff gestaltet wird, – die Wort-, Form- und Satzbildung –; insofern ist er Bildungsprinzip oder \so{innere Sprachform}.\end{sloppypar}

\begin{sloppypar}\phantomsection\label{II.III.gedaechtnisserwerb} 5. \so{Beide, der Stoff und die Form der Sprache, werden durch das Gedächtniss erworben, der Stoff unmittelbar, die Form (der Sprachgeist) als unbewusste Abstraction aus vielfacher Erfahrung und Übung}. Die Wirkung dieser unbewussten Abstraction nennen wir \so{Analogie}. \sed{In den indogermanischen Sprachen mit ihren zahllosen Unregelmässigkeiten wirkt das Analogiebedürfniss oft störend, missleitend; und so hat bei uns das Wort Analogie fast einen revolutionären Klang. Den verdient es aber nicht. Weitaus die meisten Sprachen sind regelmässiger als die unsrigen, und wer in ihnen der Analogie folgt, wird selten fehlgehen.}\end{sloppypar}

\update{Durch welche geistigen... [\textit{kein neuer Absatz}]}{Durch welche geistigen} Thätigkeiten die Abstractionen und nach ihnen die analogen Wort-, Form- und Satzgebilde zu Stande kommen, das zu erklären liegt nicht dem Sprachforscher ob, sondern dem Psychologen. Genug, die Thatsache ist da, unbestreitbar und doch schwer begreiflich. Jeder Ausländer wird uns bestätigen, wie schwer es ihm wird, sich an unsre Wortstellung, zumal an die der verbalen Satztheile, zu gewöhnen, und ich habe bisher vergeblich nach einem Lehrbuche gesucht, das sie hinlänglich darstellte. Das Verbum finitum im mittheilenden Satze an zweiter, im Fragesatze und dem ihm nachgebildeten Bedingungssatze an erster, sonst im Nebensatze an letzter Stelle; das ergänzende Hauptverbum (Infinitiv oder Participium) im Hauptsatze an letzter Stelle, mit dem Hülfsverbum alle übrigen Theile des Prädicates umklafternd: das sind im Wesentlichen die Gesetze, die bei uns jedes Kind handhaben lernt.\footnote{Das Nähere sehe man in meinem Aufsatze: Weiteres zur vergleichenden Syntax (Ztschr. f. Völkerpsychologie und Sprachwissensch. VIII, S.~144–158).} Es würde sie nicht handhaben können, wenn nicht sein Geist durch Übung unbewusst alle die Abstractionen gemacht hätte, deren ich mich eben bedienen musste. Mit Ausdrücken wie Angewöhnung, anerzogene Disposition und dergl. ist hier wenig gethan: sie erklären höchstens das Wie, nicht das Was. Und dieses Was sind eben Kategorien, an deren Erkenntniss sich \fed{{\textbar}64{\textbar}}\phantomsection\label{fp.64} eine Menge scharfsinniger Forscher abgemüht haben. Dafür ist aber auch dem Genialen nichts verwandter als das Naive. Versucht man es, sich den Sinn einer grammatischen Form oder eines Formwortes klar zu machen, so mag man sich wohl wundern, zu welch abstracten Begriffen man gelangt, und sich dann verdutzt fragen: Ist denn das wirklich auch im Hirne jedes schwatzenden Kindes vorhanden? Da\sed{{\textbar}{\textbar}64{\textbar}{\textbar}}\phantomsection\label{sp.64}rauf kann man nur mit einem entschiedenen Ja antworten. Die Erfahrung ist unanfechtbar, und mit der haben wir es allein zu thun; die Erklärung der Thatsache mag schwierig, vielleicht unmöglich sein, – jedenfalls ist sie ein Problem nicht der Sprachwissenschaft, sondern der Psychologie. Reines Gedächtnisswerk ist die Aneignung

a) der Lautkörper, d.~h. der Wörter und der etwaigen Formenelemente, und

b) der unregelmässigen Formen, das heisst derer, die sich nicht in die dem Lernenden zugänglichen Analogien hineinfügen. Die Geistesanlage des Einzelnen spielt gewiss hierbei eine bedeutende Rolle: der Eine bewältigt gruppenweise durch Analogie den Stoff, den sich ein Anderer Stück für Stück anlernen muss. Dass sich nachgehends die Association mit etwaigem Verwandten – das Analogiegefühl – einstellen werde, ist wohl psychologisch wahrscheinlich, und so wird das Sprachgefühl aller einzelnen Sprachgenossen im Wesentlichen das gleiche sein. In diesem Sinne dürfen wir den Sprachgeist einen Bestandtheil des Volksgeistes nennen. Wo nun dem Sprachgeiste Einzelnes als vereinzelt gilt, da hat auch die Einzelsprachforschung von Ausnahmen oder sporadischen Erscheinungen zu reden; denn so lange sie sich in ihren Grenzen hält, kann und will und soll sie die Dinge nur insoweit und so erklären, wie sie sich im Sprachgeiste darstellen. \sed{Was für diesen ausserhalb der Analogie steht, kann sie mit ihren Mitteln gar nicht erklären, und wenn sie nun doch der Sprachgeschichte die Erklärung entleiht, so darf sie nicht vergessen, dass sie eben mit fremden Mitteln arbeitet und der Sprache fremde, ich meine ihr fremd gewordene Stoffe beimischt, die ausgeschieden sein wollen, sobald sie ihren Aushülfedienst versehen haben.} Es ist sehr wichtig, jene zweierlei Bestandtheile scharf zu sondern: diejenigen, die nur in unmittelbarer Erinnerung wurzeln, und jene, die sich zum grossen Systeme der Analogien zusammenschliessen und aus diesem heraus jederzeit neu erzeugt werden können. Solche Erzeugnisse sind völlig zureichend erklärt, wenn ihnen ihre Stellung in jenem Systeme nachgewiesen ist, und diesen Nachweis kann von ihrem Standpunkte aus nicht die sprachgeschichtliche, sondern nur die einzelsprachliche Forschung führen. Darin eben liegt ihre selbständige Berechtigung, die man nur zu leicht verkennt, weil man sich einbildet zu wissen, warum etwas ist, wenn man weiss, wie es früher war und nach welchen Formeln es sich verändert hat. Erklären und darstellen \fed{{\textbar}65{\textbar}}\phantomsection\label{fp.65} aber kann ich nur das, was sich in meinem eigenen Geiste vorfindet. Daher setzt die Lösung der einzelsprachlichen Aufgabe voraus, dass der Geist der Sprache ein Bestandtheil unseres Geistes geworden sei, und dies ist nur durch Spracherlernung zu erreichen. \sed{Ich werde diesem Gegenstande mehr Raum widmen, als es sonst in Werken über allgemeine Sprachwissenschaft üblich ist. Der Leser wolle sich erinnern, dass ich der Sprachkenntniss und der Sprachenkenntniss mehr Werth beimesse, als einige meiner Vorgänger. (S.~29–30.)}

\sed{{\textbar}{\textbar}65{\textbar}{\textbar}}\phantomsection\label{sp.65}

\pdfbookmark[1]{IV. Capitel.}{II.IV.Capitel}
\cehead{{{\large II,}} IV. Spracherlernung.}
\cohead{§. 1. A. Durch mündlichen Umgang.}
\section*{IV. Capitel.}
\pdfbookmark[2]{§. 1. A. Durch mündlichen Umgang.}{II.IV.1}
\section*{Spracherlernung.}
\subsection*{§. 1.}\phantomsection\label{II.IV.1}
\subsection*{A. Durch mündlichen Umgang.}

Es ist höchst lehrreich, \so{Kinder} im Alter der Spracherlernung zu beobachten. Sie stellen sich dabei sehr verschieden an: manche brauchen Jahre, ehe sie die Schwierigkeiten der Aussprache und der Grammatik überwinden, und für andere scheinen diese Schwierigkeiten kaum vorhanden zu sein. Ich wüsste deutsche Kinder zu nennen, die von Beginn ihrer Redeübungen an die Gutturale, die Consonantenhäufungen ihrer Muttersprache, selbst fremdsprachliche Wörter, die sie hörten, leicht und fehlerfrei nachsprachen, gegen unsre Genusregeln, die Unregelmässigkeiten in der Pluralbildung und in der Conjugation kaum je verstiessen. Andere bauten sich ganz selbständig eine eigene Sprache mit seltsamen Gesetzen auf; so ein Knabe, der nach Art der Semiten die Consonanten als das Feste, Beständige \update{behandelte,}{behandelte} und die Vocale um so tiefer wählte, je grösser ihm die Gegenstände erschienen. Einen gewöhnlichen Stuhl nannte er Lakeil, einen Grossvaterstuhl Lukul, ein Puppenstühlchen Likill; für alles Runde hatte er die Wurzel \textit{m-m}. Der Mond oder ein Teller hiess Mem, eine grosse runde Schüssel Mom oder Mum, die Sterne aber – mit symbolischer Wiederholung – Mim-mim-mim-mim-mim. Als sein Vater im grossen Reisepelze vor ihm stand, sagte er nicht Papa, sondern Pupu. Hier war also der kindliche Geist, völlig frei schaffend, auf eine innere Wurzelbeugung verfallen, und damit scheint mir bewiesen zu sein, dass innere Veränderungen der Wurzeln nicht immer durch mechanische Prozesse entstanden sein müssen. Spuren einer ähnlichen Lautmalerei finden sich u.~A. im Malaischen und sonst vieler Orten.

\fed{{\textbar}66{\textbar}}\phantomsection\label{fp.66}

Jetzt aber handelt es sich weniger darum, wie Kinder ihre eigenen Sprachen bilden, als darum, wie sie sich die Muttersprache aneignen, und auch hierbei sind sie in gewisser Weise mitschaffend. In einem gewissen Alter beginnt das Kind sich in der Bildung verschiedener Laute zu üben; zu einem kleinen Vorrathe von Vocalen – meist wohl \textit{e}, \textit{ä}, \textit{ö}, \textit{\textsubring{ä}} und \textit{a} – finden sich nach und nach consonantische Laute ein: die Nasalen \textit{m}, \textit{ng}, \textit{n}, ferner \textit{p}, \textit{b} und wohl auch eine Art labiales \textit{r}, sowie \textit{l}, \textit{t} u.~s.~w. Die Reihenfolge mag individuell stark schwanken. Es ist nicht anzunehmen und jedenfalls nicht wahrzunehmen, dass das Kind mit diesen Lauten bestimmte Vorstellungen verbinde, – \update{\so{die}}{die} werden erst von den Eltern hineingelegt. Diese gehen auf das kindliche Lallen ein, wo immer es einem brauchbaren Worte ähnelt, wiederholen es, unwillkürlich nachbessernd, reagiren darauf und gewöhnen so den erwachenden Geist, die \sed{{\textbar}{\textbar}66{\textbar}{\textbar}}\phantomsection\label{sp.66} Bewegung der Lauterzeugung, ihre hörbare Wirkung und das darauf Erfolgende \update{zueinander}{zu einander} in Beziehung zu setzen. Das Kind sagt \textit{m\textsubring{m}-m\textsubring{m}}, die Eltern sagen \textit{mama}, \textit{mama}, und nun nimmt die Mutter das Kind auf den Arm; oder es greift nach etwas und ruft – zufällig – dabei \textit{ham!} da antworten die Eltern: haben! und geben ihm den Gegenstand, den es zu begehren schien. Bei Alledem sind Zweckmässigkeit und Absichtslosigkeit auf beiden Seiten fast gleich: das Kind wollte nicht lernen, die Eltern wollten auch neunmal unter zehnen gar nicht lehren, hüben und drüben war es ein Spiel, und doch gedieh es zum Unterrichte.

Hinfort wächst der \so{Wortschatz} des Kleinen stätig, zuweilen erstaunlich rasch, und von selbst stellen sich \so{grammatische Anregungen} ein. Denn die Eltern bedienen sich, indem sie zum Kinde oder in seiner Gegenwart von ihm reden, unzählige Male derselben stereotypen Sätze, von denen das Kind sich sein Theil aneignet. Dem Kinde wird Allerlei gezeigt und benannt, es spricht nach und gewöhnt sich je länger je mehr daran, Sinn mit den Sprachlauten zu verbinden. Zunächst erfasst es natürlich nur das Stoffliche; kennt es aber den stofflichen Inhalt der wichtigsten Satztheile, so gewöhnt es sich auch an die \so{Wortfolge}, d.~h. an eine mehr oder minder unveränderliche Ordnung der Vorstellungen. So keimen die Kategorien des Subjects, Prädicats, Objects, Adjectivums u.~s.~w., je nach der Eigenart der betreffenden Sprache: das erste wesentliche Merkmal der menschlichen Rede ist da: die \so{Gliederung}. Diese Sprache hat aber auch \so{lautliche Formenelemente}, es seien \fed{{\textbar}67{\textbar}}\phantomsection\label{fp.67} dies Wortformen oder Formwörter. Wählen wir den \update{schwie\-rigeren}{schwie\-rigen} Fall, den einer Sprache mit reich entwickelter Formenlehre. Das Kind gewöhnt sich daran, dasselbe Wort in verschiedenen Formen und dieselbe Form an verschiedenen Wörtern wieder zu erkennen. Die Sprache sei eine semitische, so gewöhnt sich das Kind daran, die Affixe gesondert aufzufassen und in den Wortstämmen die Consonanten als Träger der materiellen Vorstellungen von den formanzeigenden Vocalen zu unterscheiden: der Triconsonantismus wird im Sprachgefühle wirksam. Weiter: unsre flectirenden Sprachen sind so zu sagen defectiv; kein Wort nimmt alle Formen an, und keine Form ist allen Wörtern gemein, sondern die Formen vertreten sich gegenseitig zum Ausdrucke der nämlichen grammatischen Kategorie. Wir sagen: wehen, wehte, geweht, aber: sehen, sah, gesehen, – gehen, ging, gegangen. Und nun denke man gar weiter an die vielerlei Bedeutungen der Endungen, \textit{o}, \textit{am}, \textit{um}, \textit{is} im Lateinischen, an \update{unsre}{unsere} launenhaften Genusregeln u.~s.~w. Es ist eine Riesenarbeit, dies Alles zu bewältigen, und wohl die meisten Kinder brauchen lange Zeit, ehe sie sich in diesem Wirrsale zurechtfinden. Und doch gelingt manchen auch dies oft unglaublich rasch. Ich habe es selbst erlebt, wie ein noch nicht zweijähriger Knabe seinen Vater fragte: „Papa, hast du mir was mitgebringt? – gebrungen – gebracht!“ So schnell, fast in einem Athemzuge legte das Kind den Weg durch zwei falsche Analogiebildungen bis zur richtigen Form zurück.

\sed{{\textbar}{\textbar}67{\textbar}{\textbar}}\phantomsection\label{sp.67}

Es wäre sicher der Mühe werth, recht reichliches Material zur Kenntniss der Kindersprachen zusammenzutragen. Man sollte meinen, wenn irgend etwas, so müssten jene freien Wort- und Formschöpfungen einen Rückschluss auf den Urzustand der menschlichen Sprachen gewähren, und was der unbeeinflussten Thätigkeit des Kindes erreichbar sei, das müsse es auch den Urmenschen gewesen sein. Nun sind aber diese kindlichen Urschöpfungen so \update{mannich\-faltig,}{mannig\-faltig,} dass man schier daran verzweifeln muss, sich darnach allein eine annähernd richtige Vorstellung von den Sprachanfängen der Menschheit zu machen. Versuche in dieser Richtung tauchen immer und immer wieder in der Literatur auf; sie haben günstigsten Falles den Werth des Romanes, einer Dichtung, die wir wahr nennen, wenn sie Menschenmögliches erzählt.

Anders verhält es sich mit der kindlichen Sprachaneignung, die mir geradezu als vorbildlich gilt für jedes wahre Erlernen eines fremden Idioms. Zwischen verschiedensprachigen Menschen pflegt es schnell zu \fed{{\textbar}68{\textbar}}\phantomsection\label{fp.68} einer Art \update{\so{stummer}}{\so{stummen}}\so{ Verständigung} zu kommen; genug schon, wenn man die Aufmerksamkeit des Fremden durch ein Geräusch auf sich gelenkt hat, – mehr erzielt zunächst das Schreien des Kindes auch nicht. Was nun der Andere will, das merkt man in der Regel leicht, nicht seine Worte versteht man, sondern die Mienen und Gesten, die sie begleiten. Dass man ihn richtig verstanden habe, beweist der Erfolg. Je enger der Gedankenkreis ist, innerhalb dessen man zu suchen hat, je lebhafter Mienen- und Geberdenspiel, je flinker das eigene Auffassungsvermögen, desto besser geht diese Art der Unterhaltung von statten. Dabei fasst denn mit der Zeit der Eine auch Brocken von der Sprache des Anderen auf, die er nun seinerseits gelegentlich verwerthet, – der erste Schritt zur Spracherlernung. Die weiteren Fortschritte geschehen in geometrischer Progression; denn je mehr man schon weiss, desto leichter lernt man hinzu, desto lieber steigert man Kenntniss und Fertigkeit durch fleissiges Reden.

Nun ist es einleuchtend, warum gerade Ungebildete und unter diesen wieder Frauen und Kinder oft am schnellsten in der fremden Sprache heimisch werden. Wie sie sich vom ersten Augenblicke an zu verständigen wissen, ist oft geradezu räthselhaft. Folgendes habe ich selbst erlebt: Eine deutsche Dame brachte bei ihrer Rückkehr in die Heimath eine chinesische Dienerin mit, die nur ihre Muttersprache im Dialekte von Canton und ausserdem das s.~g. Pitchen-Englisch, die chinesisch-englische Mischsprache verstand. Zu Hause fand die Dame eine für sie gemiethete deutsche Kammerjungfer vor, die nur ihre Muttersprache, kein Wort Englisch, geschweige denn Chinesisch kannte. Seit wenigen Stunden war die Chinesin eingetroffen, da sah ich sie mit ihrer neuen Collegin beisammensitzen, scheinbar im Gespräche begriffen. Hinterdrein fragte ich die Deutsche, ob sie sich denn schon mit der Chinesin unterhalten hätte? „Ja wohl“, war die Antwort, „die hat mir schon ihre ganze Lebensgeschichte erzählt. \sed{{\textbar}{\textbar}68{\textbar}{\textbar}}\phantomsection\label{sp.68} Nachher wollte ich ihr etwas vorlesen; aber da passte sie nicht auf, da sang sie immer!“ und nun folgte eine lange Erzählung von den wechselvollen Schicksalen der Asiatin, wie sie geheirathet, ihr Mann sich den Trunk angewöhnt und sie geprügelt, wie sie ihn dann verlassen, erst bei einer amerikanischen Familie, dann bei der deutschen Herrschaft Dienste genommen, wo sie schon alles gewesen, wieviele Kinder sie habe u.~s.~w. Darüber befragte ich die Herrin der Beiden, und diese bestätigte die Geschichte Punkt für Punkt, eine Ge\fed{{\textbar}69{\textbar}}\phantomsection\label{fp.69}schichte, die ausser der A-ma nur ihr bekannt sein konnte. Das hatten also lebhafte Gesten auf der einen Seite, und lebhafte, sicher combinirende Phantasie auf der anderen Seite zu Wege gebracht.

Missionare, die sich zuerst unter einem wilden Volksstamme ansiedeln, mögen oft ähnlich daran sein; denn nicht allemal stehen ihnen für den ersten Verkehr mit den Eingebornen Dolmetscher zur Verfügung, und nur in gewissen Gebieten, z.~B. dem malaio-polynesischen, dem kongo-kaffrischen, können sie sich durch vorherige Aneignung einer nahe verwandten Sprache auf die neue Aufgabe vorbereiten. Mit welchen Schwierigkeiten sie manchmal gerade in sprachlicher Richtung zu kämpfen haben, davon zeugen ihre Berichte: hier scheinen die Laute für eine europäische Zunge unnachahmbar, dort sind ihre Unterschiede so fein, dass jeden Augenblick die garstigsten Missverständnisse entstehen; hier ist die Formenlehre so reich und unregelmässig, dass das beste Gedächtniss sie kaum fassen möchte, – dort ist die Sprache so arm und eng, dass man glaubt, sie erst für die Zwecke der christlichen Lehre \update{zurecht\-modeln}{zurecht modeln} zu müssen, – ein gefährliches Experiment! Überwunden wird aber das Alles und noch vieles Andere, hat doch der Sendbote Jahre und Jahrzehnte lang Zeit zum Lernen und Üben.

Reisende, die von Volke zu Volke wandernd Vocabulare zusammenraffen, sind oft in sehr misslicher Lage. Es kostet oft Zeit und Mühe, den Wilden nur dahin zu bringen, dass er die Namen der Dinge nennt, die man ihm zeigt. Man deutet auf die Waffe, die er bei sich führt, und er sagt: die ist gross oder ist neu, die ist mein, oder die gebe ich Dir nicht; man hält ihm die Hand hin, und er sagt: die ist weiss, oder, schon besser: Deine rechte Hand, Deine Handfläche, Deine Finger u.~s.~w. Das fasst dann der arglose Reisende lautlich auf, so gut er es versteht, und trägt es ein sub rubro Bogen, Messer oder Streitaxt, und Hand. \sed{Vor mehreren Jahren versuchte ich es, einen Melanesier von Pentacosta über seine Sprache abzuhören. Um auch im grammatischen Sinne eine dritte Person und einen Trialis zu haben, zog ich einen Bekannten hinzu. Was „ich“ und „du“, „mein“ und „dein“ heisst, war leicht ermittelt. Kam aber der Dritte in Frage, wies ich auf seine Hände, seine Augen, liess ich ihn singen u.~s.~w., so kehrte immer das Wort \textit{kōpman} wieder, und ich weiss heute noch nicht, ob dies das Pronomen der dritten Person sein sollte, oder ob es das holländische \textit{koopman} = Kaufmann, und dann weiter eine allgemeine Bezeichnung für jeden} {\textbar}{\textbar}69{\textbar}{\textbar}\phantomsection\label{sp.69} \sed{europäischen Herrn war. –} Dazu kommt nun die ganze Schaar derjenigen Missverständnisse, die auf der verschiedenen Natur der beiden Sprachen beruhen. Man versuche nur, gut aus einer europäischen Sprache in die andere zu übersetzen, so wird man gewahr werden, wie wenig sich selbst da die Begriffe decken; und nun stelle man sich vor, man hätte es mit einer Sprache zu thun, die einem ganz anderen Ideenkreise entsprossen ist. Vocabulare, wie wir sie am Schlusse von Reisewerken finden, nehmen sich oft recht stattlich aus; aber was bürgt für ihre Richtigkeit? Man muss solche Sammlungen sehr dankbar hinnehmen, denn sie sind bahnbrechend, aber \fed{{\textbar}70{\textbar}}\phantomsection\label{fp.70} auch sehr vorsichtig, denn die Bahn, die sie eröffnen, ist eine schlüpfrige. Irrthümer, wie ich sie vorhin anführte, finden sich in ihnen oft genug. Die Kunst des Inquirirens will durch Übung erworben sein, – ich wenigstens wüsste nur wenige Rathschläge zu geben.

Vor Allem sehe man den Leuten scharf auf den Mund, wie sie Lippen, Zunge und Zähne stellen, und bilde ihnen die gehörten Laute so gut es gehen will nach. Zur Lautschreibung empfiehlt sich noch immer für den ersten Gebrauch \textsc{Lepsius}’ Standard-Alphabet wegen seiner verhältnissmässigen Einfachheit. Einen Cursus der Lautphysiologie kann man nicht jedem wissenschaftlichen Reisenden zumuthen.

Zweitens verlasse man sich nicht auf eine einmalige Beobachtung, sondern frage wiederholt und mit verschiedenen Nebenumständen, z.~B. mein Auge, Dein Auge, das Auge eines Dritten, die rechte und die linke Hand des Fragers und des Gefragten geöffnet und geschlossen, von aussen und von innen. Dabei wird nebenher schon manches Grammatische mit zum Vorscheine kommen. Alles Gehörte schreibe man auf, sobald man sich überzeugt hat recht gehört zu haben.

Ferner: wo es angeht, sammle man nicht nur Wörter sondern auch Sätze mit möglichst getreuer Angabe des Sinnes oder wenigstens der Umstände, unter denen sie gesprochen wurden. Damit wird weiter dem grammatischen Zwecke gedient, so gut es in der Eile angeht. Das \update{Beste}{Beste,} was der Reisende unterwegs schaffen kann, ist eine reiche Sammlung zuverlässigen Materials, zumal auch zur Grammatik; und er möge sich immerhin sagen, dass daheim, am Studiertische, auch das scheinbar Geringfügigste zu ungeahnter Bedeutung gelangen mag. Linguistische Forschungsreisende wie \textsc{Alexander Castrén}, \textsc{von Uslar}, die beiden \textsc{Radloff}, \fed{\textsc{Mitterrutzner},} \textsc{Reinisch} u.~A. sind noch immer selten; aber Männer wie \textsc{Faidherbe}, \textsc{Heinrich Barth}, \textsc{Georg } \update{[\textit{in den Berichtigungen, S.~502}]: Schweinfurth}{\textsc{Schweinfurt},} \textsc{Brian Houghton Hodgson}, ihres Zeichens Beamte, Geographen und Naturforscher, haben auch Zeit gefunden uns Linguisten reiche Schätze heimzubringen.\footnote{\sed{Ein „Handbuch zur Aufnahme fremder Sprachen“, Berlin 1892, habe ich im Auftrage unseres Colonialamtes verfasst.}}

Ich mass vorhin der Spracherlernung der Kinder vorbildliche Bedeutung bei, und in der That kann ich nur dringend empfehlen, sich beim Eintritte in \sed{{\textbar}{\textbar}70{\textbar}{\textbar}}\phantomsection\label{sp.70} die fremde Sprache möglichst naiv zu verhalten. Man \update{plaudre}{plaudere} flott drauf los und soviel wie möglich; auf ein paar Fehler mehr oder weniger kommt es vorerst gar nicht an. Gesprächige Leute von engem Gedankenkreise sind für den Anfang die besten Lehrmeister. Was und wie sie \update{reden}{reden,} lasse man sich vor \fed{{\textbar}71{\textbar}}\phantomsection\label{fp.71} der Hand als Muster dienen und ahme es nach, bilde sich nicht \update{ein}{ein,} es besser zu können. Alles neu Erlernte verwerthe man so schnell wie möglich praktisch im Gespräche: um so fester prägt es sich ein. Soweit irgend thunlich, lasse man die zu erlernende Sprache ausschliesslich auf sich einwirken, expatriire sich sprachlich, um desto schneller und sicherer das neue geistige Heimathsrecht zu erwerben. Das Stadium, wo man noch aus der Muttersprache in die fremde übersetzen muss, will überwunden sein je eher je lieber: unwillkürlich und unmittelbar müssen sich die Gedanken in der fremden Sprache gestalten, – ein günstiges Zeichen, wenn man erst anfängt in dieser zu träumen. Sprachliche Unarten, die man sich etwa im Verkehre mit Ungebildeten angewöhnt, streifen sich bald in besserer Gesellschaft und durch fleissiges Lesen von selbst ab.

Die Methode, die ich hier in ihren Grundzügen geschildert habe, ist in manchen Gegenden zu einer Art volksthümlicher Einrichtung ausgebildet worden. An Sprachgrenzen ist es auch für die ärmeren Classen wichtig, zweier Sprachen mächtig zu sein, und das hat stellenweise zum s.~g. \so{Kindertausche} geführt. Zwei Familien verschiedener Nationalität geben sich gegenseitig Kinder in Pension, die sich natürlich im pflegeelterlichen Hause schnell die Sprache der neuen Heimath aneignen.

Gegen die \so{zwei- und mehrsprachliche Erziehung} der Kinder in vielen wohlhabenderen Häusern wird von Manchen scharf geeifert: das kindliche Hirn werde überlastet, Zeit und Kräfte könnten besser angewendet werden, Oberflächlichkeit des Denkens und Lernens, wohl gar Gemüths- und Charakterfehler seien die Folgen. Meine Erfahrungen haben nichts von Alledem bestätigt. Ganze Land- und Völkerschaften sind mehr oder weniger zweisprachig, und ich wüsste nicht, dass sie sich von ihren einsprachigen Stammverwandten nachtheilig unterschieden. Wohl alle Deutschrussen sprechen ausser ihrer Muttersprache noch die russische, wohl aller siebenbürger Sachsen ausser ihrem niederrheinischen Dialekte noch Rumänisch, viele überdies Magyarisch, und sie sind wahrlich nicht die schlechtesten ihres Stammes. Das Stammgefühl, wo es wohlgegründet ist, steigert sich oft in der Berührung mit dem Fremden, und der Verstand muss an Vielseitigkeit und Objectivität gewinnen, wenn er gewohnt ist, die Dinge in verschiedenen Sprachen zu durchdenken. Ein neunjähriges Kind sagte einmal: „Es ist doch komisch, wenn ich über eine Sache deutsch oder französisch oder englisch nachdenke: alle\fed{{\textbar}72{\textbar}}\phantomsection\label{fp.72}mal nimmt sie sich anders aus!“ So hatte also die Vielsprachigkeit das Denken des Kindes erweitert und vertieft.

\sed{{\textbar}{\textbar}71{\textbar}{\textbar}}\phantomsection\label{sp.71}

\pdfbookmark[2]{§. 2. B. Durch methodischen Unterricht.}{II.IV.2}
\cohead{§. 2. B. Durch methodischen Unterricht.}
\subsection*{§. 2.}\phantomsection\label{II.IV.2}
\subsection*{B. Durch methodischen Unterricht.}

Indem wir eine Sprache durch mündlichen Verkehr erlernen, vollzieht sich in uns, ohne dass wir es bemerken, ein Process von Associationen und Abstractionen: das Gleichartige schliesst sich zusammen, das Gemeinsame wirkt in uns als Regel, das Ausnahmsweise prägt sich dem Gedächtnisse ein. Über den Erfolg aber \update{entscheidet}{entscheiden} nicht nur Fleiss und Begabung des Lernenden, sondern auch mancherlei Zufall. Was wir unbewusst wirkende Regeln nennen, sind so zu sagen Niederschläge, Sedimente von Erfahrungen. Aus welchen Erfahrungen und in welcher Reihenfolge sie sich bilden, ist eben Sache des Zufalls, und das Gegentheil des Zufalls ist zielbewusste Methode.

Die Lehrmethode ist um so besser, je schneller, sicherer und vollständiger sie dem Lernenden den Unterrichtsstoff überliefert. Darum hat sie sich nicht allein nach diesem Stoffe, sondern auch nach den Fähigkeiten und Neigungen des Lernenden zu richten, nach seiner geistigen Reife, seinem Auffassungsvermögen, in unserm Falle zumal auch nach den Fehlneigungen, die ihm von seiner Muttersprache her anhaften. Und dann verlangen wohl auch seine besonderen beruflichen Bedürfnisse Berücksichtigung; der Diplomat, der Missionar, der Kaufmann haben sich in verschiedenen Gesellschafts- und Gedankenkreisen zu bewegen, aber sie alle wollen die fremde Sprache geläufig reden und richtig aussprechen. Dies hat der Philolog, der sich in seine Bücher vergräbt, nicht nöthig, dafür aber wieder vieles Andere. Alles dies sind unsachliche, daher in Rücksicht auf das Lehrobject unwissenschaftliche Rücksichten. Mithin ist die Methode nicht Sache der Sprachwissenschaft sondern der Pädagogik; der Sprachforscher als solcher hat nicht zu beurtheilen, ob ein Sprachunterricht zweckmässig, sondern nur, ob er sachlich richtig sei.

\largerpage[1]Auch das ist für ihn gleichgültig, ob Lehrer oder Lehrbuch. Man hat papierene Bonnen, die uns die fremde Sprache gemächlich einplaudern, und man hat wohl auch wissenschaftliche Grammatiken von Fleisch und Blut, die sich auf keinerlei Compromiss mit den Schülern einlassen. Solche sind nicht für Jeden die besten Lehrer. Es gehört mehr als \fed{{\textbar}73{\textbar}}\phantomsection\label{fp.73} linguistische Bildung, es gehört auch combinirende Phantasie dazu, sich auf Grund einer theoretischen Darstellung auch nur ein richtiges Bild vom Leben und von der Leistungskraft einer Sprache zu machen, geschweige denn sie praktisch zu bemeistern. Und wem dies alles gelingt, der verdankt es sicherlich mehr den erläuternden Beispielen als den abstracten Lehrsätzen. Mir scheint, dies werde zuweilen übersehen, und vorschnelle Urtheile seien die Folge; unsre Wissenschaft ist reich an warnenden Beispielen dieser Art.

Jede Unterrichtsmethode sollte das Übersetzen auf das nothdürftigste Mass einschränken. Die Muttersprache oder eine andere bekannte brauchen wir na\-\sed{{\textbar}{\textbar}72{\textbar}{\textbar}}\phantomsection\label{sp.72}türlich als Mittlerin der fremden gegenüber; aber diese Mittlerschaft ist ein- für allemale ein Übel, wenn auch ein nothwendiges Übel. Je öfter wir an das Heimische erinnert werden, desto schwerer werden wir im Fremden heimisch. In dieser Hinsicht wurde und wird vielleicht stellenweise noch arg gefehlt. \fed{Die lateinische Grammatik wird in deutscher Sprache vorgetragen, auch nachdem die Schüler genug Latein gelernt haben, um die Regeln in lateinischer Fassung zu verstehen. Darin verfuhren die Humanisten weiser! Beim Lesen der Classiker ist, – war wenigstens – eine Hauptfrage: Wie drückt man das in elegantem Deutsch aus? – als käme es in der lateinischen Stunde darauf an, Deutsch zu lernen.} Mit Exercitien, zumal Extemporalien, wird wohl heute noch vieler Orten wahrer Missbrauch getrieben. Der Lehrer bedarf ihrer nothwendiger als der Schüler, der ihm nur den Beweis liefern soll, dass er die gelernten Regeln und Vocabeln gut inne habe und schlagfertig anzuwenden verstehe. Der Geist des Schülers aber muss eine ganz wunderliche Turnerei treiben, immer hin- und herhüpfen zwischen den beiden Sprachen, in keiner recht zur Ruhe kommen. Das mag eine treffliche Übung sein zu mancherlei anderen Verstandesleistungen, nur gerade für die Spracherlernung ist der Gewinn zweifelhaft. Später, wenn es zum Abfassen freier Aufsätze kommt, wird die bewährte Lehre eingeprägt: Übersetzt nicht aus dem Deutschen, denkt \update{lateinisch!}{in der fremden Sprache!} Das ist aber leichter gesagt als gethan, wenn man auf nichts Anderes eingeschult ist, als auf das leidige Übersetzen. Man hätte die Krücke früher wegwerfen sollen. Und jedenfalls sollte man dies bedenken, dass der Schüler auch im besten Extemporale gerade das gethan hat, was \update{ein guter Lateiner}{ein Eingeborener nun und nimmer thut, was Einer, der die Sprache beherrschen soll,} nun und nimmermehr thun darf. Übersetze ich mit Musse einen mir vorliegenden Text, so kann ich mir \fed{{\textbar}74{\textbar}}\phantomsection\label{fp.74} überlegen, wo und wie in der Übersetzung die Sätze zu verbinden und zu trennen seien, welche Redewendung der Absicht des Verfassers am nächsten komme; am besten ist es, ich nehme den Gedanken meines Schriftstellers recht voll in mir auf, lasse die Sprache des Urtextes in den Hintergrund treten und übertrage nicht sie, sondern unmittelbar den Gedanken in die fremde Sprache. Damit komme ich der frei schaffenden Handhabung des fremden Idiomes, also der natürlichen Sprachanwendung so nahe wie nur möglich und lerne nicht nur Wort mit Wort, Form mit Form, Phrase mit Phrase in den beiden Sprachen zu vergleichen, sondern recht eigentlich Sprachgeist mit Sprachgeist. Hiesse das aber nicht, einem jugendlichen Kopfe zuviel zumuthen? Ich glaube kaum. Die Genus- und Casusregeln, die unregelmässigen Verben u.~s.~w. sind Sache des Gedächtnisses; die Aneignung eines fremden Sprachgeistes dagegen ist Sache der geistigen Schmiegsamkeit, die der Jugend vor Allen eigen ist. Ein Kind mag einen hübsch stilisirten französischen Brief schreiben, der doch von orthographischen und grammatischen Fehlern wimmelt: er ist nicht besser und nicht schlechter als etwa ein Brief einer ungebildeten \update{Französin:}{Französin;} ganz französisch gedacht und gesagt, nur mangelhaft geschrieben. Das Umgedrehte, grammatische \update{Correktheit}{Correctheit} beim hölzernsten Stile, erlebt man eher bei Erwachsenen. Es käme auf den Versuch an, ob sich jene Fähigkeit der Jugend nicht ausnutzen liesse, – man experimentirt ja soviel, auch im Unterrichtswesen.

\sed{{\textbar}{\textbar}73{\textbar}{\textbar}}\phantomsection\label{sp.73}

Bedeutende Verbesserungen in diesem Sinne haben schon jetzt an einzelnen Schulen Eingang gefunden. Quartaner müssen versuchen den Inhalt des eben gelesenen Caesar-Capitels in freier Rede oder Niederschrift wiederzugeben, aus dem Stegreife die oratio obliqua des Textes in die recta zu übertragen oder umgekehrt. So wird \update{so zu sagen}{sozusagen} aus dem Lateinischen in’s Lateinische übersetzt. Beim Übersetzen in’s Deutsche werden die langen lateinischen Perioden in kurze Einzelsätze aufgelöst u.~s.~w. Ob dies Verfahren sich überall praktisch bewähre, ob es nicht schwächeren Schülern zuviel zumuthe, mögen Schulmänner beurtheilen. Wir stehen hier auf dem Standpunkte der Sprachwissenschaft, und von dieser aus muss verlangt werden, dass man vom Unterrichte möglichst Alles ausscheide, was das Einleben des Schülers in die fremde Sprache verzögert.

\begin{sloppypar}Wer Sprachen aus Büchern erlernt, thut \update{gut}{gut,} die fremden Laute, zumal die Vocabeln und Paradigmen, vernehmbar auszusprechen. Beabsichtigt er nicht mündlich in der fremden Sprache zu verkehren, so mag \fed{{\textbar}75{\textbar}}\phantomsection\label{fp.75} er die seinem Organe schwierigen Laute durch bequemere ersetzen, z.~B. sanskrit \textit{bhū} wie behu in „behutsam“, oder das semitische \textit{'Ain} wie ein \textit{ṅ} (=\textit{ng}) aussprechen: arabisch \textit{faṅala}, \textit{faṅalat}, \update{\textit{jafṅulu},}{\textit{jafṅalu},} \textit{fuṅila} = \arabictext{fa`ala}, \arabictext{fa`alat}, \update{\arabictext{yaf`ulu},}{\arabictext{yaf`alu},} \update{\<\setarab \vocalize fu`la>}{\arabictext{fu`ila}}, um \textit{'Ain} von \textit{Alif} und \textit{Hamza} zu unterscheiden. So wirken Ohr und Auge zusammen, um die lästige Gedächtnissarbeit zu fördern.\end{sloppypar}

\pdfbookmark[2]{§. 3. C. Aus Texten}{II.IV.3}
\cohead{§. 3. C. Aus Texten.}
\subsection*{§. 3.}\phantomsection\label{II.IV.3}
\subsection*{C. Aus Texten.}

Es handle sich um eine Sprache, für deren Erlernung uns keine grammatischen noch lexikalischen Hülfsmittel zur Verfügung stehen, in der uns aber Texte vorliegen. \fed{Wir setzen voraus, dass uns die Schrift dieser Texte bekannt sei, und es nun gelte, Grammatik und Wortschatz der Sprache zu ermitteln.} Diese Arbeit stellt, wie kaum eine andere, den linguistischen Takt, den combinirenden Scharfsinn, oft auch die Ausdauer des Forschers auf die Probe und ist ebenso lohnend wie anregend. Wir müssen die Umstände, die ihre Möglichkeit bedingen, einzeln in’s Auge fassen.

Vor Allem bedarf es (ausser der Schrift) mindestens noch einer bekannten Grösse: entweder wissen wir, was der Text enthält, oder wir erkennen die Verwandtschaft der fremden Sprache mit einer uns zugänglichen: oft trifft Beides zusammen, z.~B. bei Bibelübersetzungen in Sprachen malaio-polynesischen oder kongo-kaffrischen (Bantu-)Stammes.

Die Texte, um die es sich hier handelt, rühren meist von \update{Missionären}{Missionaren} her: es sind Übersetzungen von Bibelstücken oder von bekannten Katechismen und gottesdienstlichen Büchern, dann, zumal bei den Katholiken, oft mit gegenüberstehendem Texte in einer europäischen Sprache. Zuweilen sind es auch Volkssagen und Gespräche, denen eine Übersetzung beigefügt ist. Das sind die günstigen, zum Glücke auch die gewöhnlichsten Fälle. Ist dann das Textmaterial nicht \sed{{\textbar}{\textbar}74{\textbar}{\textbar}}\phantomsection\label{sp.74} zu dürftig, die Sprache nicht gar zu schwierig: so wüsste ich nicht, wer vor der Aufgabe zurückschrecken sollte. Ich könnte einen Mann nennen, der ohne jede Anleitung, ja ohne jede höhere Schulbildung, aus blosser Liebhaberei eine ansehnliche Zahl asiatischer und afrikanischer Sprachen auf diese Art aus Bibelübersetzungen erlernt hat.

\fed{{\textbar}76{\textbar}}\phantomsection\label{fp.76}

Vielen Scharfsinnes und einer gewissen Routine bedarf es dagegen, wenn man nur den allgemeinen Inhalt des Textes ahnt. Katechismen sind zuweilen von ihren Verfassern frei entworfen; die zehn Gebote, das apostolische Glaubensbekenntniss, das Vaterunser, in katholischen Büchern dieser Art auch noch einiges Andere, was zu dem Unerlässlichen gehört, erkennt man wohl leicht; das hilft aber nicht weit. Nun entdeckt man mit Hülfe der bisher ermittelten Wörter und etwaiger Eigennamen den Inhalt anderer Stücke, in denen der Stil des Verfassers freieren Spielraum hat, z.~B. Schöpfungsgeschichte, Sündenfall u.~dgl., lernt dabei Neues hinzu, und so reist man „auf der Vetterstrasse“ weiter. Mein verewigter Vater hat in seinen „Melanesischen Sprachen“ durch die That bewiesen, wieviel Gewinn ein geübter Scharfsinn auch aus solchen, oft sehr mageren Quellen zu ziehen vermag.

Die Enträthselung von Texten mit alleiniger Hülfe bekannter Sprachen oder Dialekte ist zumal unsern paläographischen Forschern geläufig. Ihnen verdanken wir die Bekanntschaft mit manchen altgriechischen Dialekten, mit den Schwestersprachen des Lateinischen, dem Altgallischen u.~s.~w. Immerhin ist der Weg von Sprache zu Sprache kein ganz sicherer, weil oft die gleichen Wörter hüben und drüben sehr verschiedene Bedeutungen haben; man denke an englisch \textit{knight}, Ritter, deutsch: Knecht, englisch \textit{knave}, Schurke, deutsch: Knabe. Dann aber sind auch ziemlich nahe Verwandtschaften nicht immer augenfällig. Ein Spanier oder Portugiese wird sich ziemlich leicht in einem italienischen Texte zurechtfinden, – ein Italiener wahrscheinlich schon etwas schwerer in einem spanischen oder portugiesischen. Legt man ihm vollends ein rumänisches Zeitungsblatt vor, so wird er lange an den sonderbaren Conjunctionen, Präpositionen und Casusformen Anstoss nehmen, ehe er wirkliche Verwandtschaft mit dem Italienischen, – mehr als blosse Wortentlehnungen – anerkennt. Und doch steht das Rumänische in manchen Punkten seines Lautwesens dem Toskanischen näher als das Portugiesische.

Es könnte scheinen, fremdsprachigen Texten gegenüber wären wir von allem Anfange an lediglich auf gelehrtes Forschen angewiesen. Dem ist nicht so; auch hier bewährt sich für’s Erste oft ein rein naives Verhalten. Man lese ein paar Seiten, am besten laut oder halblaut, um dem Gedächtnisse auch durch’s Gehör zu Hülfe zu kommen. Um eine richtige Aussprache braucht man sich dabei nicht sonderlich zu bemühen, \fed{{\textbar}77{\textbar}}\phantomsection\label{fp.77} nur unterscheide \update{man}{man,} was verschieden geschrieben ist. Bald wird man gewahr, dass sich Wörter, wohl auch Wortstämme \sed{{\textbar}{\textbar}75{\textbar}{\textbar}}\phantomsection\label{sp.75} und Wortformen \update{wieder\-holen}{wieder\-holen,} und entdeckt gelegentlich deren Bedeutung. So findet sich denn ganz allmählich der Instinkt der Analyse ein, der Text plaudert uns vor, und von Seite zu Seite lernen wir ihn besser verstehen. Wer schnell auswendig lernt und eben nur der Sprache mächtig werden will, der dürfte auf diese Weise rascher zum Ziele kommen, als wenn er nach Forscherart gewissenhaft Collectaneen führte. \sed{Das war die Methode des grossen Autodidakten \textsc{Heinrich Schliemann}. Er bediente sich des Fénélon’schen Télémaque, bekanntlich eines der Bücher, die die meisten Übersetzungen erlebt haben. Den Urtext kannte er; und von der Übersetzung lernte er soviel auswendig, als er brauchte, um der Sprache leidlich Meister zu sein. Das Weitere that dann der Verkehr mit Eingeborenen und die Lectüre von Originalschriften.}

Aus hodegetischen Gründen kann ich nur jedem Sprachforscher empfehlen, dass er sich gelegentlich in dieser Art der Spracherlernung versuche. Wählt man eine fremdgeartete, nicht gar zu schwierige Sprache, etwa eine \fed{der} Bantufamilie, eine malaische, polynesische oder melanesische, eine \retro{uralaltaische,}{uralaltische,} so kann man, auch wenn man sich noch nie zuvor in dem betreffenden Sprachenkreise umgesehen hat, des Erfolges gewiss sein. Dabei empfängt man eine Fülle ganz neuer wissenschaftlicher Anregungen, der Scharfsinn übt sich, und man hat nach kurzer, allerdings trockener Arbeit den Genuss einer stündlich wachsenden, selbsterzeugten Erkenntniss. Zudem sieht man die Fertigkeit in dieser Forschungsmethode durch Übung sich ausserordentlich steigern. Wir wissen, nicht immer ist Sprachtalent mit sprachwissenschaftlicher Befähigung gleichbedeutend. Wird es aber in dieser Schule erzogen, so ist zu erwarten, dass mit ihm zugleich auch die Sicherheit des wissenschaftlichen Urtheils gewinne, denn ganz von selbst gesellt sich zu solcher Praxis auch das theoretische Nachdenken.

\pdfbookmark[1]{V. Capitel}{II.V.Capitel}
\cehead{{\large II,} V. Capitel. Erforschung der Einzelsprache.}
\section*{V. Capitel.}
\pdfbookmark[2]{§. 1. Erforschung der Einzelsprache.}{II.V.1}
\section*{Erforschung der Einzelsprache.}
\cohead{§. 1. Erforschung der Einzelsprache.}
\subsection*{§. 1.}\phantomsection\label{II.V.1}

Bisher handelte es sich um das Erlernen, das heisst um die blosse Aneignung einer bei Anderen schon vorhandenen Kenntniss oder Fertigkeit. Fortan haben wir es mit der Thätigkeit des Forschers zu thun, der neues Wissen gewinnen will. Sein Ziel ist nicht blos Kenntniss, sondern Erkenntniss, d.~h. Einsicht in den Zusammenhang der Dinge.

\fed{{\textbar}78{\textbar}}\phantomsection\label{fp.78}

\sed{An dieser Stelle möchte ich die Sprache nicht nur mit einem Organismus, sondern geradezu mit einer Persönlichkeit vergleichen. Denn es handelt sich} {\textbar}{\textbar}76{\textbar}{\textbar}\phantomsection\label{sp.76} \sed{weniger um eine physiologische, als um eine seelisch-intellectuelle Einheit, um eine Individualität, einen Charakter, der sich in allen, selbst den geringfügigsten Lebensäusserungen bewähren wird, selbst da, wo störende Mächte seine freie Entfaltung verkümmert haben.}

\sed{Hier zeigt sich nun der ganze Ernst der Aufgabe. Die Einzelsprache besteht aus zahllosen Künsten, äussert sich in unendlich mannichfaltigen Redegebilden, die Regeln werden von Ausnahmen durchbrochen, und diese sind für den wissenschaftlichen Beobachter nicht minder wichtig als jene. Mehr noch: innerhalb der nationalen Individualität der Sprache ist der persönlichen Individualität der Sprechenden ein mehr oder weniger weiter Spielraum gelassen, und auch dies, das Mass der Freiheit, gehört zum Charakter der Sprache.}

\largerpage[1]\sed{Jede Sprache verkörpert eine Weltanschauung, die Weltanschauung einer Nation. Sie stellt eine Welt dar, das heisst zunächst die Gesammtheit der Vorstellungen, in denen und über die sich das Denken eines Volkes bewegt; und sie ist der unmittelbarste und bündigste Ausdruck für die Art, wie diese Welt angeschaut, für die Formen, die Ordnung und die Beziehungen, in denen die Gesammtheit ihrer Objecte gedacht wird. Wer sie so versteht, – und nur der versteht sie wissenschaftlich, – zu dem redet durch sie das Volk: Dies ist mein Standpunkt, dies also mein geistiger Gesichtskreis und die Perspective, in der sich für mich die Dinge gruppiren, – und dies ist die Eigenart meines geistigen Auges, womit ich die Welt betrachte, und das sich an und in dieser Welt geschult hat.}

\sed{Es ist nicht hier der Ort, diesen Gedanken weiter zu verfolgen, der für die allgemeine Sprachwissenschaft einer der fruchtbarsten ist. Genug, jede Sprache liefert uns ein ganz individuelles und ganz einheitliches Bild. Was dem grübelnden Scharfsinn so schwer gelingt, ein folgerichtig durchgeführtes System, das hat hier, unbewusst und ungewollt, ein naiver Geist in voller Gesetzmässigkeit geschaffen, einen Riesenbau, dessen kleinster Keim, richtig gedeutet, vom Plane des Ganzen zeugen würde, und dessen Plan nun umgekehrt im letzten Keime nachgewiesen werden sollte.}

\sed{Alle wahrhaft wissenschaftliche Darstellung ist ein Nachschaffen. Der Gegenstand der einzelsprachlichen Forschung ist die Sprache als Rede: die soll aus dem nationalen Sprachvermögen erklärt werden, nachdem dieses, inductiv, aus ihr ermittelt worden ist. Sie hat nicht den Ursprung dieses Vermögens zu erklären, – das ist Sache der allgemeinen Sprachwissenschaft – auch nicht dessen zeitliche Wandelungen zu verfolgen, – das gehört der Sprachgeschichte an, – sondern sie soll dies Vermögen, wie es jeweilig ist, entdecken, beschreiben und bis in die letzten seiner Windungen hinein verfolgen. Sie soll nachschaffen, das heisst die Form ihrer Darstellung der Sache selbst ablauschen. So verstanden, ist ihre Arbeit ebenso schwierig wie reizvoll; zum Scharf- und Tiefsinne des Forschers sollte die Gestaltungskraft des Künstlers kommen.}

{\textbar}{\textbar}77{\textbar}{\textbar}\phantomsection\label{sp.77}

\sed{Es galt hier, wie immer, die Aufgabe zunächst in’s Grosse und rein ideal zu fassen, um erst dann auf die Einzelziele der Forschung und ihre Erreichbarkeit einzugehen. Denn in der That} \update{Die Fälle können...}{können die Fälle} sehr verschieden liegen, je nach der Aufgabe, die ich mir stelle, und nach den Vorkenntnissen, die ich mitbringe. Einen deutschen Schriftsteller des siebzehnten Jahrhunderts liest noch heute jeder Deutsche ohne Schwierigkeit, und bei einiger Übung und Formgewandtheit wird er seine Schreibweise täuschend nachahmen können. Kommt es nur darauf an, das Deutsche des siebzehnten Jahrhunderts auf gewisse Eigenthümlichkeiten hin zu untersuchen, so ist von Hause aus des Unbekannten im Verhältnisse zum Bekannten sehr wenig. Ebenso, wenn wir etwa einen uns geläufigen Dialekt unsrer Muttersprache zum Gegenstande der Untersuchung wählen. Solche Arbeiten bleiben auch besser der historisch-genealogischen Forschung vorbehalten, weil die Vergleichung mit anderen Entwickelungsphasen oder anderen Dialekten der Sprache immer dabei die Hauptrolle spielen wird. Nun setzen wir das andere Extrem: es gelte die Erkenntniss einer uns völlig unbekannten Sprache; da kann die wissenschaftliche Arbeit zugleich mit der praktischen Aneignung beginnen. Diese Arbeit fängt an mit dem zweckbewussten Sammeln, also mit der Anlage und Führung von Collectaneen.

\clearpage\cohead{§. 2. A. Anlegung und Führung der Collectaneen.}
\pdfbookmark[2]{§. 2. A. Anlegung und Führung der Collectaneen.}{II.V.2}
\subsection*{§. 2.}\phantomsection\label{II.V.2}
\subsection*{A. Anlegung und Führung der Collectaneen.}

Es handle sich um eine ganz fremde Sprache, so ist Alles an ihr gleich wichtig, darum gleich sammelnswerth: jedes Wort, jede grammatische Erscheinung, beide in jeder ihrer Bedeutungen und Anwendungen; – Alles will mit gleicher Sorgfalt aufgezeichnet sein, weil Alles noch neu ist. Dies ändert sich mit der Zeit; des Erkannten und Bekannten wird stündlich mehr, und die Gefahr ist dann nur, dass man zu früh im Sammeln nachlässt, weil man für bekannt oder gleichartig hält, was es nicht ist, oder weil man voreilig verallgemeinert hat. Zumal vor letzterem Fehler ist hier wie überall zu warnen, ist es doch recht eigentlich der Fehler der guten Köpfe. Auf Ausnahmen muss man immer gefasst sein, und wo sie auftreten, da gewinnen alle Beispiele der zuvor entdeckten Regel an Werth. Im Deutschen haben die Verben singen, springen, klingen, zwingen, ringen u.~s.~w. gleiche Conjugation; gäbe es nicht die Ausnahmen bringen, umringen, so dürfte sich der Grammatiker mit der Regel begnügen: Verba auf –ingen sind stark und haben die Ablautsreihe \textit{i}, \textit{a}, \textit{u}. Jene Ausnahmen aber verlangen so zu sagen als \fed{{\textbar}79{\textbar}}\phantomsection\label{fp.79} Gegengewicht ein vollständiges Verzeichniss der dieser Regel folgenden Verba.

Das fortwährende Blättern in den Collectaneen und Einschreiben ist aber nicht nur lästig, sondern geradezu hemmend, nachtheilig für den Fortschritt im \sed{{\textbar}{\textbar}78{\textbar}{\textbar}}\phantomsection\label{sp.78} Lernen. Denn es lässt uns nicht dazu kommen, unbefangen in und mit der fremden Sprache zu verkehren, uns in ihr einzubürgern. Doppelter Grund also, das Sammeln zeitweise zu unterbrechen und die durchgearbeiteten Textstücke nochmals cursorisch zu lesen. Meiner Erfahrung nach gelangt man gerade hierbei oft zu neuen, grösseren Gesichtspunkten: die Theile werden vom Ganzen belebt, erleuchtet und erwärmt, – das ahnt man von Anfang an, bald wird man es mit einer Art unbestimmten Behagens empfinden, endlich lernt man es wissenschaftlich verstehen.

\largerpage[1]Wie soll man Collectaneen anlegen und führen? Practica est multiplex, Jeder hat seine eigenen Gewohnheiten, oft so eigene, dass kein Anderer seine Sammlungen fortführen, sie verarbeiten mag. Das sollte und könnte anders sein, wenn die Collectaneen anders wären, wenn die Methodenlehre unsrer Wissenschaft es nicht unter ihrer Würde hielte, sich mit der äusserlichsten Technik zu befassen. Es ist aber sicherlich nicht gleichgültig, ob mir die Verarbeitung meiner Collectaneen einfache, doppelte oder dreifache Zeit kostet, – nicht gleichgültig, ob die Früchte meines Sammelfleisses, wenn ich nicht dazu komme sie zu verwerthen, für Andere brauchbar sind oder auf ewig der Wissenschaft verloren gehen. Dadurch mögen sich die folgenden Andeutungen rechtfertigen.

I. Die Collectaneen zerfallen naturgemäss in lexikalische und grammatikalische, und letztere sind bei einer noch unbekannten Sprache zunächst die wichtigeren. Es gilt aus den Beispielen den Sprachbau zu ermitteln, und zum Verständnisse der Beispiele gehört die Kenntniss von der Bedeutung der Wörter.

II. In der Form der Sammlungen herrscht grosse \update{Mannich\-faltigkeit.}{Mannig\-faltigkeit.} Im Allgemeinen empfiehlt es sich, das Papier nicht zu sparen, deutlich aber klein zu schreiben, damit möglichst viel Gleichartiges auf dem zugemessenen Raume untergebracht werde.

a) Bücher mit weissem \update{Papiere}{Papier} durchschiessen zu lassen ist anfangs bequem, aber nur in wenigen Fällen räthlich. Ein anerkannt tüchtiges, nicht allzu kurzes Buch, etwa die Grammatik oder das Wörterbuch einer vielbearbeiteten Sprache, mag Nachträge, stellenweise Berichtigungen, \fed{{\textbar}80{\textbar}}\phantomsection\label{fp.80} aber voraussichtlich keine weitgehende Umgestaltung verlangen: so stelle ich mir in einem durchschossenen Exemplare eine Art „neuer, vermehrter und verbesserter Auflage“ her. Gilt es, für ein Wörterbuch zu sammeln, worin die fremde Sprache die zweite Stelle einnimmt, so ist es sehr zweckmässig, ein anderes gleichartiges Buch so zu sagen als Massstab zwischenheften zu lassen. Es handle sich z.~B. darum, ein ausführliches deutsch-japanisches Wörterbuch herzustellen: so lasse ich ein ähnlich ausführliches, etwas weitläufig gedrucktes Wörterbuch, etwa ein deutsch-englisches oder deutsch-russisches, durchschiessen, weiss nun, an welcher Stelle ich jeden Eintrag zu machen habe, brauche nicht zu fürchten, dass der Raum nicht zulange, und werde immer an die gebliebenen Lücken gemahnt.

\sed{{\textbar}{\textbar}79{\textbar}{\textbar}}\phantomsection\label{sp.79}

b) Collectaneen in gebundenen, nicht durchschossenen Büchern sind höchstens Reisenden zu empfehlen.

c) Buchähnliche Collectaneen auf losen, gefalteten Bogen sind für grammatische Zwecke brauchbarer als für lexikalische. Ist der Umfang des grammatischen Materials nicht abzusehen, so muss man jedem Rubrum ein besonderes (zweiseitiges) Blatt widmen, um immer neue Blätter einschieben zu können. Steht zu erwarten, dass die Grammatik kurz ausfalle, ist etwa die Sprache besonders einfach oder der Textstoff wenig umfangreich, so kann man zum Vortheile der Handlichkeit mehrerlei auf einer Seite eintragen.

d) Zettelcollectaneen empfehlen sich zumal für lexikalische Zwecke. Nichts ist bequemer sowohl in der Anlage, wie in der Benutzung. Für syntaktische Beispielsammlungen empfehle ich Folgendes: Man trägt die Beispiele, wie sie kommen, untereinander ein auf lange Papierstreifen, die natürlich nur auf einer Seite zu beschreiben sind; dann zerschneidet man die Streifen und vertheilt die Zettel in Briefcouverts mit entsprechenden Aufschriften. Dies Zettelwesen hat das Angenehme, dass man des ewigen Hin- und Herblätterns und Löschens überhoben ist; nur leider verzetteln sich die Zettelchen auch gern!

III. Alle Einträge sind mit genauer Stellenangabe zu versehen. Satzbeispiele sind, am Besten mit \update{Über\-setzung,}{Über\-setzung} voll auszuschreiben. Dadurch erspart man sich das nochmalige Aufschlagen, überschaut das Inductionsmaterial mit einem Blicke und kann nach Befinden Theile der Collectaneen in das druckfertige Manuscript einfügen.

IV. Zunächst ist Reichhaltigkeit und Übersichtlichkeit zu erstreben. \fed{{\textbar}81{\textbar}}\phantomsection\label{fp.81} Die Anordnung der Sammlung ist anfangs ganz unvorgreiflich, das Wörterbuch alphabetisch, die grammatischen Collectaneen etwa in die Capitel unsrer Grammatiken getheilt: Lautlehre, Substantivum, Adjectivum u.~s.~w. Kennt man Verwandte der betreffenden Sprache, so mag man vorläufig deren Grammatik zum Muster nehmen; – doch darauf kommt wenig an, das System soll ja erst gefunden werden. Im Specialisiren kann man nicht leicht zu weit gehen. Vieles, was man anfänglich getrennt hat, schliesst sich mit der Zeit von selbst zu einer Einheit zusammen, und wenn man dann den allgemeinen Lehrsatz ausspricht, so weiss man, dass man ihn bis in seine letzten Folgen beweisen kann.

\cohead{§. 3. B. Prüfung und Ordnung der Collectaneen.}
\pdfbookmark[2]{§. 3. B. Prüfung und Ordnung der Collectaneen.}{II.V.3}
\subsection*{§. 3.}\phantomsection\label{II.V.3}
\subsection*{B. Prüfung und Ordnung der Collectaneen.}
Schon während des Sammelns mögen die Collectaneen nicht nur an Gehalt, sondern auch an Gestalt und Ordnung gewinnen. Vor unseren Augen verknüpfen sich die früheren Beobachtungen mit den neu hinzutretenden, Schranken, die man um einer vorläufigen Ordnung willen aufgerichtet hatte, fallen, neue Kategorien werden entdeckt, die ursprünglichen Capitelüberschriften durch passen\sed{{\textbar}{\textbar}80{\textbar}{\textbar}}\phantomsection\label{sp.80}dere ersetzt, neue Capitel angelegt. Das Wörterbuch, zunächst roh alphabetisch geordnet, lässt Wortstämme, Wurzeln, Bildungselemente erkennen, die in die grammatischen Collectaneen Aufnahme verlangen; diese ihrerseits gerathen je länger je mehr mit sich selbst in Widerspruch, ihre ganze Einrichtung wird entweder Stück für Stück umgestürzt oder im bösen Glauben beibehalten wie eine \update{fable}{Fable} convenue. Macht man nun Schicht mit der Arbeit des Sammelns, will man zur Ausarbeitung gehen, so muss der ganze Stoff noch einmal gesichtet und geordnet werden, und hier zeigt sich der Vortheil jener losen Zettel, Blätter und Bogen, die man verschieben und legen kann wie die Blätter eines Kartenspieles. Dabei ergeben sich wohl auch Mängel, zu deren Abhülfe noch Zeit ist: Beispiele sind – zumal bei Beginn der Forschung – falsch erklärt, am unrechten Orte eingetragen worden; gewisse Erscheinungen hat man über Bedarf begünstigt, andere zur Ungebühr vernachlässigt. Neue Anschauungen werden gewonnen: ein einzelnes Beispiel, das bisher in der Masse der übrigen verschwand, verbreitet plötzlich über eine ganze Gruppe von Erscheinungen ein neues Licht. So kann es kommen, dass man die Arbeit des Lesens und Sam\fed{{\textbar}82{\textbar}}\phantomsection\label{fp.82}melns nochmals aufnimmt und schliesslich ein ganz anderes Buch schreibt, als man sich erst vorgestellt hatte. Beim lexikalischen Sammeln wird man von selbst auf das Neue aufmerksam, weil es eben ein Unbekanntes ist, das man beim Nachschlagen vergeblich sucht. Über neue grammatische Erscheinungen aber schlüpft man nur zu leicht hinweg, wenn sie nicht besondere Schwierigkeiten bieten. Und auch jene Vorliebe für gewisse Theile der Grammatik auf Kosten anderer ist nur zu natürlich.

\sed{Man gewinnt bald der inductiven Arbeit eine Entdeckerfreude ab, die geradezu verführerisch werden kann. Man möchte eben immer Neues entdecken, hat seine Wonne an der Mannigfaltigkeit des Gefundenen und betäubt dabei den Sinn für die Einheit und Einfachheit. Man glaubt, Ausnahmen, Freiheiten, Willkürlichkeiten nachweisen zu können, ehe man ernstlich versucht hat, das seltsam Scheinende den bekannten Gesetzen unterzuordnen. Die meisten Sprachen sind einfacher, folgerichtiger, als sie scheinen, und andererseits doch auch viel feiner und beweglicher, als man nach flüchtiger Betrachtung glauben sollte. Beiden Seiten sollte der Sprachforscher in gleichem Masse gerecht werden; mit dem Zartsinn des Philologen sollte er dem Ausdrucke seine leisesten Abschattungen ablauschen, – und dann sollte er wieder, unerbittlich wie ein Naturforscher auf Gesetzlichkeit dringen, bis ihn sein Stoff ebenso unerbittlich daran gemahnt, dass jenes Geistesleben, das sich in der Sprache äussert, doch ganz anderen Wechselfällen und Launen unterliegt, als die Körperwelt.}

\sed{{\textbar}{\textbar}81{\textbar}{\textbar}}\phantomsection\label{sp.81}

\cehead{{\large II,} VI. Darstellung der Einzelsprache.}
\pdfbookmark[1]{VI. Capitel}{II.VI.Capitel}
\section*{VI. Capitel.}
\section*{Darstellung der Einzelsprache.}
\pdfbookmark[2]{§. 1. A. Die Grammatik.}{II.VI.1}
\cohead{§. 1. A. Die Grammatik.}
\subsection*{§. 1.}\phantomsection\label{II.VI.1}
\subsection*{A. Die Grammatik.}

Sprachkenntniss ist Kenntniss des Sprachbaues und des Sprachschatzes; die Darstellung des Sprachbaues ist Aufgabe der Grammatik, seine Kenntniss nothwendiges Erforderniss des Grammatikers, aber nicht sein einziges Erforderniss.

Sprache ist gegliederter Ausdruck des Gedankens, und Gedanke ist Verbindung von Begriffen. Die menschliche Sprache will aber nicht nur die zu verbindenden Begriffe und die Art ihrer logischen Beziehungen ausdrücken, sondern auch das Verhältniss des Redenden zur Rede; ich will nicht nur \so{etwas} aussprechen, sondern auch \so{mich} aussprechen, und so tritt zum logischen Factor, diesen vielfältig durchdringend, ein psychologischer. Ein dritter Factor kann hinzukommen: die räumlichen und zeitlichen Anschauungsformen. Die innere \so{Sprachform} ist in ihrem grammatischen Theile nichts weiter, als die Auffassung dieser drei Beziehungsarten: der logischen, der psychologischen und der \retro{räumlich-zeitlichen.}{räum\-lichzeit\-lichen. [\textit{ohne Binde\-strich}]} Die Art und Weise, wie diese drei zum Ausdrucke gebracht werden, nennen wir den \so{Sprachbau}. Grammatik nun ist die Lehre vom Sprachbaue, mithin von der Sprachform, der äusseren, das heisst der Ausdrucksform für jene Beziehungen, und also mittelbar insoweit zugleich der inneren Form, das heisst der dem Sprachbaue zu Grunde liegenden Weltanschauung. Jene äussere Sprachform, und mithin auch die innere, ist eine analytische, das heisst der Gedanke wird in seine \fed{{\textbar}83{\textbar}}\phantomsection\label{fp.83} Bestandtheile zerlegt und in diesem zerlegten Zustande zum Ausdrucke gebracht. Der Analyse entspricht als (synthetisches) Ergebniss ein organisch gegliederter Körper, das heisst ein Satz oder ein Satzwort\footnote{Z.~B. lateinisch: dormit, laudantur.}, worinnen das Ganze und die Theile zu einander in Wechselwirkung stehen. Sprachbau ist zunächst Satzbau, und dann natürlich wieder zuhöchst Satzbau.

\largerpage
Wir denken uns eine ideale Grammatik und fragen: wie muss sie beschaffen sein? Vor Allem vollständig, – das versteht sich von selbst. Dann aber auch richtig, und dies begreift etwas mehr in sich, als man wohl gemeiniglich denkt. Nur zu leicht bildet man sich ein, diesem Anspruche genügt zu haben, wenn man die einzelnen Regeln und Erscheinungen der Sprache treu und deutlich wiedergegeben und gebührend mit Beispielen belegt hat. Von der Anordnung des Stoffes verlangt man methodische Zweckmässigkeit und weiter nichts. Wir haben gute Lehrbücher, die ihre Aufgabe so und nur so auffassen. Solche \sed{{\textbar}{\textbar}82{\textbar}{\textbar}}\phantomsection\label{sp.82} mögen in allen ihren Theilen wissenschaftlich sein, – im Ganzen, ich meine als Ganze sind sie es nicht. Denn eine Darstellung ist nur dann wissenschaftlich, wenn sie sachgemäss ist, und wo die Sache ihre Ordnung in sich selbst trägt, da muss die Darstellung dieser Ordnung folgen, sonst thut sie der Sache unwissenschaftlichen Zwang an. Man sollte meinen, es wäre selbstverständlich, dass wenigstens das Nächstverwandte, dessen Zusammenhang auch dem blöden Auge einleuchtet, nicht auseinandergerissen werden darf; aber auch das Selbstverständliche verlangt Verstand. Im Lateinischen regieren bekanntlich \textit{nubo}, \textit{parco}, \textit{benedico} u.~s.~w. den Dativ, während die entsprechenden deutschen Verba accusativisches Object haben. Ich entsinne mich aber in einer weitverbreiteten lateinischen Schulgrammatik jene Verben im Capitel vom Accusative aufgezählt gefunden zu haben. Und dergleichen ist möglich, nachdem, ich weiss nicht wieviele lateinische Grammatiken in, ich weiss nicht \update{wievielen}{wie vielen} Auflagen bei uns zu Lande das Licht der Welt erblickt, und vermuthlich die Nachfolger immer von ihren Vorgängern gelernt haben. Derartige Fehler vermeidet man nun leicht, wenn man einmal darauf hingewiesen \update{worden,}{worden;} und die Abfassung einer Schulgrammatik einer wohldurchforschten Sprache in mehr oder weniger hergebrachter Form ist überhaupt keine sprachwissenschaftliche Leistung. Uns interessirt der Fall, \fed{{\textbar}84{\textbar}}\phantomsection\label{fp.84} wo der Geist des Grammatikers ein neues Gebäude aufzuführen, nicht \update{blos}{bloss} ein altes neu auszutapezieren hat, – der Fall, wo Plan und Form der Grammatik erst gefunden werden soll.

Ich wiederhole es, dieser Aufgabe ist nur der gewachsen, der die Sprache praktisch beherrscht: das wissenschaftliche Kennen, das Erkennen und Beurtheilen setzt ein Können voraus; die wissenschaftliche Darstellung wird nichts Anderes sein, als eine sachgemässe Erklärung dieses Könnens. Es ist dies ein Zustand, dessen sich der Grammatiker bewusst sein muss, und in welchem er sich Eins weiss mit dem Volke, dessen Sprache er lehren will. Somit wird die Grammatik zur Selbstschilderung, zur Selbstbespiegelung und Selbstanalyse. Diese Aufgabe, richtig erfasst, ist eine der schwierigsten, die dem Sprachforscher gestellt werden kann: er hat es nicht mit einem todten Körper zu thun, den er beliebig zerlegen und dann wieder liegen lassen kann, sondern mit der immer beweglichen Seele, mit seiner eigenen Seele; er muss zugleich ganz subjectiv und ganz objectiv sein, denn eben seine Subjectivität wird ihm zum Objecte. Darum nun fragt es sich zunächst: Wann ist sie ein geeignetes Object, d.~h. wann weiss er sich im Besitze des Sprachgeistes? Demnächst aber wird zu untersuchen sein, worin das grammatische Wissen bestehe, wie es also seinem Wesen gemäss dargestellt sein wolle.

\sed{{\textbar}{\textbar}83{\textbar}{\textbar}}\phantomsection\label{sp.83}

\pdfbookmark[2]{§. 2. a. Zeitpunkt zur Selbstprüfung.}{II.VI.2}
\cohead{§. 2. a. Zeitpunkt zur Selbstprüfung.}
\subsection*{§. 2.}\phantomsection\label{II.VI.2}
\subsection*{a. Zeitpunkt zur Selbstprüfung.}

Wer eine fremdartige Sprache erlernt, wird regelmässig Folgendes erleben: Erst findet er sich in der neuen Gedankenwelt nicht zurecht, fühlt sich höchst unheimlich darin. So vieles geschieht da, dessen \update{Sinn und Zweck}{Zweck und Sinn} ihm nicht einleuchtet, und wieder Anderes unterbleibt, was ihm selbstverständlich, unerlässlich dünkt. Ihm ist zu Muthe, wie in einem geborgten Rocke, der ihn hier drückt und zwängt und dort wieder schlottert; oder, dass ich ein anderes Bild gebrauche, denn in Bildern lässt sich hier am besten reden: die eine Sprache däucht ihm wie die ärmliche Hütte eines Wilden, die andere wie ein Museum voll seltsamer Dinge, keine wie ein wohnliches Haus. Er mag hinweggleiten über das, was ihm überflüssig dünkt, mag mit seiner Phantasie ergänzen, was ihm zu fehlen scheint: immer muss er sich sagen: hier geht es nicht nach meinem Sinne her, ich wäre nicht darauf verfallen, es so zu machen! Das nenne ich das Stadium der ersten Verblüfftheit, die der Sprache \fed{{\textbar}85{\textbar}}\phantomsection\label{fp.85} gilt. Sie dauert solange, als man die Gewohnheiten und Vorurtheile der Muttersprache oder irgend welcher anderen Sprachen mit sich schleppt und immer und immer wieder das Neue an dem Altbekannten misst. Es ist der Zustand des Heimwehes, während dessen man auf die neue Heimath schilt, und das eben beweist, dass man noch nicht heimisch geworden ist. Ich weiss nicht, wieviel Antheil daran der Verstand, wieviel der Wille hat: sicher ist, dass \update{Manche}{manche} über diesen Zustand gar nicht hinauskommen. Zu überwinden ist er aber allemal, und auch dabei kommt es auf die richtige Diät an. Man lasse nur die Kritik zu Hause, dränge alle Vergleiche mit dem Heimischen zurück, gebe sich mit ganzer Seele dem Neuen hin, suche es nicht nur zu verstehen, sondern naiv zu geniessen, und misstraue sich selbst, solange man ein Gefühl des Missbehagens in sich verspürt. Denn vielleicht in neun Fällen von zehnen beruht das Missbehagen auf Missverständniss; das fremde Geistesleben, wie es sich in einer Sprache darstellt, mag uns an das Treiben in einer Kinderstube gemahnen, nimmermehr aber an ein Narrenhaus.

Es sind zuweilen die schärfsten Geister, die sich am schwersten zu jenem naiven Verhalten herbeilassen; es sind wohl auch ungeduldige Geister, die den mühsam langen Weg durch die Praxis scheuen und meinen, sie könnten aus einer guten Grammatik ein lebensvolles Bild schöpfen, dürften sich betrachtend verhalten statt erlebend. Beiderlei Geister sind für die grammatische Arbeit noch nicht reif, denn sie sind noch keine geeigneten Objecte zur Selbstanalyse. Und als Subjecte betrachtet, nun so urtheilen die Einen gar nicht, sondern sie nehmen hin, was man ihnen sagt, und die Anderen urtheilen vorschnell.

\largerpage[1]Beiden kann man nur empfehlen: Lasst die Theorie daheim! Wer sich un- \sed{{\textbar}{\textbar}84{\textbar}{\textbar}}\phantomsection\label{sp.84} befangen mitten \update{hinein stürzt}{hineinstürzt} in das Leben und Treiben der fremden Sprache, der wird früher oder später gewahr werden, dass er sich heimisch in ihr fühlt, – oft wie mit einem Schlage, es ist, als wäre die letzte Fessel, die ihn noch hemmte, gesprengt. Jetzt scheint Alles zu kommen, wie er es ahnte, und er meint Alles geahnt zu haben, wie es kam, auch das Neue befremdet ihn nicht mehr. Jetzt weiss er sich der Sprache \so{congenial}, ihr Geist ist ein Stück seines Geistes. Nun fühlt er sich ein zweites Mal überrascht, verblüfft, aber nicht mehr über die Sprache, – die ist nach wie vor dieselbe geblieben, – sondern über sich selbst; denn in ihm muss etwas anders geworden sein. Er hat Bürgerrecht erlangt in der neuen Heimath, und nun darf er sich fragen: wodurch?

\fed{{\textbar}86{\textbar}}\phantomsection\label{fp.86}

\pdfbookmark[2]{§. 3. b. Bestandtheile des grammatischen Wissens; die beiden Systeme.}{II.VI.3}
\cohead{§. 3. b. Bestandtheile des grammatischen Wissens; die beiden Systeme.}
\subsection*{§. 3.}\phantomsection\label{II.VI.3}
\subsection*{b. Bestandtheile des grammatischen Wissens; die beiden Systeme.}

\sed{Man kann noch immer den Ausspruch hören: es gebe Sprachen ohne Grammatik. Die chinesische soll dahin gehören, und Manche, die sie dahin rechnen, gehören zu ihren hervorragendsten Kennern. Diese Ansicht ist allenfalls verzeihlich in einer Zeit, wo ein guter Theil der Grammatiken sich mit der Laut- und Formenlehre begnügt, – was allerdings weniger verzeihlich ist. So gewiss jede Sprache Gesetze hat, nach denen sie sich zur Rede aufbaut, so gewiss hat jede ihre Grammatik. Denn die Grammatik ist eben die Lehre vom Sprachbaue, – nicht bloss von den Bausteinen und dem Mörtel, sondern auch vom Bauplane, nicht bloss von den Redetheilen und ihren etwaigen Formen, sondern auch vom Satze. Eine Grammatik ohne Wortbildungs- und Formenlehre, nur aus Lautlehre und Syntax bestehend, ist möglich, ist sogar nothwendig bei den isolirenden Sprachen, die eine Wort- und Formenbildung nicht oder nicht mehr kennen. Eine Grammatik ohne Syntax ist, streng genommen, ein Unding, jedenfalls nur etwas Halbes; denn sie sagt der Sprache gerade auf dem Punkte Ade, wo die Sprache als Rede in’s Leben treten will. An dieser Stelle, bei der lebendigen Rede, müssen wir einsetzen.}

Ich \so{kann} eine Sprache, das heisst erstens: ich verstehe sie, wenn ich sie höre oder lese, – und zweitens: ich wende sie richtig an, wenn ich in ihr rede oder schreibe. Insofern ich sie verstehe, stellt sie sich mir dar als Erscheinung, oder \update{richtiger}{richtiger,} als eine Gesammtheit von Erscheinungen, die ich deute. Sofern ich sie anwende, bietet sie sich mir als Mittel, oder richtiger als eine Gesammtheit von Mitteln zum Ausdrucke meiner Gedanken. Dort war die Form gegeben und der Inhalt, der Gedanke zu suchen; hier umgekehrt: gegeben ist der Gedankeninhalt, und gesucht wird die Form, der Ausdruck. Dies leuchtet unmittelbar ein und hat längst in der bekannten Zweitheilung der Wörterbücher seinen Ausdruck gefunden: ich lese Latein, stosse auf ein mir unbekanntes Wort, \sed{{\textbar}{\textbar}85{\textbar}{\textbar}}\phantomsection\label{sp.85} und schlage im lateinisch-deutschen Wörterbuche nach. Ich will lateinisch schreiben, es fehlt mir der richtige Ausdruck für diesen oder jenen Begriff: so suche ich nach ihm im deutsch-lateinischen Wörterbuche unter dem entsprechenden deutschen Worte. Nur nebenher will ich hervorheben, dass in beiden Fällen der Weg durch das Mittel der Übersetzung geführt hat, und dass hier die deutsche Sprache etwas Fremdstoffliches, Zufälliges ist, – ein blosser Nothbehelf. Behandle ich die lateinische Sprache richtig, so behandle ich sie mit derselben Unmittelbarkeit wie ein alter Römer; mit der Vorstellung einer Frauensperson z.~B. bieten sich mir von selbst die Wörter \textit{femina}, \textit{mulier} u.~s.~w., ohne dass mir die deutschen Wörter Weib, Frau, Frauenzimmer und wie sie heissen, in den Sinn kommen. Genug, in der Lexikographie ist jene Zweitheilung althergebracht, aber eine Frucht mehr des praktischen Bedürfnisses als einer Einsicht in das Wesen der Sache. Kein Wunder daher, dass die Grammatiker so lange nicht darauf verfallen sind, in ihren Werken etwas Ähnliches einzuführen; kein Wunder darum, dass fast alle vorhandenen Grammatiken zwischen den beiden Gesichtspunkten eine seltsame Zwitterstellung einnehmen; Folgewidrigkeiten aller Art erklären sich daraus.

In der Grammatik ist die Sprache zugleich Gegenstand und Mittel der Darstellung. Als Darstellungsmittel ist sie fortlaufende Rede, und der Lauf der Rede ist bekanntlich geradlinig, also ein Vor und Nach, kein Links und Rechts, kein Oben und Unten. Als Darstellungsgegen\fed{{\textbar}87{\textbar}}\phantomsection\label{fp.87}stand ist die Sprache Vermögen, und für dieses Vermögen wäre der ideale graphische Ausdruck zweidimensional, tabellarisch, sodass man von jedem Punkte aus zwei Reihen überschauen könnte; denn Alles in der Sprache ist zugleich zu deutende Erscheinung und anzuwendendes Mittel. Es handle sich um die Conjunction \textit{quod}, die einen Objectssatz einführt: so muss ich, wenn anders ich das Lateinische grammatisch kenne, mit einem Blicke die übrigen Anwendungen von \textit{quod}, und mit einem zweiten Blicke die übrigen Formen des Objectssatzes überschauen, – ganz wie ich von einem gegebenen Punkte einer Tabelle die Augen jetzt in senkrechter, jetzt in waagerechter Linie dahingleiten lasse.

Diese tabellarische Form einer ganzen Grammatik ist ideal, wird wohl auch aus sehr äusserlichen Gründen ewig ideal bleiben. In unserm Geiste aber ist sie vorhanden, und der Grammatiker sollte auf ein Mittel sinnen, um sie thunlichst zu ersetzen, das heisst, sie in die Form der fortlaufenden Rede umzusetzen. Offenbar kann dies nur in \so{einer} Weise geschehen: er liest, so zu sagen, die Tabelle zweimal ab, das eine Mal der Länge, das andere Mal der Quere nach. So ergeben sich zwei einander nothwendig ergänzende grammatische Systeme: das eine nenne ich das \retro{\so{analytische},}{analytische,} weil in ihm die Spracherscheinungen durch Zerlegung erklärt werden; das andere nenne ich das \so{synthetische}, weil es lehrt die grammatischen Mittel zum Aufbaue der Rede zu verwerthen.
\largerpage

\sed{{\textbar}{\textbar}86{\textbar}{\textbar}}\phantomsection\label{sp.86}

Ich werde hier, das sei wiederholt, die Aufgabe zunächst ideal fassen, also von einer wissenschaftlichen Grammatik reden, die ihren Gegenstand vollständig erschöpft, ihn in der allein durch ihn selbst bedingten Form und Ordnung darstellt und, wie es jede wissenschaftliche Darstellung verlangt, ihre Lehren beweist. Was sich von vorne herein und abgesehen von der Eigenart der einzelnen Sprache bietet, ist zunächst ein weiter dreitheiliger Rahmen, dreitheilig, denn die beiden grammatischen Systeme setzen einen einleitenden allgemeinen Theil, eine Propädeutik, voraus.

\begin{styleAnmerk}
Anmerkung. Auf die innere Nothwendigkeit dieser zwei Systeme habe ich, meines Wissens zuerst, in der Ztschr. f. Völkerpsych. und Sprachwissensch. Bd. VIII, S.~130, und dann in der Ztschr. d. deutschen Morgenl. Ges. Bd. XXXII, S.~634 flg. hingewiesen. In meiner grösseren \retro{chine\-sischen}{chine\-sichen} Grammatik (Leipzig 1881) habe ich versucht, den Gedanken zu verwirklichen. Auf einer im Grunde vielleicht verwandten und doch abweichenden Anschauung beruht \textsc{Steinthal}’s Eintheilung seines geistvollen Buches „Die Mande-Negersprachen“: I. Die Elemente: 1) Lautlehre. 2) Wortlehre. II. Form und Charakter: 1) Die Begriffe. 2) Der Satz. \textsc{Steinthal} \fed{{\textbar}88{\textbar}}\phantomsection\label{fp.88} selbst sagt bei Besprechung meiner chin. Gramm. (Lotus V, pg. 143–144): Je crois aussi qu’il était indispensable d’étudier la grammaire chinoise à ce double point de vue; mais j’aurais fixé d’une autre manière le principe d’où dérivent les deux systèmes. Dann bezieht er sich auf sein Mande-Werk. Dieses ist aber meiner und wohl auch seiner Meinung nach nicht sowohl eine Grammmatik, als vielmehr eine ausführliche Sprachschilderung.
\end{styleAnmerk}

\pdfbookmark[2]{§. 4. c. Die Prolegomena.}{II.VI.4}
\cohead{§. 4. c. Die Prolegomena.}
\subsection*{§. 4.}\phantomsection\label{II.VI.4}
\subsection*{c. Die Prolegomena.}

Alles in der Sprache ist zugleich Erscheinung und Mittel, Erscheinung, die richtig gedeutet, Mittel, das richtig angewandt werden will. Es könnte also scheinen, dass das analytische und das synthetische System in ihrem Zusammenwirken die Aufgabe der Grammatik völlig erschöpften und einem ersten, einleitenden Theile nichts weiter übrig liessen, als was sonst wohl in den Vorreden und Einleitungen zu Grammatiken besprochen wird: die verwandtschaftliche Zugehörigkeit der Sprache, ihre Geschichte und Literatur, Hülfsmittel zu ihrer Erlernung u.~s.~w. Die Laut- und etwaige Schriftlehre müsste dann in einem der zwei Systeme, wo nicht in beiden besprochen werden. Dies scheint mir bedenklich.

\largerpage[1]Es ist natürlich, dass das analytische System dem synthetischen vorangehe; denn man muss die Spracherscheinungen deuten können, ehe man die Sprachmittel anwenden kann. Wo wäre nun im analytischen Systeme die Stätte für die Lautlehre? Zergliedert wird die Sprache als Erscheinung, das ist als Rede. Rede aber ist Satz, und so hat die Analyse vom Satze auszugehen. Folgerichtig schreitet sie vom Ganzen zu den Theilen, also vom Satze zu den Wörtern und Wortformen fort, und erst zuletzt gelangt sie zu den letzten Elementen, den einzelnen Lauten. Dass dies ein Unding wäre, leuchtet ein.

\sed{{\textbar}{\textbar}87{\textbar}{\textbar}}\phantomsection\label{sp.87}

Diesmal indessen brauchen wir uns mit dem \update{argumen\-tum}{Argumen\-tum} ad absurdum nicht zu begnügen, wenn anders Folgendes richtig ist: Es giebt in der Sprache Dinge, bei denen der analytische und der synthetische Gesichtspunkt völlig zusammenfallen: die letzten stofflichen Elemente, die Laute, sind wohl durch Analyse zu entdecken, aber sie sind nicht weiter zu analysiren, – wenigstens nicht vom Grammatiker. Und andrerseits finden sie sich zwar in der Synthese, d.~h. im Zustande der Zusammensetzung vor, dieser Zustand aber ist, wie er sich bietet, lediglich hinzunehmen, die Lautcomplexe sind nachzusprechen, aber sie sind nicht frei \fed{{\textbar}89{\textbar}}\phantomsection\label{fp.89} zu schaffen. Das synthetische System lehre, wie man aus Wurzeln, Stämmen und Formen Wörter, wie man aus Wörtern Sätze aufbaut; wie man aber aus Lauten Wurzeln aufbaut, kann es nicht lehren. Mindestens also hat die \so{Lehre vom Lautbefunde} den beiden Systemen vorauszugehen. Unter dieser Lehre verstehe ich die systematische Aufzählung und Beschreibung der Laute und die Angabe, an welchen Stellen und in welchen Verbindungen sie erscheinen dürfen, die Beschreibung der \so{Accente} wird sich dem anschliessen. Laut- und Tonerscheinungen aber, die durch die grammatische Synthese hervorgerufen werden, sind vielleicht folgerichtiger Weise einer späteren Stelle vorzubehalten. Dahin gehören die Gesetze des Lautwandels (\textit{sandhi}) bei der Anbildung von Formelementen oder beim Zusammentreffen von Aus- und Anlaut benachbarter Wörter, die Lehren vom Satzaccent, vom Accentwandel u.~s.~w.

Mit Recht wenden die neueren Grammatiken der Lautlehre vorwiegende Aufmerksamkeit zu. Unzählige scheinbare Unregelmässigkeiten schwinden, wenn man die zu Grunde liegenden Lautgesetze beherrscht. Wo aber diese Gesetze in der gegenwärtigen Sprache scheinbare Willkürlichkeiten zulassen, da ist anzunehmen, dass die Abweichungen auf einem älteren Zustande des Lautwesens beruhen, und dann mögen diakritische Merkzeichen für’s Auge \update{unter\-scheiden,}{unter\-scheiden} was dem Klange oder der jetzt landesüblichen Rechtschreibung nach gleich ist. Die Declinationen und Conjugationen des Koreanischen z.~B. würden sich durch solche Hülfsmittel sehr vereinfachen lassen. Ebenso die Passiva der polynesischen Sprachen auf \textit{–ia}, wenn man den Wortstämmen ein für alle Male die \update{ver\-schwun\-denen}{ver\-schwun\-denen,} hier wieder zu Tage tretenden Auslautsconsonanten anfügen wollte.

Was nun von den letzten stofflichen Elementen gilt, das gilt wohl auch von den ersten, elementaren Kräften, die den Sprachbau beherrschen. Sie sind nicht mehr zu analysiren, und sie sind auch nicht erst synthetisch herzustellen, sondern sie sind durch Analyse gefunden worden und sollen, wie sie sich bieten, in der Synthese zur Anwendung gelangen. Im allgemeinen Theile meiner chinesischen Grammatik folgt auf die Laut- und Schriftlehre ein letztes Hauptstück mit der Überschrift: die \so{Grundgesetze des Sprachbaues}. Das schien und scheint mir nicht nur zulässig und dem Lehrzwecke entsprechend, sondern \sed{{\textbar}{\textbar}88{\textbar}{\textbar}}\phantomsection\label{sp.88} geradezu durch innere Gründe wissenschaftlich erfordert. Was aber in ein solches Hauptstück aufzunehmen sei, hat sich aus der Natur der behandelten Sprache \fed{{\textbar}90{\textbar}}\phantomsection\label{fp.90} selbst zu ergeben, immerhin jedoch wird es eine kurze grammatische Charakteristik der Sprache darstellen.

\sed{Endlich gilt, streng genommen, das, was ich von den Lauten und Betonungen gesagt, auch von den etwa vorhandenen Formativen, den \so{Mitteln der Wort- und Formenbildung}. Sie sind im Sinne des Sprachbewusstseins elementar, nicht weiter zu zerlegen, nicht synthetisch zu schaffen. Nur ihre Bedeutung und Anwendung bleiben in den beiden besonderen Theilen der Grammatik zu erörtern. Immerhin würde ich im allgemeinen Theile die Formenlehre auf das Elementare beschränken, und wieweit dies reicht, ist nach der Natur der einzelnen Sprache zu beurtheilen.}

Offenbar ist hierbei der naturgemässe Weg \sed{eingehalten:} der vom Ganzen zu den Theilen. \sed{Er ist der naturgemässe; denn} \update{Die erste eigen\-lebige...}{die erste eigen\-lebige} Einheit der Sprache ist nicht das Wort, sondern der Satz; in ihm und durch ihn erhält erst das Wort seinen Werth. Wollte man Jemandem, der noch nie einen Fisch gesehen, erst ein paar Gräten, dann Schuppen, Flossen, Kiemen, Schwimmblase u.~s.~w. zeigen, so würde er die Theile sowenig begreifen lernen, wie das Ganze. Wenn es die Grammatik einer indogermanischen Sprache ähnlich macht, die todten Bruchstücke vorlegt und zerlegt, ehe sie uns eine Anschauung des lebendigen Ganzen gegeben hat: so lassen wir uns das freilich gefallen, weil wir schon vorgreiflich, von der Muttersprache her, ein annäherndes Bild des Ganzen mitbringen.

Allein, es kann nicht oft genug wiederholt werden, \so{die Sprache des Grammatikers und seiner Leser ist der darzustellenden Sprache gegenüber immer eine Zufälligkeit}; ihr einen Einfluss auf die Einrichtung der Grammatik zuzugestehen, mag wohl dem Sprachlehrer erlaubt sein, nimmermehr aber dem Sprachforscher. Den Unsinn wird noch Niemand ausgesprochen haben, dass etwa ein mathematisches oder ein medicinisches oder entomologisches Werk anders angelegt werden müsse, je nachdem es in deutscher, französischer, chinesischer oder arabischer Sprache verfasst würde. Wäre dann der Sprachforscher allein in die Zwangsjacke seiner Muttersprache gebannt?

\pdfbookmark[2]{§. 5. d. Das analytische System.}{II.VI.5}
\cohead{§. 5. d. Das analytische System.}
\subsection*{§. 5.}\phantomsection\label{II.VI.5}
\subsection*{d. Das analytische System.}

\largerpage[1]Der Grammatiker hat sich zuvörderst auf den Standpunkt eines Eingeborenen zu versetzen. Der kann seine Sprache, das heisst: er versteht sie richtig und wendet sie in der Rede richtig an, ohne sich von den Regeln, die ihn dabei leiten, Rechenschaft zu geben. In dieser Hinsicht strebt der Sprachforscher über den Standpunkt des bloss praktischen Sprachkenners hinaus.

\sed{{\textbar}{\textbar}89{\textbar}{\textbar}}\phantomsection\label{sp.89}

Er kann die Sprache, das heisst: er versteht den Sinn jeder Rede und jedes Wortes und findet in ihr für seine Gedanken den entsprechenden Ausdruck; was den Sprachgesetzen zuwider ist, widert ihn an, das Fehlerhafte weiss er zu verbessern. Damit ist schon viel gewonnen, näm\fed{{\textbar}91{\textbar}}\phantomsection\label{fp.91}lich eine Menge bekannter Grössen. Es handelt sich nicht mehr um die Arbeit, vermittels deren wir uns eine unbekannte Sprache aneignen, – das gehörte in’s Capitel von der Spracherlernung, – sondern um jene Arbeit, durch die wir einer uns bekannten Sprache ihre Gesetze abgewinnen, also die praktische Kenntniss und Fertigkeit in theoretische Erkenntniss umsetzen.

Folgendes wird hierbei der regelmässige Vorgang sein: die Entdeckung beginnt mit dem Nächstliegenden, Augenfälligsten; gleiche Erscheinungen mit gleicher Bedeutung schliessen sich so zu sagen in zwei Bündelreihen zusammen. Das Wort \textit{lapidis} z.~B. gesellt sich einerseits zu allen übrigen Casus-Formen von \textit{lapis} und andrerseits zu allen Genitiven Singularis der dritten Declination, die ja gleichfalls die Endung \textit{–is} haben und in gleichen syntaktischen Verbindungen wie \textit{lapidis} erscheinen.

Die gleiche Erfahrung wird mit den Formen des Genitivus Singularis in den übrigen Declinationen gemacht; und nun lehren Congruenzfälle, wie \textit{huius magni lapidis}, alle diese Formen als gleichwerthig erkennen. Andere syntaktische Analogien nöthigen ferner, den Genitiven des Plurals denen des Singulars gleiche Functionen zuzusprechen. So werden die Garben zu Feimen gehäuft, die Erkenntnisse immer weiter und allgemeiner, – die Anschauungen verdichten sich.

Ich kann mir ein Buch denken, das diesen Weg vom Engeren zum Weiteren einschlüge, so eine Art Odyssee, die gar lehrreich und anziehend sein müsste, wenn sie etwa den Leser der Kreuz und Quer durch die scheinbaren Wirrnisse einer besonders schwierigen Sprache hindurchführte. Nur würde ich ein solches Buch eher ein Probestück grammatischer Induction nennen, daran man lernen mag, wie die grammatischen Entdeckungen und Ermittelungen gemacht werden, – als eine eigentliche Grammatik, die bestimmt ist, den Sprachbau so darzustellen, wie er organisch beschaffen ist, nicht so, wie er Stück für Stück dem Forscher aufdämmert, und wie er sich allerdings auch Stück für Stück in der Seele jedes eingeborenen Kindes eingesiedelt hat.

Der analytische Weg ist der Weg vom Weiteren zum Engeren und zwar, wenn er nicht bloss methodisch, sondern systematisch sein soll, in zweifachem Sinne.

Erstens in Rücksicht auf den Stoff. Gegeben ist das Ganze als Erscheinung; daraus werden die Theile geschält. Das Ganze aber, die lebendige Einheit ist der Satz.

\fed{{\textbar}92{\textbar}}\phantomsection\label{fp.92}

\largerpage
Zweitens in Rücksicht auf die Gesetze und Regeln. Die allgemeineren \sed{{\textbar}{\textbar}90{\textbar}{\textbar}}\phantomsection\label{sp.90} haben voranzustehen, und aus ihnen und ihrem Zusammenspiele sind dann, soweit möglich, die besonderen zu erklären.

Jetzt dürfte es einleuchten, dass und warum sich weitere gemeingültige Vorschriften über die Einrichtung eines analytischen Systemes nicht geben lassen. Jede Form des menschlichen Sprachbaues verlangt eine besondere Form und Ordnung der analytischen Grammatik, und so ist es höchstens zu hoffen, dass man dereinst für jeden Sprachtypus einen besonderen Rahmen erfinden werde. Jedenfalls, – wenn anders ich nach meinen eigenen Erfahrungen schliessen darf – ist diese Erfindung überaus schwierig. Die besten Collectaneen helfen dabei doch nur mittelbar, zur Auffrischung des Gedächtnisses. Denn in der That sollte am \update{liebsten}{Liebsten} dem Geiste die ganze Menge des Einzelwissens immer gegenwärtig sein, während er die allgemeinen Gesichtspunkte zu gewinnen strebt. Kann doch manchmal eine scheinbar geringfügige Ausnahme plötzlich über weite Strecken der Theorie ein ganz neues Licht verbreiten.

\largerpage

\begin{sloppypar}Aus dem Begriffe des analytischen Systemes folgt, dass gleichartige Erscheinungen zusammengeordnet werden müssen. Was aber als gleichartig zu gelten habe, darüber entscheidet nicht die Vorgeschichte, die Etymologie, sondern der jeweilig wirkende Sprachgeist. Dieser wird allerdings wohl in den meisten Fällen mit der Etymologie übereinstimmen, aber er thut dies nicht immer. Uns Deutschen z.~B. ist das Gefühl für die Gleichheit \fed{der Conjunction \so{ob} mit der gleichlautenden Präposition,} der Conjunction \so{dass} mit dem Demonstrativpronomen \so{das}, \sed{der Conjunction \so{weil} mit dem Substantivum \so{Weile},} der Präposition \so{nach} mit dem Adverb \so{nahe} abhanden gekommen, weil theils die syntaktischen Functionen theils die Bedeutungen dieser Wörter die Erinnerung an ihren ursprünglichen Zusammenhang verdunkelt haben. Der einzelsprachliche Grammatiker würde also aus der Rolle fallen und sich auf den sprachgeschichtlichen Standpunkt verirren, wenn er diese Wortpaare in seinem Systeme vereinigen wollte.\end{sloppypar}

Meine grössere \retro{chinesische}{chinesiche} Grammatik ist meines Wissens der erste Versuch, die beiden grammatischen Systeme, wie ich sie verstehe, getrennt zu behandeln, und es ist wohl bisher der einzige geblieben.\footnote{Über \textsc{Steinthal}’s Eintheilung seines Buches „Die Mande-Negersprache“, s. o. die Anm. \inlineupdate{87 u. 88.}{S.~86.}} Um also überhaupt ein Beispiel anführen zu können, muss ich mich auf \fed{{\textbar}93{\textbar}}\phantomsection\label{fp.93} sie beziehen. Es zerfällt aber mein analytisches System in folgende vier Haupttheile: 1. Stellungsgesetze. 2. Hülfswörter. 3. Bestimmung der Redetheile. 4. Abgrenzung der Satztheile und Sätze. Die Gründe dieser Anordnung sind nun in Kürze folgende. Die grammatischen Erscheinungen der Sprache wollen aus Gesetzen, diese aus obersten Grundsätzen, – Grundgesetzen – hergeleitet werden. Diese Grundgesetze aber sind Gesetze der Wortstellung, und für die Wortstellung massgebend sind Anfang und Ende \sed{{\textbar}{\textbar}91{\textbar}{\textbar}}\phantomsection\label{sp.91} des Satzes. Darum ist vom begrenzten Satze auszugehen, also anzunehmen, dass die Sätze durch Interpunctionen abgetheilt seien. In den Texten sind sie das oft nicht, die lebendige Rede macht \update{aber}{oft} Pausen, und so ist diese Annahme ganz naturgemäss.

Unter den grammatischen Erscheinungen und Mitteln nehmen die Hülfswörter die zweite Stelle ein, – die zweite, weil ihre Bedeutung und Anwendung unter der Herrschaft der Stellungsgesetze steht und nur aus diesen zu erklären ist.

Nun sind die meisten Stammwörter der chinesischen Sprache dem Functionswechsel auch insofern unterworfen, als sie bald diesen bald jenen grammatischen Redetheil vertreten können: ein Wort „gut“, kann jetzt als Adjectivum, jetzt als Adverb, jetzt als Substantivum, „Güte“, jetzt wieder als verbum factivum oder denominativum, „verbessern, für gut halten“, jetzt endlich als verbum neutrum transitivum, „gut sein gegen Jemand“ dienen. Diese jeweiligen Functionen bestimmen sich nach den Stellungserscheinungen und Hülfswörtern, darum gebührt der Lehre von ihrer Ermittelung der dritte Platz. Dass die rein philologische Kunst der Satzabgrenzung, bei der ausser den grammatischen und logischen noch stilistische Erwägungen mitspielen, an letzter Stelle kommt, bedarf keiner weiteren Rechtfertigung.

Dies als Probe; denn es würde zu weit führen, wenn ich in ähnlicher Weise die innere Eintheilung der einzelnen Hauptstücke begründen wollte. Mag ich nun das Rechte getroffen haben oder nicht: immerhin wird man sehen, wie ich bemüht war, mich einzig und allein von der Natur der Sache leiten zu lassen und der Sprache ihr Gewand auf den Leib zu passen.

Dass man fremdartige Sprachen nicht in das Prokrustesbett der lateinischen Grammatik hineinzwängen dürfe, ist längst und unzählige Male ausgesprochen worden, und die Mahnung hat gute \retro{Früchte}{Früchtc} getragen. Fast überall zeigt sich ein löbliches, oft ein erfolgreiches Streben, \fed{{\textbar}94{\textbar}}\phantomsection\label{fp.94} die Grammatiken nicht-indo\-germa\-nischer Sprachen individualisirend dem Wesen dieser Sprachen nachzugestalten, – und Wesen und Erscheinung gilt in diesem Falle gleich. So hält man gesondert, was die sprachliche Form trennt, und vereinigt das Formgleiche. Gelingt dies, so entspricht die Eintheilung des Stoffes den Anforderungen eines analytischen Systemes. Dieses System verlangt aber, wie wir sahen, überdies noch eine besondere Ordnung der Theile, die ich nirgends eingehalten finde; und darum nenne ich die Systeme jener Bücher, soweit ich sie kenne, gemischte.

Das analytische System will und soll eine wissenschaftliche Darstellung des Sprachbaues in Rücksicht auf seine Erscheinungen sein. Wissenschaftlich soll es aber auch in dem Sinne sein, dass es seine Lehren beweist. Nun trete man mit diesem Anspruche an die erste beste ausführliche Grammatik etwa einer indogermanischen oder semitischen Sprache und frage sich: inwieweit ist ihm \sed{{\textbar}{\textbar}92{\textbar}{\textbar}}\phantomsection\label{sp.92} genügt? Wo findet sich der Nachweis dafür, dass z.~B. das Lateinische soviele und gerade diese Casus, Tempora, Modi hat, keinen mehr und keinen weniger? Dass \retro{die}{sie} und die Formen verschiedenen Klanges, z.~B. \textit{dicam} und \textit{amet}, gleichwer\-thig, jene, obschon von gleichem Klange, wie \textit{dicat}, \textit{amat}, \textit{dolet}, \textit{amet}, verschiedenwerthig sind? In der Regel wird ihn der Verfasser so nebenher, unversehens führen, unausgesprochen und unbeabsichtigt, in den syntaktischen Beispielen. Zuzurechnen ist ihm aber doch nur das, was er mit Wissen und Willen thut, und so bleibt auf alle Fälle sein Verhalten diesen Grundfragen gegenüber ein unwissenschaftliches, dogmatisches. Man hat mir nun Folgendes entgegengehalten: Der einzelsprachliche Grammatiker steht auf dem Standpunkte des Eingeborenen; was diesem in seinem Sprachbewusstsein gegeben ist, das darf auch er als gegeben betrachten. Ich verlange aber eben den Nachweis dafür, dass es gegeben sei. Dem römischen Kinde waren jene Dinge nicht angeboren, sondern es erwarb sie allmählich durch Erfahrungen. Welcher Art Erfahrungen dies waren, das zu zeigen ist Aufgabe des inductiven Beweises, der hier wie überall geführt sein will.

Der Einwand, dass Tadeln leichter sei als Bessermachen, ist ebenso naheliegend wie oberflächlich. Allerdings wäre ich nicht weniger in Verlegenheit, als irgend ein Anderer, wenn ich auf der Stelle ein Schema für das analytische System einer indogermanischen Sprache entwerfen sollte. Sind jedoch meine Anforderungen wissenschaftlich gerechtfertigt, \fed{{\textbar}95{\textbar}}\phantomsection\label{fp.95} so muss ihnen Genüge geschafft werden, mag es noch so viel Kopfzerbrechen kosten. Gilt es der Durchforschung der beiden classischen Sprachen bis in ihre feinsten grammatischen Eigenthümlichkeiten, so sollte man meinen, der Stoff müsse mit der Zeit ausgehen, selbst wenn der Gebrauch jedes Casus, jedes Tempus und Modus und jeder Partikel bei jedem griechischen oder römischen Schriftsteller zum Gegenstande einer Monographie gemacht werden sollte. Denn in der That scheint die philologische Grammatik schon auf’s Krümelsuchen angewiesen. Nun wird sie nicht murren, wenn ihr eine neue Aufgabe gestellt wird, eine philosophische im grossen Stile.

\pdfbookmark[2]{Zusatz.}{II.VI.zusatz}
\cohead{\edins{Zusatz.}}
\subsection*{Zusatz}\phantomsection\label{II.VI.zusatz}

Was sich im Geiste der Menschen mit einem Schlage vollzieht, das hat die Analyse des Grammatikers in seine Theile zu zerlegen. Es handele sich um ein semitisches Wort, etwa um das arabische \arabictext{kitAb} \textit{kitābun}, ein Buch, so empfängt das Gefühl gleichzeitig folgende Eindrücke: 1) den der dreiconsonantigen Wurzel \textit{ktb} = schreiben, 2) den der Vocalfolge \textit{i}—\textit{a}—\textit{u}, 3) den des Rhythmus (Quantität und Accent) \textsubscript{\LARGE \textbreve{ } \'̄ \textbreve{ }}, und 4) den des Afformatives \textit{–ŭn}. Nun sind die \sed{{\textbar}{\textbar}93{\textbar}{\textbar}}\phantomsection\label{sp.93} Wurzeln und die Afformativa wesentlich unveränderlich: bleiben also der Vocalismus \textit{i}—\textit{a} und der Rhythmus \textsubscript{\LARGE \textbreve{ } ̄}. Um die Wirkungen beider auseinanderzuhalten, müsste man natürlich den gleichen Vocalismus durch verschiedene Rhythmen und den gleichen Rhythmus durch verschiedene Vocalismen hindurch verfolgen, etwa nach dem Schema folgender Tafel:

\begin{table}[h]
\centering
\begin{tabular}{r r | l | l | l | l}
 \multicolumn{2}{r|}{Rhythmen:} & \textsubscript{\LARGE \textbreve{ }\textbreve{ }} & \textsubscript{\LARGE \raisebox{1.5mm}[1.5mm]{\parbox{0mm}{\hspace*{0.2mm}\centering´}}\textbreve{ }\textbreve{ }} & \textsubscript{\LARGE \textbreve{ } ̄} & \enspace \textsubscript{\LARGE ̄\textbreve{ }} u.~s.~w. \\
\cmidrule(l{2em}){1-6}
Vocalismen & \textit{aa} & \textit{ăă} & \textit{ắă} & \textit{ăā} & \textit{āă} \\
 & \textit{ai} & \textit{ăĭ} & \textit{ắĭ} & \textit{ăī} & \textit{āĭ} \\
 & \textit{ia} & \textit{ĭă} & \textit{\'ĭă} & \textit{ĭā} & \textit{īă}
\end{tabular}
\end{table}

\noindent u.~s.~w.; denn auch die durch Doppelung des folgenden Consonanten scharfbetonten Sylben erheischen besondere Berücksichtigung. Eine saubere Induction verlangt schlechterdings einen solchen Schematismus. Ich weiss aber nicht, ob sich schon Jemand einer so zeitraubenden Arbeit \fed{{\textbar}96{\textbar}}\phantomsection\label{fp.96} unterzogen hat. Lohnend wäre sie gewiss; denn vorhanden ist das Gesuchte sicher, vielleicht nur schwer zu finden, und wenn es gefunden, erst recht schwer in Worten zu beschreiben. – Übrigens dürfte diese Lehre wohl besser im ersten, allgemeinen Theile, als im analytischen Systeme Platz finden, und sie würde eingehende sprachgeschichtliche Untersuchungen voraussetzen. Die Vorarbeiten hätte jedenfalls die vergleichende Semitistik zu leisten.

\pdfbookmark[2]{§. 6. e. Das synthetische System.}{II.VI.6}
\cohead{§. 6. e. Das synthetische System.}
\subsection*{§. 6.}\phantomsection\label{II.VI.6}
\subsection*{e. Das synthetische System.}

Das analytische System behandelt die Frage. Wie ist die Sprache grammatisch zu verstehen? das heisst: Welches sind ihre grammatischen Erscheinungen? wie sind dieselben organisch zu ordnen? wie sind ihre \update{mannich\-faltigen}{mannig\-faltigen} Bedeutungen einheitlich zu erklären? Gegeben ist also die Erscheinung, und gesucht wird ihre Deutung. Das ist der Standpunkt dessen, der die Rede vernimmt.

Jetzt stellen wir uns auf den Standpunkt des Redenden. Gegeben ist ihm der Gedanke, den er ausdrücken will, und er sucht nach dem richtigen Ausdrucke, – nach dem grammatischen wollen wir sagen; denn nur auf die grammatische Formung, nicht auf die Wahl der Stoffwörter kommt es jetzt an. So ist, um nun den Gegensatz vollends zuzuspitzen, das grammatische Ausdrucksmittel, das der Redende zu suchen hatte, für den Hörenden gegebene grammatische Erscheinung, und der der Rede zu Grunde liegende Gedanke ist für den Hörer als Deutung zu suchen. So verschieden sind die beiderseitigen Stand\sed{{\textbar}{\textbar}94{\textbar}{\textbar}}\phantomsection\label{sp.94}punkte; sie sind geradezu entgegengesetzt. Und ebenso entgegengesetzt sind die Gesichtspunkte, unter denen das analytische und das synthetische System der Grammatik eine Sprache betrachtet. Dieser Gegensatz, richtig verstanden, muss sich in allen seinen Folgen theoretisch nachweisen und praktisch verwirklichen lassen.

Vor Allem treten die Dinge selbst in umgekehrter Ordnung vor die Blicke. Führte früher der Weg vom Ganzen in immer grösserer Specialisirung durch die Theile, so gilt es jetzt, aus den Stoff- und Formenelementen die Theile, aus den Theilen in stätig fortschreitender Erweiterung das Ganze aufzubauen.

Zweitens werden auch die Einzeldinge verschiedene Bilder darbieten, \fed{{\textbar}97{\textbar}}\phantomsection\label{fp.97} jenachdem man sie von der einen oder der anderen Seite, in diesem oder jenem Zusammenhange, aus grösserer oder geringerer Entfernung betrachtet. Denn gleich dem Maler muss auch der Grammatiker zeitweilig von dem Originale und dem Bilde zurücktreten, um zu beobachten, wie Beide „fernen“. Man begreift leicht, wie bei dieser doppelten Betrachtungsweise die Gegenstände in ganz verschiedene Werthscalen einrücken. Es kann etwas als grammatische Erscheinung höchst bedeutsam und schwierig sein, während es als grammatisches Ausdrucksmittel kaum mehr als beiläufige Erwähnung verdient. Und umgekehrt: Dinge, die zu den wirksamsten, feinsten, in der Anwendung schwierigsten grammatischen Mitteln gehören, mögen sich als Erscheinungen zu einer leicht übersehbaren Gruppe zusammenordnen. Mit anderen Worten: es ist manchmal sehr leicht, den richtigen Ausdruck zu finden, und doch sehr schwierig, zu erklären, wie dieser Ausdruck gerade zu dieser Bedeutung kommt. Und manchmal wieder mag es sehr schwierig sein, die richtigste Ausdrucksweise unter der Menge der sich bietenden zu wählen, und ist sie gefunden, so ist ohne Weiteres dem Hörer der Sinn einleuchtend, und der gewünschte Eindruck auf ihn geübt. Dies darf sogar als die Regel gelten, wenn anders Verständlichkeit und Eindringlichkeit der Zweck der Rede ist. Man weiss, wie lange oft \textsc{Lessing} an seinen Sätzen herumgefeilt hat, bis sie zu ihrer classischen Klarheit und Anmuth gereift waren.

\largerpage
Ich nenne das synthetische System eine \so{grammatische Synonymik}. Der Redende will einen Gedanken, vielleicht auch eine Stimmung ausdrücken, er will im Hörer jedenfalls Verständniss, vielleicht auch eine gewisse Stimmung oder Willensneigung erregen. Vieles dabei hängt \update{vom}{von dem} stofflichen Theile der Rede ab, von verdeutlichenden Zugaben, stimmungsvollen oder besonders treffend gewählten Substantiven, Adjectiven, Verben, Adverbien. \textsc{Lenau}’s Gedicht „die drei Indianer“ ist ein Beispiel hierfür. Doch da sind eben die Mittel nicht grammatisch, sondern lexikalisch. Wir aber haben es mit den grammatischen Mitteln zu thun, das heisst mit den Formenmitteln. Dafür, statt längerer abstracter Erörterungen, ein paar Beispiele aus unserer Muttersprache.

\sed{{\textbar}{\textbar}95{\textbar}{\textbar}}\phantomsection\label{sp.95}

Es handle sich um das \so{Object einer Handlung}, so habe ich die Wahl zwischen der activen und der passiven Redeweise.

Es handle sich um die Kategorie der \so{Allheit} oder \so{Allgemein}\fed{{\textbar}98{\textbar}}\phantomsection\label{fp.98}\so{heit}, so kann ich den Ausdruck für sie auf die Seite des Subjectes oder des Prädicates verlegen, wohl auch ihn ganz unterdrücken:

\begin{table}
\centering
\tabcolsep=1mm
\begin{tabular}{l r l}
Ein Fixstern hat & .~~~.~~~.~~~.~~~ & \ldelim\}{5}{3mm}{ } \multirow{5}{*}{eigenes Licht u.~s.~w.} \\
Jeder Fixstern hat & .~~~.~~~.~~~.~~~ \\
(Die) Fixsterne haben & \ldelim\}{2}{3mm}{ } \multirow{2}{*}{haben .~~~.~~~} \\
Alle Fixsterne \\
Fixsterne haben insgesammt
\end{tabular}
\end{table}

Es gelte der Kategorie der \so{Möglichkeit}, so stehen mir Hülfsverben und Adverbien zur Verfügung, wohl auch fragende Wendungen: Sollte etwa ...? \update{u. dgl.}{und dergl.}

Zwei Sätze sollen \update{zueinander}{zu einander} im Verhältnisse des \so{Grundes und der Folge} oder des \so{Bedingenden und Bedingten} stehen, so kann ich diese Verhältnisse durch verschiedene Mittel auf Seiten des Vorder- oder Nachsatzes anzeigen: weil –, wenn –, darum –, dann u.~s.~w. Und zudem habe ich die Wahl, ob ich den begründenden oder bedingenden Satz voran- oder nachstellen will.

In allen diesen Fällen und natürlich noch in vielen anderen hat die Logik der Grammatik Aufgaben gestellt. Aber auch die Psychologie, die Ethik und Aesthetik haben an der Formung der Sprache Antheil.

Aesthetisch, das heisst der sinnlichen Anschaulichkeit dienend, sind z.~B. unsere Diminutive und die vergrössernden, kosenden oder schmähenden Wortbildungen romanischer Sprachen, die \update{mannich\-fachen}{mannig\-fachen} Ausdrücke für örtliche Verhältnisse und viele von jenen Theilen der Stilistik, denen in der Grammatik eine Stätte gebührt.

In das ethische Gebiet gehören alle jene Fälle, wo die gesellschaftliche Stellung des Redenden und des Angeredeten über die Wahl des grammatischen Formmittels entscheidet. Beispiele sind im Deutschen die Pronomina der zweiten Person: Du, Sie und die veralteten Er, Ihr. Unglaublich tief greift im Japanischen und Koreanischen die Etiquette in die Grammatik ein.

Als psychologisch im engeren Sinne möchte ich diejenigen grammatischen Formenmittel bezeichnen, in denen sich das seelische Verhältniss des Redenden zur Rede oder seine Absicht, auf den Angeredeten einzuwirken, kundgiebt. Entschiedenheit oder Unsicherheit des Ausspruches, Erstaunen, Freude, Schmerz oder Furcht und allerhand Neben- und Hintergedanken, die wir auf Augenblicke hinter den Coulissen hervorlugen lassen: sie alle, wenn sie an der grammatischen Formung der \fed{{\textbar}99{\textbar}}\phantomsection\label{fp.99} Rede Theil haben, sind psychologische Modalitäten der ersteren Art. Hier sprechen wir recht eigentlich uns selbst aus, hauchen dem objectiven In\sed{{\textbar}{\textbar}96{\textbar}{\textbar}}\phantomsection\label{sp.96}halte der Rede etwas von unserer Seele mit ein. Frage, Bitte, Befehl, Drohung dagegen gehören zur zweiten Art. Hier versetzen wir uns in die Seele des Anderen und bemessen den Ausdruck nach dem beabsichtigten Eindrucke. Rhetorisch ist Beides. Jene Ausströmungen der eigenen Seele sind es vielleicht ungewollt, aber dafür sind sie um so eindrucksvoller, und manche Sprachen, wie die altgriechische und die deutsche, gestatten ihnen einen weiten Spielraum. Wie zart ihre Mittel sein können, dafür ein Beispiel. Der Leser höre den \textit{A} zum \textit{B} sagen: „Hast Du es auch gelesen?“ Und dann höre er den \textit{C} zum \textit{D} sagen: „Hast Du es nur gelesen?“ Beide Fragen geschahen genau in der gleichen Betonung, der Ton fiel auf \so{gelesen}. \textit{A} und \textit{C} wollen also wissen, ob \textit{B} beziehungsweise \textit{D} ein Buch oder sonstiges Schriftstück wirklich gelesen haben. Wäre es ihnen um das völlige Durchlesen zu thun gewesen, so hätten sie den Ton auf \so{hast} gelegt. Hätte \textit{A} das Wort \so{auch} betont, so wäre der Sinn ähnlich gewesen, wie wenn er gefragt hätte: „Hast auch \so{Du} es gelesen?“ Das heisst, er hätte an andere Leser gedacht. Hätte \textit{C} das Wort \so{nur} betont, so hätte er daran gedacht, dass \textit{D} wohl auch eine Abschrift entnommen oder Dritten Mittheilung gemacht haben könnte. Wie gesagt, nichts von Alledem. Und doch besagen die beiden Wörtchen jedem Verständigen, dass \textit{A} bei seiner Frage einen ganz anderen Nebengedanken gehegt habe, als \textit{C}. \textit{A} hatte nämlich erwartet, dass \textit{B} das Buch lesen sollte; vielleicht hatte er es ihm geliehen oder empfohlen. Und hätte \textit{B} das Buch nicht gelesen, so hatte \textit{A} den Vorwurf in Bereitschaft: Was hat mir nun das Ausleihen oder Empfehlen genützt? \textit{C} dagegen hatte nicht erwartet, dass \textit{D} das Buch gelesen habe. Nun gewinnt er den Eindruck, als müsse das doch der Fall sein, und ist natürlich überrascht. Antwortet nun \textit{D} verneinend, so darf auch \textit{C} mit einem Vorwurfe erwidern. Der lautet aber: Warum stellst Du Dich denn so und versetzest mich in Irrthum?

\largerpage[1]Ich habe die Erklärung dieses Beispieles sehr breit ausgesponnen. Solche Dinge sind aber auch oft fein wie Spinneweben und so durchsichtig, dass man sie selbst kaum sieht. Gewiss kann der Sprachforscher seine Sinne gar nicht genug auf solche Beobachtungen schärfen, und insofern entwächst er sein Lebtag nicht der Schule der classischen Philologen, die hierin die wahrhaft classische ist.

\fed{{\textbar}100{\textbar}}\phantomsection\label{fp.100}

Nun wird auch der Werth des synthetischen Systemes einleuchten. Zunächst der praktische. Nicht nur handelt es sich um die richtige Handhabung der Sprache bis in ihre \sed{letzten} Feinheiten hinein, sondern auch um ihr vertieftes Verständniss. Denn nie kann das Verständniss einer Rede tiefer sein, als wenn man die übrigen Mittel überblickt, deren sich etwa der Redner noch hätte zum Ausdrucke seines Gedankens bedienen können, und sich nun Rechenschaft giebt, wie die sich nach Sinn und Wirkung unterscheiden, und warum unter allen gebotenen Möglichkeiten gerade diese eine zur Thatsache geworden ist.

\sed{{\textbar}{\textbar}97{\textbar}{\textbar}}\phantomsection\label{sp.97}

Aber auch der theoretische Werth für die Beurtheilung der Sprache ist leicht ersichtlich. Denn nun erst, nachdem wir die grammatischen Mittel unter dem synthetischen Gesichtspunkte geordnet haben, lässt sich einsehen, wie reich oder arm, wie fein oder grob der Formenapparat der Sprache, und welchen Richtungen und Zwecken er vorzugsweise zugewendet ist. Zweitens aber, und das habe ich am Chinesischen erprobt, kann das synthetische System zuweilen ein ebenso unerwartetes wie entscheidendes Licht auf räthselhafte Theile des analytischen werfen, zumal auf den genetischen Zusammenhang der grammatischen Erscheinungen. Die Ausdrucksform sei an sich zweideutig, es kann z.~B. ebensogut eine Apposition wie ein Genitiv vorliegen. Es giebt aber für den Genitiv ein lautliches Zeichen (Affix oder Hülfswort), das ausschliesslich ihm, nie der Apposition dient. Nun lehrt mich die grammatische Synonymik, dass in der fraglichen Verbindung auch der Gebrauch des Genitivzeichens zulässig ist, und jetzt weiss ich mit einem Male, wie der Sprachgeist entscheidet.

\largerpage[1]Allein Manches kann für die Theorie nützlich sein, was darum noch nicht in der Theorie begründet ist. Theoretisch begründen heisst in unserem Falle nichts \update{anderes,}{Anderes,} als nachweisen, dass das synthetische System im Sprachgefühle des Volkes selbst wurzele. Ist dies der Fall, so muss ein solches Gefühl sich in Thatsachen äussern. Nun liegt der Einwand nahe: Diese Thatsachen sind ja eben die grammatischen Erscheinungen der Sprache, deren Ordnung und Erklärung im analytischen Systeme abgethan ist. Was bleibt da also für das synthetische übrig? Ist es nicht unwissenschaftlich, zu trennen, was die Sprache selbst buchstäblich und wörtlich für gleich erklärt, und zusammenzuordnen, was sie \retro{ver\-nehmlich}{vor\-nehmlich} auseinanderhält? Der Einwand hat viel Bestechendes und hält doch nicht Stich; die alltäglichste Erfahrung widerspricht ihm. \fed{{\textbar}101{\textbar}}\phantomsection\label{fp.101} \textit{A} sagt dem \textit{B} etwas. Der sagt es weiter dem \textit{C}. Er weiss, dass er sich anderer Ausdrücke \update{bedient}{bedient,} als \textit{A}, weiss jedenfalls nicht sicher, ob er des \textit{A} Rede auch wörtlich wiederholt. Und doch ist er sich vollbewusst, dass er den Sinn jener Rede getreulich wiedergiebt. Wie oft hört man Leute miteinander streiten:

Du hast so und so gesagt.

– Nein, ich habe nicht so gesagt, sondern \update{so!}{so.}

Das kommt aber doch auf Eins heraus.

– Nein, das ist ganz etwas \update{anderes!}{Anderes.}

\noindent Und das kann auch bei Kindern und Ungebildeten zu ganz feinsinnigen Erörterungen über Gegenstände der Synonymik führen. So tief wurzelt das Bewusstsein, dass man denselben Gedanken auf verschiedenerlei Weise aussprechen könne; und dies Bewusstsein ist doch auch ein Theil des Sprachgeistes.

Jetzt könnte man einwenden: Also Streit ist doch möglich, also gar so sicher ist jenes synonymische Gefühl doch nicht! Dagegen gilt zweierlei. Erstens ist auch in den Lautformen der Sprachen nicht Alles so sicher, dass man sagen \sed{{\textbar}{\textbar}98{\textbar}{\textbar}}\phantomsection\label{sp.98} könnte: Das Eine ist richtig, und das Andere ist unrichtig. \textit{A} sagt: „Ich gehe in den Bären zu Tische,“ und \textit{B} lässt die Casusendungen weg und spricht: „Ich gehe in den Bär zu Tisch.“ Sie streiten sich freilich nicht darüber, haben auch gar keinen Grund dazu; denn Einer hat so recht, wie der Andere, und vielleicht spricht morgen Jeder selbst so, wie er es heute vom Anderen gehört hat. Die meisten Menschen halten es mit der Sprache wie mit dem Gelde, achten mehr auf den Werth, als auf das Gepräge, führen in der Regel gültige Münze und streiten nur um die verdächtige. Wortstreite haben freilich noch einen besonderen Reiz als geistige Spiele, und ein Wort ist leichter gefälscht, als ein Geldstück, genügt doch schon ein unsicheres Gedächtniss, um es gegen ein ungleichwerthiges auszutauschen.

Zweitens sind die Werthgrenzen der Synonymen oft etwas verwaschen; Jeder hat seine besonderen Sprachgewohnheiten, die durchaus nicht fehlerhaft zu sein brauchen, und diese Verschiedenheiten betreffen viel öfter den Gebrauch der Wörter und Formen, als ihre äussere, lautliche Gestalt. Es handle sich um den Bedingungssatz. Im Vordersatze bedient sich der Eine lieber und öfter der fragenden Inversion, der Andere der Conjunction \so{wenn}, ein Dritter wohl gar, wo es halbwegs der Sinn zulässt, des nachdrücklichen \so{sobald}. Den Nachsatz bildet der \fed{{\textbar}102{\textbar}}\phantomsection\label{fp.102} Eine ohne einführende Conjunction, ein Zweiter meist mit \so{so}, ein Dritter mit \so{dann}. Die kleinen, feinen Bedeutungsunterschiede sollen hier nicht erörtert werden, sie ergeben sich ja aus den Ausdrücken von selbst. Gerade darum aber entsprechen den Sprachgewohnheiten ebensoviele Denkgewohnheiten, und wo jene falsch scheinen, da sind es im Grunde diese, und die Sprache hat nur ihre Schuldigkeit gethan, indem sie auch den etwaigen Denkfehler zum Ausdrucke brachte. So wahr ist das Wort: Le style c’est l’homme; denn gerade im Bevorzugen gewisser Formen und Wendungen äussert sich sowohl der Stil als auch die Denkgewohnheit des Schriftstellers.

\largerpage[1]Darin liegt nun aber eine Hauptschwierigkeit des synthetischen Systemes, dass man neben und hinter den Denkgewohnheiten den feinen Bedeutungsunterschieden nachspüren muss. Solchen Gewohnheiten huldigt wahrscheinlich Jeder, auch der Vielseitigste, und zwar Jeder nach gewissen Richtungen und auf gewissen Gebieten, und das um so mehr, je reicher die Sprache ist. Man denke z.~B. an unsere Concessivsätze mit obschon, wennschon, obgleich, wenn gleich, wenn auch, obwohl, wiewohl, obzwar, dazu an die Inversionen mit auch, gleich, schon, – und dann an die Wörter im Nachsatze: aber, doch, dennoch, trotzdem u.~s.~w., so hat man für wesentlich dieselbe Gedankenverbindung über ein Dutzend Formen und kann dicke Bücher schreiben, ohne mehr als die Hälfte jener Formen anzuwenden, und ohne dass der peinlichste Sprachkritiker daran etwas zu tadeln fände. Auch zeitweilige Gewöhnungen mögen dabei in’s Spiel kommen. Mir war vielleicht bisher der Ausdruck \so{wiewohl} ungeläufig. Jetzt \sed{{\textbar}{\textbar}99{\textbar}{\textbar}}\phantomsection\label{sp.99} lese ich einen Schriftsteller, der ihn besonders oft gebraucht, und nun bürgert er sich wohl auch in meine Red- und Schreibweise ein, bis ihn etwa bei einer ähnlichen Gelegenheit ein anderer verdrängt.

Eine reiche Sprache wie die unsrige gleicht einem riesigen Arsenale, angefüllt mit unzähligen Werkzeugen, deren immer mehrere annähernd den gleichen Zwecken dienen. Jeder Einzelne aber verfügt nur über eine \update{beschränk\-tere}{beschränk\-te} Werkstatt, ausgestattet mit einer kleineren Anzahl von Geräthen. Nach diesen greift er meist blindlings und ohne viel Besinnen, immer sicher das taugliche zu ergreifen. Eher müsste er sich besinnen, wenn er uns erklären sollte, warum er nun dieses Stück in die Hand genommen und nicht jenes. Er hat aber das Gefühl, dass die verschiedenen Geräthe verschieden wirken, und vielleicht können wir ihm sagen, worin dieses Gefühl beruht. Entweder tragen die verschiedenen \fed{{\textbar}103{\textbar}}\phantomsection\label{fp.103} Synonymen offen den Stempel ihrer etymologischen Herkunft und zeigen somit selbst an, worin sie sich unterscheiden. Das wird der Fall sein bei jenen concessiven Conjunctionen: obschon, obwohl u.~s.~w. Oder die einzelnen sind uns in gewissen verschiedenen Verbindungen geläufig. Den begründenden Vordersatz z.~B. sind wir bei gewissen Gelegenheiten mit \so{da}, bei anderen mit \so{weil} einzuleiten gewöhnt, und nun wirken unbewusste Analogien. Manchmal mag wohl auch bei gleichgültigen Fällen das Bedürfniss nach Abwechselung im Ausdrucke mitspielen, oder es mag im \update{Gegentheile}{Gegentheil} träge Gewohnheit dahin führen, dass man immer wieder zu demselben Mittel greift und am Ende den Gebrauch der übrigen verlernt. Wer es dahin kommen lässt, ist natürlich überhaupt nicht mehr als Zeuge zu gebrauchen. Man sieht aber nun, wie sehr die Philologen Recht haben, wenn sie den Sprachgebrauch einzelner Schriftsteller in gründlichen, weitläufigen Abhandlungen erörtern.

Die Gesammtgrammatik einer Sprache kann sich aus mehr äusserlichen Gründen nicht soweit versteigen. Sie braucht es aber auch aus inneren Gründen nicht, denn sie will ja aus der Sprache aller Einzelnen das gemeingültige Mittel ziehen. Und dass ihr dies möglich, dass sie nicht so ganz dem Zufalle preisgegeben, sondern wie immer an feste Grundsätze gebunden sei: das dürfte nunmehr einleuchten. – Wir wollen versuchen, diese Grundsätze zu entwickeln.

1. Offenbar kommt es hier wie immer auf möglichste Vielseitigkeit des Inductionsmateriales an. Eine grössere Anzahl verschiedener, möglichst von einander unabhängiger Zeugen will vernommen werden. Ist nun der Grammatiker auf eine Literatur angewiesen, so muss er die Bücher im Zusammenhange lesen, sich in ihre Stimmung versetzen und aus dieser Stimmung heraus die vom Schriftsteller gewählten Ausdrucksweisen beurtheilen. Warum z.~B. erzählt Sallustius hier im Imperfectum, da im historischen Perfectum, dort im Infinitiv und dort wieder im Präsens? Und wenn wir erst dem Sallustius auf die Spur gekommen sind, so werden wir voraussichtlich bei den anderen Historikern einem \sed{{\textbar}{\textbar}100{\textbar}{\textbar}}\phantomsection\label{sp.100} ähnlichen Formgebrauche begegnen und mögen getrost ihre etwaigen sprachlichen Vorlieben und Abneigungen auf Rechnung ihrer geistigen Eigenart schreiben: die Sprache ist dieselbe, aber die Sprechenden sind sehr verschieden. Allerdings betrifft dies Beispiel ganz augenfällig den künstlerischen Stil, die Erzählungsweise der Erzähler, die ja nothwendigerweise ihrer Auffassungs- und Empfindungsart entsprechen muss. Und \fed{{\textbar}104{\textbar}}\phantomsection\label{fp.104} Ähnliches gilt z.~B. von den Schlussfolgerungen und Begründungen der Juristen und Philosophen. Überall mag sich in geschlossenen Kreisen ein Handwerks- oder Schulbrauch einnisten, der die freie Entfaltung der Individualität in Schranken bannt. Alles dies muss der Grammatiker einfach hinnehmen, beobachten und verzeichnen. Wo er aber Freiheit sieht, halte er sie nicht für Willkür und Zufall, sondern lausche ihr ihre Gesetze ab.

2. Vor Allem nehme er an, dass eine Sprache, es sei die eines Volkes oder eines Einzelnen, auf die Dauer keinen Überfluss dulde. Zweierlei Ausdrücke für genau denselben Begriff sind aber unnützer Ballast; entweder wird die Sprache das Überflüssige einfach über Bord werfen, oder sie wird das Sinngleiche begrifflich gegeneinander abschatten und so den werthlosen Tand in nützlichen Reichthum verwandeln. Von Beidem weiss die Geschichte zu erzählen. Das Wort so wird jetzt kaum mehr in der Bedeutung eines Relativpronomens angewendet, und statt „wenn“ dient es höchstens noch in drohender Rede. Wer die beiden Imperfectformen von werden, ward und wurde, abwechselnd gebraucht, wird wohl die erstere mehr momentan, die zweite mehr durativ oder inchoativ verstehen: „Er sprach: Es werde Licht! und es ward Licht.“ Aber: „Es wurde viel gezecht, und er wurde gesprächig.“

3. Wie angedeutet, achte man auf die Etymologie, nehme an, dass diese, wo immer sie augenfällig ist, auch in der Seele des Redenden wirke. So müssen sich aus den Bedeutungsunterschieden zwischen \so{ob} und \so{wenn}, \so{schon} und \so{gleich} jene zwischen \so{obschon}, \so{wennschon}, \so{obgleich}, \so{wenngleich} ergeben.

4. Auch auf allgebräuchliche Redensarten habe man Acht; denn sie wirken gern vorbildlich nach dem Gesetze der Analogie. Gelten zwei Wörter in gewissen Fällen als Gegensätze, so kann es geschehen, dass sie sich nun in alle Verzweigungen ihrer Grundbedeutung hinein, sozusagen als Gegner verfolgen. Diese Beobachtung habe ich zumal im Chinesischen gemacht, und sie ist mir oft zu Statten gekommen. Ich möchte ihr aber Allgemeingültigkeit zusprechen, denn sie entspricht einem psychologischen Triebe, \update{Alles}{Alles,} was sich öfter berührt hat, mit zarten Fäden verbunden zu halten.

Für die Eintheilung des synthetischen Systemes ein im Wesentlichen gemeingültiges Schema zu finden, müsste wohl möglich sein. Ein solches müsste thunlichst Fächer für Alles und Jedes enthalten, was eine Sprache durch grammatische Mittel ausdrücken mag, einerlei wieviele dieser Fächer \fed{{\textbar}105{\textbar}}\phantomsection\label{fp.105} von der jeweilig zu behandelnden Sprache ausgefüllt werden. Mit diesem Vorbehalte und nur \sed{{\textbar}{\textbar}101{\textbar}{\textbar}}\phantomsection\label{sp.101} als das einzige mir zugängliche Beispiel glaube ich mittheilen zu sollen, wie ich das synthetische System meiner chinesischen Grammatik eingerichtet habe. Man bedenke aber zudem: es war ein erster Versuch, der gewiss noch manche Nachbesserung zulässt.

In der Einleitung gebe ich nach einer vorläufigen Verständigung über Zweck und Einrichtung des Systemes allgemeine Anweisungen über die \so{Wahl des Ausdruckes}, z.~B. über die Vorliebe des Chinesen für Kürze und Abstractheit der Redeweise.

Das Weitere zerfällt in vier Hauptstücke: I. Die Satztheile. II. Der einfache Satz. III: Der zusammengesetzte Satz. IV. Stilistik.

Zu I. In Rücksicht auf die \so{Satztheile} war zu fragen: 1. Wie werden sie, z.~B. die Substantiva, Adjectiva, Verba u.~s.~w., \so{gebildet}?

2.  Wie können sie \so{erweitert} werden? Die Antwort lautete:

\noindent A. Durch nähere \so{Bestimmungen}, und zwar entweder

a) \so{adnominale}: Apposition, Genitiv, Adjectiv, adjectivisches Participium, – oder

\largerpage[-1]b) \so{adverbiale}: Adverbien und adverbiale Redensarten für Zeit, Ort, Art und Weise; adverbiale Beziehungen der Substantiva, insbesondere durch Präpositionen und Postpositionen. Das Objectsverhältniss den adverbialen Attributen zuzuordnen, schien nach dem Geiste des Chinesischen unzulässig. Zudem ist aus logischen Gründen die Lehre von Subject und Object nicht wohl von jener über das Genus verbi zu trennen. – Die Eigenart der Sprache bestimmte mich nun,

c) der Zahl, Einheit, Vielheit, Allheit u.~s.~w. ein besonderes Capitel zu widmen.

d) \update{\so{Coordination}.}{\so{Coordination},} \sed{cumulative und alternative.}

\largerpage
3. Die dritte Frage lautete: Durch welche Mittel werden Satztheile \so{ersetzt}? Hier waren die Pronomina zu behandeln. Es ist aber denkbar, dass andere Sprachen noch Ersatzwörter anderer Art haben, demonstrative, interrogative, relative Proadverbien, z.~B. da, so, wie, wo u.~s.~w., – auch wohl Proverba von der Bedeutung:

\begin{center}
\begin{tabular}{l l}
dies \rdelim\}{3}{3mm}{ } & \ldelim\{{3}{3mm}{ } \multirow{3}{5mm}{sein thun} \\
so \\
wie?
\end{tabular}
\end{center}

\noindent – wie das fragende \textit{ainambi} des Mandschu.

\fed{{\textbar}106{\textbar}}\phantomsection\label{fp.106}

4. Endlich ist zu fragen: Wann dürfen und wann sollen Satztheile \so{weggelassen} werden? Darauf antwortet die Lehre von den Ellipsen und Kürzungen.

Zu II. Das Hauptstück vom \so{einfachen Satze} habe ich in drei Capitel getheilt: 1. Subject, Prädicat, Object. 2. Psychologisches Subject, Inversionen. 3. Copula, Modalität.

\sed{{\textbar}{\textbar}102{\textbar}{\textbar}}\phantomsection\label{sp.102}

1. Das Capitel über \so{Subject}, \so{Prädicat}, \so{Object} umfasst die Lehre von den zwei wesentlichen Bestandtheilen des einfachen Satzes: dem Subjecte und dem Prädicate, und den Erweiterungen dieses letzteren durch directe und indirecte Objecte, mithin nach der Ausdrucksweise unserer Grammatiken folgende Formen des Verbum finitum: Neutrum, Activum, Passivum, Reflexivum, Reciprocum, Causativum, Factivum und Denominativum, sowie folgende Casus: den Nominativ, Accusativ, Dativ und den sogenannten Instrumentalis für den Urheber eines passiven Verbums.

Die Grenze zwischen diesem Capitel und jenem von den adverbialen Attributen ist meiner Meinung nach je nach der Eigenart der Sprache verschieden zu ziehen und mag manchmal schwer zu finden sein. Nach dem Geiste wohl der meisten Sprachen ist das Object nichts weiter als eine Unterart des adverbialen Attributes und als solche dem Instrumentalis und den örtlichen Casus nebengeordnet. Aus logischen Gründen aber lässt sich das Object nicht wohl vom Subjecte und von dem Genus verbi getrennt behandeln. Es giebt ja auch Sprachen, die das, was wir am Verbum ausdrücken, dem Substantivum zuschieben und geradezu einen Casus Activus und einen Neutro-Passivus haben. Das Tibetische ist ein Beispiel hierfür. \sed{Da heisst: \textit{de skad bdag-gis t‘os-pa dus-gčig-na} wörtlich: „Diese Sage (\textit{skad} ohne Suffix, daher neutro-passiv) ich (\textit{bdag}) act. Instr. (\textit{-gis}) hören (\textit{t‘os-pa}) Zeit (\textit{dus}) einer (\textit{gčig}) in (\textit{-na})“ = „Diese Sage habe ich einst gehört“, oder: „Diese Sage ist einst von mir gehört worden.“ Ähnlich in einer südaustralischen Sprache (Encounter Bay): \corr{1901}{\textit{Korn’}}{Korn’} (Mann) \textit{-il} (act. instr.) \textit{lakk-in} (Aufspiessung) \textit{mãme} (Fisch, neutro-pass., weil ohne Suffix) = „Der Mann spiesst den Fisch auf“, oder: „vom Mann wird der Fisch \corr{1901}{aufgespiesst“}{aufgespiesst} (H. A. E. Meyer, Vocabulary ... \textit{preceded by a Grammar. Adelaide 1843}, Seite 38–39). Auch das Baskische unterscheidet zwischen activem und nicht activem Subjecte; das Suffix des Ersteren, –k, bezeichnet auch den Urheber eines passiven Verbums.} Andrerseits vermögen Sprachen des malaiischen Stammes nicht nur das logische Object, sondern auch Ort und Werkzeug der Handlung mittels entsprechender Passiva zu Subjecten des Satzes zu machen, und im Chinesischen selbst können gewisse Verba örtliche oder ursächliche Objecte haben und z.~B. die Bedeutungen von verweilen und bewohnen, weinen und beweinen in sich vereinigen. Es ergeben sich da für die Anordnungsfrage gewisse theoretische Zweifel, die aber für die Sache kaum erheblich sind.

2. Das zweite Capitel trägt die Überschrift: \so{Psychologisches Subject, Inversionen}. Es handelt sich hier um die Fälle, wo die gemeingültigen Stellungsgesetze durchbrochen, wohl auch mittels besonderer Constructionen umgangen, oder wo doch, unbeschadet dieser Gesetze, einzelne Satztheile besonders hervorgehoben werden sollen. Dazu \fed{{\textbar}107{\textbar}}\phantomsection\label{fp.107} können mancherlei Gründe veranlassen. Es mag gelten, das psychologische Subject, – gleichviel ob dasselbe auch zu\sed{{\textbar}{\textbar}103{\textbar}{\textbar}}\phantomsection\label{sp.103}gleich grammatisches Subject ist oder nicht, – als Gegenstand der Rede zu kennzeichnen oder auch sonst einen Satztheil nachdrücklicher hervorzuheben, als dies bei seiner gewöhnlichen Stellung im Satze möglich sein würde. Oder es soll eine schleppende oder undeutliche Satzbildung durch eine besser gegliederte oder durchsichtigere ersetzt werden. Oder endlich mag es auch anmuthiger erscheinen, ab und zu in die Eintönigkeit des Satzbaues Abwechselung zu bringen. Vieles davon werden wir später noch näher zu betrachten haben, und so möge einstweilen der Leser an gewisse Erscheinungen im Französischen denken, \update{z.~B.}{z.~B.:}

Hier était le vingt-deux – Le vingt-deux était hier.

Votre frère, je viens de le voir – Je viens de voir votre frère.

Des cigares, en voici – Voici des cigares.

3. Das Capitel: \so{Copula}, \so{Modalität} vereinigt in sich die Lehre von der logischen Modalität, das heisst von dem Verhältnisse des Prädicates als eines bejahenden, verneinenden, thatsächlichen, möglichen, nothwendigen, ausschliesslichen u.~s.~w. zum Subjecte, – und zweitens die Lehre von der psychologischen Modalität, das heisst von der Beziehung des Redenden zur Rede, ob er mittheilt, fragt, ausruft, befiehlt oder bittet, ob er mit Entschiedenheit oder mit bescheidener Zurückhaltung, vermuthend, fürchtend, hoffend, zweifelnd spricht.

a) Das Chinesische hat nun eine Anzahl allgemeiner Modalausdrücke, die ich zunächst behandele. Dann folgt

b) Ja und Nein.

c) Prädicat des Seins.

d) Possessives Prädicat. 

e) Ursächliches Prädicat. Beide Letztere gehören zu den Eigenthümlichkeiten des Chinesischen. Ferner

f) Wörter für Sein und Werden.

g) Negationen.

h) Müssen, sollen, können.

i) Vorhaben, wollen, wünschen.

k) Perfectum. Perfectum und Futurum sind nämlich im Chinesischen modal.

l) Auch, noch.

m) Nur.

\fed{{\textbar}108{\textbar}}\phantomsection\label{fp.108}

n) Wie.

o) \retro{Compa\-rativ.}{Compa\-rarativ.}

p) Superlativ.

q) Befehl, Bitte.

r) Frage- und Ausrufesätze.

Zu III. Das Hauptstück vom \so{zusammengesetzten Satze und den Satzverbindungen} bespricht zunächst in der Einleitung die Mittel, welche \sed{{\textbar}{\textbar}104{\textbar}{\textbar}}\phantomsection\label{sp.104} die Sprache besitzt, um den logischen Zusammenhang der Gedanken darzustellen, dann die Mittel, Sätze in Satztheile zu verwandeln. Es folgen nun die Capitel:

1. Subjects-, \update{Praedicats-}{Prädicats-} und Objectssatz.

2. Adnominalsätze einschliesslich der substantivischen Relativsätze.

3. Adverbialsätze, Conjunctionen. Hier werden wieder

~~~~ a) die allgemeinen Formenmittel besprochen; dann

~~~~ b) Umstand,

~~~~ c) Zeit,

~~~~ d) Grund und Absicht,

~~~~ e) Bedingung,

~~~~ f) Causalverhältnisse,

~~~~ \update{[\textit{\mbox{in den} Berichti\-gungen, S.~502}]\newline g.~Conces\-sivverhält\-nisse. (Dann):\newline h. Fort\-setzung, Steigerung.}{g)    Concessivverhältnisse,}

~~~~ \inlineupdate{g)}{h)} Fortsetzung, Steigerung,

\largerpage[1]~~~~ \inlineupdate{h)}{i)} besondere Formen der Coordination. Endlich recapitulirend:

~~~~ \inlineupdate{i)}{k)} Synonymik einiger Conjunctionen.

Zu IV. \so{Stilistik}. Es ist nicht leicht, die Grenze zwischen Grammatik und Stilistik zu ziehen. Irre ich nicht, so haben Beide ausser ihren Sondergebieten und mitten zwischen diesen liegend noch ein grosses gemeinschaftliches Areal, das der Grammatiker bebauen, an dessen Früchten aber der Stilistiker den Mitgenuss haben sollte. Die Sache verdient und verlangt eine besondere Betrachtung, denn sie geht nicht das synthetische System allein an.

\pdfbookmark[2]{§. 7. Zusatz I. Stilistik und Grammatik.}{II.VI.7}
\cohead{§. 7. I. Stilistik und Grammatik.}
\subsection*{§. 7.}\phantomsection\label{II.VI.7}
\subsection*{Zusatz I.}
\subsection*{Stilistik und Grammatik.}

Eine Sprache richtig anwenden heisst: sie so anwenden, wie dies von den Eingeborenen geschieht. Wird dies mit Bewusstsein erstrebt, so ist es geradezu Nachahmung. Jemand nachahmen heisst: seine Eigen\fed{{\textbar}109{\textbar}}\phantomsection\label{fp.109}thümlichkeiten zur Darstellung bringen. Geschieht dies in übertriebener Weise, so artet die Nachahmung in Caricatur aus. Geschieht es in unzulänglichem Grade, so bleibt die Nachahmung matt, wirkungslos. Ahmen wir endlich einen Anderen in einer Lage nach, die seinem Wesen zuwider ist, so mag das Bild noch so getroffen sein: es befriedigt nicht, denn es macht nicht den Eindruck des Natürlichen, Typischen.

In der Lage des Nachahmers befinden wir uns auch der Sprache gegenüber, solange sie uns nicht zur zweiten Natur geworden ist. Nachahmung setzt Beobachtung voraus, und diese ist Sache des Grammatikers.

Nun handhabt zwar ein Jeder seine Muttersprache in der Regel richtig, aber doch in einer besonderen, ihm eigenen Weise, bevorzugt unter den ver\-\sed{{\textbar}{\textbar}105{\textbar}{\textbar}}\phantomsection\label{sp.105}schie\-denen, sinnverwandten Ausdrücken (Wörtern, Formen, Redewendungen) die einen zum Nachtheile anderer, vielleicht zutreffenderer, bewegt sich lieber in kurzen als in längeren, lieber in abgerissenen als in verbundenen Sätzen, lieber in begründenden als in folgernden Gedankenreihen, macht häufigen oder auch gar keinen Gebrauch von rhetorischen Fragen u. dgl. mehr. Alles dies nenne ich seinen Stil; und in diesem Sinne rede ich auch vom Stile eines Schriftunkundigen, eines Kindes oder einer Bäuerin. Es ist derjenige Stil, von dem man sagt, er sei der Mensch. Er verhält sich zur nationalen Sprache, wie das Kleidungsstück, das einem Einzelnen auf den Leib geschneidert ist, zur nationalen Tracht, die ein solches Kleidungsstück verlangt oder erlaubt.

Um im Bilde zu bleiben: die Nationaltracht verlangt, gestattet oder verpönt nicht nur gewisse Kleidungsstücke, sondern auch diesen und jenen Schnitt, diese oder jene Farben. So sind auch der Stilfreiheit Grenzen gesetzt, Wege vorgezeichnet, jetzt durch den Geschmack, jetzt durch die Denkgewohnheiten, wohl auch durch das Verständnissvermögen des Volkes. Nicht Alles, was nach den Gesetzen und mit den Mitteln einer Sprache möglich ist, ist in ihrem Sinne gut, das heisst national, das heisst schliesslich doch richtig, auch im streng sprachlichen Sinne richtig.

Hier zeigt sich das Bedenkliche jener Übersetzungsliteratur, auf die wir so oft als einzige Quelle angewiesen sind. Da werden den Völkern aller Erdtheile und Farben in ihren Sprachen Dinge vorgetragen, die weit jenseits ihres geistigen Gesichtskreises liegen, Gedankenoperationen werden ihnen zugemuthet, an die sie nicht gewöhnt, zu denen sie viel\fed{{\textbar}110{\textbar}}\phantomsection\label{fp.110}leicht gar nicht befähigt sind. Ihre Sprache freilich giebt sich dazu her. Als man noch mit Gänsekielen schrieb, merkte man es bald, wenn Jemand eine fremde Feder führte: da pflegte die gemisshandelte zu schreien. Es giebt einen sprachwissenschaftlichen Instinct, der es schnell empfindet, wenn eine Sprache anders gehandhabt wird, als sie es gewöhnt ist. Das ist auch eine Misshandlung, und wo der Gänsekiel schreit, da zeigt dem Auge des feinsinnigen Betrachters die Sprache fratzenhafte Verzerrungen. Man urtheile nicht vorschnell verallgemeinernd nach unseren hochgebildeten Sprachen. Die sind vielseitig gewöhnt, darum allseitig befähigt. Sie sind seit Jahrhunderten gewöhnt, den verschiedenartigsten Zwecken zu dienen, und ahmen fremde Formen so meisterlich leicht nach, dass man die Nachahmung kaum mehr verspürt, – altgeübte Schauspielerinnen, wenn man will, die jeder Rolle gerecht werden. Jene Armen aber, deren ganzes Leben in einem engen Gedankenkreise dahinschleicht, gleichen wohl, wo ihnen Höheres zugemuthet wird, dem Bauernburschen, den man einmal aushülfsweise in die Livrée gesteckt hat. Es fragt sich: wird ihnen eine glückliche Beanlagung, eine geschickte Verwendung über die Hauptschwierigkeiten hinweghelfen? denn in der That vermag wohl jede Sprache etwas mehr, als der Alltagsbedarf erfordert.

\sed{{\textbar}{\textbar}106{\textbar}{\textbar}}\phantomsection\label{sp.106}

\largerpage[1]Besonders irreleitend können für den arglosen Forscher gewisse Grammatiken werden, zumal jene älteren, die pedantisch am lateinischen Schulmuster haften. Da müssen die wildfremdesten Sprachen das ganze grammatische Schema eines \textsc{Nebrixa} oder \textsc{Alvarez} wie eine Schuldforderung Posten für Posten in ihrer Münze bezahlen. Schlimm, wenn sie einen schuldig bleiben; bestreiten sie aber einen doppelt und dreifach, so ernten sie hohes Lob. Dabei geht Alles sehr ordentlich und ehrlich zu: die Mittel sind wirklich vorhanden, werden auch manchmal angewandt. Es fragt sich nur, wie oft und wann und wozu? Was gilt für selbstverständlich? Was muss gesagt werden? Welche Form der sprachlichen Darstellung ist dem Geschmacke und Fassungsvermögen der Hörer genehm? Weist man dies der Stilistik zu, so erklärt man damit einen Theil der Stilistik für einen Bestand\-theil der Grammatik. Und das ist er meiner Meinung nach allerdings.

\largerpage
Zunächst, wie aus dem Bisherigen folgt, ein Theil des synthetischen Systemes. Aber auch der analytischen Aufgabe kann eine gewisse Kenntniss des nationalen Stiles zu \update{Statten}{statten} kommen. Weiss ich, dass das Altchinesische Kürze der Rede, Allgemeinheit des Ausdruckes, Parallelis\fed{{\textbar}111{\textbar}}\phantomsection\label{fp.111}mus der Sätze liebt: so sind das geradezu fundamentale Kenntnisse, die meiner ganzen folgenden Lernarbeit zu Statten kommen werden.\footnote{Ich rechne es zu den Fehlern meiner chinesischen Grammatiken, dass sie des Parallelismus nicht schon im allgemeinen Theile Erwähnung gethan haben.}

Jetzt leuchtet es ein, wie erwünscht dem Sprachforscher solche Texte sein müssen, die unmittelbar aus der Rede der Eingeborenen geschöpft sind: Gespräche, Erzählungen, vielleicht rhetorische Leistungen oder selbständige schriftliche Versuche schreibkundiger Leute. Von den Texten der Missionsliteratur aber dürfen wir unter sonst gleichen Umständen denen den Vorzug geben, wo der Verfasser nach langem, innigem Verkehre mit den Eingeborenen in freier Rede auftritt: biblische oder profane Geschichten, die er nacherzählt, ein von ihm selbst entworfenes Schul- oder Beichtbuch. Wo wir auf Bibelübersetzungen angewiesen sind, da ist es Sache des Taktes, die Stellen zu finden, wo sich die Sprache am zwanglosesten geben durfte. Als Beispiel einer verständigen Wahl nenne ich das vielbenutzte Gleichniss vom verlorenen Sohne, – als Beispiel des geraden Gegentheiles das erste Capitel des vierten Evangeliums, das leider auch in einer Polyglotte als Sprachprobe herhalten musste.

\pdfbookmark[2]{§. 8. Zusatz II. Die Appendices.}{II.VI.8}
\cohead{\edins{§. 8. Zusatz II. Die Appendices.}}
\subsection*{§. 8.}\phantomsection\label{II.VI.8}
\subsection*{Zusatz II.}
\subsection*{Die Appendices.}

Nur nebenher wollen wir jener Gegenstände gedenken, die wohl auch in den Sprachlehren anhangsweise behandelt werden. Es sind dies wohl hauptsächlich folgende:

\sed{{\textbar}{\textbar}107{\textbar}{\textbar}}\phantomsection\label{sp.107}

1. Redensarten des gewöhnlichen Verkehrs, Höflichkeitsformen oder ihr Gegentheil, Titel u.~dgl.

2. Feierliche Ausdrücke des Cultus, der Zauberei oder der Dichtkunst, oft Archaismen, manchmal wohl auch von fremdher Entlehntes und Missverstandenes enthaltend, wie in den Zauberformeln der Batta und in den Gesängen der Dayak.

3. Poetik, mindestens die Formen der Gedichte.

4. Zeitrechnung, Mass-, Gewichts- und etwaiges Münzwesen, vielleicht auch Personen- und Ortsnamen, – praktisch, aber natürlich nicht grammatisch.

\fed{{\textbar}112{\textbar}}\phantomsection\label{fp.112}

5. Endlich wohl auch Literaturübersichten als Wegweiser für das weitere Studium.

Solche Zugaben sind immer dankenswerth, dienen in ihrer Art doch auch dem Sprachunterrichte und finden in der That kaum irgendwo einen geeigneten Platz, als hinter der Grammatik. Von jenen anderen aber, die besonders in Elementarbüchern beliebt sind, von Vocabularien zum Auswendiglernen, Übungsstücken und zugehörigen Glossarien, soll im nächsten Abschnitte mit die Rede sein.

\pdfbookmark[2]{§. 9. Allgemeines über die Schreibweise und Äussere Ausstattung.}{II.VI.9}
\cohead{§. 9. Allgemeines über die Schreibweise und Äussere Ausstattung.}
\subsection*{§. 9.}\phantomsection\label{II.VI.9}
\subsection*{Allgemeines über die Schreibweise und äussere Ausstattung.}

Es ist kein Wunder, wenn unsere Wissenschaft bei der grossen Menge der Gebildeten für eine der allertrockensten gilt. Die schweren, öden Stunden von Quarta und Tertia sind noch nicht vergessen, und nun urtheilt man: Nichts langweiliger als eine Grammatik! Seien wir aufrichtig: wie die Mehrzahl unserer wissenschaftlichen Sprachlehren verfasst ist, müssen wir Sprachforscher selbst klagen: Eine trockene Lectüre, so eine Grammatik! Wenn wir sie doch lesen, mit Interesse, vielleicht mit Bewunderung lesen, so ist \update{Alles}{Alles,} was den Stoff \update{belebt}{belebt,} unsere eigene Zuthat. Wir wissen aber, dass der Verfasser dasselbe gedacht und nur nicht für nöthig gehalten hat, es auszusprechen. Auf den Laien, der nichts, wenigstens nicht viel Zutreffendes hinzudenken kann, hat er keine Rücksicht genommen, der mag sich den Inhalt des Buches gedächtnissmässig einprägen. Kürze und verhältnissmässige Billigkeit der Bücher wird damit erreicht. Auch soll man den Laien und Anfängern, und das sind unter hundert Lesern neunundneunzig, zuerst kurze Bücher in die Hände geben. Es fragt sich nur, an welchen Stellen gekürzt werden dürfe. Nach der Blattzahl und Druckeinrichtung eines Buches bestimmt sich wohl seine Stärke und sein Ladenpreis, nicht aber die Zeit, die es den Leser kosten wird; kurze Bücher können für den Lernenden sehr lang werden, und dickleibige, redselige Bücher können, wie geschwätzige Bonnen, die Spracherlernung gar sehr beschleunigen. Schont man den Anfänger, indem man ihm nur das Nächstwichtigste vorträgt, \sed{{\textbar}{\textbar}108{\textbar}{\textbar}}\phantomsection\label{sp.108} so erweise man ihm noch die weitere Liebe, den Vortrag verständlich und möglichst geschmackvoll einzurichten. Müsste man nicht auf den leidigen Kostenpunkt Rücksicht nehmen, so dürfte knapp bemessener \fed{{\textbar}113{\textbar}}\phantomsection\label{fp.113} Stoff in breiter Form das Richtige sein, – das Lehrbuch müsste zugleich ein anregendes Lesebuch \update{werden.}{werden,} \sed{und, wo es der Stoff erlaubt, müsste einmaliges aufmerksames Durchlesen genügen, um den wesentlichen Inhalt des Buches in uns aufzunehmen.} In anderen Fächern, selbst in der Rechtswissenschaft, ist man längst diesem Ziele nahe gekommen. In der Grammatik mag es damit besondere Schwierigkeiten haben. Immerhin jedoch sollte der Verfasser die Länge oder Kürze seines Buches nicht nach Druckbogen, sondern etwa nach Stundenaufgaben bemessen.

\largerpage[-1]
Gerade in Deutschland wird hiergegen oft verstossen, und zwar meines Wissens in hochwissenschaftlichen Werken und in Leitfäden für \update{Studirende}{Studierende} noch viel ärger, als in den verbreiteteren Schulbüchern. Die besten Lehr- und Handbücher unserer historischen Indogermanistik leisten vielleicht hierin das Ärgste. Es ist da nachgerade ein vornehm bloss andeutender Ton eingerissen, der es verschmäht, in rechtschaffenen Sätzen zu reden, als gälte es, die Syntax in der Praxis ebenso zu vernachlässigen, wie in der Theorie. Dann bleiben wohl auch die Beispiele unübersetzt, als wäre das „Elementarbuch“ doch eigentlich nur für die erfahrenen Fachgenossen verfasst. Der Anfänger mag sich die Erklärung der Beispiele im Wörterbuche zusammensuchen, und er mag sehen, wie er das formlose Gestammel des Textes in eine menschliche Sprache übersetzt. Nun bedarf es nur noch etwa der Grimm’schen oder einer anderen Privatorthographie, um die besten Bücher zu den unlesbarsten zu machen. Niemand misshandelt die Sprache und durch sie den Leser ärger, als ein Theil unserer Sprachforscher.

So mag es doppelt gerechtfertigt sein, wenn an dieser Stelle von Dingen die Rede ist, die streng genommen lediglich zur „Mache“ gehören und sonst bei aller Welt für selbstverständlich gelten. Folgende Sätze dürfen nun wohl auch auf allgemeine Zustimmung rechnen:

1. Es ist sehr förderlich, wenn sich der Leser selbst aus der Grammatik einen Auszug anfertigt. Je kürzer, übersichtlicher, dabei vollständiger dieser ausfällt, desto mehr Gewähr bietet er für das Verständniss des Gelernten. Solche Auszüge \update{sollten aber}{aber sollten} nicht von Paragraphen zu Paragraphen, sondern besser in Zeitabständen, jedesmal nach der Bewältigung eines Lehrabschnittes, und dann womöglich aus freier Erinnerung niedergeschrieben werden. Die tabellarische Form ist, wo sie hinpasst, vorzuziehen.

2. Es mag auch zweckmässig sein, wenn der Grammatiker selbst an geeigneten Stellen seinem Buche solche auszugsweise Übersichten \fed{{\textbar}114{\textbar}}\phantomsection\label{fp.114} einschaltet und so den Leser anleitet, den Stoff nochmals, in verdichteter Gestalt, vielleicht auch in neuer Ordnung zu durchdenken.

\sed{{\textbar}{\textbar}109{\textbar}{\textbar}}\phantomsection\label{sp.109}

3. Es ist natürlich erlaubt, solche Auszüge statt anderer Lehrbücher als Leitfaden für den mündlichen Unterricht und als Repetitorien zu gebrauchen. Sie aber aus ihrer dienenden Stellung in den Rang selbständiger, auch für den Selbstunterricht bestimmter Lehrbücher zu erheben, sollte nicht gestattet sein.

4. Stil- und Satzkürzungen sollte man sich eigentlich nur in solchen Auszügen erlauben, und doch auch hier durch Anwendung von Tabellen, Formeln oder Paradigmen thunlichst vermeiden. Sonst aber möge der Grammatiker wie jeder andere Schriftsteller sich beim Vortrage seiner Lehren und in seinen kritischen Erörterungen unter dem Leser einen Hörer vorstellen, der verlangen darf, dass man in menschlicher Sprache zu ihm rede.

5. Allerdings ist der Lehrvortrag nicht der einzige Zweck der Grammatik. Zumal ausführlichere Werke wollen zugleich Nachschlagebücher sein, müssen also den Stoff möglichst übersichtlich und gedrängt bieten, und insoweit mag ihnen eine conventionell gekürzte Ausdrucksweise vergönnt sein. Von den Registern und den typographischen Hülfsmitteln, die diesem \update{Zwecke}{Zweck} ferner dienen können, ist hier nicht nöthig zu reden. Dagegen mögen im Folgenden zunächst die Arten der Grammatiken und dann in Rücksicht auf diese einige specifisch grammatische Darstellungsmittel besprochen werden.

\begin{styleAnmerk}
\sed{Anmerkung. Gerade dem Sprachforscher, der sich in möglichst kurzer Zeit möglichst viele Sprachen aneignen will und soll, sind praktische Sprachführer mit reichlichen Übungsbeispielen oft willkommener, als Bücher mit wissenschaftlichen Prätensionen. Die Theorie wird er sich schon selbst schaffen: er braucht eben die Praxis, um die Theorie daraus zu schöpfen.}
\end{styleAnmerk}

\pdfbookmark[2]{§. 10. Arten der Grammatiken.}{II.VI.10}
\cohead{§. 10. Arten der Grammatiken.}
\subsection*{§. 10.}\phantomsection\label{II.VI.10}
\subsection*{Arten der Grammatiken.}
\subsection*{a. Systematische – methodische.}

\largerpage
Was ich im Früheren über die Eintheilung und Anordnung der Grammatik gesagt habe, galt zunächst von der Systematik einer wissenschaftlichen und vollständigen Darstellung des Sprachbaues. Wir müssen uns nun nochmals den Unterschied zwischen Systematik und Methode vergegenwärtigen. Jene bezweckt für den Gegenstand eine Darstellungsform, die nur durch ihn bedingt, ihm thunlichst angeglichen ist. Die Methode dagegen bezweckt eine Lehrform, die dem Lernenden zu möglichst schnellem, gründlichem und sicherem Erfassen des Lehrstoffes verhilft.

\fed{{\textbar}115{\textbar}}\phantomsection\label{fp.115}

Ein systematisches Buch muss seiner Absicht nach wissenschaftlich sein; denn vermöge seiner Systematik erklärt es ohne Weiteres, dass ihm die sachgemässe Ordnung und Darstellung des Stoffes als oberste Regel gilt. Daneben kann es in der Ausführung sehr methodisch, vielleicht auch sehr unmethodisch sein, jenachdem es den Bedürfnissen des Lesers mehr oder minder Rechnung trägt.

\sed{{\textbar}{\textbar}110{\textbar}{\textbar}}\phantomsection\label{sp.110}

\sed{Das System verlangt, dass das Einzelne sich aus den allgemeinen Gesetzen organisch entwickele, diese also sich fort und fort in jenem wiederholen, sei es ausgesprochenermassen, sei es stillschweigend, \corr{1901}{immer}{nimmer} aber in erkennbarer Weise. Ist der Lernende fähig, eine solche Darstellung zu verstehen, so ist diese zugleich für ihn die methodisch richtigste. Darum lege ich auf jenen allgemeinen Theil der Grammatik einen sehr hohen Werth: er sollte für den Verständigen das bündigste Elementarbuch sein. Diese Methode ist der mathematischen ähnlich und darum, meinen Erfahrungen nach, mathematisch beanlagten Köpfen besonders genehm.}

Eine Grammatik kann ihrem ganzen Inhalte nach sehr wissenschaftlich und doch in der Anordnung des Ganzen völlig unsystematisch sein, und wenn anders meine früheren Ausführungen über die Zweitheilung in ein analytisches und ein synthetisches System und über die Einrichtung beider richtig sind, so trifft jener Vorwurf auch die besten unter den bisherigen wissenschaftlichen Grammatiken.

Eine methodische Sprachlehre darf von der Systematik gänzlich absehen und ihre wissenschaftlichen Grundlagen unter dem Boden versteckt lassen. Denn sie will ein Können beibringen, nicht eine Erkenntniss. Ich kann mir aber auch denken, dass sie ganz wissenschaftlich und systematisch eingerichtet sei, und den Versuch hierzu möchte ich überall da empfehlen, wo ein genügend vorgebildeter Leserkreis zu erwarten ist. Denn wer eines tieferen Verständnisses des Lehrstoffes fähig ist, der wird sich die Einzelheiten doppelt schnell und sicher aneignen, wenn der Verstand dem Gedächtnisse zu Hülfe kommt, und er wird schlussfolgernd seinen Weg da weiter finden, wo ihn etwa die führende Hand des Grammatikers verlässt. Der Anfänger freilich, auch der bestgeschulte und bestwillige, verlangt noch andere Rücksichten. Der sehnt sich nach dem Eintritt in’s frische Leben der Sprache; der unerlässlichen Gedächtnissarbeit unterzieht er sich meist nur widerwillig und ist dankbar, wenn sie zeitweilig durch unterhaltendere Pensa unterbrochen wird. Kurze Paragraphen, lange Übungsstücke, möglichst sofortiges Hantieren mit ganzen Sätzen, wohl gar mit vollständigen Texten, Ver\-theilung des Lehrstoffes in Lectionen: darin besteht das Wesen jener praktischen Sprachlehren, aus denen wohl selbst Sprachforscher von Fach lieber lernen, als aus kurzen, trockenen Elementargrammatiken. Die Methoden von \textsc{Ahn}, \update{\textsc{Ollendorf} und}{\textsc{Ollendorf},} \textsc{Toussaint-Langenscheidt} sind nicht sprachwissenschaftliche, sondern pädagogische \update{Leistungen;}{Leistungen:} aber der beste Gelehrte ist nicht immer ein guter Lehrer, und der beste Lehrer braucht nicht ein grosser Gelehrter zu sein, – Beide können bei einander lernen.

\fed{{\textbar}116{\textbar}}\phantomsection\label{fp.116}

\subsection*{b. Vollständige Grammatiken – Elementarlehrbücher.}
Eine bloss methodische Sprachlehre kann in einer Reihe von Lehrgängen (Cursen) die ganze grammatische Schulung von den Anfangsgründen bis zu den \sed{{\textbar}{\textbar}111{\textbar}{\textbar}}\phantomsection\label{sp.111} letzten Feinheiten der Sprache bieten, von Stufe zu Stufe sich wissenschaftlicher gestaltend. Beispiele hierfür liefert unsre Schulbuchliteratur in Überfülle, und regelmässig bestehen diese Sprachlehren aus sovielen einzelnen Büchern als Cursen. Das Ganze ist aber doch als Einheit gedacht: immer redet derselbe Lehrer zu denselben Schülern, weiss, was er bei ihnen voraussetzen darf, und richtet sich darnach. Die Toussaint-Langenscheidt’schen Briefe nun gar sind so zu sagen papierene Hauslehrer, die ihre Schüler wöchentlich einmal besuchen und sie unvermerkt in immer höhere Classen aufrücken lassen. Offenbar hat Jeder die Wahl, bei welcher Stufe des Wissens er den Unterricht abbrechen oder aufgeben will, das Lehrbuch braucht nicht mehr zu enthalten, als gelernt werden soll, und so hat jeder denkbare Umfang desselben wenigstens eine Art wirthschaftlicher Berechtigung: das Angebot bemisst sich nach der Nachfrage. Vom wissenschaftlichen Standpunkte aus aber lässt sich vielleicht folgende Dreitheilung rechtfertigen:
 
1. \so{Vollständige Grammatiken}, \retro{das}{dass} heisst solche, die sich die Aufgabe stellen, alle grammatischen Erscheinungen der Einzelsprache, auch die seltensten und \update{unbedeu\-tendsten}{unbedeu\-tendsten,} zu verzeichnen und zu erklären. Gelöst ist diese Aufgabe wohl nur einmal, in \textsc{Pânini}’s Wunderwerke; unternommen aber und in weitem Umfange durchgeführt ist sie noch öfters worden, so von \textsc{Sylvestre de Sacy} in seiner Grammaire arabe, von \textsc{Raphael Kühner} in seinen grossen lateinischen und griechischen Grammatiken. Durchführbar ist sie überhaupt wohl nur da, wo man es mit der Sprache einer abgeschlossenen Literatur zu thun hat. Wo die Quellen spärlich rinnen, wie etwa beim Gotischen, Altirischen, Althebräischen u.~s.~w., ist es natürlich allemal geboten, sie nach Möglichkeit auszunutzen, um wenigstens relative Vollständigkeit zu erzielen. Dabei giebt es keine Massgrenze nach oben: je reichhaltiger, desto besser. Besonders umfängliche Werke dieser Art werden immer nur als Handbücher, nicht als Lehrbücher \update{gelten;}{gelten:} man liest sie durch, dann stellt man sie beiseite, um sich in schwierigen Fällen Raths bei ihnen zu erholen. Darum müssen alle Übersichtlichkeitsmittel, die man von einem Nachschlagebuche verlangen kann, bei ihnen angewandt werden. Blosse \fed{{\textbar}117{\textbar}}\phantomsection\label{fp.117} Nachschlagebücher wollen sie aber darum doch auch nicht sein, – sonst wäre die lexikalische Anordnung für sie die geeignetste. Der Verfasser will und soll zeigen, wie sich das Viele in seinem Geiste einheitlich gestaltet hat. Hinter der Menge der Paragraphen kann dies indessen nur zu leicht verschwinden. Da empfiehlt es sich denn, den Capiteln Einleitungen vorauszuschicken, die ihre Disposition von allgemeinen Gesichtspunkten aus begründen. Immer muss der Leser empfinden, dass er es nicht mit einem Aggregate zu thun hat, sondern mit einem Systeme.

2. Eine gewisse Voll- und Selbständigkeit des grammatischen Wissens ist aber das Ziel jedes höheren Sprachunterrichtes. Wir wollen die Sprache richtig \sed{{\textbar}{\textbar}112{\textbar}{\textbar}}\phantomsection\label{sp.112} verstehen und anwenden lernen, ohne ferner der Lehrer, Dolmetscher oder Übersetzungen zu bedürfen. Diesem Zwecke muss ein Lehrbuch entsprechen, das wir entweder eine \so{ausführliche Grammatik} oder eine vollständige Sprachlehre nennen wollen, denn als solches, für seinen Zweck, ist das Lehrbuch vollständig. Als Beispiele brauche ich nur die lateinischen und griechischen Schulgrammatiken anzuführen, die man den Gymnasiasten in die Hände giebt. Nun ist es interessant zu beobachten, wie diese Bücher in den letzten Jahrzehnten immer kürzer geworden sind, und wie sie voraussichtlich noch ferner einschrumpfen werden. \textsc{Kühner}’s griechische Schulgrammatik enthält fast doppelt soviel Stoff wie die \textsc{Koch}’sche. Ich weiss nicht, \update{wieviel}{wie viel} Antheil hieran das verständige Vorbild der Franzosen hat, die dem Schüler nur soviel schwarz auf weiss geben, als er in seinen Geist aufnehmen soll. Sache des Lehrtactes ist es, das richtige Mass zu finden. Als ein Beispiel des Gegentheiles darf es gelten, wenn die Schüler βλίττειν, zeideln, a verbo lernen müssen: – wer nicht Bienenvater ist, wird kaum wissen, was das deutsche Wort bedeutet!

\retro{Wissen\-schaftlich}{Wissen\-schaftllch} und systematisch darf das vollständige Lehrbuch sein, sollte es auch sein, wenn es für Schüler von genügender Fassungskraft berechnet ist. Nur so erfüllt es den doppelten Zweck der logischen Schulung und der Anregung zu selbständigem philologischen Denken.

\largerpage
3. \so{Kurze grammatische Vorschulen} sind zumal auch zur ersten Einführung in schwierige Sprachen zu empfehlen. Sie sind in Rücksicht auf den eigentlichen Lehrstoff möglichst knapp zu bemessen, sollten nicht mehr bieten, als nöthig ist, damit der Lernende unter Beihülfe \fed{{\textbar}118{\textbar}}\phantomsection\label{fp.118} eines Lehrers oder gedruckter Übersetzungen und Schlüssel an leichtere praktische Übungen gehen könne. Wissenschaftlichkeit ist auch hierbei soweit möglich zu erstreben, strenge Systematik aber nicht nöthig. Alles kommt hier auf die Methode an, die aber auf sicherem grammatischen Urtheile beruhen muss. Der Grammatiker muss den Stoff statistisch überschauen, um das Gewöhnlichste als das Erstnothwendige auszuwählen. Er muss ihn philosophisch verdichtet beherrschen, um die allgemeinsten Gesichtspunkte gebührend zur Geltung zu bringen. Endlich muss er verstehen, sich in die Seele des Neulings zu versetzen, um diesem die Arbeit nach Kräften zu erleichtern. Im grammatischen Systeme hat er das Sprachgebäude nachgebildet; mittels der Elementarmethode führt er den Fremden durch die Räume des fertigen Baues. Er wird den kürzesten Weg einschlagen, hier länger verweilend, dort flüchtig hindurchschreitend, immer darauf bedacht, dass der Gast sich recht bald heimisch fühlen lerne. Ein Meister in dieser Kunst war mein verewigter Vater. Er hat eine beträchtliche Zahl sehr verschiedenartiger Sprachen in kurzen Grammatiken behandelt und überall das gleiche Talent sicherer Auffassung und leichtfasslicher Darstellung des Wesentlichen bewährt.

\sed{{\textbar}{\textbar}113{\textbar}{\textbar}}\phantomsection\label{sp.113}

Andere als die geschilderten drei Stufen der Grammatik wüsste ich wissenschaftlich nicht zu rechtfertigen. Dass die vollständige Grammatik obenan steht, liegt in der Natur der Sache. Vor der Elementargrammatik könnte ich mir höchstens ein Ding wie eine Fibel denken, die keinen Anspruch auch wissenschaftliche Selbständigkeit macht. Und zwischen den beiden Endpunkten ist nur \update{eine}{\so{eine}} wissenschaftlich berechtigte Mittelstufe nachzuweisen: das ausführliche, relativ vollständige Lehrbuch, das diejenigen Kenntnisse mittheilt, die zur sogenannten Beherrschung einer Sprache gehören. Jene dickleibigen Schulgrammatiken, die noch in meiner Jugend geführt wurden, waren doch eigentlich Zwitterdinger, zu einem Drittheile für das methodische Lernen, zu zwei Drittheilen für das gelegentliche Nachschlagen bestimmt.

\subsection*{c. Kritische und didaktische Grammatiken.}

Es macht einen grossen Unterschied, auf welchen Fuss sich der Grammatiker mit seinem Leser stellt. Redet er zu ihm als zu Seinesgleichen: „Dies sind meine Ansichten, das meine Gründe; prüfe und urtheile selbst!“ – oder spricht er als Lehrer zum Schüler, als Wissender \fed{{\textbar}119{\textbar}}\phantomsection\label{fp.119} zum Unwissenden: „So ist die Sprache, so will sie erlernt sein; Du hast mir zu glauben und zu folgen!“

Mit dem gehaltlichen Umfange der Grammatik hat dies weniger zu thun, als es scheint. Die einzige vollständige Grammatik, die des \textsc{Pânini}, redet im Tone des Gesetzgebers. \textsc{Wilhelm Schott}’s geistvolle Chinesische Sprachlehre dagegen ist ihrem Inhalte nach kaum mehr als ein Elementarbuch und doch ganz kritisch gehalten. Ähnliches gilt von \fed{\textsc{Brugmann}’s griechischer und} \textsc{Bickell}’s hebräischer Grammatik.

Auch erklärt es sich leicht, dass die kritische Grammatik nicht schlechthin länger zu sein braucht, als die didaktische. Denn dem Kenner gegenüber genügen oft blosse Andeutungen, wo der Neuling breite Auseinandersetzungen verlangt. Als Regel möchte aber doch Folgendes gelten: Die vollständige Grammatik sollte kritisch sein, denn sie umfasst auch das Bestreitbare, und dazu muss der Verfasser Stellung nehmen. Die wissenschaftlichen Lehrbücher dürfen kritisch sein, wenn sie sich an einen Leserkreis wenden, bei dem ein entsprechendes Verständniss vorauszusetzen ist, an Lehrer oder wohlgereifte Schüler. Dabei denke ich nicht an jene Kritik, die jedes wissenschaftliche Werk implicite, so zu sagen in der Tasche bei sich führen muss, sondern an die, welche sich offen als solche giebt, aller Welt zur Schau und Prüfung.

Vor Allem verlangt der wissenschaftliche Leser, dass sich alles Neue vor ihm rechtfertige. Neu aber brauchen nicht nur Einzelbeobachtungen, sondern können auch Gesammtanschauungen des Verfassers sein, und sie werden es dann immer sein, wenn der Stoff ein neuartiger ist, wenn es also zum ersten Male gilt, einer gewissen Form des menschlichen Sprachbaues das entsprechende \sed{{\textbar}{\textbar}114{\textbar}{\textbar}}\phantomsection\label{sp.114} grammatische Gewand anzupassen. Da versteht man das Werk am besten, wenn man sich in der Werkstatt umschauen darf.

Allein die ausführliche wissenschaftliche Darstellung eines Sprachbaues sollte nicht nur das Neue, sondern überhaupt Alles als beweisbedürftig behandeln. Denn die Wissenschaft will auch das Allbekannte begründet sehen, und die sogenannten unumstösslichen Grundpfeiler gelten ihr erst dann als feststehend, wenn sie versucht hat mit ihren Zweifeln an ihnen zu rütteln. Von der Arbeit, die nöthig war, ehe die Griechen sich der Redetheile und der Casus ihrer Sprache völlig bewusst wurden, haben wohl nur die Wenigsten eine Ahnung. Ich kann mir aber vorstellen, dass diese oder eine ähnliche analytisch-inductive Arbeit, geschickt \fed{{\textbar}120{\textbar}}\phantomsection\label{fp.120} wiederholt, einen begabten Anfänger mächtig fördern und anregen würde. In der That wäre eine Sprachlehre, die den Lernenden selbst an der Entdeckung der Sprachgesetze theilnehmen liesse, geradezu ideal, – wenn die Lernenden ihrerseits immer ideal wären. Vielleicht ist der Versuch schon gewagt worden, in einem Lehrbuche die Beispiele den Regeln vorauszuschicken; es wäre dies ein Schritt in der von mir gemeinten Richtung.

In jenen Fällen aber, wo es darauf ankommt, aus dürftigem Stoffe möglichst viele grammatische Beobachtungen zu gewinnen, kann es geradezu geboten sein, die ganze kritisch-analytische Arbeit offen vor den Augen des Lesers geschehen zu lassen. So bei den altitalischen und kleinasiatischen Sprachen, bei denen, die uns in Keilschriften erhalten sind, beim Altpreussischen und in vielen Fällen bei Sprachen wilder und halbwilder Völker.

\pdfbookmark[2]{§. 11. Die grammatische Terminologie.}{II.VI.11}
\cohead{§. 11. Die grammatische Terminologie.}
\subsection*{§. 11.}\phantomsection\label{II.VI.11}
\subsection*{Die grammatische Terminologie.}

\begin{sloppypar}Von unerquicklichen Wortstreitereien ist auch unsere Wissenschaft nicht verschont geblieben. Jede Sprache hat ihre eigenen Formenkategorien. Wie soll man die benennen? Die Mehrzahl der Grammatiken entscheidet sich für die allgebräuchlichen lateinischen Ausdrücke, fügt wohl nach Bedürfniss andere aus der griechischen, hebräischen oder, – dies seit neuerer Zeit, – aus der Sanskrit-Gram\-matik hinzu. Man redet von Aoristen, vom \textit{status constructus}, von Sandhi-Gesetzen, Dvandva-Compositis u.~s.~w. Und wo das nicht ausreicht, erfindet man neue lateinische Wörter, die den Begriff möglichst treffen: \textit{casus elativus}, \fed{\textit{illativus},} \textit{adessivus}, \textit{inessivus}, \textit{prosecutivus}, – \textit{modus benedictivus}, \textit{deprecativus} und dergleichen mehr. Alles wird nach der Analogie von den flectirenden Sprachen auf nicht flectirende übertragen; und dagegen erheben nun Andere Einspruch, als geschähe den fremden Idiomen Zwang oder, – denn darauf pflegt es hinauszulaufen, – zuviel Ehre. Da soll man hier nicht von Wörtern, sondern etwa von Stämmen oder Wurzeln, dort nicht von Verben, sondern etwa von \sed{{\textbar}{\textbar}115{\textbar}{\textbar}}\phantomsection\label{sp.115} Nomen-verbis oder von \corr{1891 und 1901}{Prädicats\-nominibus,}{Prädikats\-nominibus} da wieder nicht von Casus, sondern von Postpositionen, nicht von einem Nominativus, sondern vom Wortstamme reden, und was dessen mehr ist.\end{sloppypar}

\fed{{\textbar}121{\textbar}}\phantomsection\label{fp.121}

Es ist hier nicht der Ort, auf die Grundfrage einzugehen, ob denn die Unterschiede immer so specifisch sind, wie jene Tadler meinen. Nehmen wir an, sie wären es, so wäre meiner Meinung nach damit noch nicht entschieden, dass man die uns geläufigen Namen nicht auf ähnliche Functionen in anderen Sprachen übertragen dürfe. Wäre es wahr, dass technische Ausdrücke überall die gleiche Bedeutung haben müssen, so dürften nicht einmal die lateinischen, griechischen und deutschen Genitive mit denselben Namen bezeichnet werden; denn der Unterschied zwischen einem rein adnominalen Casus und einem Casus, der bald adnominalen, bald adverbialen Dienst versieht, ist doch fürwahr grell genug. Dafür soll nun aber auch eine Grammatik nicht bloss die Dinge benennen, sondern auch sie so erklären, dass man mit den Namen die richtigen Vorstellungen verbindet. Wollte man jenen Grundsatz überall durchführen, so wäre des Worterfindens kein Ende; denn schwerlich werden sich zwei Formen in zwei Sprachen begrifflich vollkommen decken. Oder man müsste es machen, wie es wirklich vorgeschlagen worden ist, die Formen fremder Sprachen einfach mit ihren Lauten benennen. Dann mag man z.~B. im Türkischen statt vom Nominativ und vom unbestimmten Accusative vom reinen Stamme reden. Bei den anderen Casus wäre freilich die Sache etwas umständlicher: man müsste sagen

\begin{table}[h]
\centering
\begin{tabular}{c l l l}
statt & Genitiv: & das Suffix & \textit{yṅ}, \textit{uṅ}, \textit{iṅ}, \textit{üṅ} \\
 „ & Dativ: & „~~~~~~~~„ & \textit{a}, \textit{e}, \textit{ja}, \textit{je} \\
  „ & Accusativ: & „~~~~~~~~„  & \textit{y}, \textit{u}, \textit{i}, \textit{ü}, \textit{jy}, \textit{ju}, \textit{ji}, \textit{jü}, \\
 „ & Ablativ: & „~~~~~~~~„ & \textit{dan}, \textit{den}.
\end{tabular}
\end{table}

\noindent Ähnliches würde sich dann bei den Conjugationsformen wiederholen. Ob das ein Gewinn wäre?

Eigentlich ist doch die Sprache nicht dazu da, um die Menschen zu ärgern. Nun sind wir aber sammt und sonders im Punkte unserer Sprache Gewohnheitsmenschen; das liegt im Wesen der Sache, – es wäre ja sonst nicht \so{unsere} Sprache. Die Sprachgewohnheit ist aber sehr empfindlich: alles Neue, wenn es sich nicht besonders gefällig einführt, ärgert sie geradezu. Was soll es z.~B. mit dem Gen\textit{e}tivus? Bis in die neueste Zeit haben alle Grammatiker Gen\textit{i}tivus geschrieben, die ausserdeutschen thuen es meines Wissens noch heute. Da hat man herausgefunden, dass das gut lateinische Gebilde weder seinem grammatischen Begriffe noch seinem griechischen Vorbilde recht entspreche, und nun schickt man den garstigen griechisch-lateinischen Blendling in die Welt.

\fed{{\textbar}122{\textbar}}\phantomsection\label{fp.122}

Das ist nun freilich auch ein Wortstreit, sogar ein Streit um einen Buchstaben. Die Sache liegt aber doch tiefer; denn zu Grunde liegt ihr eine Missachtung der lebendigen Sprache, eine Anmassung Einzelner an Rechten der \fed{{\textbar}{\textbar}116{\textbar}{\textbar}}\phantomsection\label{sp.116} Gesammtheit. Ein technischer Ausdruck wie Genitivus ist Gemeingut der Gebildeten, kein Einzelner, auch kein Gelehrter hat eigenmächtig darüber zu verfügen. Es ist damit wie mit der Orthographie und mit jenen wunderlichen orthographischen Jägercostümen, in denen Manche herumzustolziren lieben. Die Sache wäre harmlos, wenn sie nicht gerade von Sprachforschern ausginge.

Schliesslich noch dies: Unter allen Arten der grammatischen Terminologie möchte ich die Numerirung am wenigsten empfehlen, weil sie dem Gedächtnisse am wenigsten Vorstellungsinhalt bietet. Die von deutschen Grammatikern beliebten Casusbezeichnungen: „erster, zweiter u.~s.~w. Fall“, die Ziffern zur Bezeichnung der zehn Conjugationen im Sanskrit und der fünfzehn im Arabischen sind garstige Nothbehelfe.

\pdfbookmark[2]{§. 12. Die Beispiele.}{II.VI.12}
\cohead{§. 12. Die Beispiele.}
\subsection*{§. 12.}\phantomsection\label{II.VI.12}
\subsection*{Die Beispiele.}

Der Unterschied zwischen didaktischen und kritischen Grammatiken zeigt sich zumal in der Auswahl und Menge der Beispiele, denen in der That hüben und drüben ganz verschiedene Rollen zufallen. Im Lehrbuche dienen sie dazu, den Lehrsatz zu verdeutlichen, seine Anwendungen einzuüben, nebenbei den Wortvorrath des Lernenden zu vermehren. Der erstere Zweck ist der wichtigste, und so bedürfte das Lehrbuch, wenn es nicht zugleich ein Übungsbuch sein will, eigentlich für jeden Lehrsatz nur eines Beispieles. Dies müsste dann aber auch sehr sorgfältig ausgewählt sein, darauf berechnet, die Lehre in’s klarste Licht zu stellen.

Für die kritische Grammatik dagegen sind die Beispiele Beweisinstanzen, die nicht nur durch ihre Auswahl, sondern auch durch ihre Menge wirken müssen, darum zumal in den schwierigeren Lehren nicht wohl zu zahlreich auftreten können. Dafür steht es denn auch dem kritischen Grammatiker, der sich auf Texte stützt, frei, von den ausgeschriebenen Proben zu blossen Stellenangaben überzugehen und so dem Zweifler selbst das Nachsuchen zu überlassen. Dass er auch für die vollständigen Beispiele seine Quellen angeben muss, ist selbstverständlich.

Die Beispiele sind zu analysiren, im Lesebuche gemäss dem Bedürf\fed{{\textbar}123{\textbar}}\phantomsection\label{fp.123}nisse des Lernenden unter fortwährendem Hinweise auf früher Vorgetragenes; – im kritischen Werke soweit, als über die Beurtheilung des Falles Zweifel entstehen können.

\pdfbookmark[2]{§. 13. Paradigmen und Formeln.}{II.VI.13}
\cohead{§. 13. Paradigmen und Formeln.}
\subsection*{§. 13.}\phantomsection\label{II.VI.13}
\subsection*{Paradigmen und Formeln.}

Der grammatische Lehrsatz wird durch die Beispiele bewiesen oder veranschaulicht. Auf alle Fälle enthalten die Beispiele neben Demjenigen, worin sich \sed{{\textbar}{\textbar}117{\textbar}{\textbar}}\phantomsection\label{sp.117} der Lehrsatz äussert, noch mancherlei Zufälliges, das man für den Unterrichtszweck gern beseitigen oder doch auf das geringste Mass beschränken möchte. Ersteres geschieht durch die Formel, Letzteres durch das Paradigma.

Das Paradigma wählt aus den vielen möglichen Zufälligkeiten eine aus, um an ihr, als an der Stellvertreterin ihrer Art, alle möglichen Veränderungen vorzunehmen. So wird das Verbum amare zum Vertreter aller regelmässigen Verba der ersten Conjugation; was mit ihm geschieht, kann mit jedem gleichartigen geschehen. Am Grossartigsten ist dies wohl von den arabischen Grammatikern mit ihrer Wurzel \<\setarab \novocalize f`l>  \textit{f–{\ain}–l}, \textit{fa{\ain}ala} durchgeführt worden, deren Abwandelungen zugleich als technische Namen gelten: \arabictext{fa`ala} \textit{fa{\ain}ala}, \update{\arabictext{yaf`ulu} \textit{jaf{\AIN}ulu},}{\arabictext{yaf`alu} \textit{jaf{\AIN}alu},} \arabictext{fu`ila} \textit{fu{\ain}ila} u.~s.~w. – Die Nützlichkeit des Paradigmas für das Auswendiglernen leuchtet ein. Das Gedächtnisswerk wird damit auf das Nothwendigste eingeschränkt und zwiefach erleichtert: einmal dadurch, dass das Zufällige als Gleichbleibendes, sich vom Wesentlichen, das sich verändert, abhebt, und zweitens dadurch, dass doch immer ein concreter Körper bleibt. Z.~B. kann man sich unter \textit{amant}, sie lieben, mehr vorstellen, als unter einem abstracten \textit{–a-nt} 3.~pers. pl. ind. praes. act.

Dadurch nun unterscheidet sich die Formel vom Paradigma, dass sie schlechthin alles Zufällige ausscheidet, mithin den Lehrsatz in seiner kürzesten, reinsten Form verkörpert. Lag beim Paradigma der Vortheil auf Seiten des Gedächtnisses, so liegt er hier auf Seiten des Verstandes, und so will denn das Paradigma auswendig gelernt, die Formel hingegen frei erfunden werden.

Beide sind aber ihrer Natur nach lediglich didaktisch; denn sie beweisen nichts, verlangen vielmehr den Beweis ihrer Richtigkeit, setzen \fed{{\textbar}124{\textbar}}\phantomsection\label{fp.124} Kritik voraus und stellen, wenn sie gelungen sind, das Ergebniss einer gewonnenen allgemeinen Erkenntniss dar. Eine nothwendige Grenze ihrer Anwendbarkeit vermag ich nur da zu erkennen, wo die Grammatik dem Ausdrucke nicht logischer, sondern gemüthlicher Beziehungen gilt, wie bei einem Theile der modalen Formen und Hülfswörter. Namentlich aber auch syntaktische Lehren kann ich mir in beiderlei Darstellungsformen übertragen denken. Dafür ein Beispiel aus der chinesischen Grammatik.

% \begin{table}[h]
\resizebox{.85\textwidth}{!}{
\parbox{\textwidth}{
\begin{flushleft}
\tabcolsep=1mm
\begin{tabular}{r l l l l l}
 & \multicolumn{3}{c}{Paradigma} &  & \multicolumn{1}{c}{Formel} \\
I. & \multicolumn{1}{c}{\textit{wâng}} & \multicolumn{1}{c}{\textit{paò}} & \multicolumn{1}{c}{\textit{mîn}} & ~~.~~~~.~~~~. & \textit{A}\textsuperscript{subj.} \textit{B}\textsuperscript{v. act.} \textit{C}\textsuperscript{obj.} \(=\) \textit{Φ}(Satz) \\
 & Der König & beschützt & das Volk. & & \multicolumn{1}{c}{\textit{BC} \(=\) P(Prädicat).} \\
\end{tabular}
%
\begin{tabular}{r l l l l l}
II. & \multicolumn{1}{c}{\textit{mîn}} & \multicolumn{2}{l}{\textit{paò}} & ~.~~~~.~~~~.~~~~.~~~~. & \textit{C}\textsuperscript{subj.} \textit{B}\textsuperscript{v. pass.} \(=\) \textit{Φ}(Satz) \\
 & Das Volk & \multicolumn{2}{l}{wird beschützt.} \\
\end{tabular}
%
\begin{tabular}{r l l l l l}
III. & \multicolumn{1}{c}{\textit{mîn}} & \multicolumn{1}{c}{\textit{paò}} & \multicolumn{1}{c}{\textit{iǖ wâng}} & & \textit{C}\textsuperscript{subj.} \textit{B}\textsuperscript{v. pass.} \textit{iǖA}\textsuperscript{Urheber} \(=\) \textit{Φ}(Satz) \\
\multicolumn{6}{r}{\textit{\textsubscript{\LARGE \~{}} ABC}} \\
 & Das Volk & wird beschützt & vom König. & & \textit{B iǖ A} \(=\) \textit{P}(Prädicat).\\
 & \sed{{\textbar}{\textbar}118{\textbar}{\textbar}}\phantomsection\label{sp.118} & \\
\end{tabular}
%
\begin{tabular}{r l l l l l l}
IV. & \multicolumn{1}{c}{\textit{paò}} & \multicolumn{1}{c}{\textit{mîn}} & \multicolumn{1}{c}{\textit{čī}} & \multicolumn{1}{c}{\textit{wâng}} & & \ldelim\}{4}{3mm}{ } \multirow{4}{50mm}{\textit{P}\textsuperscript{attr.} \textit{čī A} (od. \textit{C})\textsuperscript{subst.} \(= \frac{\textit{Φ}}{\textit{n}}\textsuperscript{subst.}\)} \\
 & beschützen & das Volk & \multicolumn{1}{c}{pr. rel.} & König \\
 & \multicolumn{1}{c}{\textit{paò}} & \multicolumn{1}{c}{\textit{iǖ}} & \multicolumn{1}{c}{\textit{wâng}} & \multicolumn{1}{c}{\textit{čī}} & \multicolumn{1}{c}{\textit{mîn}} \\
 & beschützt & vom & König & pr. rel. & Volk & \multicolumn{1}{c}{(substantivischer Satztheil).} \\
V. & \multicolumn{1}{c}{\textit{paò}} & \multicolumn{1}{c}{\textit{mîn}} & \multicolumn{1}{c}{\textit{čè}} & & & \ldelim\}{5}{3mm}{ } \multirow{5}{50mm}{\textit{P čè \textsubscript{\LARGE \~{}} P čī A} (oder \textit{C})\textsuperscript{subst.} \(= \frac{\textit{Φ}}{\textit{n}}\textsuperscript{subst.}\)} \\
 & beschützt & das Volk & derjenige der \\
 & \multicolumn{3}{l}{~~~~ \(=\) Einer, der das V. beschützt} \\
 & \multicolumn{3}{l}{\textit{paò iǖ wâng čè}} \\
 & \multicolumn{4}{l}{\(=\) wer od. was vom K. beschützt wird} \\ 
VI. & \multicolumn{1}{c}{\textit{k'î}} & \multicolumn{1}{c}{\textit{wâng}} & & & & \ldelim\}{4}{3mm}{ } \multirow{4}{50mm}{\textit{k'î A} (od. \textit{C}) \textsubscript{\LARGE \~{}} \textit{P čī A} (od. \textit{C}) \(= \frac{\textit{Φ}}{\textit{n}}\textsuperscript{subst.}\)} \\
 & ein solcher & König \\
 & \multicolumn{1}{c}{\textit{kî}} & \multicolumn{1}{c}{\textit{mîn}} \\
 & ein solches & Volk \\
\multicolumn{6}{l}{ } & oder kürzer, senkrecht zu lesen: \\
 & & \multicolumn{1}{r}{\textit{S}\textsuperscript{subj.}} & \rdelim\}{2}{3mm}{ } \multirow{2}{*}{\(= \textit{Φ}\)} & & \multirow{2}{*}{\textit{k'i}} \ldelim\{{2}{3mm}{ } & \textit{P}\textsuperscript{attr.} ~~~~~~~~ \rdelim\}{3}{3mm}{ } \multirow{3}{*}{\(= \frac{\textit{Φ}\textsuperscript{subst.}}{\textit{n}}\)} \\
 & & \multicolumn{1}{r}{\textit{P}\textsuperscript{praed.}} & & & &  \textit{čī} ~~~~ \rdelim\}{2}{3mm}{ } \multirow{2}{*}{\textit{čè}} \\
\multicolumn{6}{c}{ } & \textit{S}\textsuperscript{subst.}
\end{tabular}
\end{flushleft}
}
}
% \end{table}

Die entsprechenden grammatischen Lehrsätze lauten:

1: Das Subject steht vor dem Prädicate (I, II, III).

\fed{{\textbar}125{\textbar}}\phantomsection\label{fp.125}

2. Das active Verbum steht vor seinem Objecte (I).

3. Steht ein sonst actives Verbum am Ende des Satzes, so ist es passiv (II).

4. Der active Satz kann in einen passiven verwandelt werden, indem das logische Object vor das Verbum und hinter dieses das logische Subject (der Urheber) mit der Präposition \textit{iǖ} tritt (III).

5. Jedes Prädicat kann in ein adnominales Attribut verwandelt werden, wenn es vor das logische Subject und zwischen beide die Relativpartikel \textit{čī} tritt (IV).

6. Der solchergestalt geschaffene Relativsatz wird durch einen substantivischen ersetzt, indem an Stelle von \textit{čī} und dem darauffolgenden Substantive das substantivische Relativpronomen \textit{čè} tritt (V).

7. Andrerseits kann der Relativsatz einschliesslich \textit{čī} durch das Demonstrativpronomen \textit{k’î} vertreten werden (VI).

Es wäre leicht, hierin fortzufahren, auch andere Lehren der Grammatik in ähnlicher Weise darzustellen; man brauchte nur ein zweckmässiges Zeichensystem zu \retro{erfinden,}{erfinden} z.~B. neben dem Gleichheitszeichen noch für Ähnlichkeit der Bedeutung das Ähnlichkeitszeichen und für Analogie des grammatischen Verhaltens das Parallelitätszeichen einzuführen u.~s.~w. 

Inwieweit nun solche Mittel zweckmässig seien, ist eine rein pädagogische \sed{{\textbar}{\textbar}119{\textbar}{\textbar}}\phantomsection\label{sp.119} Frage. Manchen mögen die abstracten Formeln verwirren und ärgern; Andere werden es vorziehen, sich die Paradigmen selbst zu verfertigen, statt sie sich fix und fertig vorlegen zu lassen. Mathematisch beanlagte Köpfe mögen nur der Anregung bedürfen, um selber ein zutreffendes Formelsystem zu erfinden, und der Nutzen wird dann nicht ausbleiben; denn was wir uns selbst erzeugt haben, bleibt unverlierbar unser.

Zudem hat die graphische Formel für den Kundigen den Vorzug der unmittelbaren, von der Sprache des Darstellers unabhängigen Anschaulichkeit; sie beseitigt also das, was in jedem sprachlich ausgedrückten Lehrsatze zufällig ist. So ist sie, wo sie überhaupt Anwendung leiden kann, das vollkommenste Mittel wissenschaftlicher Darstellung. Der Versuch, sie im weitestmöglichen Umfange anzuwenden, sollte unternommen und dann bis zum Gelingen weiter verfolgt werden. Vor der Hand haben wir noch die Mathematiker und die Chemiker um ihre Zeichensysteme zu beneiden.

\fed{{\textbar}126{\textbar}}\phantomsection\label{fp.126}

Die \update{Anubandhas}{\so{Anubandhas}} der Inder verfolgen das gleiche Ziel. Es sind dies künstlich erfundene Affixe, die den Sylben, Wurzeln, Suffixen oder Stämmen angefügt, die einschlägigen grammatischen Regeln andeuten. Irre ich nicht, so liegt der Vortheil hierbei nicht nur in der Kürze des Ausdruckes, sondern auch in einer Erleichterung der Gedächtnissarbeit. Dem Gedächtnisse wird sich ein Lautcomplex sammt anubandha mindestens nicht viel schwerer, zuweilen sogar leichter einprägen, als ohne diese Zugabe; denn was zu wenig Körper hat, merkt sich am Schwersten. Der Fehler dürfte eher in den zu Grunde liegenden mechanischen Anschauungen, als in dem Zeichensysteme zu suchen sein. Dass dies durchweg aussprechbar ist, also das Ohr mit an der Gedächtnissarbeit theilnehmen lässt, kann man vom Standpunkte der Lehrmethode aus nur loben. Gräulich geschmacklos ist es aber doch, die herrlichen Gebilde der Sanskritsprache durch solche aufgeklebte Etiquetten zu verunstalten.

\pdfbookmark[2]{§. 14. Uebungsstücke.}{II.VI.14}
\cohead{\edins{§. 14. Uebungsstücke.}}
\subsection*{§. 14.}\phantomsection\label{II.VI.14}
\subsection*{Uebungsstücke.}

Es ist nothwendig, die grammatischen Lehrsätze nicht nur Stück für Stück zu kennen, wie sie im Buche stehen, sondern sie auch in ihrem gemischten Zusammenwirken erkennen und anwenden zu lernen. Diesem Zwecke und dem einer gründlichen Generalrepetition dienen sorgfältig ausgewählte und erklärte Textstücke. Jede Elementargrammatik, die den Bedürfnissen des Selbstunterrichtes Rechnung tragen will, sollte mit solchen Zugaben versehen sein; die meisten sind es ja auch thatsächlich, wenn schon die Übungsbücher unter besonderem Titel erscheinen mögen, und ein Lehrer die Erklärungen beifügt, die sonst das Lehrbuch geben müsste.

\sed{{\textbar}{\textbar}120{\textbar}{\textbar}}\phantomsection\label{sp.120}

Allein auch in grösseren Grammatiken kann es sich empfehlen, gelegentlich zusammenhängende längere Übungsstücke einzuschalten. Dann nämlich, wenn die Lehren durch das Ineinandergreifen verschiedener Lehrsätze verwickelt werden. So wird die Lehre von der oratio obliqua im Lateinischen sehr veranschaulicht werden, wenn man denselben Text in \update{oratio}{Oratio} obliqua und recta gegenüberstellt.

\fed{{\textbar}127{\textbar}}\phantomsection\label{fp.127}

\pdfbookmark[2]{§. 15. Die Sprache des Grammatikers und die darzustellende Sprache.}{II.VI.15}
\cohead{\fed{§. 15. Die Sprache des Grammatikers und die darzustellende Sprache.}}
\subsection*{§. 15.}\phantomsection\label{II.VI.15}
\subsection*{Die Sprache des Grammatikers und die darzustellende Sprache.}

\largerpage[-1]
Schon früher habe ich angedeutet, dass es für die wissenschaftliche Auffassung und Einrichtung einer Grammatik ganz gleichgültig sein muss, in welcher Sprache und für wes Landes Kinder sie geschrieben wird. Lernen wir eine indogermanische Sprache, so bringen \update{wir}{wir,} ohne es zu wissen und zu bemerken, eine Menge Vorstellungen mit an’s Werk, die uns Niemand erst beizubringen braucht, die uns eben als selbstverständlich gelten. Dem Grammatiker lassen wir es dann hingehen, wenn er sich und uns mit derlei Dingen nicht weiter aufhält. Er selbst muss sich aber sagen, dass sein Buch insoweit sowohl unvollständig, als auch unwissenschaftlich ist. Beides, denn er hat aus Gründen, die mit der Natur des Gegenstandes nichts zu schaffen haben, eine Auswahl getroffen und sehr wesentliche Dinge übergangen. Jene philosophischen und allgemeinen Grammatiken aus unserer Urgrossväter Zeiten treiben von der Rumpelkammer her noch immer ihren Spuk. Immer noch thut man, als wäre das uns Gewohnte gemeingültig und selbstverständlich; mag man sich zehnmal durch Erfahrungen in anderen Gebieten der Sprachenwelt vom Gegentheile überzeugt haben, man bleibt beim hergebrachten Brauche.

Wenn und soweit die Grammatik nur ein Lehrmittel sein will, ist nun auch hiergegen gar nichts einzuwenden. Der Lehrer soll ja den Schüler in der fremden Sprache heimisch machen, und wir mögen zweifeln, ob sich der Fremde, der uns besucht, schneller eingewöhnt, wenn wir ihm etwa auf Schritt und Tritt zurufen wollten: „Siehst Du, das ist ganz wie bei Dir zu Hause; das ist ein Stuhl, da setzt man sich \update{drauf,}{darauf,} u.~s.~f.“ Er würde fragen: „Bin ich denn vom Monde gefallen?“ und sich erst recht fremd fühlen.

In der Wissenschaft aber giebt es nichts Selbstverständliches. Was nicht gesagt ist, gilt als nicht gedacht, was nicht bewiesen wird, gilt als nicht erwiesen. Gerade wir Sprachforscher können in solchen Dingen gar nicht radical genug sein. Muttersprachliche Gewohnheiten und Schulerinnerungen wollen uns einengen und einschläfern und beherrschen uns schliesslich, wie alte treue Diener herrschen, durch die Macht der Bequemlichkeit.

\fed{{\textbar}128{\textbar}}\phantomsection\label{fp.128}

Auf das Übersetzungselend im Sprachunterrichte brauche ich hier nicht zurückzukommen; ich habe darüber seines Orts genug geklagt. Dafür sei denn \sed{{\textbar}{\textbar}121{\textbar}{\textbar}}\phantomsection\label{sp.121} jetzt darauf hingewiesen, dass uns die Feinheiten einer fremden Sprache oft aus einer treffenden Übersetzung viel greller und schneller entgegenleuchten, als aus der sorgfältigsten Definition. Meisterschaft im Übersetzen ist für den Sprachforscher eine sehr wichtige Gabe, denn sie gewährt ihm eines der wirksamsten Darstellungsmittel.

\pdfbookmark[2]{§. 16. B. Das Wörterbuch.}{II.VI.16}
\cohead{§. 16. B. Das Wörterbuch.}
\subsection*{§. 16.}\phantomsection\label{II.VI.16}
\subsection*{B. Das Wörterbuch.}

Das Wörterbuch im gewöhnlichen Sinne dient dem Zwecke des Nachschlagens und nur diesem; seine wissenschaftlichen Vorzüge sind Vorzüge seiner einzelnen Theile, vielleicht aller Theile, das heisst aller Artikel, nicht aber des Ganzen als einer Einheit. Es sei so geordnet, dass es die Arbeit des Aufsuchens möglichst erleichtert: daher ist die alphabetische Ordnung die bevorzugte. Die Artikel sind harmonisch zu gestalten, im Innern übersichtlich zu \retro{ordnen,}{ordnen} nicht aber untereinander organisch zu verknüpfen, – eine Häusercolonie, nicht ein Palast.

Ist das Wörterbuch nur Nachschlagebuch, so dient es allein dem Bequemlichkeitszwecke. Dann hat es in der Wissenschaft überhaupt keine Stätte, es sei denn diejenige, die man im Studierzimmer dem Sopha gönnt.
 

\sloppypar{
Nun aber gehört zu einer Sprache der Sprachschatz nicht minder, als der Sprachbau, folglich zur Darstellung einer Sprache das Wörterbuch nicht minder, als die Grammatik. Warum soll also die eine wissenschaftlich sein, und das andere nicht? Wäre etwa nur das Formenwesen einer Sprache ein organisches Ganze, und der Wortschatz ein zufällig angesammelter Haufen? Dann wäre auch jener Vorrath von Vorstellungen nichts besseres, über die ein Volk verfügt, und der im Wortvorrathe seinen Ausdruck findet. Es ist leicht nachzuweisen, dass die Sprachen nicht nur dadurch bestimmt werden, wie ein Volk denkt, sondern auch durch das, worüber es denkt. Es ist auch von vorn herein nothwendig, dass Beides, Stoff und Form der Rede, einander beeinflussen, dass Verlauf und Darstellung des Denkens abhängig sind vom Gegenstande, dass also die bevorzugten Gegenstände der Rede, die Geistes- und Lebensbedürfnisse eines Volkes, einen Einfluss üben müssen auf die Gestaltung der Grammatik. Dies weiter zu verfolgen, mag einer späteren Erörterung \fed{{\textbar}129{\textbar}}\phantomsection\label{fp.129} vorbehalten bleiben. Genug für jetzt: dieselbe Volksindividualität schafft Beides, die Grammatik und den Wortschatz.
}

Jetzt stellen wir uns auf den einzelsprachlichen Standpunkt, das heisst auf den des nationalen Sprachgefühles. Da dürfte es nun einleuchten, dass hier eine grundsätzliche Scheidung zwischen dem Wortvorrathe und dem grammatischen Formenwesen kaum besteht. Die Hülfswörter gehören zu Beiden; die Mittel der Wortbildung sind, was ihr Name besagt, Formenmittel, die zur \sed{{\textbar}{\textbar}122{\textbar}{\textbar}}\phantomsection\label{sp.122} Stofferzeugung dienen; und wo der etymologische Zusammenhang noch zu Tage liegt, da verbindet sich in diesem Gefühle das abgeleitete Substantivum, Adjectivum oder Adverb mit dem Verbum ebenso innig, wie sich die verschiedenen Formen desselben Verbums zusammen gesellen: „Bau, Gebäude, baulich“ stehen dem Verbum „bauen“ nicht ferner, als dem Infinitive das Imperfectum „ich baute“, oder das Participium „gebaut“. Erst recht auffällig ist dies in den semitischen Sprachen, deren Vocalisationswesen zwischen Wort- und Formenbildung so zu sagen auf der Kippe steht; und halten wir weiter Umschau in der Sprachenwelt, so wird die Grenze zwischen den beiden Begriffen wohl noch unsicherer. Jene zwei Gesichtspunkte, die im analytischen und synthetischen Systeme der Grammatik zur Geltung kommen sollen, walten, wenn auch meist roh genug, in der bekannten Zweitheilung der Wörterbücher.

Unbestreitbar gehört die Wortbildung zum Sprachbaue, folglich die Lehre von ihr in die Grammatik. Diese Lehre ist aber nur dann vollständig, wenn sie besagt, in welchen Fällen ein jedes Wortbildungsmittel zulässig ist. Und so lässt sich denn vom rein wissenschaftlichen Standpunkte aus nichts dagegen einwenden, dass eine Grammatik den gesammten Wortschatz einer Sprache in sich aufnehme. In der That ist dies wenigstens einmal versucht worden: in \textsc{Jacob Grimm}’s deutscher Grammatik, die allerdings eine historisch-vergleichende ist. Wo aber die Sprachen und ihr Wortbildungswesen noch lebendig sind, da verbietet sich ein solches Unternehmen von selbst. Die Composita eines einzelnen deutschen, griechischen oder altindischen Schriftstellers können wir aufzählen, nicht aber die zulässigen Composita ihrer drei Sprachen. Und ebenso ist es mit anderen, noch frei verwerthbaren Bildungsmitteln, den deutschen auf \textit{–ung}, \textit{–lich}, \textit{–bar} u.~a.~m.

Stehe nun das Sprachgefühl des Volkes der Frage, ob Grammatik oder Wörterbuch, noch so gleichgültig gegenüber: der Sprachforscher \fed{{\textbar}130{\textbar}}\phantomsection\label{fp.130} muss hier einen Unterschied machen, und zwar aus wissenschaftlichen Gründen nicht minder, als aus praktischen.

Erstens: so innig in einer Sprache Stoff und Form einander durchdringen mögen, so wesentlich verschieden sind doch Beider Functionen. Dies gilt von unseren Sprachen, denen man zugleich jene Verquickung und die feine Scheidung Beider nachrühmt. Erst recht aber wird es von jener grossen Menge der so genannten agglutinirenden Sprachen gelten, die an unveränderliche Wortstämme nach unverbrüchlichen Gesetzen saubere Formativelemente anfügen, – für jede Function immer dasselbe.

Zweitens: Die einzelsprachliche Grammatik lehrt das Zulässige, mithin das, was in jedem Augenblicke thatsächlich werden kann. Das Wörterbuch hingegen, – hierin immer auf dem positiven historischen Standpunkte fussend, – darf und kann nur besagen, was wirklich zur Thatsache geworden ist. Dies \sed{{\textbar}{\textbar}123{\textbar}{\textbar}}\phantomsection\label{sp.123} gilt in doppelter Hinsicht. Einmal bezüglich der Gebilde. Die Grammatik erklärt: die und die dürfen geschaffen werden. Das Wörterbuch besagt: die und die sind wirklich bereits geschaffen worden. Dann aber auch von den Bedeutungen. Die Grammatik lehrt: das und das ist ein für allemale die Bedeutung des einzelnen Bildungsmittels. Das Wörterbuch dagegen giebt Aufschluss darüber, in welchen besonderen Bedeutungen dies Bildungsmittel in den einzelnen Fällen angewandt wird. Ich wähle ein Beispiel aus dem Sanskrit. In der Grammatik lernen wir, dass das Suffix \textit{–in }possessive Bedeutung hat: \textit{açvín} = Pferde besitzend, \textit{hastín} = behandet, \textit{çaçín} = Hasen habend. Dass \textit{açvín} einen Reiter, nicht etwa einen Hauderer, der Dual \textit{açvínau} die indischen Dioskuren, \textit{hastín} den Elephanten und nicht etwa einen Wegweiser, \textit{çaçín} den Mond, nicht einen Wildprethändler bedeutet: dies erfahren wir aus dem Wörterbuche; in der Grammatik, auch der vollständigsten, ist dafür keine Stätte, es wäre denn in zufälligen Beispielen. So sehr verlangen die Beiden einander als nothwendige Ergänzungen.

Die wissenschaftliche Berechtigung, ja Nothwendigkeit des Wörterbuches ist, denke ich, hiermit bewiesen. Damit ist aber auch eine Aufgabe gestellt, für das Wörterbuch eine wissenschaftliche, \retro{das}{dass} heisst organisch einheitliche Form zu finden.

Auch hierbei wird jene zweifache Betrachtung der Sprache als einer Gesammtheit zu deutender Erscheinungen und als einer Gesammtheit anzuwendender Mittel in erster Reihe entscheidend sein, und es ergiebt sich darnach vorläufig folgende Eintheilung:

\fed{{\textbar}131{\textbar}}\phantomsection\label{fp.131}

I. Der Wortschatz in seiner Erscheinung. In dieser Hinsicht wiederum ist eine zweifache Eintheilungsweise denkbar und folglich, wo sie möglich ist, geboten:

a. Zu Grunde gelegt wird die Wurzel- und Stammverwandtschaft; die Anordnung ist \update{\so{etymologisch,}}{\so{etymologisch;}} alle wurzel- und stammverwandten Wörter werden in ein Rubrum geordnet.

b. Zu Grunde gelegt wird die Wortbildungsweise; die Anordnung ist \update{\so{morphologisch},}{\so{morphologisch};} alle durch die gleichen Mittel gebildeten Wörter werden zusammengestellt.

II. Der Wortschatz als Mittel zum Ausdrucke der Vorstellungen. Die Anordnung ist im systematischen Sinne \so{encyklopädisch}, das Werk ist eine \so{Synonymik}. – Damit sind nun meiner Meinung nach die durch die Natur der Sache gebotenen Möglichkeiten erschöpft; denn Reimlexika können doch hier nicht in Frage kommen.

Ich habe aber wieder zunächst die Sache ideal aufgefasst, ohne Rücksicht auf das Erreichbare und Zweckmässige. Auch daran müssen wir nunmehr denken.

\so{Zu Ia und Ib}. Die einzelsprachliche Forschung hat es nur mit dem zu thun, \sed{{\textbar}{\textbar}124{\textbar}{\textbar}}\phantomsection\label{sp.124} was im Sprachgefühle des Volkes vorhanden ist. Offenbar aber verhält sich, je nach der Beschaffenheit der Sprachen, dieses Gefühl der Etymologie und Morphologie gegenüber sehr verschieden. Je einfacher die Analyse, je durchsichtiger die Structur der Wörter ist, desto lebhafter wird es sein, und desto gewisser darf man annehmen, dass in diesem Punkte das Sprachgefühl aller Volksgenossen übereinstimme. Je veränderlicher dagegen die Wurzeln und Stämme, je unregelmässiger und schwankender anscheinend die Bildungsmittel sind, desto weniger darf man in solchen Dingen dem Sprachgefühle zutrauen. In einer wahrhaft isolirenden Sprache endlich kann höchstens vergleichende geschichtliche Forschung zu etymologischen und morphologischen Einsichten führen. Allerdings glaube ich und denke an einer späteren Stelle nachzuweisen, dass überall eine Art lautsymbolisches Gefühl diejenigen Wörter, die sowohl lautlich als auch in ihrer Bedeutung einander ähneln, wie recken und strecken, glühen, glimmen, glänzen und glitzern, miteinander verbinde, und dass es zwischen diesem eingebildeten Zusammenhang und dem geschichtlich begründeten etymologischen keinen Unterschied mache. Hier wäre allerdings ein Gesichtspunkt gewonnen, der zugleich wissenschaftlich und anscheinend praktisch ist. Ob aber mehr als anscheinend \fed{{\textbar}132{\textbar}}\phantomsection\label{fp.132} praktisch? Gerade jenes Gefühl wird je nach der Individualität von sehr verschiedener Stärke sein, nur in den grellsten Fällen bei Allen das gleiche. So dürften am Ende die etymologischen und morphologischen Wörterbücher für Einzelsprachen nur da am Platze sein, wo die Dinge besonders klar liegen, vor Allem also da, wo eine lebensfrische Agglutination in freier Bildsamkeit waltet. In anderen Fällen bleiben sie besser der sprachgeschichtlichen Forschung überlassen.

\so{Zu II}. Die lexikalische Synonymik dagegen ist überall ein unabweisbares Erforderniss; wo sie mangelt, da ist auch die Sprache nur sehr mangelhaft erkannt. Jene alphabetischen Wörterbücher, in denen die bekannte Sprache an erster Stelle steht, genügen aber, was auch sonst ihre Vorzüge sein mögen, dem Zwecke der Synonymik nur unvollkommen. Die Welt von Vorstellungen, über die ein Volk verfügt, die \update{Art}{Art,} wie es ordnet, unterscheidet, feinsinnig oder grobsinnlich darstellt: von Alledem erfährt man höchstens auf Umwegen. Ich schlage nach: Schiff, Boot, Kahn, Fähre und finde dafür überall nur dasselbe eine Wort: so weiss ich wohl, was ich auch aus der Völkerkunde erfahren könnte, dass das betreffende Volk in der Nautik noch sehr weit zurück ist. Das mag nun noch angehen. Wie aber, wenn es sich um übersinnliche Vorstellungen handelt? Hierin sind die Begriffe der Völker so verschieden, dass sich die Sprachen gar nicht Wort für Wort gegenüberstellen lassen, ohne dass die heillosesten Missverständnisse erweckt würden. In der That, ein ideales Synonymen-Wörterbuch müsste nicht nur in der Anordnung, sondern auch in der inneren Ausgestaltung eine Art nationaler Encyklopädie sein. Man müsste daraus erfahren, was und \sed{{\textbar}{\textbar}125{\textbar}{\textbar}}\phantomsection\label{sp.125} wie der Eingeborene bei jedem Worte denkt, und diesem Zwecke wird am besten, weil am objectivsten, eine reiche Phraseologie dienen.

Diese, die Lehre von der Verwendung eines jeden Wortes im Zusammenhange der Rede, ist aber ein unerlässlicher Bestandtheil eines jeden Wörterbuches, das Vollständigkeit beansprucht. Bedeutung und Gebrauch eines Wortes bedingen einander wechselseitig: weil das Wort im Unterschiede von seinen sinnverwandten diese besondere Geltung hat, darum ist es in den und den Verbindungen berechtigt; und weil es in diesen Verbindungen erscheint, darum wird ihm seine besondere Bedeutung beigelegt.

Zum Schlusse noch dies. Umsturzpläne habe ich mit den vorstehenden Betrachtungen nicht verfolgt. Die herrschende Verfassung der \fed{{\textbar}133{\textbar}}\phantomsection\label{fp.133} Wörterbücher ist praktisch berechtigt und nothwendig. Bei uns Culturvölkern kennt kein Einzelner den ganzen Wortschatz seiner eigenen Sprache, geschweige denn den einer fremden. Ohne Nachschlagebücher kommt man nicht aus, und wenn ich allen den äusseren Rücksichten Rechnung trage, die dabei mit entscheiden, so glaube ich, die systematische Synonymik habe immer noch die meiste Aussicht auf Verwirklichung. Eine solche, gut durchgeführt, müsste sogar eine anziehende Lectüre abgeben. Denn wie ein Volk lebt und denkt und empfindet, das spräche es hier in bündigster Form selber aus.

\pdfbookmark[2]{§. 17. C. Berücksichtigung zeitlicher und örtlicher Besonderheiten etc.}{II.VI.17}
\cohead{§. 17. C. Berücksichtigung zeitlicher und örtlicher Besonderheiten etc.}
\subsection*{§. 17.}\phantomsection\label{II.VI.17}
\subsection*{C. Berücksichtigung zeitlicher und örtlicher Besonderheiten in Grammatik und Wörterbuch.}

Die einzelsprachliche Forschung findet ihre Grenzen in denen der Einzelsprache, das heisst der Sprachgemeinschaft. Innerhalb dieser hat sich die Sprache mit der Zeit verändert, in der Regel gau- oder stammweise in Mundarten gespalten, zeitliche und örtliche Verschiedenheiten treten hervor. Es ist eine reine Thatfrage, inwieweit das Sprachgefühl diese Verschiedenheiten als zulässig anerkennt, ob es den Archaismus für todt erklärt, oder ihm ein Greisenleben gönnt, ob es einen Provinzialismus in den Kehricht der \update{patois}{Patois} und \update{jargons}{Jargons} wirft, oder ihm Berechtigung einräumt. Die Entscheidungen, die dieses Sprachgefühl fällt, mögen noch so launenhaft sein: die einzelsprachliche Forschung hat sich ihnen ohne Widerrede zu fügen.

Jene Grenzen aber bezeichnen nicht nur den Umkreis, der nicht überschritten werden darf, sondern auch den Raum, der nach allen Richtungen hin und in allen seinen Theilen ausgebeutet werden muss. Der einzelsprachliche Forscher soll die Sprache nicht schulmeistern, – das überlasse er der rechtsverbindlichen Entscheidung unfehlbarer \retro{Akademien,}{Akademien.} – sondern er soll sie hinnehmen, wie er sie findet. Eine Grammatik der jetzigen hochdeutschen Schrift- \sed{{\textbar}{\textbar}126{\textbar}{\textbar}}\phantomsection\label{sp.126} und Umgangssprache z.~B. sollte die gleichberechtigten mundartlichen Aussprachen des \textit{au}, \textit{ei}, \textit{r}, \textit{s}, \textit{g}, \textit{f}, \textit{w} u.~s.~w., das alterthümliche „sintemal“ und den gleichwerthigen Gebrauch von „nachdem“ in Österreich berücksichtigen, ja sogar jene norddeutschen Unarten, an die man sich nachgerade gewöhnt hat, wie die Plurale „Jungens, Kerls“. Wenn wir uns über die Sprachsudeleien unserer Zeitungen ärgern, so wollen wir nicht vergessen, dass ein grosser Theil \fed{{\textbar}134{\textbar}}\phantomsection\label{fp.134} unserer gebildeten Landsleute längst \retro{dagegen}{dagegegen} abgestumpft ist und Anklänge an’s Plattdeutsche oder an das Neuhebräische achtloser hingehen lässt, als einen Bavarismus oder Austriacismus von althochdeutschem Adel.

Hier finden nun doch jene Sprachakademien, wie sie andere Länder besitzen, ihre volle Rechtfertigung. Ihre Einrichtung verhält sich zu unseren Zuständen, wie ein sachverständiges Collegium zu einem Parlamente, „wo rohe Kräfte sinnlos walten“, wo nach Majoritäten entschieden wird, und, um nochmals mit Schiller zu reden:

\begin{center}
„Verstand ist stets bei Wenigen nur gewesen“.
\end{center}

Was wir gute, correcte Sprache nennen, ist ebenso der Mode unterworfen, wie die Kleidertrachten und die gesellschaftlichen Gewohnheiten, übt auch auf die Mehrzahl der Gebildeten einen ähnlich knechtenden Zwang. Aus lauter Angst, etwas Verpöntes zu sagen, verzichtet man lieber auf einen grossen Theil des Erlaubten. So sind es weniger die unwillkommenen Einschmuggelungen und Neuerungen, als die Einbussen an altem, \update{ächtem}{echtem} Sprachgute, die wir zu fürchten haben, und eine Akademie sollte vor Allem dafür sorgen, dass lebendig bleibe, was lebensberechtigt ist. Jene Neuerungen kann sie ohnehin nicht verhüten; die gelesensten unter den französischen Romanschriftstellern schreiben die Sprache ihrer Zeit, nicht die eines Voltaire oder Rousseau, machen Schule so gut wie diese, mag die Pariser Akademie dazu sagen, was sie will. \update{Büreau\-kratische}{Bureau\-kratische} Drillung lassen sich nur junge Cultursprachen gefallen.

Nun hat sich wohl der Akademiker auf den Standpunkt des Sprachforschers, nicht aber der Sprachforscher auf den Standpunkt des obrigkeitlich verfügenden Akademikers zu stellen. Hat die Akademie entschieden, so muss er sich ihrem Urtheile unterwerfen; hält er die gefallene Entscheidung für falsch, so mag er ihre Unrichtigkeit nachweisen; umstossen kann er sie nicht, gültig bleibt sie, bis eine andere kommt, und wenn sie noch so albern wäre. Verzeichnen soll er aber auch das Abweichende, wenn es aus leidlich guten Quellen stammt, z.~B. provinzielle oder persönliche Spracheigenthümlichkeiten eines angesehenen Schriftstellers, an die sich voraussichtlich schon ein weiter Leserkreis gewöhnt haben wird. Der Akademiker mag dann prüfen, ob die Neuerung wirklich schon in das Sprachgut übergegangen sei, und wenn sie das ist, so muss er sie anerkennen. So ist die Sprech- und Schreibweise „unpass, unpässlich“ der Etymologie zuwider und sinnlos, und doch so verbreitet, dass man ihr nicht mehr \sed{{\textbar}{\textbar}127{\textbar}{\textbar}}\phantomsection\label{sp.127} die Anerkennung versagen darf und höchstens \fed{{\textbar}135{\textbar}}\phantomsection\label{fp.135} fragen muss, ob das etymologisch berechtigte „unbass, unbässlich“ noch in einem Theile unseres Vaterlandes vom Sprachbewusstsein festgehalten werde. Ähnlich ist es mit dem „Alpdrücken“ und „Albdrücken“, mit „allmälig“ und „allmählich“, mit dem oberdeutschen „gieng, fieng, hieng“, neben dem verbreiteteren „ging, fing, hing“, mit „frug“ und „fragte“, „fünfzig“ und „funfzig“, „eilf“ und „elf“ und vielen anderen Fällen, wo sich das Sprachgefühl der Gebildeten entweder gleichgültig verhält oder landschaftenweise theilt.

Nicht ohne Grund redet man von bühnenmässiger Sprache und gesteht dieser eine Art massgebender Bedeutung zu. In der That ist das Theater die einzige Stätte, wo eine dialektfreie, von Allen gleichmässig gutgeheissene Aussprache gepflegt wird, und eine Akademie wird sich eher nach dem Bühnenbrauche richten, als die Bühne nach der Akademie. Im Théâtre français wird eine Aussprache des Französischen gepflegt, die im ganzen Lande für classisch gilt, und derengleichen wohl noch heute in Predigten und gerichtlichen Reden gehört wird.

\pdfbookmark[2]{§. 18. D. Sprache und Schrift.}{II.VI.18}
\cohead{§. 18. D. Sprache und Schrift.}
\subsection*{§. 18.}\phantomsection\label{II.VI.18}
\subsection*{D. Sprache und Schrift.}

Die wissenschaftliche Schriftenkunde schlägt nur zum Theil in die Sprachwissenschaft ein. Handelt es sich um die Frage: Wie kamen die Menschen zur Erfindung der Schrift? welches sind die Vorläufer, welches die ältesten Formen der Schrift? so dürfte die Antwort in Kürze dahin lauten:

1. Dem Menschen, auch dem rohesten, wohnt ein Trieb zu bildnerischem Schaffen inne. Der äussert sich in jenen vielbewunderten Zeichnungen der Buschmänner, in den fratzenhaften Ahnen- und Götzenbildern der Papuas, wie in den rohen Bildern, mit denen bei uns zu Lande die Kinder die Wände bemalen.

\largerpage[-1]2. Gefördert wird dieser Trieb durch das eitele Gefallen, sich irgendwo verewigt zu wissen. Daher die Vorliebe für dauerhafte Stoffe. Ob Steppennomaden einen Steinhaufen errichten, oder ob ich meinen Namen in die Rinde einer Buche eingrabe: immer ruht im Hintergrunde derselbe Gedanke: Non omnis moriar, es ist dafür gesorgt, dass ich nicht vergessen werde.

3. Die eigene Vergesslichkeit haben wir aber nicht weniger zu befürchten, als die anderer Leute. Was wir uns merken wollen, dafür \fed{{\textbar}136{\textbar}}\phantomsection\label{fp.136} suchen wir ein Merkmal oder schaffen es uns selbst, und dem Beauftragten, der sich für uns etwas merken soll, geben wir sicherheitshalber ein Merkzeichen mit. So üben es Kaffernvölker mit den Boten, die sie an Nachbarstämme senden. Sie schneiden eine Anzahl Ruten, soviele als Mittheilungen zu machen sind, und angesichts einer jeden lernt der Bote einen Theil seines Auftrages auswendig. Am Orte \sed{{\textbar}{\textbar}128{\textbar}{\textbar}}\phantomsection\label{sp.128} seiner Bestimmung wird dann das Ruthenbündel seinem Gedächtnisse zu Hülfe kommen. Hier, wie bei den Knoten, die wir in das Taschentuch oder in die Uhrkette schlingen, ist die Bedeutung des Merkzeichens von Fall zu Fall verschieden.

4. Ein Fortschritt ist es, wenn den Zeichen ständige Bedeutungen beigelegt werden. So war es mit den Knotenschnüren, Quipus, der altperuanischen Staatshistoriker; und ähnlicher Knotenzeichen wollen sich auch die Chinesen vor der Erfindung der Schrift bedient haben. So war es und ist es wohl stellenweise noch jetzt in Europa mit den Kerbhölzern, die unter den Bauern vollbeweisende, unverfälschbare Schuldurkunden ersetzen, und mit jenen stab- oder bandförmigen Aufzeichnungen, womit in Nordamerika die indianischen Sänger und Erzähler ihrem Gedächtnisse zu Hülfe kommen.

5. Sind solche Zeichnungen erkennbar, bildlich oder symbolisch, stellen also die Zeichen Nachahmungen des zu Bezeichnenden dar, so haben wir die Vorläufer der ältesten bekannten Schriften. Derart sind z.~B. jene vielbewunderten Pictographien nordamerikanischer Indianer, und auch die ältesten chinesischen Schriftdenkmäler weisen Ähnliches auf. Da werden die zwei Wörter, die bedeuten sollen: „Der Sohn schnitzt ....“ zu einem Zeichen verbunden, das ein Kind mit einem Messer in der Hand darstellt.

Alles dies liegt noch vor der Schrift und ausserhalb derselben, im günstigsten Falle an ihrer Schwelle. Giebt es denn aber eine solche Schwelle? Wo fängt die Schrift an? wo hören Bild und Symbol auf? Die Antwort lautet: Bei der Lesbarkeit. Bild und Symbol kann man deuten, aber nicht lesen. Es zeichne Jemand ein Haus und daneben einen Baum und fordere uns nun auf, das zu lesen. Der Eine sagt: „Ein Haus und ein Baum“; der Andere: „Ein Haus neben einem Baume“, ein Dritter vielleicht wieder anders, – und wenn die Leute verschiedensprachig sind, so redet jeder in seiner Sprache. Jeder hat Recht, nur nicht darin, dass er gelesen habe: er hat eben nur gedeutet. Wo stehen die Artikel \fed{{\textbar}137{\textbar}}\phantomsection\label{fp.137} „ein“, wo die Conjunction „und“ oder die Präposition „neben“? Was sagt uns, dass das Haus eher zu nennen sei, als der Baum und nicht umgekehrt? Das Wesentliche ist dies, dass sich die Zeichnung durch den Gesichtssinn ohne Weiteres an unsern Geist gewendet, und dieser ihren Inhalt in Sprache übertragen hat, mit anderen Worten, dass ihre Darstellung nicht sprachlich, sondern sachlich war. Die Schrift dagegen stellt Sprache dar, ist nur durch Vermittelung der Sprache zu verstehen. Überraschend grell zeigt sich dies bei den Ziffern, zumal in Fällen wie 18, lateinisch duodeviginti, 93, französisch quatre-vingt-treize. Mag man da auch von einem Lesen reden, in der That ist es doch nur ein Deuten. Und doch ist der Unterschied zwischen Zeichen, die zur Sinnlichkeit, und solchen, die zum reinen Verstande reden, so gross, wie man nur irgend verlangen kann.

Noch eines besonderen Unterschiedes zwischen Schrift und Bild müssen \sed{{\textbar}{\textbar}129{\textbar}{\textbar}}\phantomsection\label{sp.129} wir gedenken. Die Schrift, auch die Bilderschrift, stilisirt, die Zeichnungen müssen sich dem einmal angenommenen Ductus fügen und in die Zeilen einreihen. So erscheint in den ägyptischen Hieroglyphen der Löwe nicht grösser, als die Eule oder die Schwalbe, und bei aller Schönheit der Zeichnungen giebt sich doch der Text durch sein zeilenmässiges Aussehen und durch die Gruppirung der einzelnen Zeichen ohne Weiteres als solcher, also als Schriftstück zu erkennen. Das Gleiche gilt von den ältesten chinesischen und zumal von den keilförmigen assyrisch-babylonischen Schriften.

Nur soviel zur Frage: Wann und wie kommt die Sprache zur Schrift? Zur Beantwortung musste jetzt die Psychologie, jetzt die Culturanthropologie herbeigezogen werden; die Sprachwissenschaft hatte erst im letzten Augenblicke mit dreinzureden.

Noch weniger wird die Sprachwissenschaft von jenen anderen Fragen berührt: auf welchen Stellen und zu welchen Zeiten die Menschheit zur Erfindung \update{ächter}{echter} Schriften gelangt, und wie diese dann weiter verbreitet und verändert worden seien? Gewiss hat die Eigenart der Sprachen auf die Entwickelung der Schrift bei den verschiedenen Völkern einen Einfluss geübt; dann aber dient eben die Sprachkunde als Hülfswissenschaft der Schriftkunde. Der Semit mochte sich bei der unvollkommenen Vocalisation seiner Schrift beruhigen. Wenn der Türke, der Uigure, der Perser, der Malaie das semitische Schriftsystem fast unverändert auf ihre so gar anders gebauten Sprachen anwandten, so war das doch eine tadelns\fed{{\textbar}138{\textbar}}\phantomsection\label{fp.138}werthe Trägheit. Die Griechen, die Kalmüken, die Mandschu und – wenn die indischen Schriften semitischen \update{Ursprunges}{Ursprungs} sein sollten – die Inder haben es dagegen verstanden, die fremdartige Schrift nach den Anforderungen ihrer Sprachen umzugestalten. Die Chinesen mit ihrer einsylbig isolirenden Sprache thaten weise daran, bei der Wortschrift stehen zu bleiben. Die hat sich thatsächlich als eine Art Pasigraphie bewährt, zunächst für das dialektisch gespaltene Riesenreich, dann auch für die culturverwandten Nachbarn in Japan, Korea und Annam. Ein Jeder liest und schreibt dieselben Zeichen und spricht sie seiner Zunge gemäss aus. Einsylbig und isolirend ist auch die annamitische Sprache, und doch in Grammatik und Wortschatz sehr weit von der chinesischen verschieden. Als nun die Annamiten der chinesischen Anregung folgten, sich gleichfalls eine Schrift schufen, so lag es nahe, dass sie \retro{das}{dass} System der Wortschrift wählten und nur neue Zeichen nach chinesischem Muster erfanden. Das Japanische ist formenreich aber lautarm. Jede Sylbe besteht ursprünglich entweder aus einem einfachen Vocale oder aus einem Consonanten sammt Vocale; man darf zweifeln, ob die Sprache zur Zeit der Schriftschöpfung mehr als siebzig verschiedene Sylben gekannt habe. So wies die Eigenart der Sprache geradezu darauf hin, eine Sylbenschrift herzustellen, einen Theil der chinesischen Wortzeichen als Sylbenzeichen zu verwerthen. Seltsam \sed{{\textbar}{\textbar}130{\textbar}{\textbar}}\phantomsection\label{sp.130} und eigentlich schön entwickelte sich die Schrift bei den Koreanern. Die hatten durch buddhistische Sendlinge das indische Buchstabensystem kennen gelernt, während sonst ihre Bildung auf chinesischer Grundlage ruht. Die Sprache \update{erforderte}{erfordert} ihrer Natur nach eine Lautschrift: dies sprach für das indische Muster. Für das chinesische aber sprach die Gewohnheit, senkrechte Zeilen, Pinselductus und in Rechtecke eingefügte zusammengesetzte Zeichen zu sehen, – eine chinesisch geschulte Aesthetik. Beides wusste man sinnig zu vereinigen und dabei noch das indische Vorbild durch Vereinfachung zu übertreffen. Es dürfte nicht möglich sein, Buchstaben- und Sylbenschrift glücklicher \update{mit einander}{miteinander} zu verquicken.

Ganz abseits von unserem Wege liegen jene für praktische Zwecke erfundenen \so{Kunstschriften}, die Kurz-, Geheim-, Blindenschriften u.~s.~w. Mit Spannung müssen wir aber den zu erhoffenden Vervollkommnungen des Phonographen folgen. Erreicht je diese geniale Erfindung ihr Ideal, stellt sie Laute und Töne in vollkommenster Reinheit dar, so gewinnen wir damit ein unschätzbares Hülfsmittel. Schockweise können wir dann \fed{{\textbar}139{\textbar}}\phantomsection\label{fp.139} in unseren Studierzimmern eingeborene Sprachmeister beherbergen, die uns vorplaudern, so oft wir wollen. Doch das sind zur Zeit noch Träumereien.

Ob und inwieweit die Sprache durch die Schrift beeinflusst werden könne, ist nicht hier zu erörtern, sondern Sache der Sprachgeschichte. Für jetzt aber interessiert uns die Frage: \so{Wie werden die Sprachen von den ihnen zugehörigen Schriften aufgefasst?}

Zunächst: \so{bei welchen Einheiten}? Satzschriften giebt es nicht, und zwar aus leicht erklärlichen Gründen, denn selbst in der einfachsten Sprache sind unzählig viele verschiedene Sätze möglich.

\largerpage[1]\so{Wortschriften} kennen wir nur zwei: die chinesische und die von ihr abgeleitete annamitische. Sollten die alten Inschriften und Bücher der Mexikaner und Yukateken in einer Art Schriften dieser Gattung verfasst sein, so darf man von vornherein sagen: Der Versuch war unvollkommen und musste es nach der Natur jener formenreichen Sprachen bleiben.

Ein Mittelding zwischen Wort- und Lautschrift ist jenes System, das man nach seinem Hauptvertreter das \so{Hieroglyphische} nennen mag. Hier können die Wörter bald durch blosse Bilder oder Symbole, bald durch Lautzeichen, bald durch eine Verbindung beider ausgedrückt werden; zur Bezeichnung der Affixe dienen theils Symbole, theils Buchstaben oder Sylbenzeichen. Nächst den hieroglyphischen und hieratischen Schriften der alten Aegypter gehören die ältesten Keilschriften hierher. Das System ist inconsequent, daher unvollkommen, und man thut der chinesischen Schrift unrecht, wenn man sie eine hieroglyphische nennt. Allerdings bestehen etwa neun Zehntel ihrer Zeichen aus Verbindungen ideographischer Bestandtheile mit phonetischen. Aber der Typus der \sed{{\textbar}{\textbar}131{\textbar}{\textbar}}\phantomsection\label{sp.131} Wortschrift ist doch in vollster Reinheit gewahrt und in erstaunlicher Vollkommenheit ausgebildet. \sed{Über die sogenannten Hieroglyphen der mittelamerikanischen Culturvölker, der Mayas und Azteken, herrscht noch Zweifel. An Entzifferungsversuchen fehlt es nicht; aber nichts Geringeres als das Schriftsystem selbst ist unter den Forschern streitig. Bekanntlich hat der blödsinnige Fanatismus der ersten Missionare den grössten Theil dieser Literaturen vernichtet und es nicht für der Mühe werth erachtet, der Welt sichere Kunde von dem Schriftwesen der Teufelsbücher zu hinterlassen.}

Es ist erklärlich, wenn sich die Lautschrift zunächst an das \update{Greifbarere}{Greifbare} hält, also an die \so{Sylbe}. Selbst die altsemitischen Buchstaben waren doch implicite Sylbenzeichen, wenn auch als solche mehrdeutig. Die Lautanalyse war gelungen; der Fehler lag aber darin, dass man einen Theil der gewonnenen Elemente unbeachtet liegen liess. Man wird an jene Geistesart erinnert, die zum Zerlegen geschickter ist, als zum Aufbauen. Das noch mangelhaftere Tifinagh der Berbern verschmäht nun gar auch die dürftigsten Andeutungen der Vocale; es ist als wenn wir \fed{{\textbar}140{\textbar}}\phantomsection\label{fp.140} etwa die Wörter Lob, lebe, labe, Liebe, Leib, Elbe durch blosses lb schreiben wollten. Jene bekannten zwei Schrifterfinder, der Tscheroki-Indianer \textsc{Sequoyah} und der Vei-Neger \textsc{Momoro Dualu Bukere}, erfanden für ihre Sprachen Syllabare. Das Gleiche thaten die Tungusenvölker der Kitan und Aisin, als sie zeitweilig China beherrschten.

\largerpage[1]Von einer eigentlichen \so{Buchstabenschrift} kann erst dann die Rede sein, wenn möglichst jeder von der Sprache unterschiedene Laut sein Zeichen erhält. So stellen die Vocalzeichen der syrischen, hebräischen, arabischen und äthiopischen Schrift einen wesentlichen Fortschritt dar, und man muss bekennen, dass für semitische Sprachen diese Art der Vocalschreibung besonders sachgemäss war. Nun wird aber natürlich bei der Analyse die Sylbe leichter gewonnen, als der einzelne Laut\footnote{Ein zweijähriges Kind sagte zu mir: „Sprich mal Schas!“ Schas! „Und nun sprich Sef!“ Sef! „Und nun sprich Fa!“ Fa! „Nun sprich Schasséfa!“ – sollte heissen Josepha.}, und mehr oder minder wird wohl auch der Consonant vom benachbarten Vocale beeinflusst. So mag es sich erklären, wenn die Türken, Mongolen und Mandschu in gewissen Fällen verschiedene Zeichen anwenden, je nachdem auf den Consonanten ein harter oder weicher Vocal folgt, und wenn die Mandschu bei solchen Gelegenheiten den weichen Vocal nicht mehr mit dem Weichheitszeichen versehen. Ganz syllabarisch ist die Vocalbezeichnung in den Schriften indischen Systems und Ursprungs. Da wird das kurze \textit{ă} oder sein Vertreter als selbstverständliches Zubehör des Consonanten behandelt, gut noch, wenn consonantische Ligaturen oder besondere Zeichen (\textit{virâma} der Inder, \textit{pangolat} der Batta) seine Abwesenheit andeuten. Schlimm aber ist es, wie dabei die Sylbentheilung weder der Aussprache noch der Etymologie Rechnung \sed{{\textbar}{\textbar}132{\textbar}{\textbar}}\phantomsection\label{sp.132} trägt. Sanskrit \textit{mantra}, Spruch, Wurzel \textit{man}, denken, wird \textit{ma-ntra} {\skt ma.ntra} abgetheilt.

Sehr viele Schriften vernachlässigen die \so{Wortabtheilung}. So die altsemitischen und griechischen, die indischen und die japanische. Bei letzterer mag der Grund äusserlicher Art sein und im chinesischen Vorbilde liegen. In anderen Fällen ist aber doch die Sache bedeutsamer und scheint auf dem Sprachgefühle selbst zu beruhen, das im Satzganzen noch nicht zu einer strengen Wortscheidung gelangt war. Wo, wie im Sanskrit, Aus- und Anlaut benachbarter Wörter einander beeinflussen, wo ferner vielsylbige Composita gebräuchlich sind, und das \fed{{\textbar}141{\textbar}}\phantomsection\label{fp.141} Verbum selbst enklitisch (unbetont) hinter sein Object gefügt wird, da ist die Worttrennung schon eine That der Abstraction. Umgekehrt war den Uralaltaiern die Worttrennung erleichtert, einmal durch das Gesetz der Vocalharmonie, und zweitens durch die unwandelbare Setzung des Haupttones auf die letzte oder erste Sylbe.

Fast überall zeigt bei Lautschriften der Schriftbrauch die Neigung, sich \so{orthographisch zu festigen}. Erst hatte ein Jeder die gegebenen Zeichen angewandt, so gut er es verstand, der eigenen Aussprache und dem eigenen Gehöre folgend. Dann erhob wohl die Meinung der Lesenden gewisse Schriftsteller zu gemeingültigen Mustern, und auch ohnedem fliesst dem Schreiber das, was er zu lesen gewöhnt ist, unwillkürlich in die Feder. So ragt denn schliesslich über den einzelnen Mundarten, von diesen auch im Punkte der Rechtschreibung befreit, eine Schriftsprache hervor, und von dieser gilt in ganz besonderem Sinne der Satz: Litera scripta manet. Sie ist nämlich starr im Vergleich zu der \update{immer flüssigen}{immerflüssigen} Umgangssprache und gleicht jenen Rechtssatzungen, die Mephisto verhöhnt als eine ewige, vererbliche Krankheit. Die \so{historischen Orthographien}, von denen ich hier rede, sind den Neuerern ein Dorn im Auge, den ABC-Schützen ein bitteres Leid. Und in der That, wenn man sie mit der lebendigen, frisch fortschreitenden Volkssprache vergleicht, so nehmen sie sich aus wie Verkörperungen eines stupiden passiven Widerstandes. Was giebt es Tolleres, als wenn im Englischen jene vier Buchstaben \textit{ough}

~~~~wie \textit{ū} in through,

~~~~wie \textit{ō} in though,

~~~~wie \textit{å} in thought, ought, brought,

~~~~wie \textit{au} in plough,

~~~~wie \textit{of}, \textit{öf} in rough, enough, cough lauten,

\begin{sloppypar}\noindent oder wenn im Französischen dieselbe Sylbe: \textit{sans}, \textit{sang}, \textit{s’en}, \textit{sens}, \textit{sent}, \textit{cent}, \mbox{(\textit{per-})\textit{çant}}, \mbox{(\textit{pas-})\textit{sant}} geschrieben wird?\end{sloppypar}

Solche Dinge sehen aus wie wüster Unfug, und man begreift, wie in England eine ansehnliche Partei seit Jahrzehnten unverzagt für die Einführung einer \sed{{\textbar}{\textbar}133{\textbar}{\textbar}}\phantomsection\label{sp.133} phonetischen Orthographie kämpft. Sie hat vor ihren deutschen Strebensgenossen dreierlei voraus: erstens stramme Parteizucht, während bei unseren \retro{Neuerern}{Neueren} fröhliche Anarchie zu herrschen scheint, – zweitens eine verständige Politik, indem sie sich vorzugsweise an jene wendet, die den Jammer der historischen Orthographie am \fed{{\textbar}142{\textbar}}\phantomsection\label{fp.142} Schmerzlichsten empfinden, während sich bei uns die Gelehrten gegenseitig mit ihren neuen Orthographien überraschen, – endlich drittens einen Gegenstand, gegen den anzukämpfen der Mühe lohnt, während die \retro{deutsche}{deutschǝ} Orthographie schon seit einem Jahrhunderte wohl manche Freiheiten, aber nur geringe Schwierigkeiten bot.

So günstig scheinen in England die Dinge zu liegen; und doch ist noch keine Aussicht, die Reformer obsiegen zu sehen. Bibeln, Zeitungen, Volksschriften, Unterhaltungsbücher, Gesetze werden nach wie vor in der alten Orthographie gedruckt, und die Gelehrten hüten sich wohl, ihre Werke durch launenhafte Verkünstelungen unlesbar zu machen. Denn das ist es eben. Längst hat sich an eine andere Sprache das Ohr, an eine andere das Auge gewöhnt, und beide wollen in ihren Gewohnheiten nicht gestört sein. In einem lesekundigen Volke beansprucht auch die Schriftsprache alle Rechte einer Volkssprache, – sie ist eben nur die zweite dieser Art. Je weiter nun der Abstand zwischen der Orthographie und den Lauten der mündlichen Rede, desto gewaltiger müsste der Sprung sein, der die erstere wieder an die Seite der anderen brächte. Sprünge aber duldet die Geschichte nicht gerne, und wo sie ihr abgetrotzt werden, da weiss sie sich zu rächen. Strafe genug wäre es nun doch für die Umsturzmänner, wenn mit einem Male alle jene Schöpfungen der Vergangenheit, die eben noch die Gegenwart belebten und bereicherten, in antiquarische Todtenkammern wandern müssten. Die bisherigen Verbesserungen der deutschen und französischen Orthographie wurden vom Volke geduldet, weil sie sich schrittchenweise einführten, und den meisten meiner Landsleute wird es so gehen wie mir, dass ihnen ein \sed{deutsches} Buch aus der ersten Hälfte des achtzehnten Jahrhunderts mit seiner Consonantenverschwendung noch immer heimischer vorkommt, als etwa ein Grimm’sches oder Schleicher’sches mit seiner zur Schau getragenen Sparsamkeit.

\largerpage[-1]
Noch zwei besondere Vorzüge sind aber den historischen Orthographien nachzurühmen. Erstens stehen sie, wie schon angedeutet, oberhalb der jeweiligen Einzelmundarten, während phonetische Orthographien doch immer nur der Aussprache eines Theiles der Gebildeten folgen können. Jetzt hat auch Schweden seine orthographische Reformpartei. Die verwirft u.~A. das \textit{hj} und \textit{hv}, schreibt \textit{vad}, \textit{vem}, \textit{jerta}, \textit{jul} statt \textit{hvad} (was), \textit{hvem} (wer), \textit{hjerta} (Herz), \textit{hjul} (Rad) u.~s.~w., während doch noch viele Schweden das \textit{h} deutlich hören lassen, ebenso wie vieler Orten \fed{{\textbar}143{\textbar}}\phantomsection\label{fp.143} in England bei \textit{which} = welches, das durch Aspiration unterschieden wird von \textit{witch} = \update{Hexe.}{Hexe,} \sed{\corr{1901}{\textit{weather}}{wether} = Wetter, von \textit{whether} = ob, \textit{were} = waren,} {\textbar}{\textbar}134{\textbar}{\textbar}\phantomsection\label{sp.134} \sed{wäre, von \textit{where} = wo u.~s.~w.} Warum bevorzugen nun die Verbesserer gerade die \update{nachläs\-sigere}{nachläs\-sige} Aussprache?

Der zweite Vorzug, dem ersten im Grunde verwandt, liegt allerdings auf theoretischer, und zwar auf sprachgeschichtlicher Seite; er wird also an späterer Stelle zu besprechen sein.

Dass sich der Sprachforscher der zeit- und landesüblichen Schreibweise der fremden Sprache zu fügen und nur die etwaigen Abweichungen der Aussprache anzugeben hat, liegt auf der Hand. Wie soll er es aber mit den \so{fremden Schriften} halten? Folgendes scheint mir das Richtige:

1. Wortschriften und solche Lautschriften, die die Aussprache nur ungenügend anzeigen, müssen für den Unkundigen durch eine beigegebene Transscription ergänzt werden.

2. Zureichende Lautschriften, d.~h. solche, aus denen die Laute der Sprache entweder unmittelbar oder vermittels gewisser Regeln zu erkennen sind, bedürfen der Beigabe einer Umschreibung nicht. Die Urschrift ist in der Regel vorzuziehen, schon um der Übung willen. Doch können Ersparnissrücksichten für die Wahl der Transscription sprechen. \sed{Ganz feste Orthographien haben wohl die wenigsten Völker; bei manchen, wie die Koreaner, herrscht die wildeste Anarchie: Jeder schreibt, wie es ihm einfällt. Aber auch das kann seinen Werth haben. Denn Jeder wird doch wahrscheinlich nach Möglichkeit so schreiben, wie er es zu sprechen und zu hören gewohnt ist: Die mundartlichen Verschiedenheiten kommen zur Geltung, und wie wichtig können diese für die wissenschaftliche Beurtheilung der Sprache werden. Man nehme also das unvermeidliche Ungemach dankbar hin, verzeichne die Dinge, wie sie sich bieten, und sehe zu, ob nicht am Ende doch in der scheinbaren Wildniss Gesetzmässigkeit zu entdecken und manche gute Frucht zu pflücken ist.}

3. Semitische Literatursprachen sollten je in der ihnen eigenen oder in einer anderen semitischen Schrift wiedergegeben werden. Ein Alphabet, das Vocale und Consonanten als gleichwerthig behandelt, schickt sich schlecht für Sprachen semitischen Baues.

4. Schriften, deren Entzifferung noch nicht abgeschlossen ist, durch Transscriptionen zu ersetzen, ist wohl in der Regel bedenklich. Jedenfalls muss die Umschreibung die Orthographie des Originals genau wiedergeben, also auch anscheinende Varianten unterscheiden. Hierin ist die Geschichte der altbaktrischen Forschungen lehrreich.

5. Kommt es auf dialektische Besonderheiten der Aussprache an, so ist selbstverständlich eine genaue Lautschreibung der blossen Transscription vorzuziehen.

6. Wo das Schriftwesen besonders verwickelt oder gar der Willkür unterworfen ist, wie das Japanische, da wird am Besten die Schriftlehre von der \sed{{\textbar}{\textbar}135{\textbar}{\textbar}}\phantomsection\label{sp.135} Sprachlehre getrennt gehalten, erst die Sprache in Transscriptionen gelehrt und geübt, und dann zur Lesekunst verschritten.

7. Die Schriftsysteme der Lautphysiologen sind nur in soweit zu empfehlen, als es sich um die Hervorbringung und den Klang der Laute handelt. Wo sie aber Laute als zusammengesetzte darstellen, die in der \fed{{\textbar}144{\textbar}}\phantomsection\label{fp.144} Sprache selbst als einfache gelten, wie dies oft bei \textit{tš}, \textit{dž}, zuweilen bei \textit{ts}, \textit{dz}, \textit{tr} u.~a.~m. der Fall ist: da halte man sich an die einheimische \update{Auffassung,}{Auffassung} und wähle womöglich einfache Zeichen. Wenn also z.~B. die Sprache im Anlaute sonst nur einfache Anlaute kennt, so behandelt sie auch solche Consonanten, wo sie anlauten, als einfache. Ebenso dann, wenn sie sie in anderen Hinsichten als einfache behandelt, z.~B. nicht sie, sondern nur die anderen Consonantenverbindungen „Position machen“ lässt. Darum gilt sanskrit \sed{palatales \textit{k’}, jetzt \textit{tš} gesprochen,} und slavisches \textit{č} = \textit{tš} als \so{ein} Laut, und gelten umgekehrt lateinisches \textit{x} und griechisches ξ, ψ als Doppellaute.

\sed{8. In historischen Orthographien werden oft gleiche Laute in verschiedenen Wörtern mit verschiedenen Zeichen geschrieben, wohl auch umgekehrt die nämlichen Zeichen bald so, bald anders ausgesprochen. Da ist es wichtig, den ursprünglichen Lautwerth zu ermitteln. So gewinnt z.~B. das Siamesische eine ganz andere phonetische Gestalt, wenn man prüft, wie es seine Schriftzeichen zur Umschreibung indischer Fremdwörter verwendet hat. Offenbar gehören solche Untersuchungen der sprachgeschichtlichen Forschung an; aber ihr Zweck ist immerhin, die Einzelsprache so kennen zu lernen, wie sie war, als sie sich zur Literatursprache festigte. Was damals das Ohr vernahm, bietet sich heute nur im Abbilde dem Auge des Lesers, wo aber gelesen wird, da nimmt eben auch das Auge am Sprachbewusstsein Theil, Laut- und Schriftbilder sammeln sich in zwei parallelen Inventarien, und die optische Sprache ist ebenso thatsächlich, ist ebensogut eine lebende Sprache, wie die akustische.}

