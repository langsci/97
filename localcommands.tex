%add all your local new commands to this file

\newcommand{\smiley}{:)}

\newcommand{\sed}[1]{{\color{lsLightWine}#1}}
\newcommand{\fed}[1]{{\color{lsDarkBlue}#1}}
\newcommand{\edins}[1]{{\color{lsMidBlue}#1}}
\newcommand{\othersrc}[1]{{\color{gray}#1}}
\newcommand{\edtext}[1]{{\color{lsMidBlue}#1}}

\newcommand{\inlineupdate}[2]{\fed{#1}\edins{{\textbar}{\textbar}}\sed{#2}}

%\newcommand{\update}[2]{\todo[color=cyan]{\tiny \textsuperscript{1891}#1}\sed{#2}}
%\newcommand{\retro}[2]{\todo[color=cyan]{\tiny \textsuperscript{1901}#2}\sed{#1}}
%\newcommand{\retro}[2]{\marginpar{\tiny \textsuperscript{1901}#1}{\color{cyan}#2}}
%\newcommand{\update}[2]{\marginpar{\tiny \textsuperscript{1891}#2}{\color{cyan}#1}}

%\newcommand{\corr}[3]{\marginpar{\vspace{-5pt}\color{black}\raggedright \scriptsize#3\linebreak \tiny#1}{\color{orange}#2}}
\newcommand{\arbup}[4]{\marginpar{\vspace{-5pt}\color{black}\raggedright \scriptsize#4\linebreak \tiny#1}{\color{#2}#3}}
\newcommand{\corr}[3]{\arbup{#1}{lsMidBlue}{#2}{#3}}
\newcommand{\retro}[2]{\arbup{1901}{lsDarkBlue}{#1}{#2}}
\newcommand{\update}[2]{\arbup{1891}{lsLightWine}{#2}{#1}}
%\newcommand{\ain}{\<\setarab \novocalize \tiny `>}
\newcommand{\arabictext}[1]{\<\setarab \fullvocalize #1>}
\newcommand{\ain}{\texttt{{Ꜣ}}\hspace*{-.25mm}}
\newcommand{\AIN}{\texttt{\hspace*{-.35mm}\textsl{Ꜣ}}\hspace*
{-.25mm}}%for use after f

\renewcommand{\labelitemi}{–}

\newenvironment{styleAnmerk}{\footnotesize}{}

\newenvironment{register}{\begin{multicols}{2}\small \setlength{\parindent}{-2mm}}{\end{multicols}}

\newenvironment{inhaltsverzeichniss}{\small}{}

\selectlanguage{german}

\setcounter{secnumdepth}{-1}
\titleformat{\chapter}[block]{\centering\normalfont\LARGE\bfseries}{}{1em}{}
\titleformat{\section}[block]{\centering\normalfont\Large\bfseries}{}{1em}{}
\titleformat{\subsection}[block]{\centering\normalfont\bfseries}{}{1em}{}

\renewcommand{\tikzmark}[2]{\tikz[overlay,remember picture,baseline=(#1.base)] \node (#1) {#2};}

\newcommand{\contentchap}[2]{#1 & \multicolumn{2}{b{0.8\linewidth}}{Capitel. \so{#2}} \\}
\newcommand{\contentteil}[1]{\multicolumn{3}{b{0.8\linewidth}}{#1} \\}
\newcommand{\contentnumsec}[3]{ & #1 & #2 \dotfill & \hfill \pageref{#3} \\}
\newcommand{\contentsec}[2]{ & \multicolumn{2}{b{0.8\linewidth}}{\hangindent=0.7cm #1 \dotfill} & \hfill \pageref{#2} \\}

\setlength\marginparwidth{13mm}

\newcommand{\pushbar}{\hspace*{19mm}\LARGE {\textbar}}
